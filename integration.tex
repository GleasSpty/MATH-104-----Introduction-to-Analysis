\chapter{Integration}\label{chp5xx}

So, first things first---fuck the Riemann integral.  Seriously.  The only argument pro-Riemann integral is that it is easier.  What a ridiculous argument.  This is math, dude.  If you choose to do things because they're easy, you're in the wrong subject.  Moreover, I would argue that this is not even true---if you set things up right, you can literally \emph{define} the (Lebesgue) integral to be the area (measure) under the curve.  Or, if you prefer, you can take a limit over the size of a partition of the sum of the areas of the rectangles corresponding to the subsets of the partition (the Riemann integral).  Are you really going to sit here and try to argue that this is easier to teach?  I call bullshit.  And besides, if you're going to become a mathematician, you have to learn the Lebesgue integral at some point anyways\textellipsis why learn something only to have to relearn it later?

Okay, so now that my rant is out of the way, let's actually do some mathematics.

\section{Measure theory}

All of integration theory ultimately boils down to measure theory.  The definition of the integral itself is relatively easy.  In fact, the definition of abstract measure spaces is even easier.  There's really no question that writing down the definition of the Lebesgue integral is \emph{significantly} easier than that of the Riemann integral.  What is a bit tricky, however, is constructing specific measures.  In our case, we will primarily be concerned with constructing Lebesgue measure (on $\R ^d$), and this is really the only part that is a bit tricky.  Fortunately, there is a \emph{huge} theorem that will just spit out Lebesgue measure for us, the \emph{\nameref{HaarHowesTheorem}}.

\subsection{Measures}

The intuition behind measure is actually quite easy---a measure is just an axiomatization of our intuition about notion of things like length, area, and volume.  Before we define ``measure'', however, it will be convenient to introduce a couple of terms.
\begin{dfn}{Subadditivity and additivity}{}
Let $X$ be a set, let $\meas :2^X\rightarrow [0,\infty ]$, and let $\collection{M}\subseteq 2^X$.  
\begin{enumerate}
\item $\meas$ is \term{subadditive}\index{Subadditive} on $\collection{M}$ iff for $\{ M_m:m\in \N \} \subseteq \collection{M}$ we have
\begin{equation}\label{5.1.2}
\meas \left( \bigcup _{m\in \N}M_m\right) \leq \sum _{m\in \N}\meas (M_m);
\end{equation}
\item $\meas$ is \term{additive}\index{Additive (measure)} on $\mcal{M}$ iff for $\{ M_m:m\in \N \} \subseteq \mcal{M}$ a \emph{disjoint} collection we have
\begin{equation}\label{5.1.3}
\meas \left( \bigcup _{m\in \N}M_m\right) =\sum _{m\in \N}\meas (M_m).
\end{equation}
\end{enumerate}
$\meas$ is simply just subadditive (resp.~additive) if it is subadditive (resp.~additive) on all of $2^X$.
\begin{rmk}
You might think that we should always have additivity, or at the very least, we should have finite additivity:  if $S$ and $T$ are disjoint, then $\meas (S\cup T)=\meas (S)+\meas (T)$.  Unfortunately, this is \emph{false} for Lebesgue measure, our measure of primary interest---see \cref{exm5.2.56}.  We will have additivity on a very large class of sets, however, the so-called \emph{measurable sets}---see \cref{MeasurableSet}.
\end{rmk}
\end{dfn}
\begin{exr}{}{exr5.1.4}
Let $\collection{M}$ be a collection of sets that is closed under union, intersection, and complementation.  Show that if $\meas$ is additive on $\collection{M}$ then it is subadditive on $\collection{M}$.
\begin{rmk}
There is something to show here.  While \eqref{5.1.3} itself is obviously a stronger condition than \eqref{5.1.2}, it is also only assumed for \emph{disjoint} collections.  The problem then is to show that, if \eqref{5.1.3} holds for disjoint collections, then \eqref{5.1.2} holds for \emph{all} collections.
\end{rmk}
\end{exr}

\begin{dfn}{Measure}{OuterMeasure}
Let $X$ be a set.  A \term{measure}\index{Measure} on $X$ is a function $\meas :2^X\rightarrow [0,\infty ]$ such that
\begin{enumerate}
\item $\meas (\emptyset )=0$;
\item (Nondecreasing)\label{Measure.Monotonicity} $\meas :\coord{2^X,\subseteq}\rightarrow [0,\infty ]$ is nondecreasing;\footnote{Concretely, this means that $\meas (S)\leq \meas (T)$ if $S\subseteq T$.} and
\item (Subadditivity) $\meas$ is subadditive.
\end{enumerate}
A set equipped with measure is a \term{measure space}\index{Measure space}.
\begin{rmk}
Note that we allow the measure of sets to be infinite.  This is incredibly important---for example, we will want $\meas (\R )=\infty$ (for Lebesgue measure anyways).
\end{rmk}
\begin{rmk}
As a consequence of this, we needn't worry about convergence in the third axiom.  As a matter of fact, we definitely want to allow this sum to diverge---think about what the measure of $\bigcup _{m\in \Z}(m,m+1)$ should be.
\end{rmk}
\begin{rmk}
Most other sources will likely refer to $\meas$ as an \term{outer-measure}\index{Outer-measure}.  For them, a measure would be considered a \emph{measure} iff it were additionally additive (and not just subadditive).  However, as we work exclusively with what they would refer to as an ``outer-measure'', we simply just the the term ``measure'' as ``outer-measure'' is unnecessarily verbose.\footnote{Incidentally, as we shall see shortly, $\meas$ restricted to the collection of measurable sets (\cref{MeasurableSet}) will be additive (\namerefpcref{CaratheodorysTheorem}), that is, a ``measure'' in their sense of the term.  That is, every measure in our sense of the word determines a ``measure'' in their sense of the term, and so the distinction is not really that big of a deal---I just find it quite a bit cleaner to be working with $\meas$ defined on all of $2^X$ than merely just the measurable sets.}
\end{rmk}
\begin{rmk}
Measure is sometimes also called \term{exterior-measure}\index{Exterior-measure}.
\end{rmk}
\end{dfn}
\begin{exm}{The Zero Measure}{ZeroMeasure}{}
Let $X$ be a set and define $\meas :2^X\rightarrow [0,\infty ]$ by $\meas (S)\coloneqq 0$.  How terribly interesting.
\end{exm}
\begin{exm}{The Infinite Measure}{InfiniteMeasure}
Let $X$ be a set and define $\meas :2^X\rightarrow [0,\infty ]$ by
\begin{equation}
\meas (S)\coloneqq \begin{cases}0 & \text{if }S=\emptyset \\ \infty & \text{otherwise.}\end{cases}
\end{equation}
Dear god, this example is even more interesting than the last one.
\end{exm}
\begin{exm}{The Unit Measure}{UnitMeasure}
Let $X$ be a set and define $\meas :2^X\rightarrow [0,\infty ]$ by
\begin{equation}
\meas (S)\coloneqq \begin{cases}0 & \text{if }S=\emptyset \\ 1 & \text{otherwise.}\end{cases}
\end{equation}
\begin{rmk}
Note that this is \emph{never} additive (unless of course $X$ is either empty or a single point).  This makes it useful for producing counter-examples (see \cref{exm5.1.42}), and not much else.
\end{rmk}
\end{exm}
\begin{exm}{The counting measure}{}
Let $X$ be a set and for $S\subseteq X$ define $\meas (S)\coloneqq \abs{S}$, that is, the cardinality of $S$.
\begin{rmk}
This is the \term{counting measure}\index{Counting measure} on $X$.
\end{rmk}
\begin{rmk}
This is actually incredibly important, as we shall see (\cref{prp5.2.228}) that sums are just integrals with respect to the counting measure.
\end{rmk}
\end{exm}

In measure theory, things almost always matter only `up to' sets of measure $0$.  This concept is so important that there is a term for it.
\begin{mdf}{Almost-everywhere XYZ\hfill}{AlmostEverywhereXYZ}
Let $f\colon \coord{X,\meas}\rightarrow Y$ be a function on a measure space.  Then, $f$ is \term{almost-everywhere XYZ}\index{Almost-everywhere XYZ} iff
\begin{equation}
\meas \left( \left\{ x\in X:f(x)\text{ is not XYZ.}\right\} \right) =0.
\end{equation}
\begin{rmk}
If we need to clarify the measure we're working with, we will write ``$\meas$-almost-everywhere XYZ''.
\end{rmk}
\begin{rmk}
Of particular importance is the condition ``$f=g$ almost-everywhere'',\footnote{Though see the following exercises for other important ``almost-everywhere'' definitions.} which we shall denote as $f\sim _{\AlE}g$\index[notation]{$f\sim _{\AlE}g$}, and explicitly means that
\begin{equation}
\meas \left( \left\{ x\in X:f(x)\neq g(x)\right\} \right) =0.
\end{equation}
\end{rmk}
\end{mdf}
\begin{exr}{}{exr5.1.20}
Let $X$ be a measure space and let $Y$ be a set.  Show that $\sim _{\AlE}$ is an equivalence relation on $\Mor _{\Set}(X,Y)$.
\end{exr}
\begin{exr}{}{}
Let $X$ be a measure space, let $\coord{Y,\leq}$ be a preordered set, and for $f,g\in \Mor _{\Set}(X,Y)/\sim _{\AlE}$ define $f\leq g$ iff $f(x)\leq g(x)$ almost-everywhere.  Show that $\leq$ is well-defined.
\begin{rmk}
Recall that $\Mor _{\Set}(X,Y)/\sim _{\AlE}$ is our notation for the quotient set, that is, the set of equivalence classes---see \cref{dfnA.1.42}.
\end{rmk}
\end{exr}
\begin{exr}{}{}
Let $X$ be a measure space, let $Y$ be a topological space, let $\lambda \mapsto f_{\lambda}\in \Mor _{\Set}(X,Y)/\sim _{\AlE}$, let $f_{\infty}\in \Mor _{\Set}(X,Y)$, and define ``$\lambda \mapsto f_{\lambda}$ converges to $f_{\infty}$ in $\Mor _{\Set}(X,Y)$'' iff $\lambda \mapsto f_{\lambda}(x)$ converges to $f_{\infty}(x)$ almost-everywhere.  Show that this definition of convergence is well-defined and defines a topology on $\Mor _{\Set}(X,Y)/\AlE$ via \namerefpcref{KelleysConvergenceTheorem}
\begin{rmk}
Note that this topology isn't terribly useful.  For example, the integral will not be continuous with respect to this topology---see \cref{MonotoneCounterexample}.
\end{rmk}
\end{exr}
\begin{important}
Whenever $X$ is equipped with a measure, \emph{all relations on $\Mor _{\Set}(X,Y)$ are defined only up to measure zero}.  For example, if we write $f=g$, what we really mean is that $\meas \left( \{ x\in X:f(x)\neq g(x)\} \right) =0$.  Similarly, if we write $f\leq g$, what we really mean is that $\meas \left( \{ x\in X:f(x)\not \leq g(x)\} \right) =0$.  Etc..  Unfortunately, there is no reasonable way to make this into a category without imposing extra conditions on our functions.\footnote{And it is certainly possible to do so---see \cref{MeasurableFunction}---but this condition is overly restrictive to be of much practical use---see \cref{CantorFunction}.}
\end{important}
\begin{exm}{Composition is not well-defined almost-everywhere}{}
$\sim _{\AlE}$ is an equivalence relation by \cref{exr5.1.20}.  Our claim is that composition is \emph{not} well-defined with respect to this equivalence relation.  Precisely, we give an examples of functions $f_1\sim f_2$ and $g_1\sim g_2$, with $g_1\circ f_1\not \sim g_2\circ f_2$.

Define $X\ceqq \{ x_0\}$, $Y\ceqq \{ y_0\}$, and $Z\ceqq \{ z_1,z_2\}$, and equip $X$ with the Infinite Measure, $Y$ with the Zero Measure, and $Z$ with the Infinite Measure.\footnote{Though the measure $Z$ is equipped with doesn't matter so much.}

There is only one function from $X$ to $Y$---let $f_1=f_2$ be this unique function.  On the other hand, let $g_i\colon Y\rightarrow Z$ be the unique function that sends $y_0$ to $z_i$.  $f_1$ and $f_2$ are equal everywhere, and so certainly equal almost-everywhere.  On the other hand, $Y$ has the Zero Measure, and so any two functions with domain $Y$ are going to be equal almost-everywhere.  On the other hand, $g_i\circ f_i$ is the unique function from $X$ to $Z$ that sends $x_0$ to $z_i$, and so $g_1\circ f_1$ and $g_2\circ f_2$ disagree on all of $X$, which has infinite measure!  In particular, $g_1\circ f_1\not \sim g_2\circ f_2$.
\end{exm}

While we will be assigning a measure to every set, not all of them will be considered to be \emph{measurable}.  In general, we will \emph{not} have additivity; however, when we restrict our (outer) measures to the collection of what are called \emph{measurable sets}, we \emph{will} have additivity.  This is more or less the point of talking about measurable sets---additivity is a nice thing to have.
\begin{dfn}{Measurable (set)}{MeasurableSet}
Let $\meas :2^X\rightarrow [0,\infty ]$ be a measure on a set $X$ and let $M\subseteq X$.  Then, $M$ is \term{measurable}\index{Measurable (set)} iff
\begin{equation}
\meas (S)=\meas (S\cap M)+\meas (S\cap M^{\comp})
\end{equation}
for all sets $S\subseteq X$.
\begin{rmk}
Think about what this means:  $M$ is chopping up $S$ into two pieces, the set of points in $M$ and the set of points not in $M$.  $M$ is measurable, then, if the measure of $S$ is the sum of the measure of these two pieces \emph{for all} $S$.  In particular, $S$ itself is definitely not required to be measurable.\footnote{For one thing, this would make the definition circular.}
\end{rmk}
\begin{rmk}
By subadditivity, we \emph{always} have that $\meas (S)\leq \meas (S\cap M)+\meas (S\cap M^{\comp})$.  Therefore, in fact, $M$ is measurable iff
\begin{equation}
\meas (S)\geq \meas (S\cap M)+\meas (S\cap M^{\comp})
\end{equation}
for all $S\subseteq X$.
\end{rmk}
\begin{rmk}
You might say that the motivation for the definition is that, for sets $S,T$ that satisfy this property, we should have at least finite additivity (i.e.~$S,T$ disjoint implies $\meas (S\cup T)=\meas (S)+\meas (T)$).  It turns-out that this is true, but in fact, perhaps surprisingly, we have much more than this---we actually have \emph{(countable) additivity}--see \cref{CaratheodorysTheorem}.  By asking for finite additivity, we get countable additivity for free!
\end{rmk}
\begin{wrn}
Warning:  There definitely exist sets that are not measurable in general!  In fact, nonmeasurable sets are easy to find in `artificial' spaces---see \cref{exm5.1.42}.  But such pathologies exist even for the nicest of measures.  Indeed, see \cref{exm5.2.47} for a set that is not measurable with respect to Lebesgue measure.
\end{wrn}
\begin{rmk}
This is also what is sometimes referred to \term{Carathéodory measurable}.
\end{rmk}
\end{dfn}
\begin{exr}{}{exr5.1.20x}
Show that if $\meas (M)=0$, then all subsets of $M$ are measurable.
\begin{rmk}
In particular, $M$ itself is measurable.
\end{rmk}
\end{exr}

\begin{thm}{Carathéodory's Theorem}{CaratheodorysTheorem}\index{Carathéodory's Theorem}
Let $\meas :2^X\rightarrow [0,\infty ]$ be a measure.  Then,
\begin{enumerate}
\item \label{CaratheodorysTheorem.i}$\meas$ is additive on the collection of measurable sets;
\item \label{CaratheodorysTheorem.ii}the countable union of measurable sets is measurable;
\item \label{CaratheodorysTheorem.iii}the countable intersection of measurable sets is measurable;
\item \label{CaratheodorysTheorem.iv}the complement of a measurable set is measurable;
\item \label{CaratheodorysTheorem.v}$\emptyset$ and $X$ are measurable.
\end{enumerate}
\begin{rmk}
Note that \namerefpcref{DeMorgansLaws} imply that \cref{CaratheodorysTheorem.ii} are \cref{CaratheodorysTheorem.iii} are equivalent if \cref{CaratheodorysTheorem.iv} is true.  A collection of sets which satisfies \cref{CaratheodorysTheorem.ii}--\cref{CaratheodorysTheorem.v} is called a \term{$\sigma$-algebra}\index{$\sigma$-algebra}.  We do not use this language, but it is important to know for consulting other references.
\end{rmk}
\begin{proof}
\Step{Show \cref{CaratheodorysTheorem.iv}}
The definition of measurability is $S\leftrightarrow S^{\comp}$ symmetric, so \cref{CaratheodorysTheorem.iv} is automatically true.

\Step{Show \cref{CaratheodorysTheorem.v}}
The empty-set is measurable by the previous exercise, and hence by \cref{CaratheodorysTheorem.iv}, $X$ is measurable as well, which establishes \cref{CaratheodorysTheorem.v}.

\Step{Reduce the proof of \cref{CaratheodorysTheorem.iii} to the proof of \cref{CaratheodorysTheorem.ii}}
As was explained in a remark, we need not show \cref{CaratheodorysTheorem.iii} itself---it will now follow if we can show \cref{CaratheodorysTheorem.ii}.

\Step{Prove \cref{CaratheodorysTheorem.ii} for finite unions}
We first show that the union of finitely many measurable sets is measurable.  It suffices of course to then just show that the union of two measurable sets is measurable.  So, let $M_1,M_2\subseteq X$ be measurable and let $S\subseteq X$.  Then,
\begin{equation*}
\begin{split}
\meas (S) & =\footnote{Because $M_2$ is measurable.}\meas (S\cap M_2)+\meas (S\cap M_2^{\comp}) \\
& =\footnote{Because $M_1$ is measurable (applied twice).}\meas (S\cap M_2\cap M_1)+\meas (S\cap M_2\cap M_1^{\comp}) \\ & \qquad +\meas (S\cap M_2^{\comp}\cap M_1)+\meas (S\cap M_2^{\comp}\cap M_1^{\comp}) \\
& \geq \footnote{By subadditivity and the fact that $M_1\cup M_2=(M_1\cap M_2)\cup (M_1\cap M_2^{\comp})\cup (M_1^{\comp}\cap M_2)$.}\meas (S\cap (M_1\cup M_2))+\meas (S\cap (M_1\cup M_2)^{\comp})
\end{split}
\end{equation*}
Thus, indeed, $M_1\cup M_2$ is measurable.

\Step{Complete the proof of \cref{CaratheodorysTheorem.ii}}
Let $\{ M_m:m\in \N \}$ be a countable collection of measurable sets.  We wish to show that
\begin{equation}
\bigcup _{m\in \N}M_m
\end{equation}
is measurable.  First of all, define
\begin{equation}\label{5.1.13}
M_m'\coloneqq M_m\setminus \bigcup _{k=0}^{m-1}M_k.
\end{equation}
Note that each $M_m'$ is measurable because we already know that finite unions, complements, and hence also finite intersections, of measurable sets are measurable.
\begin{exr}[breakable=false]{}{}
Show that (i)~$M_m'\subseteq M_m$, (ii)~the collection $\{ M_m':m\in \N \}$ is disjoint, and (iii)~$\bigcup _{m\in \N}M_m=\bigcup _{m\in \N}M_m'$.
\begin{rmk}
This trick (the one in \eqref{5.1.13}, that is) is important.  Don't forget it.
\end{rmk}
\end{exr}
Thus, as
\begin{equation}
\bigcup _{m\in \N}M_m=\bigcup _{m\in \N}M_m',
\end{equation}
it suffices to prove this step in the case where $\{ M_m:m\in \N \}$ is itself disjoint (just rename $M_m$ to now be $M_m'$).  Thus, we now without loss of generality assume that $\{ M_m:m\in \N \}$ is disjoint.

Now define
\begin{equation}
N_m\coloneqq \bigcup _{k=0}^mM_k\text{ and }N\coloneqq \bigcup _{k\in M}M_m.
\end{equation}
so that, by the previous step, we have that $N_m$ is measurable.

Then, for $S\subseteq X$,
\begin{equation}
\begin{split}
\meas (S\cap N_m) & =\footnote{Because $E_m$ is measurable.}\meas (S\cap N_m\cap M_m) \\ & \qquad +\meas (S\cap N_m\cap M_m^{\comp}) \\
& =\footnote{Because the collection $\{ M_m:m\in \N \}$ is disjoint.}\meas (S\cap M_m)+\meas (S\cap N_{m-1}) \\
& =\footnote{Apply this trick inductively.}\sum _{k=0}^m\meas (S\cap M_k).
\end{split}
\end{equation}
Thus,
\begin{equation}
\begin{split}
\meas (S) & =\meas (S\cap N_m)+\meas (S\cap N_m^{\comp}) \\
& =\sum _{k=0}^m\meas (S\cap M_k)+\meas (S\cap N_m^{\comp}) \\
& \geq \footnote{Because $N^{\comp}\subseteq N_m^{\comp}$.}\sum _{k=0}^m\meas (S\cap M_k)+\meas (S\cap N^{\comp}).
\end{split}
\end{equation}
Hence, taking the limit of this inequality as $m\to \infty$, we have
\begin{equation}\label{5.1.16}
\begin{split}
\meas (S) & \geq \sum _{m\in \N}\meas (S\cap M_m)+\meas (S\cap N^{\comp}) \\
& \geq \footnote{By subadditivity.}\meas (S\cap N)+\meas (S\cap N^{\comp}),
\end{split}
\end{equation}
and so indeed $N$ is measurable.

\Step{Prove \cref{CaratheodorysTheorem.i}}
Let $N$ and $S$ be as in the previous step and take $S\coloneqq N$.  Then, by \eqref{5.1.16}, we have that
\begin{equation}
\meas \left( \bigcup _{m\in \N}M_m\right) \geq \sum _{m\in \N}\meas (M_m).
\end{equation}
The other inequality is automatic from subadditivity, and so indeed, we have equality.
\end{proof}
\end{thm}
\begin{exr}{}{exr5.1.21}
Let $\meas :2^X\rightarrow [0,\infty ]$ be a measure and let $S\subseteq T\subseteq X$.  Show that if $S$ is measurable with finite measure, then
\begin{equation}
\meas (T\setminus S)=\meas (T)-\meas (S).
\end{equation}
\begin{rmk}
The only reason you need $S$ to have finite measure otherwise the right-hand side of this equation will be undefined if $T$ also has infinite measure.  In particular, the rearranged equation $\meas (S)+\meas (T\setminus S)=\meas (T)$ holds even in the case $\meas (S)=\infty$.
\end{rmk}
\begin{rmk}
Note that you do \emph{not} need $T$ to be measurable for this to hold.
\end{rmk}
\end{exr}
\begin{exr}{``Continuity from below''}{exr5.1.27}
Let $M_0\subseteq M_1\subseteq \cdots$ be an nondecreasing countable collection of measurable sets.  Show that
\begin{equation}
\meas \left( \bigcup _{k=0}^\infty M_k\right) =\lim _m\meas \left( \bigcup _{k=0}^mM_k\right) =\lim _m\meas (M_k).
\end{equation}
\end{exr}
\begin{exr}{``Continuity from above''}{exr5.1.29}
Let $M_0\supseteq M_1\supseteq \cdots$ be a nonincreasing countable collection of measurable sets.  Show that, \emph{if at least one $M_k$ has finite measure},
\begin{equation}
\meas \left( \bigcap _{k=0}^\infty M_k\right) =\lim _m\meas \left( \bigcap _{k=0}^mM_k\right) =\lim _m\meas (M_k).
\end{equation}
\begin{rmk}
Though we technically have not defined measure on $\R$ yet, it is easy to see intuitively why we would neither expect nor want this to hold if the measure of each $M_k$ were infinite.  For example, take $M_k\coloneqq (-\infty ,-k)$.  Then, on one hand, $\meas (M_k)=\infty$ for all $k$, but yet $\meas (\bigcap _{k=0}^\infty M_k)=\meas (\emptyset )=0$.
\end{rmk}
\end{exr}

Just as we have a notion of measurable set, so too do we have a notion of measurable \emph{function}
\begin{dfn}{Measurable (function)}{MeasurableFunction}
\\
Let $f\colon \coord{X_1,\meas _1}\rightarrow \coord{X_2,\meas _2}$ be a function between measure spaces.  Then, $f$ is \term{measurable}\index{Measurable (function)} iff
\begin{enumerate}
\item \label{MeasurableFunction.i}the preimage of every measurable set is measurable; and
\item \label{MeasurableFunction.ii}the preimage of a set of measure $0$ has measure $0$.
\end{enumerate}
\begin{rmk}
This is a \emph{very strong} condition.  For example, there are \emph{uniform-homeomorphisms} on $\R$ that are not measurable---see \cref{CantorFunction} (the Cantor Function).
\end{rmk}
\begin{rmk}
Neither of these conditions imply one another---see the following examples.
\end{rmk}
\end{dfn}
\begin{exm}{A function which preserves measurability but not measure $0$}{}
Precisely, we give a function that has the property that the preimage of every set is measurable but the preimage of a set of measure $0$ does not have measure $0$.

Let $X_1\coloneqq \{ x_1\}$ be a one point set and let $\meas _1$ be the Infinite Measure (\cref{InfiniteMeasure}) on $X_1$.  Let $X_2\coloneqq \{ x_2\}$ be a one point set and let $\meas _2$ be the Zero Measure (\cref{ZeroMeasure}) on $X_2$.  Let $f\colon X_1\rightarrow X_2$ be the only function that exists from $X_1$ to $X_2$.

Every subset of $X_1$ is measurable, and so trivially the preimage of every subset of $X_2$ is measurable.  On the other hand, $\meas _2(\{ x_2\} )=0$, but $\meas _1\left( f^{-1}(\{ x_2\} )\right) =\infty \neq 0$.
\end{exm}
\begin{exm}{A function which preserves measure $0$ but not measurability}{}
Precisely, we give a function that has the property that the preimage of every set of measure $0$ has measure $0$ but the preimage of some measurable set is not measurable.

Let $X_1\coloneqq \{ x_1,x_2\} \eqqcolon X_2$ be a two point set.  Equip $X_1$ with the Unit Measure (\cref{UnitMeasure}) and equip $X_2$ with the Infinite Measure.  Let $f\coloneqq \id _{\{ x_1,x_2\}}$.
\begin{exr}[breakable=false]{}{}
Show that $\{ x_1\} ,\{ x_2\} \subseteq X_1$ are \emph{not} measurable with respect to the Unit Measure.
\end{exr}

There is only one subset of $X_2$ with measure $0$, namely $\emptyset$, and of course the preimage of the empty-set (the empty-set itself) also has measure $0$.  On the other hand, $\{ x_1\}$ is measurable with respect to the Infinite Measure, but not the Unit Measure, and so its preimage (namely itself) is not measurable.
\end{exm}

\begin{important}
I should probably mention at this point that the way I am presenting measure theory goes against the orthodoxy.  (You can skip this comment if you don't plan to look in other sources.)  Every other author I am aware of \emph{only works with measurable sets}.  For them, a \emph{measure space} is a set, together with a $\sigma$-algebra, the ``measurable sets'', along with a set function that is additive on the $\sigma$-algebra (and sends the empty-set to $0$).  For them, measures are merely tools for constructing measures (a l\`{a} Mr.~Carathéodory).  Of course, their ``measure''\footnote{The quotes are to indicate that this is what they call it---I myself have not defined this term.} is just the restriction of the measure to the collection of all measurable sets.  You might say they simply `forget' that they had ever assigned a measure to the nonmeasurable sets.  I find this unnecessarily complicated and messy.  For one thing, if you do things the way I have presented them, you don't have to worry about $\sigma$-algebras (at least not explicitly).  The disadvantage is that now I have to add the hypothesis ``These sets are measurable'' to a lot of my theorems.  Meh.  It's a trade-off, but I quite like not having to ever worry about $\sigma$-algebras explicitly.
\end{important}

\begin{important}
At some point in the near future, we will be doing arithmetic with $\infty$---for example, what should the measure of $\R \times \{ 0\}$ in $\R ^2$ be?  Of course, from our definition of product measures, this will turn out to be $\infty \cdot 0$.  We hence declare that
\begin{equation}
\infty \cdot 0\coloneqq 0\eqqcolon 0\cdot \infty .
\end{equation}
There are other arithmetic notions we have to technically define (e.g.~$x+\infty=\infty$ for $x$ finite), but this is the only 						nonobvious one.  On thing we leave \emph{un}defined, however, is the quantity $\infty +(-\infty )$ (and likewise for $-\infty +\infty$).
\end{important}

\subsection{Measures on topological and uniform spaces}

One can go ahead and develop the theory for measures on arbitrary sets, but in practice, we will only be working with measures defined on spaces with \emph{a lot} of extra structure.  This motivates us to investigate measures on topological and uniform spaces, in which case we are of course going to require our measures to be compatible with this extra structure.

\subsubsection{Measures on topological spaces}

There are definitely measures on topological spaces for which the open sets are not measurable---see \cref{exm5.1.42}---in fact, this can even be the case for regular measures (the example in \cref{exm5.1.42} is regular---see \cref{RegularMeasure} for the definition of a regular measure), but most measures on topological spaces which are not cooked-up for the sole purpose of producing counter-examples have the property that open sets are measurable.  We have a name for such measures:  \emph{Borel measures}.
\begin{dfn}{Borel measure}{BorelMeasure}
Let $X$ be a topological space and let $\meas :2^X\rightarrow [0,\infty ]$ be a measure.  Then, $\meas$ is \term{Borel}\index{Borel measure} iff every open set is measurable.  A topological space equipped with a Borel measure is a \term{Borel measure space}\index{Borel measure space}.
\begin{rmk}
By \nameref{CaratheodorysTheorem}, it follows that the ``$\sigma$-algebra'' generated by the open sets likewise consists of measurable sets.  The term for the sets in this $\sigma$-algebra is \term{Borel set}, hence the term, \emph{Borel measure}---a Borel measure is a measure in which the Borel sets are measurable (which is of course equivalent to the open sets being measurable).
\end{rmk}
\begin{rmk}
In particular, as the complements of measurable sets are measurable, closed sets are likewise measurable for Borel measures.  Then, as countable unions and countable intersections of measurable sets are measurable, it follows that in fact all $G_{\delta}$ and $F_{\sigma}$ (\cref{GDeltaFSigma}) sets are measurable.
\end{rmk}
\end{dfn}

\begin{dfn}{Regular measure}{RegularMeasure}
Let $X$ be a topological space and let $\meas :2^X\rightarrow [0,\infty ]$ be a measure.  Then, $\meas$ is \term{regular}\index{Regular measure} iff
\begin{enumerate}
\item $\meas$ is finite on quasicompact subsets;
\item (Outer-regular) for $S\subseteq X$,
\begin{equation}
\meas (S)=\inf \{ \meas (U):S\subseteq U,\ U\text{ open.}\} ;\text{ and }
\end{equation}
\item (Inner-regular on open sets) for $U\subseteq X$ open,
\begin{equation*}
\meas (U)=\sup \{ \meas (K):K\subseteq U,\ K\text{ quasicompact.}\} .
\end{equation*}
\end{enumerate}
A topological space equipped with regular measure is a \term{regular measure space}\index{Regular measure space}.
\begin{wrn}
Warning:  As mentioned before, regular measures need not be Borel---see the following counter-example (\cref{exm5.1.42}).
\end{wrn}
\end{dfn}
\begin{exm}{A regular measure that is not Borel}{exm5.1.42}
Define $X\coloneqq \{ x_1,x_2\}$, and equip it with the discrete topology and the Unit Measure (\cref{UnitMeasure}).  Of course the measure of every quasicompact subset is finite.  It is automatically outer-regular and inner-regular on open sets because every subset is both open and quasicompact.  It is hence regular.  On the other hand,
\begin{equation}
\meas (\{ x_1,x_2\} )=1<1+1=\meas (\{ x_1\} )+\meas (\{ x_2\} ).
\end{equation}
Therefore, one of the open sets $\{ x_1\}$ or $\{ x_2\}$ must have been nonmeasurable.
\end{exm}
Of course, there is a counter-example to the other potential implication as well.
\begin{exm}{A Borel measure that is not regular}{}
Consider the counting measure on $\R$.  The set $[0,1]$ is quasicompact, but has infinite measure.  Therefore, the counting measure on $\R$ is not regular.  On the other hand, the cardinality of the union of two disjoint sets is the sum of the cardinalities of those two sets,\footnote{This is how we defined addition of cardinals!} and so we always have $\meas (S)=\meas (S\cap M)+\meas (S\cap M^{\comp})$, that is to say, every set is measurable, and so certainly the open sets are measurable.
\end{exm}

In general topology, I really dislike imposing unnecessary countability assumptions.  On the other hand, countability is something fundamental in measure theory, simply because of the conditions of additivity and subadditivity---see \cref{OuterMeasure}.  Furthermore, as explained there, we \emph{don't want} any stronger additivity assumptions.  Thus, in the context of \emph{measures} on topological spaces, it makes sense to impose countability conditions on our spaces.  This leads us to the following definition.
\begin{dfn}{$\sigma$-quasicompact}{SigmaQuasicompact}
A topological space is \term{$\sigma$-quasicompact}\index{$\sigma$-quasicompact} iff it is the countable union of quasicompact sets.  A topological space is \term{$\sigma$-compact}\index{$\sigma$-compact} iff it is the countable union of compact sets.
\begin{rmk}
Recall that compact is synonymous (by definition) with $T_2$ and quasicompact.  In particular, compact sets are \emph{closed} (and so measurable for Borel measures).
\end{rmk}
\end{dfn}
\begin{exm}{A subspace of a $\sigma$-quasicompact space that is not $\sigma$-quasicompact}{}
The example is the Sorgenfrey Line $S$ from \cref{SorgenfreyPlane}.\footnote{Refer to the example for more detail, but in brief, the Sorgenfrey Line is, as a set, the real numbers, whose topology has as a base the set of all closed-open intervals $[a,b)$.}
\begin{exr}{}{}
Show that $S$ is not $\sigma$-quasicompact.
\end{exr}

We showed in \cref{SorgenfreyPlane} that $S$ is perfectly-$T_4$.  Thus, by the \namerefpcref{TychonoffEmbeddingTheorem}, it is (homeomorphic to) a subspace of a product of $[0,1]$, which in particular is a compact space, hence trivially $\sigma$-quasicompact.  Thus, this product is $\sigma$-quasicompact, but this subspace homeomorphic to $S$ is not $\sigma$-quasicompact.
\end{exm}
\begin{exr}{}{SigmaProductSigma}
Show that the product of two $\sigma$-quasicompact spaces is $\sigma$-quasicompact.  Show that the product of two $\sigma$-compact spaces is $\sigma$-compact.  Find an example of an infinite product of $\sigma$-compact spaces that is not $\sigma$-quasicompact.
\end{exr}
\begin{dfn}{Topological measure space}{TopologicalMeasureSpace}
A \term{topological measure space}\index{Topological measure space} is a $\sigma$-compact topological space equipped with a regular Borel measure.  If $\coord{X,\meas}$ is a topological measure space, then we say that $\meas$ is \term{topological}\index{Topological measure}.
\end{dfn}
\begin{exr}{}{SigmaFinite}
Show that a topological measure space can be written as the countable \emph{disjoint} union of measurable sets each of which is contained in a compact set.
\begin{rmk}
In particular a topological measure space can be written as the countable disjoint union of measurable sets of finite measure.  A measure space that can be written as the countable union of measurable sets of finite measure is called \term{$\sigma$-finite}\index{$\sigma$-finite}.
\end{rmk}
\begin{rmk}
By regularity and $\sigma$-compactness, you by definition have that topological measure spaces can be written as the countable union of compact sets.  It is your job to turn that union into a disjoint one.
\end{rmk}
\begin{rmk}
I told you once upon a time not to forget a trick.  You haven't forgotten it, have you?
\end{rmk}
\end{exr}
\begin{prp}{}{prp5.1.56}
Let $X$ be $\sigma$-compact topological space, let $\meas$ be a measure on $X$, and let $K\subseteq X$ be quasicompact.  Then, if $\meas$ is Borel, then $K$ is measurable.
\begin{rmk}
In particular, quasicompact sets are measurable in topological measure spaces.
\end{rmk}
\begin{proof}
Suppose that $\meas$ is Borel.  As $X$ is $\sigma$-compact, we may write $X=\bigcup _{m\in \N}K_m$ for $K_m\subseteq X$ compact.  Each $K_m$ is in particular closed, and so $K\cap K_m$ is quasicompact by \cref{prp3.5.6} (intersection of a quasicompact set and a closed set is quasicompact), hence closed as it is a subspace of a $T_2$ space (namely $K_m$), hence measurable.
\end{proof}
\end{prp}

The following is nice characterization of measurability in topological measure spaces.  It is arguably the reason why we give the conditions ``$\sigma$-compact, regular, Borel'' a name in the first place.
\begin{prp}{}{prp5.1.39}
Let $\coord{X,\meas}$ be a topological measure space and $M\subseteq X$.  Then, the following are equivalent.
\begin{enumerate}
\item \label{prp5.1.39.i}$M$ is measurable.
\item \label{prp5.1.39.ii}For every $\varepsilon >0$, there is an open set $U_{\varepsilon}$ and a closed set $C_{\varepsilon}$ such that
\begin{equation}
C_{\varepsilon}\subseteq M\subseteq U_{\varepsilon}\text{ and }\meas (U_{\varepsilon}\setminus C_{\varepsilon})<\varepsilon .
\end{equation}
\item \label{prp5.1.39.iii}There is a $G_{\delta}$ set $G$ and an $F_{\sigma}$ set $F$ such that
\begin{equation}
F\subseteq M\subseteq G\text{ and }\meas (G\setminus F)=0.
\end{equation}
\end{enumerate}
\begin{proof}
Write $X=\bigcup _{m\in \N}K_m$ for $K_m\subseteq X$ compact.

\blankline
\noindent
$(\cref{prp5.1.39.i}\Rightarrow \cref{prp5.1.39.ii})$  Suppose that $M$ is measurable.  Define $M_m\coloneqq M\cap K_m$.  As $M_m\subseteq K_m$, $M_m$ has finite measure because $\meas$ is regular.  Let $\varepsilon >0$.  By outer-regularity, there is some open $U_m$ containing $M_m$ such that
\begin{equation}
\meas (M_m)\leq \meas (U_m)<\meas (M_m)+\tfrac{\varepsilon}{2^m}.
\end{equation}
Because $M$ is measurable, it in turn follows that\footnote{See \cref{exr5.1.21}.  We are also using here the fact that $M_m$ is measurable, because $M$ and $K_m$ are measurable, $K_m$ being measurable because it is compact, hence closed, and the measure is Borel (by hypothesis).}
\begin{equation}
\meas (U_m-M_m)<\tfrac{\varepsilon}{2^m}.
\end{equation}
Define $U\coloneqq \bigcup _{m\in \N}U_m$.  Then,
\begin{equation}
\meas (U\setminus M)\leq \sum _{m\in \N}\meas (U_m-M_m)<2\varepsilon .
\end{equation}
Applying this same logic to $M^{\comp}$, we can find an open set $V$ containing $M^{\comp}$ such that
\begin{equation}
\meas (V\setminus M^{\comp })<2\varepsilon .
\end{equation}
Then, $V^{\comp}$ is then a closed subset of $M$ and
\begin{equation}
\begin{split}
\meas (U\setminus V^{\comp}) & =\meas (U\cap V) \\
& \leq \meas (U\cap V\cap M)+\meas (U\cap V\cap M^{\comp}) \\
& \leq \meas (V\setminus M^{\comp})+\meas (U\setminus M)<4\varepsilon .
\end{split}
\end{equation}

\blankline
\noindent
$(\cref{prp5.1.39.ii}\Rightarrow \cref{prp5.1.39.iii})$ Suppose that for every $\varepsilon >0$, there is an open set $U_{\varepsilon}$ and a closed set $C_{\varepsilon}$ such that
\begin{equation}
C_{\varepsilon}\subseteq M\subseteq U_{\varepsilon}\text{ and }\meas (U_{\varepsilon}\setminus C_{\varepsilon})<\varepsilon .
\end{equation}
For $\varepsilon \coloneqq \frac{1}{m}$, $m\in \Z ^+$, let $U_m\supseteq M$ be open and $C_m\subseteq M$ be closed and such that $\meas (U_m\setminus C_m)<\frac{1}{m}$.  Now define
\begin{equation}
G\coloneqq \bigcap _{m\in \N}U_m\text{ and }F\coloneqq \bigcup _{m\in \N}C_m.
\end{equation}
Then, $G$ is $G_{\delta}$, $F$ is $F_{\sigma}$ (\cref{GDeltaFSigma}), $F\subseteq M\subseteq G$, and
\begin{equation}
\meas (G\setminus F)\leq \footnote{Simply because $G\subseteq U_m$ and $F\supseteq C_m$ for all $m\in \Z ^+$.}\meas (U_m\setminus C_m)<\tfrac{1}{m},
\end{equation}
and hence $\meas (G\setminus F)=0$.

\blankline
\noindent
$(\cref{prp5.1.39.iii}\Rightarrow \cref{prp5.1.39.i})$ Suppose that there is a $G_{\delta}$ set $G$ and an $F_{\sigma}$ set $F$ such that
\begin{equation}
F\subseteq M\subseteq G\text{ and }\meas (G\setminus F)=0.
\end{equation}
Note that closed sets are measurable because the measure is Borel, and hence $F_{\sigma}$ sets are measurable, being the countable union of closed (that is, measurable) sets.  Thus, $M=F\cup (M\setminus F)$ will be measurable if we can show that $M\setminus F$ is measurable.  However, $\meas (M\setminus F)\leq \meas (G\setminus F)=0$, and so $M\setminus F$ is measurable by \cref{exr5.1.20x} (sets of measure zero are measurable), as desired.
\end{proof}
\end{prp}

By definition, regularity requires inner-regularity on open sets.  In fact, however, for topological measure spaces, we also get inner-regularity on measurable sets of finite measure.
\begin{prp}{}{InnerRegularFinite}
Let $\coord{X,\meas}$ be a topological measure space.  Then, $\meas$ is inner-regular on measurable sets measure.
\begin{proof}
\Step{Prove the result for sets of finite measure}
Let $S\subseteq X$ be measurable.  Suppose that $\meas (S)<\infty$.  By \cref{SigmaFinite}, write $X=\bigcup _{m\in \N}F_m$ for $F_m$ measurable and contained in a compact set with $\{ F_m:m\in \N \}$ disjoint.  Let $S\subseteq X$ be measurable and of finite measure.  Define $S_m\coloneqq S\cap F_m$, so that each $S_m$ is measurable, contained in a compact set, and with $\{ S_m:m\in \N\}$ disjoint.
    
Let $\varepsilon >0$.  Then, by \cref{prp5.1.39}, there are open sets $U_{m,\varepsilon}$ and closed sets $C_{m,\varepsilon}$ such that
\begin{equation}\label{eqn5.1.77}
C_{m,\varepsilon}\subseteq S_m\subseteq U_{m,\varepsilon}\text{ and }\meas (U_{m,\varepsilon}\setminus C_{m,\varepsilon})<\frac{\varepsilon}{2^m}.
\end{equation}
Note that as $C_m\subseteq S_m$ and $S_m$ is contained in a compact set, $C_m$ itself is compact, being a closed subset of a compact set.

As $\meas$ is inner-regular on open sets, there are quasicompact sets $K_{m,\varepsilon}'\subseteq U_{m,\varepsilon}$ such that
\begin{equation}\label{eqn5.1.78}
\meas (U_{m,\varepsilon})-\frac{\varepsilon}{2^m}<\meas (K_{m,\varepsilon}')\leq \meas (U_{m,\varepsilon}).
\end{equation}
Note that $K_{m,\varepsilon}'$ is measurable by \cref{prp5.1.56}.  Define $K_{m,\varepsilon}\ceqq K_{m,\varepsilon}'\cap C_{m,\varepsilon}$.  Note that $K_{m,\varepsilon}$ is quasicompact by \cref{prp3.5.6} (intersection of quasicompact set with closed set is quasicompact), and hence actually compact because $C_{m,\varepsilon}$ is compact (because subspaces of $T_2$ spaces are $T_2$).  We then have that
\begin{equation*}
\begin{split}
\meas (U_{m,\varepsilon}\setminus K_{m,\varepsilon}) & \ceqq \meas (U_{m,\varepsilon}\cap (K_{m,\varepsilon}'\cap C_{m,\varepsilon}')^{\comp}) \\
& \leq \meas (U_{m,\varepsilon}\cap K_{m,\varepsilon}'^{\comp})+\meas (U_{m,\varepsilon}\cap C_{m,\varepsilon}^{\comp}) \\
& =\meas (U_{m,\varepsilon}\setminus K_{m,\varepsilon}')+\meas (U_{m,\varepsilon}\setminus C_{m,\varepsilon}) \\
& <\footnote{Here we are applying \cref{exr5.1.21} together with \eqref{eqn5.1.77} and \eqref{eqn5.1.78}.  This is why we neeed $K_{m,\varepsilon}'$ to be measurable.}\tfrac{\varepsilon}{2^m}+\tfrac{\varepsilon}{2^m}=\tfrac{\varepsilon}{2^{m-1}}.
\end{split}
\end{equation*}

Now, $L_{m,\varepsilon}\coloneqq \bigcup _{k=0}^mK_{k,\varepsilon}$ is likewise quasicompact and
\begin{equation}
\begin{split}
\MoveEqLeft
\meas (S)-\meas (L_{m,\varepsilon})=\footnote{Here we are using the fact that the $F_k$s are disjoint, so that the $S_k$s and $K_{k,\varepsilon}$s are in turn disjoint.}\sum _{k=0}^\infty \meas (S_k)-\sum _{k=0}^m\meas (K_{k,\varepsilon}) \\
& =\sum _{k=m+1}^\infty \meas (S_k)+\sum _{k=0}^m\left( \meas (S_k)-\meas (K_{k,\varepsilon})\right) \\
& \leq \sum _{k=m+1}^\infty \meas (S_k)+\sum _{k=0}^m\left( \meas (U_{k,\varepsilon})-\meas (K_{k,\varepsilon})\right) \\
& <\sum _{k=m+1}^\infty \meas (S_k)+\sum _{k=0}^m\frac{\varepsilon}{2^{k-1}} \\
& \leq \sum _{k=m+1}^\infty \meas (S_k)+4\varepsilon
\end{split}
\end{equation}
As $\meas (S)=\sum _{k=0}^\infty \meas (S_k)$ is finite, we can make this arbitrarily small by taking $m$ sufficiently large.  As $L_{m,\varepsilon}\subseteq S$ is quasicompact, we thus have that
\begin{equation}
\meas (S)=\sup \{ \meas (K):K\subseteq S,\ K\text{ quasicompact}\} ,
\end{equation}
and so $\meas$ is inner-regular on $S$.

\Step{Prove the result for sets of infinite measure}
Let $S\subseteq X$ be measurable.  Suppose that $\meas (S)=\infty$.  This time, write $X=\bigcup _{m\in \N}K_m$ for $K_m\subseteq X$ compact and $K_m\subseteq K_{m+1}$.\footnote{By definition, $\sigma$-compact just means we can write $X=\bigcup _{m\in \N}K_m'$ for $K_m\subseteq X$ compact, not necessarily with $K_m'\subseteq K_{m+1}'$.  Given this, we may define $K_m\ceqq \bigcup _{k=0}^mK_m'$ so that now $X=\bigcup _{m\in \N}$, $K_m\subseteq X$ is compact, and $K_m\subseteq K_{m+1}$.}  Define $S_m\ceqq S\cap K_m$.  Then, $K_m\setminus S_m$ has finite measure, and so by outer-regularity, there is some open $U_m\subseteq X$ such that $K_m\setminus S_m\subseteq U_m$ and $\meas \left( U_m\setminus (K_m\setminus S_m)\right) <1$.\footnote{Here, we have taken $\varepsilon =1$.}  Now define $L_m\ceqq K_m\cap U_m^{\comp}$.  $L_m$ is compact and $L_m\subseteq S_m$.  Furthermore,
\begin{equation}
\begin{split}
\meas (S_m\setminus L_m) & \ceqq \meas (S_m\cap (K_m^{\comp}\cup U_m) \\
& =\meas (S_m\cap U_m) \\
& \leq \meas (U_m\setminus (K_m\setminus S_m))<1.
\end{split}
\end{equation}
It follows that $\meas (L_m)>\meas (S_m)-1$.   As each $S_m$ is measurable, $S_m\subseteq S_{m+1}$, and $S=\bigcup _{m\in \N}S_m$, we have
\begin{equation}
\lim _m\meas (S_m)=\meas \left( \bigcup _{m\in \N}S_m\right) =\meas (S)=\infty ,
\end{equation}
and so in turn we have $\lim _m\meas (L_m)=\infty$, so that
\begin{equation}
\sup \left\{ \meas (K):K\subseteq S,\ K\text{ quasicompact.}\right\} =\infty ,
\end{equation}
as desired.
\end{proof}
\end{prp}
A related result is the following.
\begin{prp}{}{Semifinite}
Let $X$ be a $\sigma$-quasicompact topological space, let $\meas$ be a regular measure, and let $S\subseteq X$.  Then, if $\meas (S)=\infty$, then there is a subset $T\subseteq S$ with $0<\meas (T)<\infty$.
\begin{rmk}
This condition, that every set of infinite measure has a subset of finite positive measure is sometimes called \term{semifinite}\index{Semifinite}.
\end{rmk}
\begin{rmk}
In particular, topological measure spaces are semifinite.
\end{rmk}
\begin{proof}
Suppose that $\meas (S)=\infty$.  Write $X=\bigcup _{m\in \N}K_m$ for $K_m$ quasicompact.  Define $S_m\coloneqq S\cap K_m$.  As $K_m$ is quasicompact, it has finite measure, and so $S_m$ has finite measure.  We also have that
\begin{equation}
\infty =\meas (S)=\meas \left( \bigcup _{m\in \N}S_m\right) \leq \sum _{m\in \N}\meas (S_m),
\end{equation}
which means that the measure of at least some $S_m$ is strictly positive.
\end{proof}
\end{prp}
\begin{prp}{}{exr5.1.84}
Let $\coord{X,\meas}$ be a topological measure space and let $S\subseteq X$.  Then, if $\meas$ is inner-regular for $S$, then
\begin{enumerate}
\item \label{exr5.1.84.i}
\begin{equation}
\meas (S)=\sup \left\{ \meas (K):K\subseteq S,\ K\text{ compact.}\right\} ;
\end{equation}
and
\item \label{exr5.1.84.ii}
\begin{equation}
S=\bigcup _{m\in \N}K_m\cup Z,
\end{equation}
where $K_0\subseteq K_1\subseteq \cdots$ is a nondecreasing countable collection of compact sets and $\meas (Z)=0$.
\end{enumerate}
\begin{rmk}
To clarify, when we say that ``$\meas$ is inner-regular for $S$'', we mean that
\begin{equation*}
\meas (S)=\sup \left\{ \meas (K):K\subseteq S,\ K\text{ quasicompact.}\right\} .
\end{equation*}
Thus, the problem is to show you can can approximate $S$ not just with quasicompact sets, but in fact actual compact sets.
\end{rmk}
\begin{rmk}
The second part says that, `modulo a set of measure $0$', $S$ is a (countable nondecreasing) union of compact sets. 
\end{rmk}
\begin{rmk}
In particular, in topological measure spaces, this works for $S$ open (\cref{RegularMeasure}) and for $S$ measurable (\cref{InnerRegularFinite}).
\end{rmk}
\begin{proof}
Suppose that $\meas$ is inner-regular for $S$.

\cref{exr5.1.84.i} Define
\begin{equation}
\mcal{S}\ceqq \left\{ \meas (K):K\subseteq S,\ K\text{ compact.}\right\} .
\end{equation}
Certainly $\meas (S)$ is an upper-bound for $\mcal{S}$ because $\meas$ is nondecreasing.  To show that it is the \emph{least} upper-bound, let $\varepsilon >0$.  By hypothesis, there is then a \emph{quasi}compact subset $K\subseteq S$ with $\meas (S)-\varepsilon <\meas (K)\leq \meas (S)$.  Write $X=\bigcup _{m\in \N}L_m$ as a nondecreasing union of compact subsets $L_m\subseteq X$ and define $K_m\ceqq K\cap L_m$, so that $K=\bigcup _{m\in \N}K_m$, and hence $\meas (K)=\lim _m\meas (K_m)$.  Thus, there is some $m_0\in \N$ such that, whenever $m\geq m_0$, it follows that $\abs{\meas (K_m)-\meas (K)}<\varepsilon$.  From this, we have
\begin{equation}
\meas (S)-2\varepsilon <\meas (K)-\varepsilon <\meas (K_m)\leq \meas (S),
\end{equation}
and hence $\meas (S)=\sup (\mcal{S})$, as desired.

\blankline
\noindent
\cref{exr5.1.84.ii} Using the result of the previous part, let $K_0\subseteq S$ be compact and such that $\meas (S)-2^{-0}<\meas (K_0)$.  Similarly, let $K_1\subseteq S\setminus K_0$ be compact and such that $\meas (S\setminus K_0)-2^{-1}<\meas (K_1)$.  Similarly, let $K_2\subseteq S\setminus (K_0\cup K_1)$ be compact and such that $\meas (S\setminus (K_1\cup K_2)-2^{-2}<\meas (K_2)$.  Proceeding inductively, for every $m\in \N$, choose $K_m\subseteq S\setminus \bigcup _{k=0}^{m-1}K_k$ compact and such that
\begin{equation}
\meas \left( S\setminus \bigcup_{k=0}^{m-1}K_k\right) -2^{-m}<\meas (K_m).
\end{equation}

Define $L_m\ceqq \bigcup _{k=0}^mK_m$, $L\ceqq \bigcup _{m\in \N}L_m=\bigcup _{m\in \N}K_m$, and $Z\ceqq S\setminus L$.  Then, $L_0\subseteq L_1\subseteq \cdots$ is a nondecreasing countable collection of compact sets and
\begin{equation}
S=\bigcup _{m\in \N}L_m\cup Z,
\end{equation}
and so it only remains to show that $\meas (Z)=0$  However,
\begin{equation}
\meas (Z)\ceqq \meas \left( S\setminus \bigcup _{m\in \N}\right) \leq \meas \left( S\setminus \bigcup _{k=0}^mK_k\right) <2^{-m}.
\end{equation}
As this holds for every $m\in \N$, we have $\meas (Z)=0$, as desired.
\end{proof}
\end{prp}

In a topological measure space, we can approximate \emph{every} set from the outside by open sets, and we can approximate open sets from the inside by quasicompact sets.  One would then hope that every set is measurable `modulo a set of measure $0$'.  Unfortunately, this need not be the case.
\begin{exm}{A topological measure space in which not every subset is measurable `modulo a set of measure $0$'}{}
Precisely, what we mean by ``measurable `modulo a set of measure $0$''' is as follows.
\begin{important}
Let $S\subseteq X$.  Then, there is a measurable set $M\subseteq X$ such that $\meas (S\cap M^{\comp})+\meas (S^{\comp}\cap M)=0$.\footnote{In the `Venn diagram' of $S$ and $M$, $S\cap M^{\comp}\cup S^{\comp}\cap M$ is everything outside of $S\cap M$.  We are saying that, in particular, this set has measure $0$.  This is what we mean when we say that \emph{every} set is measurable `modulo' a set of measure $0$.}
\end{important}
We present an example of a topological measure space $X$ in which this is \emph{false}.

Define $X\ceqq \{ x_1,x_2\}$, equip $X$ with the indiscrete topology, and let $\meas$ be the Unit Measure on $X$.  $X$ is a finite set, and so certainly $\sigma$-compact (it is the union of its points, and points are always compact spaces in the subspace topology).  The only open sets are $\emptyset$ and $X$, which are always measurable, and so the measure is Borel.  Every subset has finite measure, and so in particular quasicompact sets have finite measure.

Let $S\subseteq X$.  If $S=\emptyset$, then as $\emptyset$ is itself open, we certainly have $\meas (S)=\inf \left\{ \meas (U):S\subseteq U,\ U\text{ open.}\right\}$.  Otherwise, the only open set containing $S$ is $X$ itself, in which case we have $1=\meas (S)=\inf \left\{ \meas (U):S\subseteq U,\ U\text{ open.}\right\}$ as well.  Thus, $\meas$ is outer-regular.

A similar check shows that it is inner-regular on open sets (there are only two cases to check of course, $U=\emptyset$ and $U=X$).

Thus, $\coord{X,\meas}$ is a topological measure space.

On the other hand, the only measurable subsets are $\emptyset$ and $X$ itself, and for either $M=\emptyset$ or $M=X$, we have
\begin{equation}
\meas (\{ x_1\} \cap M^{\comp})+\meas (\{ x_1\} ^{\comp}\cap M)=1.
\end{equation}

Thus, it is not the case that $\{ x_1\}$ is measurable `up to sets of measure $0$'.
\end{exm}

One likewise might expect that nonempty open sets have positive measure in topological measure spaces.  This is not true.
\begin{exm}{A measure on $\R$ for which $\coord{\R ,\meas}$ is a topological measure space, but is not strictly-positive---the Dirac Measure}{}\footnote{\term{Strictly-positive}\index{Strictly-positive (measure)} means just that---that every nonempty open set has positive measure.  The term here presumably comes from the Dirac Delta Function.}\index{Dirac Measure}
Define $\meas :2^{\R}\rightarrow [0,\infty ]$ by
\begin{equation}
\meas (S)\coloneqq \begin{cases}0 & \text{if }0\notin S \\ 1 & \text{if }0\in S.\end{cases}
\end{equation}

$\meas ((0,1))=0$ for example, so this is definitely not strictly-positive.  What we need to check is that it is a topological measure.

We first check that it is regular.  $\meas$ itself is finite, and so certainly finite on quasicompact sets.  Let $U\subseteq \R$ be open.  If $0\in U$, then as $\{ 0\}$ is quasicompact, we have that
\begin{equation}
\begin{split}
\meas (U) & =1 \\
& =\sup \{ \meas (K):K\subseteq U,\ K\text{ quasicompact}\} .
\end{split}
\end{equation}
On the other hand, if $0\notin U$, then no subset of $U$ will contain $0$, and so once again we have
\begin{equation}
\begin{split}
\meas (U) & =0 \\
& =\sup \{ \meas (K):K\subseteq U,\ K\text{ quasicompact}\} .
\end{split}
\end{equation}
Thus, $\meas$ is inner-regular on opens.  Now let $S\subseteq \R$ be arbitrary.  If $0\in S$, then every open set which contains $S$ will also contain $0$, and so we definitely have
\begin{equation}
\meas (S)=1=\inf \{ \meas (U):S\subseteq U\, U\text{ open}\} .
\end{equation}
On the other hand, if $S$ does not contain $0$, then $\{ 0\} ^{\comp}$ is an open set containing $S$, and so
\begin{equation}
\meas (S)=0=\inf \{ \meas (U):S\subseteq U\, U\text{ open}\} .
\end{equation}
Thus, $\meas$ is outer-regular, and hence regular.

We now check that $\meas$ is Borel.  To show this, obviously it suffices to show that every subset of $\R$ is measurable with respect to $\meas$.

So, let $M,S\subseteq \R$.  We would like to show that $M$ is measurable, and so we need to show that
\begin{equation}
\meas (S)=\meas (S\cap M)+\meas (S\cap M^{\comp}).
\end{equation}
There are two cases:  either $0\in M$ or $0\notin M$.  By $M\leftrightarrow M^{\comp}$, we may as well assume that $0\in M$.  Thus, we need to show that
\begin{equation}
\meas (S)=\meas (S\cap M).
\end{equation}
There are two cases:  $0\in S$ or $0\notin S$.  In the former case, this equation reads $1=1$, an in the later case it reads $0=0$.  Either way, it is true, and so $M$ is measurable.
\end{exm}

Before moving onto discussion of measures on uniform spaces, we must first at least discuss the property of \emph{topological-additivity}.
\begin{dfn}{Topological-additivity}{TopologicalAdditivity}
Let $X$ be a topological space and let $\meas$ be a measure on $X$.  Then, $\meas$ is \term{topologically-additive}\index{Topologically-additive} iff whenever $S,T\subseteq X$ are separated by neighborhoods, it follows that $\meas (S\cup T)=\meas (S)+\meas (T)$.
\begin{rmk}
Note that the Unit Measure on a two point space with the discrete topology (see \cref{exm5.1.42}) is an example of a regular measure that is not topologically-additive.  The point is, we do not get topological-additivity for free.  On the other hand, note that Borel measures are automatically topologically-additive---see \cref{exr5.1.56}.  In particular, topological measure spaces are topologically-additive.
\end{rmk}
\end{dfn}
\begin{exr}{}{exr5.1.56}
Let $\meas$ be a Borel measure on $X$.  Show that if $S,T\subseteq X$ are topologically-distinguishable, then $\meas (S\cup T)=\meas (S)+\meas (T)$.
\begin{rmk}
In particular, Borel measure are topologically-additive.  In fact, the converse is true for regular measures on $T_2$ spaces---see the next result.
\end{rmk}
\end{exr}
\begin{prp}{}{prp5.1.57}
Let $\meas$ be a regular topologically-additive measure on a $T_2$ space.  Then, $\meas$ is Borel.
\begin{rmk}
Thus, by the previous exercise, for $T_2$ spaces, Borel is equivalent to topological additivity.
\end{rmk}
\begin{wrn}
Warning:  This will fail if the space is not $T_2$---see \cref{exm5.1.43}.
\end{wrn}
\begin{proof}\footnote{Proof adapted from \cite[pg.~194]{Cohn}.}
Let $U\subseteq X$ be open and let $A\subseteq X$ be arbitrary.  We wish to show that
\begin{equation}
\meas (A)\geq \meas (A\cap U)+\meas (A\cap U^{\comp}).
\end{equation}
If $\meas (A)=\infty$, this is automatically satisfied, so we may as well assume that $\meas (A)<\infty$.

Let $\varepsilon >0$.  Then, by outer-regularity, there is some open set $U_\varepsilon$ that contains $A$ and
\begin{equation}
\meas (A)\leq \meas (U_{\varepsilon})<\meas (A)+\varepsilon .
\end{equation}
Then, by inner-regularity on opens, there is some quasicompact $K_{\varepsilon}\subseteq U_{\varepsilon}\cap U$ such that
\begin{equation}
\meas (U\cap U_{\varepsilon})-\varepsilon <\meas (K_{\varepsilon})\leq \meas (U\cap U_{\varepsilon}).
\end{equation}
As $X$ is $T_2$, $K_{\varepsilon}$ is closed, and so that $U_{\varepsilon}\cap K_{\varepsilon}^{\comp}$ is open, and so there is some compact $L_{\varepsilon}\subseteq U_{\varepsilon}\cap K_{\varepsilon}^{\comp}$ such that
\begin{equation}
\meas (U_{\varepsilon}\cap K_{\varepsilon}^{\comp})-\varepsilon <\meas (L_{\varepsilon})\leq \meas (U_{\varepsilon}\cap K_{\varepsilon}^{\comp}).
\end{equation}
Hence,
\begin{equation}
\begin{split}
\meas (A) & >\meas (U_{\varepsilon})-\varepsilon \geq \footnote{Because $K_{\varepsilon}\cup L_{\varepsilon}\subseteq U_{\varepsilon}$.}\meas (K_{\varepsilon}\cup L_{\varepsilon})-\varepsilon \\
& =\footnote{You can separate by neighborhoods disjoint compact subsets of $T_2$ spaces (\cref{exr4.6.39}).  Then we apply the fact that $\meas$ is topologically-additive.}\meas (K_\varepsilon )+\meas (L_{\varepsilon})-\varepsilon \\
& >\meas (U\cap U_{\varepsilon})+\meas (U_{\varepsilon}\cap K_{\varepsilon}^{\comp})-3\varepsilon \\
& \geq \footnote{Because $U_{\varepsilon}\cap U^{\comp}\subseteq U_{\varepsilon}\cap K_{\varepsilon}^{\comp}$.}\meas (U\cap U_{\varepsilon})+\meas (U_{\varepsilon}\cap U^{\comp})-3\varepsilon \\
& \geq \footnote{Because $A\subseteq U_{\varepsilon}$.}\meas (A\cap U)+\meas (A\cap U^{\comp})-3\varepsilon .
\end{split}
\end{equation}
As $\varepsilon$ is arbitrary, we have that $\meas (A)\geq \meas (A\cap U)+\meas (A\cap U^{\comp})$, and so $U$ is measurable.
\end{proof}
\end{prp}

\subsubsection{Measures on uniform spaces}

Of course, there is an analogue of topological-additivity (\cref{TopologicalAdditivity}) for uniform spaces.
\begin{dfn}{Uniform-additivity}{UniformAdditivity}
Let $X$ be a uniform space and let $\meas$ be a measure on $X$.  Then, $\meas$ is \term{uniformly-additive}\index{Uniformly-additive} iff whenever $S,T\subseteq X$ are uniformly-separated by neighborhoods, it follows that $\meas (S\cup T)=\meas (S)+\meas (T)$.
\begin{wrn}
Warning:  The term ``uniform'' most of the time is something strictly stronger than something only topological.  This is not the case here:  topological-additivity is superficially stronger than uniform-additivity because it is easier to be separated by neighborhoods than it is to be uniformly-separated by neighborhoods.  In particular, topologically-additive measures on uniform spaces are automatically uniformly-additive.
\end{wrn}
\begin{rmk}
The second condition is a generalization of the defining condition of what is called a \term{metric measure}\index{Metric measure}.  In particular, if $X$ is a metric space, then any uniform measure is (by definition) a metric measure.
\end{rmk}
\end{dfn}
\begin{exr}{}{}
Can you find an example of a regular measure $\meas$ on a uniform space $X$ with $S,T\subseteq X$ uniformly-separated by neighborhoods, but $\meas (S\cup T)\neq \meas (S)+\meas (T)$.
\begin{rmk}
The point is:  do we really need to assume uniform-additivity, or can we get it for free?
\end{rmk}
\begin{rmk}
Hint:  We have already encountered a counter-example that will do the trick.
\end{rmk}
\end{exr}
\begin{prp}{}{prp5.1.61}
Let $\meas$ be a regular measure on a $T_0$ uniform space.  Then, the following are equivalent.
\begin{enumerate}
\item \label{enm5.1.61.i}$\meas$ is topologically-additive.
\item \label{enm5.1.61.ii}$\meas$ is uniformly-additive.
\item \label{enm5.1.61.iii}$\meas$ is Borel.
\end{enumerate}
\begin{proof}
Let $X$ be the $T_0$ uniform space that $\meas$ is a regular measure on.

\blankline
\noindent
$(\text{\cref{enm5.1.61.i}}\Leftrightarrow \text{\cref{enm5.1.61.iii}})$ $T_0$ uniform spaces are $T_2$ (\cref{crl4.4.16}).  Therefore, by \cref{exr5.1.56,prp5.1.57}, topological-additivity is equivalent to being Borel.

\blankline
\noindent
$(\text{\cref{enm5.1.61.i}}\Rightarrow \text{\cref{enm5.1.61.ii}})$ Suppose that $\meas$ is topologically-additive.  Let $S,T\subseteq X$ be uniformly-separated by neighborhoods.  Then, $S$ and $T$ are separated by neighborhoods, and so $\meas (S\cup T)=\meas (S)+\meas (T)$.  Thus, $\meas$ is uniformly-additive.

\blankline
\noindent
$(\text{\cref{enm5.1.61.ii}}\Rightarrow \text{\cref{enm5.1.61.i}})$ Suppose that $\meas$ is uniformly-additive.  Let $S,T\subseteq X$  be separated by neighborhoods.  By definition, we want to show that $\meas (S\cup T)=\meas (S)+\meas (T)$.  By subadditivity, it suffices to show that $\meas (S\cup T)\geq \meas (S)+\meas (T)$.  If either one of the sets has infinite measure, then this inequality reads $\infty \geq \infty$, and so is automatically satisfied.  Thus, we may as well assume that $\meas (S)$ and $\meas (T)$ are finite.  Then, by outer-regularity, for every $\varepsilon >0$, there is an open set $W_{\varepsilon}$ with $S\cup T\subseteq W_{\varepsilon}$ and
\begin{equation}
\meas (S\cup T)\leq \meas (W_{\varepsilon})<\meas (S\cup T)+\varepsilon .
\end{equation}

On the other hand, because $S$ and $T$ are separated by neighborhoods, we know that there are disjoint open sets $U$ and $V$ with $S\subseteq U$ and $T\subseteq V$.  Define $U_\varepsilon \coloneqq U\cap W_{\varepsilon}$ and $V_\varepsilon \coloneqq V\cap W_{\varepsilon}$, so that $U_{\varepsilon}$ and $V_{\varepsilon}$ are both disjoint and open.  By inner-regularity on open sets, there are quasicompact $K_{\varepsilon}\subseteq U_{\varepsilon}$ and $L_{\varepsilon}\subseteq V_{\varepsilon}$ such that
\begin{equation}
\meas (U_{\varepsilon})-\varepsilon <\meas (K_{\varepsilon})\leq \meas (U_{\varepsilon})
\end{equation}
and
\begin{equation}
\meas (V_{\varepsilon})-\varepsilon <\meas (L_{\varepsilon})\leq \meas (V_{\varepsilon}).
\end{equation}
As $X$ is $T_2$, $K_{\varepsilon}$ and $L_{\varepsilon}$ are closed (\cref{exr4.6.40}), and so as $X$ is uniformly-$T_3$ (\cref{crl4.4.16}), $K_{\varepsilon}$ and $L_{\varepsilon}$ are uniformly-separated by neighborhoods.   Thus, by hypothesis, we have $\meas (K_{\varepsilon}\cup L_{\varepsilon})=\meas (K_{\varepsilon})+\meas (L_{\varepsilon})$.

Now,
\begin{equation}
\begin{split}
\meas (S)+\meas (T) & \leq \meas (U_{\varepsilon})+\meas (V_{\varepsilon}) \\
& <\meas (K_{\varepsilon})+\meas (L_{\varepsilon})+2\varepsilon \\
& =\meas (K_{\varepsilon}\cup L_{\varepsilon})+2\varepsilon \\
& \leq \meas (U_{\varepsilon}\cup V_{\varepsilon})+2\varepsilon \\
& \leq \meas (W_{\varepsilon})+2\varepsilon <\meas (S\cup T)+2\varepsilon .
\end{split}
\end{equation}
As $\varepsilon$ is arbitrary, it follows that $\meas (S)+\meas (T)\leq \meas (S\cup T)$, as desired.
\end{proof}
\end{prp}
\begin{exm}{A topologically-additive regular measure on a uniform space that is not Borel}{exm5.1.43}\footnote{Topologically-additive measures on uniform spaces are automatically uniformly-additive---see the remark in \cref{UniformAdditivity}.}
\begin{rmk}
The point is, you cannot drop the hypothesis of $T_0$ in \cref{prp5.1.61}.
\end{rmk}
Define $X\coloneqq \{ x_1,x_2,x_3\}$, and equip it the uniformity defined by the uniform base with just one cover, $\mcal{B}\coloneqq \{ \{ x_1,x_2\} ,\{ x_3\} \}$.  (This is a uniform base because this single cover star-refines itself.)  The neighborhood bases defined by this uniform base are
\begin{equation}
\begin{split}
\mcal{B}_{x_1} & =\{ \Star _{\mcal{B}}(x_1)\} =\{ \{ x_1,x_2\} \} \\
\mcal{B}_{x_2} & =\{ \Star _{\mcal{B}}(x_2)\} =\{ \{ x_1,x_2\} \} \\
\mcal{B}_{x_3} & =\{ \Star _{\mcal{B}}(x_3)\} =\{ \{ x_3\} \} .
\end{split}
\end{equation}
In particular, the open sets are\footnote{Recall that (\cref{prp3.1.11}), for a topology defined by a neighborhood base, a set $U$ is open iff every point $x\in U$ has an element $B\in \mcal{B}_x$ such that $x\in B\subseteq U$.  As our set only contains $3$ points, you can simply verify by hand that these are the only open sets.}
\begin{equation}
\emptyset ,X,\{ x_1,x_2\} ,\{ x_3\} .
\end{equation}

We define a measure $\meas :2^X\rightarrow [0,\infty ]$ by
\begin{equation}
\meas (S)\coloneqq \begin{cases}0 & \text{if }S=\emptyset ,\{ x_3\} \\ 1 & \text{otherwise.}\end{cases}
\end{equation}
Every set has finite measure, and so certainly every quasicompact set does.  There are only four open sets to check, and they are all quasicompact (because they are finite), and so certainly $\meas$ is inner-regular on each one of them.  Measures always are outer-regular on open sets, and so we only need to check outer-regularity on the remaining $8-4=$ sets that are not open.  Once again, there are only four things to check, and so you can just do so by hand.

We now check that this measure is uniformly-additive.  If $S$ and $T$ are uniformly-separated by neighborhoods, then, without loss of generality, we have that $S\subseteq \{ x_1,x_2\}$ and $T\subseteq \{ x_3\}$.  We automatically have that $\meas (S\cup T)=\meas (S)+\meas (T)$ if either $S$ or $T$ is empty, so we may without loss of generality assume that $T=\{ x_3\}$ and $S\subseteq \{ x_1,x_2\}$ is nonempty.  Then,
\begin{equation}
\meas (S\cup T)=1=1+0=\meas (S)+\meas (T),
\end{equation}
and so the measure is uniformly-additive.

On the other hand, it is not Borel, because $U\coloneqq \{ x_1,x_2\}$ is not measurable, as we now check.  Take $S\coloneqq \{ x_1\}$.  Then,
\begin{equation}
\meas (S)=1\neq 1+1=\meas (S\cap U)+\meas (S\cap U^{\comp}).
\end{equation}
\end{exm}

\subsubsection{A summary}

Before finally moving on to the \nameref{HaarHowesTheorem}, let us briefly summarize.
\begin{enumerate}
\item Borel means that open sets are measurable---see \cref{BorelMeasure}.
\item Regular means quasicompact sets have finite measure, inner-regularity on opens, and outer-regularity on everything---see \cref{RegularMeasure}.
\item A topological space is $\sigma$-(quasi)compact iff it is the countable union of (quasi)compact sets---see \cref{SigmaQuasicompact}.
\item A topological measure space is a $\sigma$-compact topological space equipped with a regular Borel measure---see \cref{TopologicalMeasureSpace}.
\item A subset $M$ of a topological measure space is measurable iff for every $\varepsilon >0$ there is an open set $U_{\varepsilon}$ containing $M$ and a closed set $C_{\varepsilon}$ contained in $M$ such that $\meas (U_{\varepsilon}\setminus C_{\varepsilon})<\varepsilon$.
\item Topological-additivity means that the measure is additive on two sets which are separated by neighborhoods---see \cref{TopologicalAdditivity}.
\item Uniform-additivity means that the measure is additive on two sets which are uniformly-separated by neighborhoods---see \cref{UniformAdditivity}.
\item Topological-additivity and uniform-additivity are both equivalent to being Borel for $T_2$ spaces---see \cref{prp5.1.61}.
\item On the other hand, neither regular topologically-additive nor regular uniformly-additive measures need be Borel (\cref{exm5.1.43}).
\end{enumerate}

\subsection{The Haar-Howes Theorem}

It turns out that there is a theorem, the \nameref{HaarHowesTheorem},\footnote{Warning:  This is the name I have chosen to call it, because I am unaware of another name for this result.  In particular, don't expect others to know what you're talking about if you reference this theorem by name.  (It is a generalization of the existence and essential uniqueness of haar measure\index{Haar measure}, in case you are more familiar with that.)  While the proof of the result I cobbled together from other sources, the formulation of the result I found essentially in \cite{Howes}, in which he claims ``the author'' established essential uniqueness, as well as existence, which he and another mathematician (Izkowitz) established independently.}  that is quite general and will just spit out regular uniformly-additive for us.  This is how we will construct Lebesgue measure.  That being said, it doesn't just work for any old uniform space.  We're going to need extra structure:
\begin{dfn}{Isogeneous space}{IsogeneousSpace}
An \term{isogeneous space} is a uniform space $X$ equipped with a group of uniform-homeomorphisms $\Phi$ such that
\begin{equation}\label{5.1.28}
\uniformity{B}_\Phi \coloneqq \{ \mcal{B}_U\}
\end{equation}
where
\begin{equation*}
\mcal{B}_U\coloneqq \{ \phi (U):\phi \in \Phi\} \text{ and }U\subseteq X\text{ nonempty open}.
\end{equation*}
is a uniform base for $X$.
\begin{rmk}
That is to say, for every open subset $U\subseteq X$, you obtain a single uniform cover in $\uniformity{B}_{\Phi}$, and that single cover is given by `translating' $U$ around using $\Phi$.
\end{rmk}
\begin{rmk}
$\Phi$ is called the \term{group of symmetries} of $X$.  It is a subgroup of $\Aut _{\Uni}(X)$.  $\uniformity{B}_{\Phi}$ is the \term{isogeneous base}\index{Isogeneous base} and each $\mcal{B}_U$ is an \term{isogeneous cover}\index{Isogeneous cover}.
\end{rmk}
\begin{rmk}
The example you should have in mind here is that of a topological group $G$.  In this case, take $\Phi$ to be the set of all uniform-homeomorphisms given by left multiplication $\Phi \coloneqq \{ L_g:g\in G\}$.\footnote{$L_g\colon G\rightarrow G$ is defined by $L_g(x)\ceqq gx$.}  Then, the corresponding isogeneous base is the canonical one, $\uniformity{B}\coloneqq \{ \mcal{B}_U\}$ with $\mcal{B}_U\coloneqq \{ gU:g\in U\}$ for $U$ an open neighborhood of the identity.\footnote{What happened to the uniform covers for $U\subseteq G$ open but not necessarily containing the identity?}  Of course, you can also choose right-translations over left-translations if you so desire.
\end{rmk}
\begin{rmk}
The example of metric spaces with $\Phi$ the group of isometries should also guide your intuition---see \cref{exr5.1.109}.
\end{rmk}
\begin{rmk}
What do you think the morphisms of isogeneous spaces should be?
\end{rmk}
\end{dfn}
\begin{exr}{}{exr5.1.48}
Let $X$ be a uniform space with uniform topology $\mcal{U}$, let $\mcal{B}$ be a base for the topology, and let $\Phi$ be a subgroup of $\Aut _{\Uni}(X)$.  Show that
\begin{equation}\label{5.1.49}
\left\{ \mcal{B}_B:B\in \mcal{B}\right\}
\end{equation}
is a uniform base iff $\uniformity{B}_\Phi$ is, and that these two uniform bases both define the same uniformity.
\begin{rmk}
The point is that, in order to show that $\Phi$ makes $X$ into an isogeneous space, we only need to check that the smaller collection of covers in \eqref{5.1.49}, the covers coming from only elements of the base instead of \emph{all} open sets, form a uniform base.
\end{rmk}
\end{exr}
\begin{exr}{}{}
Let $G$ be a topological group and let $\Phi \coloneqq \{ \phi _g:G\in G\}$, where $\phi _g:G\rightarrow G$ is defined by $\phi _g(x)\coloneqq gx$.  Show that $\coord{G,\Phi}$ is an isogeneous space.
\end{exr}
\begin{exr}{}{exr5.1.109}
Let $\coord{X,\metric}$ be a metric space and let $\Phi \subseteq \Aut _{\Uni}(X)$ be the group of isometries.  Show that $\coord{X,\Phi}$ is an isogeneous space.
\begin{rmk}
A function $f\colon \coord{X,\metric}\rightarrow \coord{Y,\metric}$ between metric spaces is an \term{isometry}\index{Isometry} iff it is an isomorphism in the category of metric spaces.  Concretely, this means that it is bijective\footnote{This condition implies that $f$ must be injective, but it need not be surjective.} and satisfies $\metric{f(x_1)}{f(x_2)}=\metric{x_1}{x_2}$ for all $x_1,x_2\in X$.
\end{rmk}
\begin{rmk}
Hint:  Apply \cref{exr5.1.48} to the base consisting of all $\varepsilon$-balls.
\end{rmk}
\end{exr}
\begin{dfn}{Uniformly-measurable}{UniformlyMeasurable}
Let $\meas :2^X\rightarrow [0,\infty ]$ be a measure on a set $X$ and let $\mcal{U}$ is a cover of $X$.  Then, $\mcal{U}$ is \term{uniformly-measurable}\index{Uniformly-measurable cover} iff $\meas$ is constant on $\mcal{U}$.  A uniform base consisting of uniformly-measurable covers is a \term{uniformly-measurable base}\index{Uniformly-measurable base}.
\begin{rmk}
Think about what having a uniform base of uniformly-measurable covers means for a metric space---if we take as a uniform base the collection of all covers by $\varepsilon$-balls, then the statement that this uniform base is uniformly-measurable\footnote{It doesn't have to be of course---it will depend on our choice of measure.} is just the statement that every $\varepsilon$-ball has to have the same measure.
\end{rmk}
\begin{rmk}
Note that you definitely do not want to require \emph{every} uniform cover be uniformly-measurable.  For example, in a metric space, by upward-closedness the collection of all $\varepsilon$-balls together with a single $2\varepsilon$-ball will also be a uniform-cover---we definitely do not want to require that a $2\varepsilon$-ball has the same measure as an $\varepsilon$-ball.
\end{rmk}
\end{dfn}
\begin{dfn}{Isogeneous measure}{}
Let $\coord{X,\Phi}$ be an isogeneous space.  An \term{isogeneous measure}\index{Isogeneous measure} is a uniformly-additive regular measure for which $\uniformity{B}_\Phi$ is a uniformly-measurable base.
\begin{rmk}
Explicitly, this means that
\begin{enumerate}
\item $\meas$ is uniformly-additive;
\item $\meas$ is regular; and
\item $\meas (\phi (U))=\meas (U)$ for $\phi \in \Phi$ and $U\subseteq X$ open.
\end{enumerate}
\end{rmk}
\end{dfn}
\begin{exr}{}{}
Let $\meas$ be an isogeneous measure on an isogeneous space $\coord{X,\Phi}$.  Show that $\meas (\phi (S))=\meas (S)$ for all $\phi \in \Phi$ and $S\subseteq X$.
\begin{rmk}
In other words, this holds for \emph{all} $S$, not just open $S$.
\end{rmk}
\end{exr}
\begin{exr}{}{}
Let $\meas$ be an isogeneous measure on an isogeneous space $\coord{X,\Phi}$, let $M\subseteq X$ be measurable, and let $\phi \in \Phi$.  Show that $\phi (M)$ is measurable.
\begin{rmk}
In other words, the symmetries of the isogeneous space preserve measurability (in particular, for Lebesgue measure, we will have that isometries of $\R ^d$ preserve measurability---see \cref{LebesgueMeasure} (the definition of Lebesgue measure)).
\end{rmk}
\end{exr}

And finally now the key result that will allow us to define basically every measure we work with in these notes (and much more).
\begin{thm}{Haar-Howes Theorem}{HaarHowesTheorem}\index{Haar-Howes Theorem}
Let $\coord{X,\Phi}$ be a locally compact\footnote{Equivalently, $T_0$ and locally quasicompact.  Locally compact implies locally $T_2$ implies $T_1$ implies $T_0$.  Conversely, $T_0$ implies uniformly-completely-$T_3$ implies $T_2$ and subspaces of $T_2$ spaces are $T_2$.} isogeneous space and let $K\subseteq X$ be quasicompact with nonempty interior.  Then, there exists a unique isogeneous measure $\meas$ on $X$ such that $\meas (K)=1$.
\begin{rmk}
You should think of $K$ has a set with which we can `compare' all other sets to get a ``measure'' of `size'.  The condition that it be quasicompact you can think of the condition that the measure of $K$ be finite, and the condition that it have nonempty interior you can think of the condition that the measure of $K$ be positive.  If $\meas (K)$ is neither infinite nor zero, then we can `normalize' to get $\meas (K)=1$.
\end{rmk}
\begin{rmk}
Classically, the term ``Haar measure'' is reserved for $G$ a $T_0$ locally quasicompact group with $\Phi$ the set of all left-translations.  In particular, \emph{Lebesgue measure} will be the Haar measure for the topological group $\coord{\R ,^d+,0,-}$.\footnote{If the group is commutative, the symmetries by left translation and right translations are the same, so left vs.~right does not matter.}
\end{rmk}
\begin{rmk}
Yes, there are $\sigma$-compact spaces which are not locally quasicompact and vice-versa---see \cref{exr5.1.162}.
\end{rmk}
\begin{proof}
\Step{Make hypotheses and introduce notation}
Let $K_0\subseteq X$ be a quasicompact subset with nonempty interior.  Denote the uniform topology on $X$ by $\topology{U}$ and denote the collection of all quasicompact subsets of $X$ by $\collection{K}$

\Step{Define $(K:U)$ for $K\in \collection{K}$ and $U\in \topology{U}$}
The cover $\cover{B}_U\coloneqq \{ \phi (U):h\in \Phi\}$ is an open cover of $K$, and so there is a finite subcover.  Let $(K:U)$ denote the cardinality of the smallest such subcover.

\Step{Define $\mrm{H}_U:\collection{K}\rightarrow \R _0^+$}
For $U\in \topology{U}$, define $\mrm{H}_U:\collection{K}\rightarrow \R _0^+$ by
\begin{equation}
\mrm{H}_U(K)\coloneqq \frac{(K:U)}{(K_0:U)}.\footnote{$K_0$ is nonempty, and so cannot be covered by anything empty.  Therefore, $(K_0:U)\geq 1$, and in particular, is not $0$.}
\end{equation}

\Step{Show that $\mrm{H}_U(K)\leq (K:\Int (K_0))$}
We now check that $\mrm{H}_U(K)\leq (K:\Int (K_0))$, that is, $(K:U)\leq (K:\Int (K_0))(K_0:U)$.  Let us temporarily write $m\coloneqq (K:\Int (K_0))$ and $n\coloneqq (K_0:U)$.  There are thus $\phi _1,\ldots ,\phi _m\in \Phi$ such that $\{ \phi _1(\Int (K_0)),\ldots ,\phi _m(\Int (K_0))\}$ covers $K$, and $\phi _1',\ldots ,\phi _n'\in \Phi$ such that $\{ \phi _1'(U),\ldots ,\phi _n'(U)\}$ covers $K_0$.  Therefore,
\begin{equation}
\begin{split}
K & \subseteq \bigcup _{k=1}^m\phi _k(\Int (K_0))\subseteq \bigcup _{k=1}^m\phi _k(K_0) \\
& \subseteq \bigcup _{k=1}^m\phi _k\bigg( \bigcup _{l=1}^n\phi _l'(U)\bigg) =\bigcup _{k=1}^m\bigcup _{l=1}^n[\phi _k\circ \phi _l'](U)
\end{split}
\end{equation}
Hence, $K$ is covered by $mn$ elements of $\mcal{B}_U$, and hence $(K:U)\leq mn\coloneqq (K:\Int (K_0))(K_0:U)$.

\Step{Define $\mrm{H}:\mcal{K}\rightarrow \R _0^+$}[Haar.4x]
Define $\mcal{H}\coloneqq \prod _{K\in \mcal{K}}[0,(K:\Int (K_0))]$.  Each $\mrm{H}_U$ may be thought of as a point in $\mcal{H}$, whose component at $K\in \mcal{K}$ is $\mrm{H}_U(K)\in [0,(K:K_0)]$.\footnote{That was sort of the point of the previous step.}  Thus, for $U\in \mcal{U}$, let us define
\begin{equation}
C_U\coloneqq \Cls \left( \left\{ \mrm{H}_V:\topology{U}\ni V\subseteq U\right\} \right) 
\end{equation}
and
\begin{equation}
\collection{C}\coloneqq \{ C_U:U\in \topology{U}\} .
\end{equation}
We wish to show that the intersection of any finitely many elements of $\collection{C}$ is nonempty.  Then, because $\mcal{H}$ is quasicompact by \namerefpcref{TychonoffsTheorem}, it will follow that the intersection over \emph{all} elements in $\collection{C}$ will be nonempty (\cref{prp4.2.32}).

This is actually really easy, however, because for $U_1,\ldots ,U_m\in \topology{U}$, we have that
\begin{equation}
\mrm{H}_{U_1\cap \cdots \cap U_m}\in \bigcap _{k=1}^mC_{U_k}.
\end{equation}
Therefore, by quasicompactness, there is some
\begin{equation}
\mrm{H}\in \bigcap _{U\in \topology{U}}C_U.
\end{equation}
Though $\mrm{H}$ is an element of the product $\mcal{H}\ceqq \prod _{K\in \collection{K}}[0,(K,\Int (K_0))]$, we can regard $\mrm{H}$ as a function $\mrm{H}\colon \collection{K}\rightarrow \R _0^+$ by defining $\mrm{H}(K)\coloneqq \mrm{H}_K$, that is, the value of $\mrm{H}$ at $K\in \collection{K}$ is the $K$-component of $\mrm{H}\in \prod _{K\in \collection{K}}[0,(K,\Int (K_0))]$.

\Step{Show that $\mrm{H}(K_1)\leq \mrm{H}(K_2)$ if $K_1\subseteq K_2$}[Haar.4]
Let $K_1,K_2\in \collection{K}$ be such that $K_1\subseteq K_2$.  We first show that, for each $U\in \topology{U}$, $\mrm{H}_U(K_1)\leq \mrm{H}_U(K_2)$.  But this is trivial, because the covering of $K_2$ with $(K_2:U)$ elements of $\cover{B}_U$ is also a covering of $K_1$ with $(K_2:U)$ elements of $\mcal{B}_U$, so that $(K_1:U)\leq (K_2:U)$, and hence $\mrm{H}_U(K_1)\leq \mrm{H}_U(K_2)$.

Thinking of elements $f$ of $\mcal{H}$ as functions from $\collection{K}$ to $\R$, consider the map\footnote{For each $K_1,K_2\in \collection{K}$ with $K_1\subseteq K_2$, we have such a map.} that sends $f\in \mcal{H}$ to $f(K_2)-f(K_1)$.  This is a composition of continuous functions, and hence continuous.\footnote{The first map from $\mcal{H}$ into $\R \times \R$ is the projection of $f\in \mcal{H}$ onto the $K_1^{\text{th}}$ coordinate in the first coordinate and the projection of $f\in \mcal{H}$ onto $K_2^{\text{th}}$ coordinate in the second coordinate.  This map is continuous because it is continuous in each coordinate (each coordinate is continuous because projections are continuous).  The first map is followed by the map from $\R \times \R$ into $\R$ given by subtraction, which is continuous because we know that $\coord{R,+,0,-}$ is a topological group.}  This map is also nonnegative on each $C_U$ because $\mrm{H}_U(K_1)\leq \mrm{H}_U(K_2)$ for each $U\in \mcal{U}$ (we need continuity so that we know it is nonnegative on the \emph{closure} of $\{ \mrm{H}_V:V\subseteq U\}$).  As $\mrm{H}$ is an element of each $C_U$, it follows that this map is also nonnegative at $\mrm{H}$, so that $\mrm{H}(K_1)\leq \mrm{H}(K_2)$.

\Step{Show that $\mrm{H}(K_1\cup K_2)\leq \mrm{H}(K_1)+\mrm{H}(K_2)$}[Haar.5]
Let $K_1,K_2\in \collection{K}$.  We first show that $\mrm{H}_U(K_1\cup K_2)\leq \mrm{H}_U(K_1)+\mrm{H}_U(K_2)$ for each $U\in \topology{U}$.  This is trivial, because a covering of $K_1$ with $(K_1:U)$ elements of $\cover{B}_U$ together with a covering of $K_2$ with $(K_2:U)$ elements of $\cover{B}_U$ is a cover of $K_1\cup K_2$ with $(K_1:U)+(K_2:U)$ elements of $\cover{B}_U$, so that $(K_1\cup K_2:U)\leq (K_1:U)+(K_2:U)$.  It follows that $\mrm{H}_U(K_1\cup K_2)\leq \mrm{H}_U(K_1)+\mrm{H}_U(K_2)$.

Proceeding similarly as in \cref{Haar.4}, the map that sends $f\in \mcal{H}$ to $f(K_1)+f(K_2)-f(K_1\cup K_2)$ is continuous and nonnegative on each $C_U$, and hence is nonnegative for $\mrm{H}\in \mcal{H}$.  Thus, $\mrm{H}(K_1\cup K_2)\leq \mrm{H}(K_1)+\mrm{H}(K_2)$.

\Step{Show that $\mrm{H}_U(K_1\cup K_2)=\mrm{H}_U(K_1)+\mrm{H}_U(K_2)$ if $K_1$ and $K_2$ are uniformly-separated by neighborhoods}[Haar.8]
Let $K_1,K_2\in \mcal{K}$ be uniformly-separated by neighborhoods with respect to $\cover{B}_U$.  We have already shown that $\mrm{H}_U(K_1\cup K_2)\leq \mrm{H}_U(K_1)+\mrm{H}_U(K_2)$, so it suffices to show that $\mrm{H}_U(K_1)+\mrm{H}_U(K_2)\leq \mrm{H}_U(K_1\cup K_2)$.  In other words, it suffices to show that $(K_1:U)+(K_2:U)\leq (K_1\cup K_2:U)\eqqcolon m$.  Let $\phi _1(U),\ldots ,\phi _m(U)\in \cover{B}_U$ be a cover of $K_1\cup K_2$.  By hypothesis,\footnote{This is the definition of uniformly-separated---see \cref{UniformlySeparated}.} every single one of these can only intersect $U_1$ or $U_2$, but not both.  Thus, after relabeling if necessary, the first $k$ of these guys will form a cover of $K_1$ and the latter $m-k$ will form a cover of $K_2$.  Thus, $(K_1:U)\leq k$ and $(K_2:U)\leq m-k$, and so $(K_1:U)+(K_2:U)\leq k+(m-k)=m\coloneqq (K_1\cup K_2:U)$, which completes this step.

\Step{Show that $\mrm{H}(K_1\cup K_2)=\mrm{H}(K_1)+\mrm{H}(K_1)$ if $K_1$ and $K_2$ are uniformly-separated}[Haar.9]
Proceeding similarly as in \cref{Haar.5}, the map that sends $f\in \mcal{H}$ to $f(K_1)+f(K_2)-f(K_1\cup K_2)$ is continuous and vanishes on each $C_U$, and hence vanishes for $\mrm{H}\in \mcal{H}$.  Thus, $\mrm{H}(K_1\cup K_2)=\mrm{H}(K_1)+\mrm{H}(K_2)$.

\Step{Define the measure $\meas$ on all open subsets of $X$}
For $U\subseteq X$ open, define
\begin{equation}\label{5.1.46}
\meas (U)\coloneqq \sup \{ \mrm{H}(K):K\subseteq U,\ K\in \mcal{K}\} .
\end{equation}

\Step{Extend $\meas$ to all subsets of $X$}
Now, for an arbitrary subsets $S$ of $X$, define
\begin{equation}\label{5.1.38}
\meas (S)\coloneqq \inf \{ \meas (U):S\subseteq U,\ U\in \topology{U}\} .
\end{equation}
\begin{exr}[breakable=false]{}{}
Show that this agrees with \eqref{5.1.46} when $S$ is open, so that this is indeed an extension.
\end{exr}

\Step{Show that $\meas$ is a measure}
\begin{exr}[breakable=false]{}{}
Check that $\meas (\emptyset )=0$ and that $\meas$ is nondecreasing.
\end{exr}

We now check that it is subadditive.  To prove this, we will first need a lemma.
\begin{lma}[breakable=false]{}{}
Let $X$ be locally compact, let $K\subseteq X$ be quasicompact, and let $U_1,U_2\subseteq X$ be open and such that $K\subseteq U_1\cup U_2$.  Then, there are quasicompact subsets $K_1,K_2\subseteq X$ such that (i)~$K_1\subseteq U_1$, (ii)~$K_2\subseteq U_2$, and (iii)~$K=K_1\cup K_2$.
\begin{rmk}
Note that nothing here is necessarily disjoint.
\end{rmk}
\begin{proof}
For each $x\in U_1$ and $y\in U_2$, let $U_x$ and $V_x$ be open neighborhoods of $x$ and $y$ contained in $U_1$ and $U_2$ respectively.  By \cref{prp5.2.4}, there are open sets $U_x'\subseteq X$ and $V_x'\subseteq X$ with compact closure such that $x\in U_x'\subseteq \Cls (U_x')\subseteq U_1$ and similarly for $V_x'$.  Thus, without loss of generality, suppose that each $U_x$ and $V_x$ has compact closure and $\Cls (U_x)\subseteq U_1$ and $\Cls (V_y)\subseteq U_2$.

The $U_x$s and $V_y$s together cover $U_1\cup U_2$, and so cover $K$, and so there are finitely many $x_1,\ldots ,x_m\in U_1$ and $y_1,\ldots ,y_n\in U_2$ such that
\begin{equation}
U_{x_1}\cup \cdots \cup U_{x_m}\cup V_{y_1}\cup \cdots \cup V_{y_n}.
\end{equation}
Define
{\small
\begin{subequations}
\begin{align}
K_1 & \ceqq \left( \Cls (U_{x_1})\cup \cdots \cup \Cls (U_{x_m})\right) \cap K \\
K_2 & \ceqq \left( \Cls (V_{x_1})\cup \cdots \cup \Cls (V_{x_n})\right) \cap K.
\end{align}
\end{subequations}
}
These are both compact, $K_i\subseteq U_i$, and $K=K_1\cup K_2$, as desired.
\end{proof}
\end{lma}
As $X$ is a $T_0$ uniform space, it is in particular $T_2$, and so it is not just locally quasicompact, but in fact locally compact, so that we may indeed apply the lemma.

We now show subadditivity for \emph{open} sets.  (We will then prove subadditivity in general.)  So, let $\{ U_m:m\in \N \}$ be a countable collection of open sets of $X$.  Let $K\subseteq \bigcup _{m\in \N}U_m$.  Then, there is some $m_K\in \N$ such that $K\subseteq \bigcup _{k=1}^{m_K}U_k$.  By applying this lemma inductively then, we may find quasicompact sets $K_1,\ldots ,K_m$ such that (i)~$K_k\subseteq U_k$ for $0\leq k\leq m$ and $K=\bigcup _{k=1}^mK_k$.  Using the fact that we have already proved finite `subadditivity' (of $\mrm{H}$) for quasicompact sets (\cref{Haar.5}), we find that
\begin{equation}
\mrm{H}(K)\leq \sum _{k=1}^m\mrm{H}(K_k)\leq \sum _{k=1}^m\meas (U_k)\leq \sum _{m\in \N}\meas (U_m).
\end{equation}
Taking the $\sup$ over $K\in \collection{K}$ such that $K\subseteq \bigcup _{m\in \N}U_m$, we find that
\begin{equation}
\begin{split}
\MoveEqLeft
\meas \bigg( \bigcup _{m\in \N}U_m\bigg) \\
& \coloneqq \sup \left\{ \mrm{H}(K):K\subseteq \bigcup _{m\in \N}U_m,\ K\in \collection{K}\right\} \\
& \leq \sum _{m\in \N}\meas (U_m).
\end{split}
\end{equation}

Having proved subadditivity for open sets, we now prove it for arbitrary sets.  So, let $\{ S_m:m\in \N \}$ be an arbitrary countable collection of subsets of $X$.  If $\sum _{m\in \N}\meas (S_m)=\infty$, then there is nothing to show, and so we may as well suppose that $\sum _{m\in \N}\meas (S_m)<\infty$.  Let $\varepsilon >0$ and for each $m\in \N$ pick an open set $U_m$ such that (i)~$S_m\subseteq U_m$ and (ii)~$\meas (S_m)\leq \meas (U_m)<\meas (S_m)+\frac{\varepsilon}{2^m}$.  Then, using subadditivity for open sets, we find
\begin{equation}
\begin{split}
\meas \bigg( \bigcup _{m\in \N}S_m\bigg) & \leq \meas \left( \bigcup _{m\in \N}U_m\right) \leq \sum _{m\in \N}\meas (U_m) \\
& <\sum _{m\in \N}\left[ \meas (S_m)+\tfrac{\varepsilon}{2^m}\right] \\
& =\sum _{m\in \N}\meas (S_m)+2\varepsilon .
\end{split}
\end{equation}
Hence, as $\varepsilon >0$ was arbitrary, we have that
\begin{equation}
\meas \left( \bigcup _{m\in \N}S_m\right) \leq \sum _{M\in \N}\meas (S_m).
\end{equation}
Thus, $\meas$ is a measure on $X$.

\Step{Show that each $\cover{B}_U$ is uniformly-measurable with respect to $\meas$}[Haar.11]
Let $\phi \in \Phi$.  We want to show that $\meas (\phi (U))=\meas (U)$.  Then, for any other $\phi '\in \Phi$, we will have that $\meas (\phi (U))=\meas (U)=\meas (\phi '(U))$, so that indeed every element of $\cover{B}_U$ has the same measure.

However, $K$ is a quasicompact set contained in $U$ iff $\phi (K)$ is a quasicompact set contained in $\phi (U)$.  Therefore, by the definition of $\meas (U)$ \eqref{5.1.46} it suffices to show that $\mrm{H}(K)=\mrm{H}(\phi (K))$ for all $K\in \collection{K}$.  To show this, we first show that $\mrm{H}_U(K)=\mrm{H}_U(h(K))$ for all $U\in \topology{U}$.  That is, we would like to show that $(K:U)=(\phi (K):U)$.  However, every cover of $K$ by elements of $\cover{B}_U$, $\phi _1(U),\ldots ,\phi _m(U)$, gives a cover of $\phi (K)$ by elements of $\cover{B}_U$ of the same cardinality, $\phi (\phi _1(U)),\ldots ,\phi (\phi _m(U))$.  It thus follows that $\mrm{H}_U(K)=\mrm{H}_U(h(K))$.

For $\phi \in \Phi$ fixed, consider the map from $\mcal{H}$ to $\R$ that sends $f$ to $f(\phi (K))-f(K)$.  We just showed that this is $0$ on each $\mrm{H}_U\in T$, and so it is $0$ on $C_U$, and so it is $0$ on $\mrm{H}$, that is, $\mrm{H}(K)=\mrm{H}(\phi (K))$.

\Step{Show that if $S$ and $T$ are uniformly-separated, then $\meas (S\cup T)=\meas (S)+\meas (T)$}[Haar.12]
Note that we always have that $\meas (S\cup T)\leq \meas (S)+\meas (T)$, and so it suffices to show that $\meas (S\cup T)\geq \meas (S)+\meas (T)$.

We first prove this for open sets.  So, let $U,V\in \topology{U}$.  If either $U$ or $V$ s has infinite measure, then this inequality just reads $\infty \geq \infty$, and is so automatically satisfied.  Thus, without loss of generality, assume that $\meas (U),\meas (V)<\infty$.  Let $\varepsilon >0$.  Then, there is are some $K,L\in \collection{K}$ such that $K\subseteq U$, $L\subseteq V$, and
\begin{equation*}
\meas (U)-\varepsilon <\mrm{H}(K)\leq \meas (U)\text{ and }\meas (V)-\varepsilon <\mrm{H}(L)\leq \meas (V).
\end{equation*}
If $U$ and $V$ are uniformly-separated, then certainly $K$ and $L$ are uniformly-separated, and so by \cref{Haar.9}, we have that
\begin{equation}
\mrm{H}(K\cup L)=\mrm{H}(K)+\mrm{H}(L),
\end{equation}
and so
\begin{equation}
\begin{split}
\meas (U\cup V) & \geq \mrm{H}(K\cup L)=\mrm{H}(K)+\mrm{H}(L) \\
& >\meas (U)+\meas (V)-2\varepsilon .
\end{split}
\end{equation}
Hence, $\meas (U\cup V)\geq \meas (U)+\meas (V)$.

We now do the general case.  Once again, if either $S$ or $T$ has infinite measure, we are done, so we may as well suppose that $\meas (S),\meas (T)<\infty$.

Our first order of business it to show that there are \emph{some} open sets containing $S$ and $T$ respectively which are uniformly-separated.

Look at any open cover $\cover{B}$ which uniformly-separates $S$ and $T$, and take an open star-refinement $\cover{C}$ of this\footnote{Note that every $\cover{B}_U$ is an open cover---both $\cover{B}$ and $\cover{C}$ are secretly of the form $\cover{B}_U$ and $\cover{B}_V$ for some $U,V\subseteq X$ open---so there is no need to say ``open'' here---it is just to clarify.  (We don't write $\cover{B}_U$ or $\cover{B}_V$ to simplify the notation (and also because we will want to write $U$ for something else).)}.  Define $U\coloneqq \Star _{\cover{C}}(S)$ and $V\coloneqq \Star _{\cover{C}}(T)$.  We wish to show that $U$ and $V$ are uniformly-separated with respect to $\cover{C}$.  Because $\mcal{B}$ uniformly-separates $S$ and $T$, by definition (see \cref{UniformlySeparated}), we have that $\Star _{\cover{B}}(S)$ and $\Star _{\cover{B}}(T)$ are disjoint.  Therefore, it suffices to show that $\Star _{\cover{C}}(U)\subseteq \Star _{\cover{B}}(S)$ (and similarly for $V$).  So, suppose that $C\in \cover{C}$ intersects $U$.  Then, by definition of $U$, it must intersect some element $C'\in \cover{C}$ which intersects $S$.  Let $B\in \cover{B}$ be such that $\Star _{\cover{C}}(C')\subseteq B$.  We then have that
\begin{equation}
C\subseteq \footnote{Because $C$ and $C'$ intersect.}\Star _{\cover{C}}(C')\subseteq B\subseteq \footnote{Because $B$ contains $C'$, which intersects $S$.}\Star _{\cover{B}}(S).
\end{equation}
Thus, indeed, $\Star _{\cover{C}}(U)\subseteq \Star _{\cover{B}}(S)$.

So, let $U,V\in \topology{U}$ be open sets containing $S$ and $T$ respectively which are uniformly-separated.  Let $\varepsilon >0$, and choose $W\in \topology{U}$ that contains $S\cup T$ and satisfies
\begin{equation}
\meas (S\cup T)\leq \meas (W)<\meas (S\cup T)+\varepsilon .
\end{equation}
Let us replace $U$ and $V$ by $U\cap W$ and $V\cap W$---upon doing so, it will still be the case that $U,V\in \topology{U}$, it will still be the case that $S\subseteq U$ and $T\subseteq V$, and it will still be the case that $U$ and $V$ are uniformly-separated, but now we will also have that $\meas (U\cup V)\leq \meas (W)$.  Then we have
\begin{equation}
\begin{split}
\meas (S)+\meas (T)+\varepsilon & \leq \meas (U)+\meas (V)+\varepsilon \\
& =\meas (U\cup V)+\varepsilon \\
& \leq \meas (W)+\varepsilon <\meas (S\cup T)+2\varepsilon .
\end{split}
\end{equation}
As $\varepsilon$ is arbitrary, we have that
\begin{equation}
\meas (S)+\meas (T)\leq \meas (S\cup T),
\end{equation}
as desired.

In particular, we have now shown that $\uniformity{B}_\Phi$ is a uniformly-measurable base for $\meas$ and that $\meas$ is additive for uniformly-separated sets, so that indeed $\meas$ is a uniformly-additive on $X$.

It remains to show that $\meas$ is regular.

\Step{Show that $\mrm{H}(K)\leq \meas (K)$}
Let $U\in \topology{U}$ contain $K$.  Then, by the definition of $\meas (U)$, \eqref{5.1.46}, we have that $\mrm{H}(K)\leq \meas (U)$.  Taking the infimum over all such $U$, we obtain $\mrm{H}(K)\leq \meas (K)$.

\Step{Show that $\meas$ is regular}
The first thing we check is that $\meas (K)<\infty$ for $K$ quasicompact.  By \cref{prp5.2.4}, there is some open $U\subseteq X$ containing $K$ with compact closure.  Hence, for $K\subseteq K'\subseteq U$, with $K'$ quasicompact, we have
\begin{equation}
\mrm{H}(K')\leq \footnote{$\mrm{H}$ is nondecreasing by \cref{Haar.4}.}\mrm{H}(\Cls (U))<\infty .\footnote{Recall that $\mrm{H}$ is always finite, by definition.}
\end{equation}
Taking the supremum over such $K'$, we have that $\meas (U)\leq \mrm{H}(\Cls (U))$, and so, as $\meas (K)\leq \meas (U)$ (because $\meas$ is a measure), $\meas (K)$ is finite.

$\meas$ is outer-regular by definition \eqref{5.1.38}.

We now turn to inner-regular on open subsets.  This is \emph{almost} true by the definition \eqref{5.1.46}, but we don't have $\mrm{H}(K)=\meas (K)$.  However, we actually don't need this---we only need $\mrm{H}(K)\leq \meas (K)$.  To show this, let $U\in \topology{U}$ contain $K$.  Then, by the definition of $\meas (U)$, \eqref{5.1.46}, we have that $\mrm{H}(K)\leq \meas (U)$.  Taking the infimum over all such $U$, we obtain $\mrm{H}(K)\leq \meas (K)$.

Thus, $\meas$ is regular.

\Step{Conclude that $\meas$ is an isogeneous measure}
We know that $\meas$ is regular from the previous step and uniformly-additive from \cref{Haar.12}.  By \cref{Haar.11}, $\uniformity{B}_{\Phi}$ is a uniformly-measurable base, and hence $\meas$ is an isogeneous measure.

\Step{Show that $\meas (K_0)=1$}
First of all, from the definition, we have that $\mrm{H}_U(K_0)=1$ for all $U\in \topology{U}$.  Perhaps we could try harder and show that we already do have that $\meas (K_0)=1$; however, this enough is to show simply that $\meas (K_0)>0$, and so by simply dividing by $\meas (K_0)$ if necessary, we obtain a regular uniformly-additive measure with uniformly-measurable base $\uniformity{B}_\Phi$ and $\meas (K_0)=1$.

\Step{Define $S\Subset T$}
For the rest of this proof, let us write $S\Subset T$ iff there is some uniform cover $\cover{B}\in \uniformity{B}_\Phi$ for which $\Star _{\cover{B}}(S)\subseteq T$.

\Step{Show that if $S\Subset T$, then $\Cls (S)\subseteq \Int (T)$}[Haar.17]
Suppose that $S\Subset T$ and let $x\in X$ be an accumulation point of $S$.  By definition, there is some uniform cover $\cover{B}\in \uniformity{B}$ such that $\Star _{\cover{B}}(S)\subseteq T$.  Because $x$ is an accumulation point of $S$ and every element of $\cover{B}$ is open, every element of $\cover{B}$ that contains $x$ (of which there must be at least one, say $B\in \cover{B}$), must intersect $S$, and so we have that $x\in B\subseteq \Star _{\cover{B}}(S)\subseteq T$, which implies that $x\in \Int (T)$.

\Step{Show that $\meas$ is unique}
Let $\meas '$ be another regular uniformly-additive on measure $X$ with uniformly-measurable base $\uniformity{B}_\Phi$ and $\meas (K_0)=1$.  As both $\meas$ and $\meas '$ are outer-regular, it suffices to show that they agree on open sets.  Then, because they are both inner-regular on open sets, it suffices to show that they both agree on quasicompact subsets.  However, on account of \cref{prp5.2.4} and outer-regularity, it in turn suffices to show that they agree on open sets with compact closure.  So, let $\collection{G}$ denote the collection of all open sets with compact closures.  Furthermore, let us define
\begin{equation}
\begin{multlined}
Z(\collection{G})\coloneqq \\ \left\{ S\in 2^X:\text{for every }\varepsilon >0\text{ there are }C_{\varepsilon}\text{ closed}\right. \\ \left. \text{and }U_{\varepsilon}\in \mcal{G}\text{ such that }\right. \\ \left. C_\varepsilon \subseteq S\subseteq U_{\varepsilon}\text{ and}\right. \\ \left. \meas (U_{\varepsilon}\setminus C_{\varepsilon}),\meas '(U_{\varepsilon}\setminus C_{\varepsilon})<\varepsilon \right\} .
\end{multlined}
\end{equation}

To prove that they agree on $\collection{G}$, we will first show that they agree on $Z(\collection{G})\cap \collection{G}$.  To show that this is in fact sufficient, we prove that
\begin{equation}\label{5.1.75}
\meas (U)=\sup \{ \meas (V):V\in \collection{G}\cap Z(\collection{G}),\ V\Subset U\} 
\end{equation}
for $U\in \collection{G}$ open (the exact same proof will work for $\meas '$).  Of course, we only need to show that $\leq$ inequality (because $\meas (U)\geq \meas (V)$ for $V\Subset U$).  As $\meas$ is inner-regular on open sets (and so in particular on elements of $\collection{G}$), we have that
\begin{equation}
\meas (U)=\sup \left\{ \meas (K):K\subseteq U,\ K\in \collection{K}\text{.}\right\} ,
\end{equation}
and so to show that this suprema is at most the suprema in \eqref{5.1.75}, it suffices to show that, for every $K\subseteq U$ quasicompact, there is some $V_K\in \collection{G}\cap Z(\collection{G})$ with $K\subseteq V_K\Subset U$.

So, let $U\subseteq X$ be open and let $K\subseteq U$ be quasicompact.  We want to find such a $V_K$.  As $X$ is $T_0$, it is uniformly-completely-$T_3$, and so we may uniformly-separate quasicompact sets from closed sets, and so there is some uniform cover $\cover{B}\in \uniformity{B}_\Phi$ such that $\Star _{\cover{B}}(K)$ is disjoint from $\Star _{\cover{B}}(U^{\comp})$.  Because $X$ is a locally compact $T_2$ isogeneous space, by taking a star-refinement if necessary, we can without loss of generality assume that each element of $\cover{B}$ has compact closure (i.e.~is an element of $\collection{G}$).\footnote{Take any open set $U$ with compact closure (which exists by local compactness).  Then, $\cover{B}_U$ will be a cover whose elements are in $\collection{G}$.  Take a common star-refinement of this and $\cover{B}$.  The closures of elements of this new cover will be contained in the elements of $\cover{B}_U$, which themselves will be compact as closed sets of compact sets are compact.}  Take a star-refinement $\cover{C}\in \uniformity{B}_{\Phi}$.  Once again, every element of $\cover{C}$ has compact closure.  By quasicompactness of $K$, there are finitely many $C_1,\ldots ,C_m\in \cover{C}$ that cover $K$.  Define $C\coloneqq C_1\cup \cdots \cup C_m\in \collection{G}$.  Furthermore,
\begin{equation*}
K\subseteq C\subseteq \Star _{\cover{C}}(C)\subseteq \Star _{\cover{B}}(K)\subseteq \Star _{\cover{B}}(U^{\comp})^{\comp}\subseteq U.
\end{equation*}
Thus, we have shown that for $U\in \collection{G}$ and $K\subseteq U$ quasicompact, there is some $V_K\in \collection{G}$ with $K\subseteq V_K\Subset U$.  However, we still need to show that we can find such a $V_K\in \collection{G}\cap Z(\collection{G})$.  From what we have just shown, it suffices to show that, for every $U,V\in \collection{G}$ with $U\Subset V$, there is some $W\in Z(\collection{G})\cap \collection{G}$ with $U\Subset W\Subset V$.

So, let $U,V\in \mcal{G}$ with $U\Subset V$.  We showed before in \cref{Haar.17} that this implies that $\Cls (U)\subseteq \Int (V)$.  Then, because we may uniformly separate quasicompact sets from closed sets in uniformly-completely-$T_3$ spaces, there is some uniform cover $\cover{B}\in \uniformity{B}_{\Phi}$ such that $\Star _{\cover{B}}(\Cls (U))$ is disjoint from $\Star _{\cover{B}}(V^{\comp})$.  Define
\begin{equation}
W\coloneqq \Star _{\cover{B}}(\Cls (U)).
\end{equation}
$W$ is open as every element of $\cover{B}$ is open.  It also has compact closure as
\begin{equation}
W\subseteq \Star _{\cover{B}}(V^{\comp})^{\comp}\subseteq (V^{\comp})^{\comp}=V,
\end{equation}
and so its closure is contained in $\Cls (V)$, which is compact.  By definition, we have that $U\Subset W$.  We check that also $W\Subset V$.  To show this, we show that $\Star _{\cover{B}}(W)\subseteq V$.  So, let $B\in \cover{B}$ intersect $W$.  We proceed by contradiction:  suppose that $B$ intersects $V^{\comp}$.  Then, it is contained in $\Star _{\cover{B}}(V^{\comp})$, which is disjoint from $W$:  a contradiction.  Thus, $U\Subset W\Subset V$ and $W\in \mcal{G}$.  It remains to show that $W\in Z(\collection{G})$.  This, however, follows from the fact that that $W$ is measurable with respect to both $\meas$ and $\meas '$ (because it is open---see \cref{prp5.1.39}), the fact that it has finite measure for both $\meas$ and $\meas '$ (because its closure is quasicompact and the measure is regular), and \cref{prp5.1.39} (we can force $U_{\varepsilon}$ there to have compact closure by intersecting it with $V$).  This finally establishes \eqref{5.1.75}, and so finishes our proof that it suffices to show that $\meas$ and $\meas '$ agree on $\collection{G}\cap Z(\collection{G})$.

For the rest of the proof, take note that everything we know about $\meas '$ is likewise true about $\meas$.  Therefore, everything we prove to be true about $\meas '$ will also be true about $\meas$.  Thus, hereafter, if we prove facts about either $\meas$ or $\meas '$, we shall prove them about $\meas '$---they are then automatically true about $\meas$.

For $U\in \collection{G}\cap Z(\collection{G})$, every cover $\mcal{B}_V$ has a finite subcover of $U$ (because $\Cls (U)$ is compact), and so just as we did for $K$ compact, we may define $(U:V)$ to be the the cardinality of the smallest such subcover.  We shall use this notation in a moment.

\begin{exr}[breakable=false]{}{}
Let $U\in \collection{G}\cap Z(\collection{G})$.  Show that, for every $\varepsilon >0$, there is some open $U_\varepsilon$, such that
\begin{equation}
\meas '\left( \Star _{\cover{B}_{U_{\varepsilon}}}(U)-U\right) <\varepsilon 
\end{equation}
\end{exr}
Now, fix $U_0\in Z(\collection{G})\cap \collection{G}$ and let $U$ open be arbitrary.
\begin{exr}[breakable=false]{}{}
Show that there is some $U'\Subset U$ such that, for all $V$ sufficiently small (with respect to $\Subset$),
\begin{equation}
\frac{\meas '(U_0)}{\meas '(U)}\leq \frac{(U_0':V)}{(U':V)}
\end{equation}
whenever $\Star _{\cover{B}_{U'}}(U_0)\subseteq U_0'$.
\begin{rmk}
Hint:  See (10.2) in \cite{Howes}.
\end{rmk}
\end{exr}
\begin{exr}[breakable=false]{}{}
Show that there is some $U''\Subset U'$ such that, for all $V$ sufficiently small (with respect to $\Subset$),
\begin{equation}
\frac{\meas '(U_0)}{\meas '(U'')}\geq \frac{(U_0':V)}{(U':V)}
\end{equation}
whenever $\Star _{\cover{B}_{U'}}(U_0')\subseteq U_0$.
\begin{rmk}
Hint:  See (10.3) in \cite{Howes}.
\end{rmk}
\end{exr}
\begin{exr}[breakable=false]{}{}
Combine the last three exercises (and the fact that $\meas (K_0)=1=\meas '(K_0)$) to show that $\meas '$ and $\meas$ agree on $\collection{G}\cap Z(\collection{G})$.
\begin{rmk}
Hint:  You should be able to combine these results to get an expression for $\meas '(U_0)$ that, besides factors of $\meas '(U'')$ and $\meas '(U')$, will be completely independent of $\meas '$.  As explained above, everything true of $\meas '$ must also be true of $\meas$, and so, we will have the same expression for $\meas$, with exception of the fact that the factors $\meas '(U')$ and $\meas '(U'')$ will be different.
\end{rmk}
\end{exr}
\end{proof}
\end{thm}
Dayyyuuummm.  That was a hard theorem.  Probably the hardest in these notes.  But holy Jesus was it worth it.  Check out this epic definition.
\begin{exr}{}{}
Define $X\ceqq \R ^d$\footnote{With the usual uniformity.} and
\begin{equation}
\Phi \ceqq \left\{ \phi \colon X\rightarrow X:\phi \text{ an isometry.}\right\} .
\end{equation}
Show that $\coord{X,\Phi}$ is an isogeneous space.
\begin{rmk}
You have to check that (i)~$\R ^d$ is $\sigma$-compact, (ii)~$\R ^d$ is $T_0$, (iii)~$\R ^d$ is locally quasicompact, and (iv)~the group of isometries actually generate a uniform base for $\R ^d$ via \eqref{5.1.28}.  These are actually all quite trivial.\footnote{Though definitely convince yourself that they are true!}  For example, (iv)~is the most nontrivial, but this is true essentially just because the isometric image of an $\varepsilon$-ball is---gasp---another $\varepsilon$-ball!
\end{rmk}
\end{exr}
\begin{dfn}{Lebesgue measure}{LebesgueMeasure}
\term{Lebesgue measure}\index{Lebesgue measure} $\meas$ on $\R ^d$ is unique isogeneous measure with respect to the symmetry group of all isometries\footnote{In the case of $\R ^d$, all rotations, reflections, and translations are isometries.  (In fact, one can show that these generate all isometries.)} such that $\meas ([0,1]\times \cdots \times [0,1])=1$.
\begin{rmk}
This is actually much better than defining Lebesgue measure to be `classical' Haar measure (i.e.~Haar measure with respect to the topological group structure).  With this definition we \emph{automatically} get that Lebesgue measure is invariant under rotations for free, whereas with the ``classical'' definition, this requires some work.
\end{rmk}
\begin{rmk}
Besides using Haar measure to define Lebesgue measure, it is also common to use something called \emph{Carathéodory's Extension Theorem}, which, while not that bad, has the problem that it lacks a uniqueness result.\footnote{At least in general.  I think in certain nice cases it can be made to work.}  Moreover, Carathéodory doesn't even give us a regular measure---it actually just gives us a measure.  We would then have to go through by hand and check that the measure defined in this way is finite on quasicompact sets, inner-regular on open sets, outer-regular, has a uniformly-measurable base, is uniformly-additive, is invariant under translation, is invariant under rotation, is invariant under reflection, and that translations, rotations, and reflections actually give us all the isometries.  Ew.  The theory is just so much prettier when all of this hard work is done for us by \emph{one} result, instead of by fifty-bajillion separate ones.
\end{rmk}
\end{dfn}

We can even define counting measure using the \nameref{HaarHowesTheorem}.\footnote{Though this is a bit like using a sledgehammer (or maybe a nuke?) `swat' a fly.}
\begin{dfn}{Counting measure}{}
Let $X$ be a set equipped with the discrete uniformity and take $\Phi \coloneqq \Aut _{\Set}(X)$ to be the group of all bijections from $X$ to itself.  Then, 
\begin{equation}
\begin{split}
\{ \mcal{B}_{\{ x_0\}}:x_0\in X\} & \coloneqq \left\{ \{ h(\{ x_0\} ):h\in H\} \right\} \\
& =\left\{ \left\{ \{ x\} :x\in X\right\} \right\} \footnote{This is the uniform base which has a single cover, namely, the cover by singletons.  I say this because what is going on here is actually very easy, even though the notation may be a bit hard to parse.}
\end{split}
\end{equation}
is a uniform base, and so, because $\mcal{B}\coloneqq \{ \{ x_0\} :x_0\in X\}$ is a base for this topology, by \cref{exr5.1.48}, $\coord{X,H}$ is an isogeneous space.  Therefore, by the \nameref{HaarHowesTheorem}, there is a unique isogeneous measure $\meas$, the \emph{counting measure}, on $X$ such that $\meas (\{ x_0\} )=1$.
\end{dfn}

\subsection{The product measure}

As you know, the integral is `supposed' to be the ``area under the curve''.  The functions we will be integrating will take values in the reals, and so the integral will spit out a real number.  This is probably nothing new to you.  What is almost certainly new to you, however, is that now the \emph{domains} of the functions we will be integrating will be topological measure spaces.\footnote{You can do things much more generally than this, but for us,there is no need.  As a rule, it is usually best to do things in the nicest theory that encompasses every example you're interested in, and for us, we will not be interested in measures that are not topological (except perhaps when it comes to counter-examples).}  That is, we will be integrating functions $f\colon X\rightarrow \R$, for $X$ a topological measure space.  We will then simply define the integral to be the measure of the set\footnote{At least when $f$ is nonnegative---we'll have to work just a teensy bit harder to take the `signed area' if the function is negative somewhere.}
\begin{equation}
\left\{ \coord{x,y}\in X\times \R :0\leq y<f(x)\right\} .
\end{equation}
To do this, of course, we must first define a measure on $X\times \R$.  In fact, we will much more generally define a measure on $X_1\times X_2$ for $X_1$ and $X_2$ any topological measure spaces:  the \emph{product measure}.

The product measure is---you guessed it---a measure on the product.  As the integral is the `area under the curve' and the `area under the curve' is a subset of $X\times [0,\infty ]$ (for $f\colon X\rightarrow [0,\infty ]$), it is necessary for our development to put a measure on $X\times [0,\infty ]$.\footnote{Of course, you don't \emph{have} to do things this way, and indeed, this is not how it's usually done.  Doing things the other way, however, is arguably one reason why people think the Lebesgue integral is ``not geometric''.  On the other hand, I can't believe anybody would argue that the ``area under the curve'' is not geometric or intuitive---this is in fact how we explain things to high school students, after all.}
\begin{thm}{Product measure}{ProductMeasure}
Let $\coord{X_1,\meas _1}$ and $\coord{X_2,\meas _2}$ be topological measure spaces.  Then, there exists a unique topological measure $\meas _1\times \meas _2$\index[notation]{$\meas _1\times \meas _2$} on $X_1\times X_2$, the \term{product measure}\index{Product measure}, that satisfies $[\meas _1\times \meas _2](K_1\times K_2)=\meas _1(K_1)\meas _2(K_2)$ for $K_i\subseteq X_i$ compact.

Furthermore, it satisfies
\begin{enumerate}
\item $[\meas _1\times \meas _2](S_1\times S_2)=\meas _1(S_1)\meas _2(S_2)$ for all $S_i\subseteq X_i$\footnote{That is, you assume that `area is base times height' for \emph{quasicompact} rectangles, and then you get that `area is base times height' for \emph{all} rectangles---you don't even need $S_1$ and $S_2$ to be measurable.};
\item
{\scriptsize
\begin{equation}\label{RectangleInnerRegularity}
\begin{split}
\MoveEqLeft {}
[\meas _1\times \meas _2](U)=\sup \left\{ \sum _{k=0}^m\meas _1(K_{1,k})\meas _2(K_{2,k}):m\in \N \right. \\
& \qquad \qquad \left. K_{i,k}\subseteq X_i\text{ compact},\right. \\
& \qquad \qquad \left. \left\{ K_{1,k}\times K_{2,k}:0\leq k\leq m\right\} \text{ is disjoint},\right. \\
& \qquad \qquad \left. \bigcup _{k=0}^mK_{1,k}\times K_{2,k}\subseteq U\right\} \\ & \qquad \text{for }U\subseteq X_1\times X_2\text{ open};
\end{split}
\end{equation}
}
and
\item
{\scriptsize
\begin{equation}\label{RectangleOuterRegularity}
\begin{split}
\MoveEqLeft {}
[\meas _1\times \meas _2](S)=\inf \left\{ \sum _{m\in \N}\meas _1(U_{1,m})\meas _2(U_{2,m}):\right. \\ & \qquad \qquad \left. U_{i,m}\subseteq X_i\text{ open, }S\subseteq \bigcup _{m\in \N}U_{1,m}\times U_{2,m}\right\} .
\end{split}
\end{equation}
}
\end{enumerate}
\begin{rmk}
So those formulas look ridiculous, but I promise, it's not that complicated.  The first simply says that we can approximate open sets from the inside with compact sets of a special form (namely, finite disjoint unions of compact rectangles)---essentially just a nice version of inner-regularity.  The second simply says that we can approximate \emph{any} set from the outside with open sets of a special form (namely, countable unions of open rectangles)---essentially just a nice version of outer-regularity.  Perhaps one thing to note is that in we do not need to require disjointness explicitly in \eqref{RectangleOuterRegularity}---not having disjointness makes the sum larger, and so the infimum `doesn't care' about these cases, so to speak.
\end{rmk}
\begin{rmk}
Also note that it says \emph{compact} in \eqref{RectangleInnerRegularity} instead of just \emph{quasicompact} (though certainly we still have equality if we had used the word ``quasicompact'')---this is for essentially the same fact as given in \cref{exr5.1.84}.  This will be used repeatedly throughout the proof, probably nearly every time we make use of inner-regularity of $X_1$ or $X_2$, and so we mention this once only here, instead of referencing it every time we use it.
\end{rmk}
\begin{proof}
\Step{Check that $X_1\times X_2$ is $\sigma$-compact}
You (hopefully) already showed in \cref{SigmaProductSigma} that the product of two $\sigma$-compact spaces is $\sigma$-compact, and so $X_1\times X_2$ is $\sigma$-compact.

\Step{Define $\meas _1\times \meas _2$}
We define
{\scriptsize
\begin{equation}\label{eqn5.2.3}
\begin{split}
\meas _{\mrm{K}}(K_1\times K_2) & \coloneqq \meas _1(K_1)\meas _2(K_2)\text{ for }K_1\subseteq X_1,\ K_2\subseteq X_2\text{ compact} \\
\meas _{\mrm{U}}(U) & \coloneqq \sup \left\{ \sum _{k=0}^m\meas _K(K_{1,k}\times K_{2,k}):m\in \N \right. \\
& \quad \left. K_{1,k}\subseteq X_1,\ K_{2,k}\subseteq X_2\text{ compact},\right. \\
& \quad \left. \left\{ K_{1,k}\times K_{2,k}:0\leq k\leq m\right\} \text{ is disjoint},\right. \\
& \quad \left. \bigcup _{k=0}^mK_{1,k}\times K_{2,k}\subseteq U\right\} \text{ for }U\subseteq X_1\times X_2\text{ open} \\
[\meas _1\times \meas _2](S) & \coloneqq \inf \{ \meas _{\mrm{U}}(U):S\subseteq U,\ U\text{ open.}\} .
\end{split}
\end{equation}
}
We will eventually show that all these formulas agree in the case that more than one applies to a given set (e.g.~the second and third both apply to any open set).  After which time, we shall simply denote $\meas _1\times \meas _2$ for everything.

\Step{Show that $\meas _1\times \meas _2$ is nondecreasing}
\begin{exr}[breakable=false]{}{}
Check that $\meas _{\mrm{K}}$ is nondecreasing on sets of the form $K_1\times K_2$ for $K_i\subseteq X_i$ compact.
\end{exr}
\begin{exr}[breakable=false]{}{}
Check that $\meas _{\mrm{U}}$ is nondecreasing on open sets.
\end{exr}
\begin{exr}[breakable=false]{}{}
Check that $\meas _1\times \meas _2$ itself is nondecreasing.
\end{exr}

\Step{Show that $\meas _{\mrm{U}}(U_1\times U_2)=\meas _1(U_1)\meas _2(U_2)$ for $U_i\subseteq X_i$ open}[stpProductMeasure.3]
Let $U_i\subseteq X_i$ be open.  Let us first do the case where one has measure $0$.  Without loss of generality, take $\meas _1(U_1)=0$.  Then, whenever we have
\begin{equation}
\bigcup _{k=0}^mK_{1,k}\times K_{2,k}\subseteq U_1\times U_2,
\end{equation}
we must have that $K_{1,k}\subseteq U_1$ for all $0\leq k\leq m$,\footnote{Okay, you caught me.  This is not necessarily true if $K_{2,k}=\emptyset$.  Congratulations.  Would you like a cookie?} which forces $\meas _1(K_{1,k})=0$, and so in turn it forces
\begin{equation}
\sum _{k=0}^m\meas _{\mrm{K}}(K_{1,k}\times K_{2,k})=0,
\end{equation}
and hence $\meas _{\mrm{U}}(U_1\times U_2)=0=\meas _1(U_1)\meas _2(U_2)$.

Let us now suppose that one of $U_1$ and $U_2$ has infinite measure and the other is positive.  Without loss of generality, suppose that $\meas _1(U_1)=\infty$ and $\meas _2(U_2)>0$.  Then, by inner-regularity on opens, for every $M>0$, there is a compact subset $K_1\subseteq U_1$ with $\meas _1(K_1)>M$.  Similarly, there is a compact subset $K_2\subseteq U_2$ with $\meas _2(K_2)>0$.  Then,
\begin{equation}
\begin{split}
\meas _{\mrm{U}}(U_1\times U_2) & \geq \meas _{\mrm{K}}(K_1\times K_2)\ceqq \meas _1(K_1)\meas _2(K_2) \\
& >M\meas _2(K_2),
\end{split}
\end{equation}
and so, as $M>0$ is arbitrary and $\meas _2(K_2)>0$, $\meas _{\mrm{U}}(U_1\times U_2)=\infty$.

Finally, consider the case where both $0<\meas (U_1),\meas (U_2)<\infty$.  Let $\varepsilon >0$ be such that
\begin{equation}
(\meas _1(U_1)+\meas _2(U_2))\cdot \min \{ \meas _1(U_1),\meas _2(U_2)\} >\varepsilon >0
\end{equation}
and choose $K_i\subseteq U_i$ compact so that
\begin{equation}
\meas _i(U_i)-\frac{\varepsilon}{\meas _1(U_2)+\meas _2(U_2)}<\meas (K_i)\leq \meas (U_i).
\end{equation}
Then,
\begin{equation}\label{eqn5.2.16}
\begin{split}
\MoveEqLeft
\meas _1(U_1)\meas _2(U_2)\geq \meas _1(K_1)\meas _2(K_2) \\
& >\left( \meas _1(U_1)-\frac{\varepsilon}{\meas _1(U_2)+\meas _2(U_2)}\right) \\
& \quad \cdot \left( \meas _2(U_2)-\frac{\varepsilon}{\meas _1(U_1)+\meas _2(U_2)}\right) \\
& =\meas _1(U_1)\meas _2(U_2)-\varepsilon \\ & \quad +\frac{\varepsilon ^2}{(\meas _1(U_1)+\meas _2(U_2))^2} \\
& >\meas _1(U_1)\meas _2(U_2)-\varepsilon ,
\end{split}
\end{equation}
whence it follows\footnote{Note that, after throwing away the intermediate steps, this inequality reads $\meas _1(U_1)\meas _2(U_2)-\varepsilon <\meas _1(K_1)\meas _2(K_2)\leq \meas _1(U_1)\meas _2(U_2)$, which, by our good old buddy from \cref{chp1}, \cref{prp1.4.11}, is precisely the statement that $\meas _1(U_1)\meas _2(U_2)$ is the supremum of \eqref{eqn5.2.3}.} that $\meas _{\mrm{U}}(U_1\times U_2)=\meas _1(U_1)\meas _2(U_2)$.

\Step{Show that $[\meas _1\times \meas _2](U)=\meas _{\mrm{U}}(U)$}
Let $S\subseteq X_1\times X_2$ be open.  We must show that $[\meas _1\times \meas _2](S)=\meas _{\mrm{U}}(S)$.  Of course, $\meas _{\mrm{U}}(S)\in \{ \meas _{\mrm{U}}(U):S\subseteq U,\ U\text{ open}\}$, which gives us that $[\meas _1\times \meas _2](S)\leq \meas _{\mrm{U}}(S)$.\footnote{Because $[\meas _1\times \meas _2](S)$ is defined to be the infimum of this set---see \eqref{eqn5.2.3}.}  On the other hand, because $\meas _{\mrm{U}}$ is nondecreasing on open sets, we have that $\meas _{\mrm{U}}(S)$ is a lower-bound for $\{ \meas _{\mrm{U}}(U):S\subseteq U,\ U\text{ open}\}$, which gives us the other inequality.

\Step{Show that $[\meas _1\times \meas _2](S_1\times S_2)=\meas _1(S_1)\meas _2(S)$ for $\meas _i(S_i)<\infty$}[stpProductMeasure.5]
Let $S_i\subseteq X_i$ with $\meas _i(S_i)<\infty$.  We want to show that $[\meas _1\times \meas _2](S)=\meas _1(S_1)\meas _2(S_2)$.  In other words, we want to show that
\begin{equation*}
\meas _1(S_1)\meas _2(S_2)=\inf \{ \meas _{\mrm{U}}(U):S_1\times S_2\subseteq U,\ U\text{ open}\} .
\end{equation*}
Let $\varepsilon >0$.  Choose $\delta >0$ such that $\delta (\meas _1(S_1)+\meas _2(S_2))+\delta ^2<\varepsilon$.  Because $\meas _i$ is regular, there is some open $U_i\subseteq X_i$ with $S_i\subseteq U_i$ and
\begin{equation}
\meas _i(S_i)\leq \meas _i(U_i)<\meas _i(S_i)+\delta .
\end{equation}
Then, $S_1\times S_2\subseteq U_1\times U_2$, $U_1\times U_2$ is open, and
\begin{equation}
\begin{split}
\MoveEqLeft
\meas _1(S_1)\meas _2(S_2)\leq \meas _1(U_1)\meas _2(U_2)=\footnote{By \cref{stpProductMeasure.3}.}\meas _{\mrm{U}}(U_1\times U_2) \\
& <\meas _1(S_1)\meas _2(S_2)+\delta (\meas _1(S_1)+\meas _2(S_2))+\delta ^2 \\
& <\meas _1(S_1)\meas _2(S_2)+\varepsilon ,
\end{split}
\end{equation}
whence it follows\footnote{Similarly as in \eqref{eqn5.2.16}, note that, after throwing away the intermediate steps, this inequality reads $\meas _1(S_1)\meas _2(S_2)\leq \meas _{\mrm{U}}<\meas _1(S_1)\meas _2(S_2)+\varepsilon$, which, by our good old buddy from \cref{chp1}, \cref{prp1.4.11}, is precisely the statement that $\meas _1(S_1)\meas _2(S_2)$ is the infimum of \eqref{eqn5.2.3}.} that $[\meas _1\times \meas _2](S_1\times S_2)=\meas _1(S_1)\meas _2(S_2)$.

\Step{Show that $\meas _{\mrm{U}}(K_1\times K_2)=\meas _{\mrm{K}}(K_1\times K_2)$}
Now suppose that $U\coloneqq K_1\times K_2$ is open for $K_i\subseteq X_i$ compact.\footnote{This probably seems a bit awkward, but keep in mind that $\meas _{\mrm{K}}$ was only defined for compact rectangles and $\meas _{\mrm{U}}$ was only defined for open sets, so in order to even ask the question ``Are these equal?'', we must be dealing with a set that is both a compact rectangle \emph{and} open.}  We must show that $\meas _{\mrm{K}}(U)=\meas _{\mrm{U}}(U)$.  That is, we must show that $\meas _{\mrm{U}}(U)=\meas _1(K_1)\meas _2(K_2)$.  However, from the definition of $\meas _{\mrm{U}}$ as a supremum, we have that $\meas _{\mrm{U}}(U)\geq \meas _1(K_1)\meas _2(K_2)$.  To show the other inequality, let
\begin{equation}
\bigcup _{k=0}^mK_{1,m}\times K_{2,m}\subseteq K_1\times K_2
\end{equation}
be a disjoint union for $K_{i,k}\subseteq X_i$ compact.\footnote{Imagine a finite disjoint union of rectangles contained inside another big rectangle.}  We need to show that
\begin{equation}
\sum _{k=1}^m\meas _1(K_{1,k})\meas _2(K_{2,k})\leq \meas _1(K_1)\meas _2(K_2),
\end{equation}
as the supremum of the left-hand side over all possibilities for the $K_{i,k}$s is precisely $\meas _{\mrm{U}}(U)$---see \eqref{eqn5.2.3}.  $K_{i,k}\subseteq K_i$ for $0\leq k\leq m$, and so
\begin{equation}\label{eqn5.2.20}
\left( \bigcup _{k=0}^mK_{1,k}\right) \times \left( \bigcup _{k=0}^mK_{2,k}\right) \subseteq K_1\times K_2.\footnote{This is the statement that the `bounding rectangle' of the union of rectangles will still be contained inside the big rectangle.}
\end{equation}
Now, here is where the notation becomes atrocious, but the idea is only a little bit tricky.  The intuition is that we break up the bounding rectangle into a grid of smaller rectangles so that each of the original rectangles is a disjoint union of the `grid' rectangles.  Let $S\subseteq \{ 1,\ldots ,k-1,k+1,\ldots ,m\} \eqqcolon \mcal{S}_k$ and define
\begin{equation}
K_{i,k,S}\coloneqq K_{i,k}\cap \bigcap _{l\in S}K_{i,l}\cap \bigcap _{l\notin S}K_{i.l}^{\comp},
\end{equation}
so that
\begin{equation}
K_{i,k}=\bigcup _{S\subseteq \mcal{S}_k}K_{i,k,S}
\end{equation}
and
\begin{equation}
L_i\coloneqq \bigcup _{k=0}^mK_{i,k}=\bigcup _{k=0}^m\bigcup _{S\subseteq \mcal{S}_k}K_{i,k,S}
\end{equation}
are \emph{disjoint} unions.  Then, we have that
\begin{equation*}
\begin{split}
\MoveEqLeft
\sum _{k=0}^m\meas _1(K_{1,k})\meas _2(K_{2,k}) \\
& =\sum _{k=0}^m\bigg( \sum _{S\subseteq \mcal{S}_k}\meas _1(K_{1,k,S})\bigg) \big( \sum _{S\subseteq \mcal{S}_k}\meas _2(K_{2,k,S})\bigg) \\
& \leq \bigg( \sum _{k=0}^m\sum _{S\subseteq \mcal{S}_k}\meas _1(K_{1,k,S})\bigg) \bigg( \sum _{k=0}^m\sum _{S\subseteq \mcal{S}_k}\meas _2(K_{2,k,S})\bigg) \\
& =\meas _1(L_1)\meas _2(L_2)\leq \footnote{By \eqref{eqn5.2.20}.}\meas _1(K_1)\meas _2(K_2),
\end{split}
\end{equation*}
as was to be shown.

\Step{Show that $[\meas _1\times \meas _2](K_1\times K_2)=\meas _{\mrm{K}}(K_1\times K_2)$}
In other words, we want to show that $[\meas _1\times \meas _2](K_1\times K_2)=\meas _1(K_1)\meas _2(K_2)$.  However, this is true by \cref{stpProductMeasure.5} as $\meas _i(K_i)<\infty$ by regularity.

At this point, we have shown that all of $\meas _{\mrm{K}}$, $\meas _{\mrm{U}}$, and $\meas _1\times \meas _2$ agree on their common domains of definition, and so hereafter we shall only write $\meas _1\times \meas _2$.

\Step{Show that $\meas _1\times \meas _2$ is a measure}
That $[\meas _1\times \meas _2](\emptyset )=0$ follows immediately from the definition.  We already showed that it is nondecreasing.
\begin{exr}[breakable=false]{}{}
Check that $\meas _1\times \meas _2$ is subadditive.
\end{exr}

\Step{Show that $\meas _1\times \meas _2$ is regular}
That it is outer-regular follows immediately from the definition.\footnote{Indeed, the definition was designed for the purpose of being regular.}

Inner-regularity also follows quite easily from the definition.
\begin{equation*}
\begin{split}
[\meas _1\times \meas _2](U) & \coloneqq \sup \left\{ \sum _{k=0}^m[\meas _1\times \meas _2](K_{1,k}\times K_{2,k}):m\in \N \right. \\
& \quad \quad \left. K_{1,k}\subseteq X_1,\right. \\
& \quad \quad \left. K_{2,k}\subseteq X_2\text{ compact},\right. \\
& \quad \quad \left. \left\{ K_{1,k}\times K_{2,k}:0\leq k\leq m\right\} \text{ is disjoint},\right. \\
& \quad \quad \left. \bigcup _{k=0}^mK_{1,k}\times K_{2,k}\subseteq U\right\} \\
& \leq \sup \{ [\meas _1\times \meas _2](K):K\subseteq U,\ K\text{ compact}\} \\
& \leq \footnote{Because $\meas _1\times \meas _2$ is nondecreasing.}[\meas _1\times \meas _2](U),
\end{split}
\end{equation*}
and so all of these inequalities must be equalities.

We now check that it is finite on quasicompact sets.  Let $K\subseteq X_1\times X_2$ be quasicompact.  Define $K_i\coloneqq \pi _i(K)\subseteq X_i$.  This is quasicompact by the \nameref{ExtremeValueTheorem}, and so $\meas _i(K_i)<\infty$.  However, $K\subseteq K_1\times K_2$, and hence $[\meas _1\times \meas _2](K)\leq \meas _1(K_1)\meas _2(K_2)<\infty$.

\Step{Show that $\meas _1\times \meas _2$ is additive on finite disjoint unions of compact rectangles}
Let $K_{i,k}\subseteq X_i$ be compact for $0\leq k\leq m$ be such that $\bigcup _{k=0}^mK_{1,k}\times K_{2,k}$ is a disjoint union.  We wish to show that
\begin{equation}
\begin{multlined}
[\meas _1\times \meas _2]\left( \bigcup _{k=0}^mK_{1,k}\times K_{2,k}\right) \\ =\sum _{k=0}^m\meas _1(K_{1,k})\meas _2(K_{2,k}).
\end{multlined}
\end{equation}
The $\leq$ equality follows from subadditivity.  To show the other inequality, let $\varepsilon >0$, and let $U\subseteq X_1\times X_2$ be open and such that
\begin{equation*}
\begin{split}
\bigcup _{k=0}^mK_{1,k}\times K_{2,k} & \subseteq U\text{ and }[\meas _1\times \meas _2](U) \\
& <[\meas _1\times \meas _2]\left( \bigcup _{k=0}^mK_{1,k}\times K_{2,k}\right) +\varepsilon .
\end{split}
\end{equation*}
Then,
\begin{equation}
\begin{split}
\MoveEqLeft {}
[\meas _1\times \meas _2]\left( \bigcup _{k=0}^mK_{1,k}\times K_{2,k}\right) >[\meas _1\times \meas _2](U)-\varepsilon \\
& \geq 
\footnote{Because of the definition of $\meas _{\mrm{U}}(U)=[\meas _1\times \meas _2](U)$ as the suprema of this sum over all finite disjoint unions of compact rectangles contained in $U$.}\sum _{k=0}^m\meas _1(K_{1,k})\meas _2(K_{2,k})-\varepsilon ,
\end{split}
\end{equation}
and so as $\varepsilon >0$ is arbitrary, we have the other inequality.

\Step{Show that $\meas _1\times \meas _2$ is Borel}
Now, let $U\subseteq X_1\times X_2$ be open.  We wish of course to show that $U$ is open.  Write $X_i=\bigcup _{m\in \N}J_{i,m}$ for $J_{i,m}$ compact and define $U_{k,l}\ceqq U\cap (J_{1,k}\times J_{2,l})$.  Then, as $U=\bigcup _{k,l\in \N}U_{k,l}$, it suffices to show that $U_{k,l}$ is measurable for $k,l\in \N$.  Thus, for this step, without loss of generality, suppose that $X_1$ and $X_2$ themselves are compact.

We first show that $K_1\times K_2$ is measurable for $K_i\subseteq X_i$ compact.  So, let $S\subseteq X_1\times X_2$ be arbitrary.  We wish to show that
\begin{equation}
\begin{split}
[\meas _1\times \meas _2](S) & \geq [\meas _1\times \meas _2](S\cap (K_1\times K_2)) \\ & \quad +[\meas _1\times \meas _2](S\cap (K_1\times K_2)^{\comp}).
\end{split}
\end{equation}
If $[\meas _1\times \meas _2](S)=\infty$, we are done, so we may as well assume that $[\meas _1\times \meas _2](S)<\infty$.  Let $\varepsilon >0$, and let $U\subseteq X_1\times X_2$ be open and such that
\begin{equation*}
S\subseteq U\text{ and }[\meas _1\times \meas _2](S)\leq [\meas _1\times \meas _2](U)<[\meas _1\times \meas _2](S)+\varepsilon .
\end{equation*}
$K_1\times K_2$ is closed, and so $U\cap (K_1\times K_2)^{\comp}$ is open.  Therefore, there is a disjoint union
\begin{equation}
\bigcup _{k=0}^mL_{1,k}\times L_{2,k}\subseteq U\cap (K_1\times K_2)^{\comp}
\end{equation}
for $L_{i,k}\subseteq X_i$ compact such that
\begin{equation*}
[\meas _1\times \meas _2](U\cap (K_1\times K_2)^{\comp})-\varepsilon <\sum _{k=0}^m[\meas _1\times \meas _2](L_{1,k}\times L_{2,k}).
\end{equation*}
Similarly, there is a disjoint union
\begin{equation}
\bigcup _{k=0}^nM_{1,k}\times M_{2,k}\subseteq U\cap \left( \bigcup _{k=0}^mL_{1,k}\times L_{2,k}\right) ^{\comp}
\end{equation}
for $M_{i,k}\subseteq X_i$ compact such that
\begin{equation}
\begin{multlined}
[\meas _1\times \meas _2]\left( U\cap \left( \bigcup _{k=0}^mL_{1,k}\times L_{2,k}\right) ^{\comp}\right) -\varepsilon \\ <\sum _{k=0}^n[\meas _1\times \meas _2](M_{1,k}\times M_{2,k}).
\end{multlined}
\end{equation}
Hence,
\begin{equation*}
\begin{split}
\MoveEqLeft
[\meas _1\times \meas _2](S)>[\meas _1\times \meas _2](U)-\varepsilon \\
& \geq [\meas _1\times \meas _2]\left( \bigcup _{k=0}^mL_{1,k}\times L_{2,k}\cup \bigcup _{k=0}^nM_{1,k}\times M_{2,k}\right) -\varepsilon \\
& =\footnote{By the previous step, because these two collections of rectangles are disjoint.}\sum _{k=0}^m[\meas _1\times \meas _2](L_{1,k}\times L_{2,k}) \\ & \qquad +\sum _{k=0}^n[\meas _1\times \meas _2](M_{1,k}\times M_{2,k})-\varepsilon \\
& >[\meas _1\times \meas _2]\left( U\cap (K_1\times K_2)^{\comp}\right) \\ & \qquad +[\meas _1\times \meas _2]\left( U\cap \left( \bigcup _{k=0}^mL_{1,k}\times L_{2,k}\right) ^{\comp}\right) \\ & \qquad -3\varepsilon \\
& \geq [\meas _1\times \meas _2](U\cap (K_1\times K_2)^{\comp}) \\ & \qquad +[\meas _1\times \meas _2](U\cap (K_1\times K_2))-3\varepsilon .
\end{split}
\end{equation*}
As $\varepsilon >0$ was arbitrary, we obtain the desired inequality.

As $X_1$ and $X_2$ are compact, $[\meas _1\times \meas _2](X_1\times X_2)=\meas _1(X_1)\meas _2(X_2)<\infty$, and so $[\meas _1\times \meas _2](U)<\infty$.  So, let $\varepsilon >0$, and let $K_{i,k}\subseteq X_i$ be compact such that $\{ K_{1,k}\times K_{2,k}:0\leq k\leq m\}$ is disjoint and
\begin{equation}
\begin{multlined}
[\meas _1\times \meas _2](U)-\varepsilon \\ <\sum _{k=0}^m[\meas _1\times \meas _2](K_{1,k}\times K_{2,k})\leq [\meas _1\times \meas _2](U).
\end{multlined}
\end{equation}
Applying \cref{exr5.1.21} and using the fact that $\cup _{k=0}^mK_{1,k}\times K_{2,k}$ is measurable, we have that
\begin{equation}
[\meas _1\times \meas _2]\left( U\setminus \left( \bigcup _{k=0}^mK_{1,k}\times K_{2,k}\right) \right) <\varepsilon .
\end{equation}
Picking such a disjoint union of rectangles for $\varepsilon =\frac{1}{m}$ and taking the union of these over $m\in \Z ^+$ gives a measurable set $F$ with $[\meas _1\times \meas _2](U\setminus F)=0$, and so $U\setminus F$ is measurable, and so $U=F\cup (U\setminus F)$ is measurable.

\Step{Show uniqueness}
To show that two regular measures agree, by outer-regularity, it suffices to show that they agree on on open sets.  By inner-regularity, to show this, it suffices to show that
\begin{equation}
\begin{split}
\MoveEqLeft
\sup \{ \meas (K):K\subseteq U,\ K\text{ quasicompact}\} \\
& =\sup \left\{ \sum _{k=0}^m[\meas _1\times \meas _2](K_{1,k}\times K_{2,k}):m\in \N \right. \\
& \quad \left. K_{1,k}\subseteq X_1,\ K_{2,k}\subseteq X_2\text{ quasicompact},\right. \\
& \quad \left. \left\{ K_{1,k}\times K_{2,k}:0\leq k\leq m\right\} \text{ is disjoint},\right. \\
& \quad \left. \bigcup _{k=0}^mK_{1,k}\times K_{2,k}\subseteq U\right\} .\footnote{Here, $\meas$ denotes \emph{any} regular borel measure on $X_1\times X_2$ such that $\meas (K_1\times K_2)=\meas _1(K_1)\meas _2(K_2)$ for $K_i\subseteq X_i$ compact.}
\end{split}
\end{equation}
The left-hand side is automatically at least as large as the right-hand side because the set inside the $\sup$s on the left-hand side is a superset of the set inside the $\sup$ right-hand side.  On the other hand, for $K\subseteq X_1\times X_2$ quasicompact, $K_i\coloneqq \pi _i(K)$ is quasicompact by the \nameref{ExtremeValueTheorem}, and as $K\subseteq K_1\times K_2$, we have that $\meas (K)\leq \meas _1(K_1)\meas _2(K_2)$, which shows that the left-hand side is no larger than the right-hand side, which gives equality.

\Step{Show `rectangle outer-regularity'}
We now seek to show \eqref{RectangleOuterRegularity}, which we reproduce here for convenience.
\begin{equation}\label{eqn5.2.40}
\begin{multlined}
[\meas _1\times \meas _2](S)=\inf \left\{ \sum _{m\in \N}\meas _1(U_{1,m})\meas _2(U_{2,m}):\right. \\ \left. U_{i,m}\subseteq X_i\text{ open, }S\subseteq \bigcup _{m\in \N}U_{1,m}\times U_{2,m}\right\} .
\end{multlined}
\end{equation}

We first reduce it to the case where $S$ is open.  So, suppose we have proven the result for open sets, and let $S\subseteq X_1\times X_2$ be arbitrary.  Note that the right-hand side of \eqref{eqn5.2.40} is at least as large as $\inf \left\{ [\meas _1\times \meas _2](U):S\subseteq U,\ U\text{ open.}\right\}$, and so in particular, if $[\meas _1\times \meas _2](S)=\infty$, there is nothing to prove, so suppose that $[\meas _1\times \meas _2](S)<\infty$.

Let $\varepsilon >0$ and pick $U\supseteq S$ open and such that
\begin{equation}
[\meas _1\times \meas _2](S)\leq [\meas _1\times \meas _2](U)<[\meas _1\times \meas _2](S)+\varepsilon .
\end{equation}
If we have proven the result for $U$ open, then there are $U_{i,m}\subseteq X_i$ open such that $U\subseteq \bigcup _{m\in \N}U_{1,m}\times U_{2,m}$ and
\begin{equation}
\begin{split}
[\meas _1\times \meas _2](U) & \leq \sum _{m\in \N}\meas _1(U_{1,m})\meas _2(U_{2,m}) \\
& <[\meas _1\times \meas _2](U)+\varepsilon .
\end{split}
\end{equation}
Hence,
\begin{equation}
\begin{split}
\MoveEqLeft {}
[\meas _1\times \meas _2](S)\leq [\meas _1\times \meas _2](U) \\
& \leq \sum _{m\in \N}\meas _1(U_{1,m})\meas _2(U_{2,m}) \\
& <[\meas _1\times \meas _2](U)+\varepsilon \\
& <[\meas _1\times \meas _2](S)+2\varepsilon ,
\end{split}
\end{equation}
which gives \eqref{eqn5.2.40}, as desired.

It thus suffices to prove \eqref{eqn5.2.40} for $S=U\subseteq X$ open.  So, let $U\subseteq X_1\times X_2$ be open.  First suppose that $[\meas _1\times \meas _2](U)=\infty$.  We wish to show that $\sum _{m\in \N}\meas _1(U_{1,m})\meas _2(U_{2,m})=\infty$ whenever $\bigcup _{m\in \N}U_{1,m}\times U_{2,m}\supseteq U$ and $U_{i,m}\subseteq X_i$ is open.  As this sum is at least as large as $[\meas _1\times \meas _2]\left( \bigcup _{m\in \N}U_{1,m}\times U_{2,m}\right)$, it suffices to show that this measure is infinite.  However, this is infinite because it contains $U$, which is of infinite measure.  Thus, it suffices to prove the result in the case $U$ has finite measure.

So now suppose that $[\meas _1\times \meas _2](U)<\infty$.  Let $K_0\subseteq U$ be a finite disjoint union of compact rectangles such that $[\meas _1\times \meas _2](U\setminus K_0)<2^{-0}$.  $U\setminus K_0$ is open, and so again pick a finite disjoint union of compact rectangles $K_1\subseteq U\setminus K_0$ such that $[\meas _1\times \meas _2]((U\setminus K_0)\setminus K_1)<2^{-1}$.  Proceeding inductively, let $K_{m+1}\subseteq U\setminus (K_0\cup K_1\cup \cdots \cup K_m)$ be a finite disjoint union of compact rectangles such that $[\meas _1\times \meas _2](U\setminus (K_0\cup \cdots \cup K_m)\setminus K_{m+1})<2^{-(m+1)}$.  It follows that $[\meas _1\times \meas _2]\left( U\setminus \bigcup _{m\in \N}K_m\right) =0$, and so, as we can approximate the countably many disjoint compact rectangles appearing in $\bigcup _{m\in \N}K_m$ with open rectangles as good as we like, if we can prove the results for $G_{\delta}$ sets of measure $0$ (with the open sets appearing in the intersection of finite measure), we will be done.

So, let $G\subseteq X_1\times X_2$ be $G_{\delta}$ of measure $0$, and write $G=\bigcap _{m\in \N}U_m$ for $U_m\subseteq X_1\times X_2$ open of finite measure.  By replacing $U_m$ with $\bigcap _{k=0}^mU_k$, we can without loss of generality assume that this is a nonincreasing sequence $U_m\supseteq U_{m+1}$.  Then, as these sets have finite measure, we have that
\begin{equation}
0=[\meas _1\times \meas _2](G)=\lim _m[\meas _1\times \meas _2](U_m).
\end{equation}
Thus, for every $\varepsilon >0$, we can pick $U_{m_0}$ with measure at most $\varepsilon$.  Now, using the fact that we have proven the result for open sets of finite measure, cover $U_{m_0}$ with open rectangles the sum of whose measures is within $\varepsilon$ of $[\meas _1\times \meas _2](U_{m_0})$.  Then, the sum of these measures will be within $2\varepsilon$ of $0$, proving the result of $G$, thereby completing the proof of this step.

\Step{Show that $[\meas _1\times \meas _2](S_1\times S_2)=\meas _1(S_1)\meas _2(S_2)$ for $S_i\subseteq X_i$}
We have actually already done the case where $\meas _i(S_i)<\infty$---see \cref{stpProductMeasure.5}. So, suppose that at least one of $S_1$ and $S_2$ has infinite measure.  Without loss of generality, suppose that $\meas _1(S_1)=\infty$.

Let us first suppose that $\meas _2(S_2)>0$.  In this case, we wish to show that $[\meas _1\times \meas _2](S_1\times S_2)=\infty$.  Let $U\supseteq S_1\times S_2$ be open.  By outer-regularity, it suffices to show that $[\meas _1\times \meas _2](U)=\infty$.

First of all, for $\coord{x,y}\in U$, let $U_{x,y}\times V_{x,y}\subseteq U$ be an open neighborhood of $\coord{x,y}$.  Define
\begin{equation}
P_x\ceqq \left\{ y\in X_2:\coord{x,y}\in U\right\} .
\end{equation}
for $x\in S_1$.  As $U\supseteq S_1\times S_2$, we have that
\begin{equation}
\begin{split}
P_x & \supseteq \left\{ y\in X_2:\coord{x,y}\in S_1\times S_2\right\} \\
& =\begin{cases}S_2 & \text{if }x\in S_1 \\ \emptyset & \text{if }x\notin S_1.\end{cases}
\end{split}
\end{equation}
Note that $P_x$ is open (and has positive measure for $x\in S_1$ because $S_2$ has positive measure).  So, for $x\in S_1$, let $M_x<\meas _2(P_x)$ be arbitrary and choose some $L_x\subseteq P_x$ compact such that $\meas (L_x)>M_x$.  $\{ V_{x,y}:y\in P_x\}$ is an open cover of $L_x$, and so there are finitely many $y_1,\ldots ,y_{n_x}\in P_x$ such that
\begin{equation}
L_x\subseteq V_{x,y_1}\cup \cdots \cup V_{x,y_{n_x}}.
\end{equation}
Define
\begin{equation*}
U_x\ceqq U_{x,y_1}\cap \cdots \cap U_{x,y_{n_x}}\text{ and }V_x\ceqq V_{x,y_1}\cup \cdots \cup V_{x,y_{n_x}},
\end{equation*}
and
\begin{equation}
U_1\ceqq \bigcup _{x\in S_1}U_x.
\end{equation}
Note that for every $x\in S_1$
\begin{equation}\label{eqn5.2.45}
\begin{split}
U_x\times V_x & =\bigcup _{k=1}^{n_x}\left( U_{x,y_1}\cap \cdots \cap U_{x,y_{n_x}}\right) \times V_{x,y_k} \\
& \subseteq \bigcup _{k=1}^{n_x}U_{x,y_k}\times V_{x,y_k}\subseteq U.
\end{split}
\end{equation}

As $x\in U_x$, $U_1\supseteq S_1$, and so $\meas _1(U_1)=\infty$ because $\meas _1(S_1)$.  So, for every $M>0$, let $K_M\subseteq U_1$ be compact and such that $\meas _1(K_M)>M$.  $\{ U_x:x\in S_1\}$ is an open cover of $K_M$, and so there are finitely may $x_1,\ldots ,x_m\in S_1$ such that
\begin{equation}
K_M\subseteq U_{x_1}\cup \cdots \cup U_{x_m}.
\end{equation}
Make this union disjoint by defining $U_k'\ceqq U_{x_k}\setminus \bigcup _{l=0}^{k-1}U_{x_l}$, so that we have
\begin{equation}
\begin{split}
\sum _{k=1}^m\meas _1(U_k') & =\meas _1\left( \bigcup _{k=0}^mU_k'\right) =\meas _1\left( \bigcup _{k=0}^mU_{x_k}\right) \\
& \geq \meas _1(K_M)>M.
\end{split}
\end{equation}

Note that
\begin{equation}
\meas _2(V_{x_k})\geq \meas _2(L_{x_k})>\min \{ M_{x_1},\ldots ,M_{x_m}\} .
\end{equation}
As $M_{x_k}<\meas _2(P_{x_k})$ was arbitrary, in fact we have
\begin{equation*}
\meas _2(V_{x_k})\geq \min \{ \meas _2(P_{x_1}),\ldots ,\meas _2(P_{x_m})\} \geq \meas _2(S_2).
\end{equation*}

By \eqref{eqn5.2.45}, we have that $\bigcup _{k=1}^mU_k'\times V_{x_k}\subseteq U$ (because $U_k'\subseteq U_{x_k}$), and so, as this union is disjoint (because the $U_k'$s are disjoint), we have that
\begin{equation}
\begin{split}
[\meas _1\times \meas _2](U) & \geq [\meas _1\times \meas _2]\left( \bigcup _{k=1}^mU_k'\times V_{x_k}\right) \\
& =\sum _{k=1}^m[\meas _1\times \meas _2](U_k'\times V_{x_k}) \\
& =\sum _{k=1}^m\meas _1(U_k')\meas _2(V_{x_k}) \\
& \geq \meas _2(S_2)\sum _{k=1}^m\meas _1(U_k') \\
& >\meas _2(S_2)M.
\end{split}
\end{equation}
As $\meas _2(S_2)>0$ and $M>0$ is arbitrary, it follows that $[\meas _1\times \meas _2](U)=\infty$, as desired.

Now assume that $\meas _2(S_2)=0$.  As $[\meas _1\times \meas _2](S_1\times S_2)\leq [\meas _1\times \meas _2](X_1\times S_2)$, it suffices to show that $[\meas _1\times \meas _2](X_1\times S_2)=0$.

Write $X_1=\bigcup _{m\in \N}K_m$ for $K_m\subseteq X_1$ compact.  Then,
\begin{equation}
\begin{split}
[\meas _1\times \meas _2](X_1\times S_2) & \leq \sum _{m\in \N}[\meas _1\times \meas _2](K_m\times S_2) \\
& =\footnote{By \cref{stpProductMeasure.5}---the case when both sets have finite measure.}\sum _{m\in \N}\meas _1(K_m)\meas _2(S_2)=0,
\end{split}
\end{equation}
and so $[\meas _1\times \meas _2](X_1\times S_2)=0$, as desired.
\end{proof}
\end{thm}
Be careful:  If you've studied product measures before with the use of $\sigma$-algebras, there is going to be what will seem like a little bit weird behavior.
\begin{exm}{A measurable set $M\subseteq X_1\times X_2$ for which $M_{x_1}\subseteq X_2$ is not measurable for any $x_1$}{exm5.2.153}
First of all, we have used the notation here
\begin{equation}\label{eqn5.2.154}
M_{x_1}\coloneqq \left\{ x_2\in X_2:\coord{x_1,x_2}\in M\right\} .
\end{equation}

Let $\meas _1$ be the Zero Measure on $X_1$ (which, for the sake of concreteness, you can take to be $\R$ if you like), and let $\meas _2$ be Lebesgue measure on $\R$.  Let $N\subseteq X_2$ be a nonmeasurable set,\footnote{Such a set exists by \cref{exm5.2.47}.}  and define $M\coloneqq X_1\times N$.

The product measure $\meas _1\times \meas _2$ is the Zero Measure again, and so every subset of $X_1\times X_2$ is measurable.  In particular, $M$ is measurable.  On the other hand, $M_{x_1}=N\subseteq X_2$ is not measurable for all $x_1\in X_1$. 
\begin{rmk}
On the other hand, this is true for \emph{almost-every} $x_1$---\cref{exr5.2.298}.
\end{rmk}
\end{exm}

The product of two isogeneous spaces is canonically an isogeneous spaces, and if they are both $T_0$ locally quasicompact, so too will the product be.  Hence, the \nameref{HaarHowesTheorem} tells us that the product will obtain a measure.  The question, then, is whether this measure agrees with the product measure as defined above.  Fortunately, the answer is in the affirmative.
\begin{dfn}{Product isogeneous structure}{}
Let $\coord{X_1,\Phi _1}$ and $\coord{X_2,\Phi _2}$ be isogeneous spaces.  Then, the \term{product isogeneous structure}\index{Product isogeneous structure} on $X_1\times X_2$ is given by
\begin{equation}
\Phi _1\times \Phi _2\coloneqq \{ \phi _1\times \phi _2:\phi _1\in \Phi _1,\ \phi _2\in \Phi _2\},
\end{equation}
where $\phi _1\times \phi _2:X_1\times X_2\rightarrow X_1\times X_2$ is defined by
\begin{equation}
[\phi _1\times \phi _2](\coord{x_1,x_2})\coloneqq \coord{\phi _1(x_1),\phi _2(x_2)}.
\end{equation}
\begin{exr}[breakable=false]{}{}
Show that this in fact gives an isogeneous structure.
\begin{rmk}
In other words, you need to show that
\begin{equation}
\begin{multlined}
\left\{ \mcal{B}_{U_1\times U_2}:\right. \\ \left. U_i\subseteq X_i\text{ nonempty open.}\right\}
\end{multlined} 
\end{equation}
is a uniform base, where
\begin{equation*}
\begin{multlined}
\mcal{B}_{U_1\times U_2}\coloneqq \{ \phi _1(U_1)\times \phi _2(U_2): \\ \phi _1\in \Phi _1,\ \phi _2\in \Phi _2\} .
\end{multlined}
\end{equation*}
\end{rmk}
\end{exr}
\end{dfn}
\begin{prp}{}{prp5.2.65}
Let $\coord{X_1,\Phi _1}$ and $\coord{X_2,\Phi _2}$ be $T_0$ locally quasicompact isogeneous spaces, let $K_i\subseteq X_i$ be quasicompact with nonempty interior, and let $\meas _i$ be the unique isogeneous measure on $X_i$ such that $\meas _i(K_i)=1$.  Then, $\meas _1\times \meas _2$ is the unique isogeneous measure on $\coord{X_1\times X_2,\Phi _1\times \Phi _2}$ such that $[\meas _1\times \meas _2](K_1\times K_2)=1$.
\begin{proof}
We leave this as an exercise.
\begin{exr}[breakable=false]{}{}
Prove this yourself.
\end{exr}
\end{proof}
\end{prp}

Finally, we mentioned tangentially in a remark of the \nameref{HaarHowesTheorem} that neither one of locally (quasi)compact and $\sigma$-(quasi)compact imply the other.  We end this section with the relevant counter-examples.
\begin{exr}{}{exr5.1.162}
\begin{enumerate}
\item \label{exr5.1.162.i}Find an example of a topological space that is locally compact but not $\sigma$-quasicompact.
\item \label{exr5.1.162.ii}Find an example of a topological space that is $\sigma$-compact but not locally quasicompact.
\end{enumerate}
\begin{rmk}
Hint:  See \cite[58]{Steen} for \cref{exr5.1.162.ii}.  \cref{exr5.1.162.i} should be quite a bit easier.
\end{rmk}
\end{exr}

\subsection{Lebesgue measure}

Before continuing on with integration, we prove some properties about Lebesgue measure.  Knowing that Lebesgue measure on $\R ^d$ is just the product of the Lebesgue measure on $\R$ is actually incredibly useful,\footnote{Lebesgue measure on $\R ^d$ is defined to be the unique isogeneous measure with respect to the group of isometries on $\R ^d$ (that assign measure $1$ to the unit `cube').  On the other hand, by \cref{prp5.2.65}, this is just the product of Lebesgue measure on $\R$.} and so now that we know this, we take advantage of this fact.
\begin{prp}{}{}
The Lebesgue measure of a point is $0$.
\begin{wrn}
Warning:  This is most definitely not true for general measures.  Counter-example?
\end{wrn}
\begin{proof}
The argument is the exact same in $\R ^d$ as it is in $\R$, so we write down the argument in $\R$ and save us from some $\cdots$.

The first thing to notice is that, by translation invariance, the measure of every point is the same.  Let us denote this measure by $M$.

Points are closed, and hence measurable, and so $\meas$ is additive on points.  Therefore, we have that
\begin{equation}
\begin{split}
1 & =\meas ([0,1])\geq \meas \bigg( \bigcup _{x\in [0,1]\cap \Q}\{ x\} \bigg) \\
& =\sum _{x\in [0,1]\cap \Q}\meas (\{ x\} )=\sum _{m\in \N}M.
\end{split}
\end{equation}
This equation forces $M=0$.
\end{proof}
\end{prp}
\begin{crl}{}{}
$\meas ([0,1)\times \cdots \times [0,1))=1$.
\begin{rmk}
Sets of the form
\begin{equation}\label{5.2.16}
[a_1,b_1)\times \cdots \times [a_d,b_d)
\end{equation}
are important in measure theory because, for example,
\begin{equation}
[0,2)=[0,1)\cup [1,2)
\end{equation}
is a \emph{disjoint} union.  If we tried replacing everything here with all open intervals we we would have that $(0,2)=(0,1)\cup (1,2)$, which is just plain false, and if we tried replacing everything here with all closed intervals the union would not be disjoint ($[0,1]$ and $[1,2]$ intersect at $1$).  The disjointness is important in measure theory of course because of additivity (on measurable sets).  Sets of the form \eqref{5.2.16} are called \term{half-open rectangles}\index{Half-open rectangle} or \term{closed-open rectangles}\index{Closed-open rectangle}.
\end{rmk}
\begin{proof}
Now that we know that
\begin{equation}
\meas ([0,1)\times \cdots \times [0,1))=\meas ([0,1))\cdots \meas ([0,1)),
\end{equation}
it suffices to prove this result in $\R$.

That it is true in $\R$ follows from the fact that
\begin{equation}
1=\meas ([0,1])=\meas ([0,1))+\meas (\{ 1\} )=\meas ([0,1)).
\end{equation}
\end{proof}
\end{crl}
\begin{prp}{}{}
$\meas ([a_1,b_1]\times \cdots \times [a_d,b_d])=(b_1-a_1)\cdots (b_d-a_d)$.
\begin{proof}
Now that we know that it is a product measure, it suffices to show the one-dimensional case, and so we prove that $\meas ([a,b])=b-a$.

\begin{exr}[breakable=false]{}{}
Prove the cases where at least one of $a$ of $b$ is infinite.
\end{exr}

By translation invariance, it suffices to show that $\meas ([0,b])=b$.

Because the measure of points is $0$, it suffices to show that $\meas ([0,b))=b$.

\begin{exr}[breakable=false]{}{}
Show that $\meas ([0,m))=m$ for $m\in \N$.
\end{exr}
\begin{exr}[breakable=false]{}{}
Show that $\meas ([0,\frac{p}{q}))=\frac{p}{q}$ for $p,q\in \Z ^+$.
\end{exr}
Then, we have that
\begin{equation}
q=\meas ([0,q)) \leq \meas ([0,b))\leq \meas ([0,r))=r
\end{equation}
for all $p,q\in \Q ^+$ with $q\leq b\leq r$.  It follows that $\meas ([0,b))=b$.
\end{proof}
\end{prp}
\begin{exr}{}{exr5.2.77}
Show that any open set in $\R ^d$ can be written at the countable disjoint union of half-open rectangles.
\end{exr}
\begin{exr}{}{exr5.2.28}
Let $X$ be a regular measure space and for $S\subseteq X\times \R ^d$ and $a\in \R$, define
\begin{equation}
aS\coloneqq \left\{ \coord{x,ay}:\coord{x,y}\in S\right\} ,
\end{equation}
where $x\in X$ and $y\in \R ^d$.  Show that
\begin{enumerate}
\item
\begin{equation}
\meas (aS)=\abs{a}^d\meas (S);
\end{equation}
and
\item if $S$ is measurable, then $aS$ is measurable.
\end{enumerate}
\begin{rmk}
Hint:  Prove it for $\R ^d$ alone first (i.e.~if $X$ is just a point, so that $X\times \R ^d=\R ^d$).  For this case, prove it for $S$ open using the previous exercise and the fact that we already know it is true for half-open rectangles.  Then prove it in general for $\R ^d$ by outer-regularity.  Then generalize this to the case $X\times \R ^d$ for $X$ not-necessarily a point using the definition of product measures.
\end{rmk}
\end{exr}
\begin{exr}{}{exr5.1.254}
Let $T:\R ^d\rightarrow \R ^d$ be a linear transformation and let $S\subseteq \R ^d$.  Show that
\begin{equation}
\meas (T(S))=\abs{\det (T)}\meas (S).
\end{equation}
\begin{rmk}
Hint:  Recall your matrix decompositions from linear algebra and use the fact that we already know (by \namerefpcref{HaarHowesTheorem}) that Lebesgue measure is invariant under isometries together with the result of the previous exercise.
\end{rmk}
\end{exr}

\begin{exr}{}{exr5.2.33}
Show that countable sets have Lebesgue measure $0$.
\begin{rmk}
In particular, $\Q$ has $0$ measure!
\end{rmk}
\begin{rmk}
Note that the converse is false---see \cref{CantorSet}.
\end{rmk}
\end{exr}

\subsubsection{Generalized Cantor sets}

We next turn to a couple relatively famous subsets of the real line with particularly interesting properties.  One such such, the \emph{Cantor Set}, will serve as our first example of an uncountable set of measure zero.\footnote{We already know that countable sets must have Lebesgue measure zero.  The existence of the Cantor Set thus shows that the converse is false.}  Another such set, the \emph{Cantor-Smith-Volterra Set}, will serve as our first example of a set with empty interior but positive measure.\footnote{We already know (Why?) that sets of Lebesgue measure zero must have empty interior.  The existence of the Cantor-Smith-Volterra Set thus shows that the converse is false.}  Both of these sets are in fact just particular cases of a family of sets known as \emph{Generalized Cantor sets}, and we begin with these.
\begin{exm}{Generalized Cantor sets}{GeneralizedCantorSets}
For $m\in \N$, let $\alpha _m\in (0,1)$.  For every distinct choice of the $\alpha _m$s, we obtain a set $C$ defined as below.  Sets of this form are known as \term{generalized Cantor sets}\index{Generalized Cantor sets}.

We shall define a decreasing sequence of sets $[0,1]\eqqc C_0\supset C_1\supset C_2\supset \cdots$ recursively in stages, and then finally we will define $C\ceqq \bigcap _{m\in \N}C_m$.

We will see that $C_k$ is a disjoint union of $2^k$ closed intervals, all of the same length.  $C_{k+1}$ will then be obtained from $C_k$ by removing the middle $\alpha _k^{\text{th}}$ of each of these $2^k$ intervals.  That is, $\alpha _k$ is the fraction of the length that you are removing, so, for example, if the length of each interval in $C_k$ is $\frac{1}{6}$ and $\alpha _k=\frac{1}{5}$, then you will be removing the middle open intervals of length $\frac{1}{5}\cdot \frac{1}{6}=\frac{1}{30}$.  This is the basic idea anyways.  We now turn to the precise construction.

First of all, note that the set obtained from the interval $[a,b]$ of length $L\ceqq b-a$ by removing the middle $\alpha ^{\text{th}}$ of the interval is
\begin{equation}
\begin{multlined}
[a,b]\setminus \left( \tfrac{a+b}{2}-\alpha \tfrac{L}{2},\tfrac{a+b}{2}+\alpha \tfrac{L}{2}\right) \\ =\left[ a,\tfrac{a+b}{2}-\alpha \tfrac{L}{2} \right] \cup \left[ \tfrac{a+b}{2}+\alpha \tfrac{L}{2},b\right] .
\end{multlined}
\end{equation}
Note that this is the disjoint union of two closed intervals each with length $\frac{1-\alpha}{2}L$.  Using this, we now turn to the actual definitions.

Certainly, $C_0\ceqq [0,1]$ is the disjoint union of $2^0=1$ closed intervals all of the same length.  Inductively, assuming that $C_k$ is the disjoint union of $2^k$ closed intervals each of length $L$, we write
\begin{equation}
C_k=\bigcup _{i=1}^{2^k}[a_i,b_i]
\end{equation}
and define
\begin{equation}
\begin{multlined}
C_{k+1}\ceqq \\ \bigcup _{i=1}^{2^k}\left( \left[ a_i,\tfrac{a_i+b_i}{2}-\alpha _kL \right] \cup \left[ \tfrac{a_i+b_i}{2}+\alpha _kL,b_i\right] \right) .
\end{multlined}
\end{equation}
Note that $C_{k+1}$ is again the disjoint union of $2^{k+1}$ intervals all of the same  length $\frac{1-\alpha _k}{2}L$, and so this recursive definition is makes sense.  As mentioned before, we now define
\begin{equation}
C\ceqq \bigcap _{m\in \N}C_m.
\end{equation}

Having now defined $C$, we investigate some of its properties, the first of which being its Lebesgue measure.

From the above, if $C_k$ is the disjoint union of $2^k$ intervals each of length $L$, then $C_{k+1}$ will be the disjoint union of $2^{k+1}$ intervals each of length $\frac{1-\alpha _k}{2}L$.  As $\meas (C_0)=1$, we thus inductively find that
\begin{equation}
\meas (C_m)=2^m\cdot \tfrac{1}{2^m}\prod _{k=0}^{m-1}(1-\alpha _k)=\prod _{k=0}^{m-1}(1-\alpha _k).
\end{equation}
By \cref{exr5.1.29} (``continuity from above'') then, it follows that
\begin{equation}
\begin{split}
\meas (C) & \ceqq \meas \left( \bigcap _{m\in \N}C_m\right) =\lim _m(C_m) \\
& =\lim _m\prod _{k=0}^{m-1}(1-\alpha _k)\eqqc \prod _{m\in \N}(1-\alpha _m).
\end{split}
\end{equation}

We next show that $C$ is uncountable.  We do this in two steps.
\begin{exr}[breakable=false]{}{}
Show that $C$ is perfect.
\begin{rmk}
Recall that this means that $C$ is equal to its set of accumulation points---see \cref{PerfectSet}.
\end{rmk}
\end{exr}

Thus, in summary,
\begin{important}
$C$ is a nonempty perfect subset of $[0,1]$ with measure $\prod _{m\in \N}(1-\alpha _m)$.\footnote{In particular, it is closed, hence compact, and uncountable (\cref{prp2.5.49}).}
\end{important}
\end{exm}
Having introduced the family of generalized Cantor sets, we turn to two important special cases.
\begin{exm}{An uncountable set of zero measure---the Cantor Set}{CantorSet}
The \term{Cantor Set}\index{Cantor Set} is a generalized cantor set with $\alpha _m=\frac{1}{3}$ for all $m\in \N$.  In fact, there is nothing particularly special about $\frac{1}{3}$---what is important is that all the $\alpha _m$s are equal.  We do this more general case instead.  So, let $L\in (0,1)$, define $\alpha _m\ceqq L$ for all $m\in \N$, and denote by $C$ the resulting generalized Cantor set.

We know from \cref{GeneralizedCantorSets} that $C$ is nonempty and perfect, hence uncountable by \cref{prp2.5.49}.  Furthermore, as $0<1-L<1$,
\begin{equation}
\meas (C)=\lim _m\prod _{k=0}^{m-1}(1-L)=\lim _m(1-L)^m=0.
\end{equation} 
Thus, $C$ is indeed an uncountable set of measure $0$.
\end{exm}
\begin{exm}{A set with empty interior of positive measure---the Cantor-Smith-Volterra Set}{CantorSmithVolterraSet}
The Cantor-Smith-Volterra Set is in fact a generalized Cantor set, though that is not how we shall define it.\footnote{You'll see why in a moment---the definition is straightforward, and finding the appropriate $\alpha _m$s is unnecessary tedium.}
\begin{exr}[breakable=false]{}{}
Modify construction of a generalized Cantor set starting with $C_0\coloneqq [0,1]$ again, but upon constructing $C_{k+1}$ from $C_k$, remove the middle open interval of length $\frac{1}{4^{k+1}}$ from each closed interval of $C_k$.  The resulting set is the \term{Cantor-Smith-Volterra Set}\index{Cantor-Smith-Volterra Set}.  Show that it has measure $\frac{1}{2}$, but has empty interior.
\end{exr}
\begin{rmk}
Note that generalized Cantor sets were constructed by removing a given \emph{fraction} of each interval at each step, whereas in this case, what you are removing is not `relative', but rather, `absolute':  at the $k^{\text{th}}$ step, you remove intervals of length $\frac{1}{4^k}$ \emph{period}.
\end{rmk}
\begin{rmk}
The Cantor-Smith-Volterra Set is still uncountable of course, just as the Cantor Set likewise had empty-interior.  It's just that it's not surprising for a set with empty interior to have measure $0$---what is surprising however is a set with empty interior of \emph{positive measure}.  Similarly, it's not surprising for a uncountable set to have positive measure---what is surprising is an uncountable set of \emph{measure zero}.
\end{rmk}
\begin{exr}[breakable=false]{}{}
Show that the Cantor-Smith-Volterra Set is in fact a generalized Cantor set.
\begin{rmk}
That is, find $\alpha _m\in (0,1)$ for $m\in \N$ so that the resulting generalized Cantor set is the Cantor-Smith-Volterra set.
\end{rmk}
\end{exr}
\end{exm}

\horizontalrule

\begin{exm}{A set that is not Lebesgue-measurable}{exm5.2.47}\footnote{Construction adapted from \cite[pg.~407]{Pugh}.}
Define $\q :\R \rightarrow [0,1)$ by $\q (x)\coloneqq x-\lfloor x\rfloor$.  In other words, this is just the ``fractional part'' of the real number (intuitively, you drop of the integer in front of its decimal expansion).  Note that $1$ gets sent to $0$, and so, if you like, you can think of the image as a circle, with $1$ `glued to' $0$, if this helps your intuition.

Now fix $\theta \in \Q ^{\comp}$ and define $R:\R \rightarrow \R$ by $R(x)\coloneqq x+\theta$.\footnote{The ``$R$'' is for ``rotation'' because in the ``circle'' picture, this will correspond to a rotation by angle $\theta$.}  For $x_1,x_2\in \R$, define $x_1\sim x_2$ iff there is some $m\in \Z$ such that $x_1=R^m(x_2)\coloneqq x_2+m\theta$.
\begin{exr}[breakable=false]{}{}
Show that $\sim$ is an equivalence relation on $\R$.
\end{exr}
Denote by $O_x$ the equivalence class of $x\in \R$ with respect to $\sim$.\footnote{The ``$O$'' is for ``orbit''---you can imagine the point $x$ ``orbiting'' around the circle as you apply $R$ to it over-and-over.}
\begin{exr}[breakable=false]{}{}
Show that $\q (O_x)$ is dense in $[0,1)$.
\begin{rmk}
Hint:  This is why we needed $\theta$ to be \emph{irrational}.
\end{rmk}
\end{exr}

Now, let $N\subseteq \R$ be any set which contains exactly one point from each equivalence class with respect to $\sim$.  We claim that $\q (N)\subseteq [0,1)$ is not measurable.  We proceed by contradiction:  suppose that $\q (N)$ is measurable.
\begin{exr}[breakable=false]{}{}
Use the fact that $\sim$ is an equivalence relation to show that $\q (R^m(N))$ and $\q (R^n(N))$ are disjoint iff $m\neq n$.
\begin{rmk}
Hint:  See \cref{prpA.1.12}---equivalence classes form a partition of the set.
\end{rmk}
\end{exr}
\begin{exr}[breakable=false]{}{}
Show that
\begin{equation}
[0,1)=\bigcup _{m\in \Z}\q (R^m(N))
\end{equation}
\begin{rmk}
Hint:  Once again, uses the fact that equivalence classes form a partition.
\end{rmk}
\end{exr}
\begin{exr}[breakable=false]{}{}
Let $S\subseteq \R$.  Show that $\q (S)$ is measurable iff $\q (R(S))$ is measurable.
\end{exr}
\begin{exr}[breakable=false]{}{}
Let $S\subseteq \R$.  Show that $\meas (\q (S))=\meas (\q (R(S)))$.
\end{exr}
Thus, these exercises give us that
\begin{equation}
[0,1)=\bigcup _{m\in \Z}\q (R^m(N))
\end{equation}
is a disjoint union of measurable sets, all of which have the same measure $M\coloneqq \q (N)$.  If $M=0$, then, by additivity, we have $\meas ([0,1))=0$:  a contradiction.  On the other hand, if $M>0$, by additivity again, we have $\meas ([0,1))=\infty$:  a contradiction.  Therefore, it cannot be the case that $N$ is measurable.
\end{exm}
\begin{exm}{Two disjoint sets $S,T$ with $\meas (S\cup T)\neq \meas (S)+\meas (T)$}{exm5.2.56}
Let $N$ be as in the previous example.  We showed there that it was not measurable.  Therefore, there must exist some $S\subseteq \R$ such that
\begin{equation}
\begin{split}
\MoveEqLeft
\meas \left( (S\cap N)\cup (S\cap N^{\comp})\right) \\
& =\meas (S) \\
& \neq \meas (S\cap N)+\meas (S\cap N^{\comp}).
\end{split}
\end{equation}
\end{exm}
As a matter of fact, a similar sort of trick can be adapted to prove that \emph{every} set of positive measure has a nonmeasurable subset.
\begin{prp}{}{prp5.2.58}
Let $S\subseteq \R ^d$.  Then, if $\meas (S)>0$, then $S$ has a nonmeasurable subset.
\begin{wrn}
Warning:  The analogous result for ``measurable'' in place of ``not measurable'' is false.  Obviously, every set of positive measure (in fact, every set period) has a measurable subset:  the empty-set.  Instead, what you might guess is that every set of positive measure has a \emph{measurable} subset of \emph{positive measure}.  This, however, is false---see \cref{exm5.1.289}.
\end{wrn}
\begin{proof}\footnote{Proof adapted from \cite[pg.~53]{BigRudin}.}
Suppose that $\meas (S)>0$.  If $\meas (S)=\infty$, then because Lebesgue measure is semifinite (\cref{Semifinite}), there is some subset $T\subseteq S$ with $0<\meas (T)<\infty$.  If $T$ contains a set that is not measurable, then of course so does $S$, so it suffices to prove the case where $S$ has \emph{finite} measure.  Thus, without loss of generality, suppose that $\meas (S)<\infty$.

Consider the collection of cosets\footnote{See \cref{Cosets} for the definition of cosets, though this definition will not actually be important for us here.} $\{ x+\Q ^d:x\in \R ^d\}$.  Let $N\subseteq \R ^d$ be a set that contains precisely one element of each coset.  It follows that\footnote{This follows from the fact that the cosets form a partition of $\R$, which in turn follows from the fact that equivalence classes form partitions---see \cref{crlA.1.13}.}
\begin{equation}\label{eqn5.2.112}
\R ^d=\bigcup _{r\in \Q ^d}(r+N)
\end{equation}
is a disjoint union, and so it in turn follows that
\begin{equation}
S=\bigcup _{r\in \Q ^d}\left[ S\cap (r+N)\right]
\end{equation}
is a disjoint union.

We proceed by contradiction:  suppose that every subset of $S$ is measurable.  Then, the previous equality implies that
\begin{equation}
\meas (S)=\sum _{r\in \Q ^d}\meas (S\cap (r+N))
\end{equation}
We show that $\meas (S\cap (r+N))=0$ for all $r\in \Q$, which will gives us that $\meas (S)=0$:  a contradiction.

As $\meas (S\cap (r+N))<\infty$ and $\R$ is inner-regular on measurable sets (\cref{InnerRegularFinite}), it suffices to show that every quasicompact subset of $S\cap (r+N)$ has measure $0$.  So, let $K\subseteq S\cap (r+N)$ be quasicompact.

Define
\begin{equation}
H\coloneqq \bigcup _{s\in \Q ^d\cap [0,1]}(s+K).
\end{equation}
As $K\subseteq r+N$, this is a disjoint union (for the same reason that \eqref{eqn5.2.112} is a disjoint union), and so we have
\begin{equation}
\meas (H)=\sum _{s\in \Q ^d\cap [0,1]}\meas (s+K)=\footnote{By translation invariance.}\sum _{s\in \Q ^d\cap [0,1]}\meas (K).
\end{equation}
$H$ is bounded, and so has finite measure.  Thus, the above equality forces $\meas (K)=0$, and we are done.
\end{proof}
\end{prp}
\begin{exm}{A set with no measurable subsets of positive measure}{exm5.1.289}
Let $\q \colon \R \rightarrow [0,1)$ and $N$ be as in \cref{exm5.2.47}.  Recall that $\q (x)\ceqq [x]-\floor{x}$ and that $N$ is a set that contains exactly one element of each equivalence class with respect to the equivalence relation $x\sim y$ iff there is some $m\in \Z$ such that $x_1-x_2=m\theta$, where $\theta \in \Q ^{\comp}$ is some fixed irrational number.  We claim that $\q (N)$ has no measurable subset of positive measure.

So, let $M\subseteq \q (N)$ be measurable.  We wish to show that $\meas (M)=0$.  Define $M'\ceqq \q ^{-1}(M)$, so that $\q (M')=M$.  In \cref{exm5.2.47}, we showed that
\begin{equation}
[0,1)=\bigcup _{m\in \Z}\q (R^m(N))
\end{equation}
is a disjoint union, where $R\colon \R \rightarrow \R$ is defined by $R(x)\ceqq x+\theta$.  It follows that
\begin{equation}
\bigcup _{m\in \Z}\q (R^m(M'))
\end{equation}
is a disjoint union contained in $[0,1)$, and furthermore, a disjoint union of measurable sets as $M$ is measurable.  Therefore,
\begin{equation}
\begin{split}
1 & \geq \sum _{m\in \Z}\meas (\q (R^m(M'))=\sum _{m\in \Z}\meas (\q (R^0(M')) \\
& =\sum _{m\in \Z}\meas (M),
\end{split}
\end{equation}
which implies that $\meas (M)=0$, as desired.
\end{exm}

We mentioned awhile ago in the definition of measurable functions (\cref{MeasurableFunction}) the existence of a uniform-homeomorphism of $\R$ that preserves neither measurability not measure $0$.  It is time we return to this.
\begin{exm}{A uniform-homeomorphism of $\R$ that preserves neither measurability nor measure $0$---the Cantor Function}{CantorFunction}\index{Cantor Function}
We first define a uniformly-continuous function $f\colon [0,1]\rightarrow [0,1]$.  This function will be nondecreasing, and so $g\coloneqq f+\id _{[0,1]}:[0,1]\rightarrow [0,2]$ will be increasing, and hence injective.  It will turn out that $f(0)=0$ and $f(1)=1$, so that $g(0)=0$ and $g(1)=2$, and so we will then extend $g$ to all of $\R$ `periodically', defining $g(x)$ to be $g(x-1)+2$ for $x\in [1,2]$, etc..\footnote{Of course, we have to shift the graph up to maintain continuity.} 
\begin{rmk}
$f$ is the \term{Devil's Staircase}\index{Devil's Staircase}, and $g$ is the \term{Cantor Function}.\footnote{Some people use both these terms to refer to $f$.  I prefer this convention so that each has its own name.  Furthermore, $f$ is also known by the names of the \term{Cantor-Lebesgue Function}\index{Cantor-Lebesgue Function}, the \term{Lebesgue Function}\index{Lebesgue Function}, among other things---see the Wikipedia page for a more complete list of aliases.}
\end{rmk}

We define $f\colon [0,1]\rightarrow [0,1]$ as the uniform limit of a sequence of continuous functions.  It will then be continuous, because $\Mor _{\Top}([0,1],\R )$ is complete, and hence uniformly-continuous because $[0,1]$ is quasicompact (by the \namerefpcref{prp4.2.73}).  We then must check that $f$ is nondecreasing.  Once we do so, we will have that $g$ is increasing, and hence injective.  Then, because the domain $[0,1]$ is quasicompact and the codomain $[0,2]$ is $T_2$, we will have that its inverse is continuous (\cref{exr3.6.46} does this for us), and hence uniformly-continuous, and hence a uniform-homeomorphism.  Extending $g$ `periodically' has no effect on this, that is to say, the period extension to all of $\R$, $g\colon \R \rightarrow \R$, will still be a uniform-homeomorphism.

So, let us get started.\footnote{It will probably help to find a picture on this internet of what this is supposed to look like.  (Sorry.  Making diagrams takes awhile and I have not had the time.)  Let $C_m$ denote the `$m^{\text{th}}$ step' in the construction of the Cantor Set (see \cref{CantorSet}).  In words, $f_m$ is supposed to be the function that is constant $[0,1]\setminus C_m$ and increases linearly on the intervals that remain in $C_m$.}  Define $f_0:[0,1]\rightarrow [0,1]$ to be the identity function.  We define $f_m:[0,1]\rightarrow \R$ for $m>0$ recursively:
\begin{equation}\label{5.2.22}
f_m(x)\coloneqq \begin{cases}\tfrac{1}{2}f_{m-1}(3x) & \text{if }x\in [0,\tfrac{1}{3}] \\ \tfrac{1}{2} & \text{if }x\in [\tfrac{1}{3},\tfrac{2}{3}] \\ \tfrac{1}{2}+\tfrac{1}{2}f_{m-1}(3x-2) & \text{if }x\in [\tfrac{2}{3},1].\end{cases}
\end{equation}
\begin{exr}[breakable=false]{}{}
Show that $f_m$ is well-defined for all $m\in \N$.
\begin{rmk}
You must check that the two expressions agree for $x=\frac{1}{3}$ and $x=\frac{2}{3}$.
\end{rmk}
\end{exr}
\begin{exr}[breakable=false]{}{}
Show that $f_m(0)=0$ and $f_m(1)=1$ for all $m\in \N$.
\end{exr}
\begin{exr}[breakable=false]{}{}
Show that $f_m(x)\in [0,1]$, so that indeed $f_m$ defines a function into $[0,1]$.
\end{exr}
\begin{exr}[breakable=false]{}{}
Show that $f_m$ is continuous.
\begin{rmk}
Hint:  Use induction and apply the \namerefpcref{PastingLemma}.
\end{rmk}
\end{exr}
\begin{exr}[breakable=false]{}{}
Show that $f_m$ is nondecreasing.
\end{exr}

We now check that $m\mapsto f_m\in \End _{\Top}([0,1])$ is Cauchy.\footnote{Recall that $\End _{\Top}([0,1])\coloneqq \Mor _{\Top}([0,1],[0,1])$---see \cref{Endomorphism}.  To simplify notation, we shall denote the norm on $\End _{\Top}([0,1])$ simply as $\norm$.} To do this, we will show by induction that
\begin{equation}\label{5.2.28}
\norm{f_{m+1}-f_m}\leq \left( \frac{1}{2}\right) ^{m+1}.
\end{equation}
It will then follow from the Triangle Inequality that
\begin{equation}
\begin{split}
\norm{f_n-f_m} & \leq \sum _{k=m+1}^n\left( \frac{1}{2}\right) ^k\leq \sum _{k=m+1}^\infty \left( \frac{1}{2}\right) ^k \\
& =\frac{1}{1-\tfrac{1}{2}}-\frac{1-(\tfrac{1}{2})^{m+1}}{1-\tfrac{1}{2}}=\left( \frac{1}{2}\right) ^m,
\end{split}
\end{equation}
from which Cauchyness follows immediately.

Using the definition \eqref{5.2.22}, we have that
\begin{equation}
f_1(x)\coloneqq \begin{cases}\tfrac{1}{2}x & \text{if }x\in [0,\tfrac{1}{3}] \\ \tfrac{1}{2} & \text{if }x\in [\tfrac{1}{3},\tfrac{2}{3}] \\ \tfrac{1}{2}(3x-1) & \text{if }x\in [\tfrac{2}{3},1],\end{cases}
\end{equation}
and so
\begin{equation}
\left| f_1(x)-f_0(x)\right| =\begin{cases}\tfrac{1}{2}x & \text{if }x\in [0,\tfrac{1}{3}] \\ \abs{x-\tfrac{1}{2}} & \text{if }x\in [\tfrac{1}{3},\tfrac{2}{3}] \\ \tfrac{1}{2}(1-x) & \text{if }x\in [\tfrac{2}{3},1].\end{cases}
\end{equation}
It follows that
\begin{equation}
\norm{f_1-f_0}=\tfrac{1}{6}\leq \left( \tfrac{1}{2}\right) ^{0+1}.
\end{equation}
This completes the base case.  As for the inductive case, fix $m\geq 1$ and assume that \eqref{5.2.28} holds for $0\leq n<m$.  We prove that it holds for $m$ as well.  From the definition \eqref{5.2.22} again, we have that
{\small
\begin{equation*}
\begin{split}
\MoveEqLeft
\abs{f_{m+1}(x)-f_m(x)} \\
& =\tfrac{1}{2}\cdot \begin{cases}\abs{f_m(3x)-f_{m-1}(3x)} & \text{if }x\in [0,\tfrac{1}{3}] \\ 0 & \text{if }x\in [\tfrac{1}{3},\tfrac{2}{3}] \\ \abs{f_m(3x-2)-f_{m-1}(3x-2)} & \text{if }x\in [\tfrac{2}{3},1].\end{cases}
\end{split}
\end{equation*}
}
Thus, by the induction hypothesis,
\begin{equation}
\norm{f_{m+1}-f_m}\leq \tfrac{1}{2}\norm{f_m-f_{m-1}}=\left( \frac{1}{2}\right) ^{m+1}.
\end{equation}

We may now define $f\coloneqq \lim _mf\in \End _{\Top}([0,1])$.  You showed that each $f_m$ is nondecreasing, and so, as limits preserve inequalities, $f$ itself is nondecreasing.  Now define $g(x)\coloneqq f(x)+x$.  As described at the beginning of the example, $g\colon [0,1]\rightarrow [0,2]$ is a uniform-homeomorphism.

We now show that $g$ preserves neither measurability nor measure $0$.  Let $C\subset [0,1]$ denote the ($L=\tfrac{1}{3}$) Cantor Set and let $C_m$ denote the set defined in the $m^{\text{th}}$ step of the construction of the Cantor Set---see \cref{CantorSet} if you don't know what we're referring to.  For convenience of notation, define $D_m\coloneqq [0,1]\setminus C_m$ and $D\coloneqq [0,1]\setminus C$.
\begin{exr}[breakable=false]{}{}
Show that $\restr{f_k}{D_m}=\restr{f_m}{D_m}$ is constant\footnote{Note that this constant will depend on the component.} on each component of $D_m$ for all $k\geq m$.  In particular, the image of $f_k$ on $D_m$ is a finite set of points for all $k\geq m$.
\begin{rmk}
Though it's perhaps not clear from the formulas, the $f_m$s were defined so that precisely this is true:  even though $f(0)=0$ and $f(1)=1$, $f$ does all of increasing on a set of measure $0$, namely the Cantor Set.
\end{rmk}
\end{exr}
From this, it follows that $f$ itself is constant on each component of $D$.

Recall that each $D_m$ is a disjoint union of open intervals.  As $f$ is constant on each one of these intervals, the measure of the image of each one of these intervals under $g$ is just the length of that interval (the image is the interval itself plus whatever constant $f$ happened to be on that interval).  Furthermore, by injectivity, the images of each of these intervals must be disjoint, and hence, $\meas (g(D))$ is the sum of the measures of all these intervals, namely, $\meas (D)=1$.  As $g([0,1])=[0,2]$, we thus have that
\begin{equation}
\begin{split}
\meas (g(C)) & \geq \meas (g([0,1]))-\meas (g([0,1]\setminus C)) \\
& =\meas ([0,2])-\meas (g(D))=2-1=1.
\end{split}
\end{equation}
Thus, $\meas (g(C))\geq 1$, despite the fact that $\meas (C)=0$.\footnote{This shows that the preimage of a set of measure $0$ under $h\ceqq g^{-1}$ need not have measure $0$, even though $h$ is a uniform-homeomorphism.}

Every subset of $\R$ of positive measure contains a nonmeasurable set (\cref{prp5.2.58}), so let $N\subseteq g(C)$ be some such set.  Then, $g^{-1}(N)\subseteq C$ is measurable because it has measure $0$.  Thus, $M\ceqq g^{-1}(N)$ is measurable, but yet $g(M)=N$ is not.\footnote{This shows that the preimage of a measurable set under $h\ceqq g^{-1}$ need not be measurable, even though $h$ is a uniform-homeomorphism.}
\end{exm}

Before we finally move on to the integral, we tie up a loose end:  we defined $0^0\coloneqq 1$ way back in \cref{chp2} (see \cref{Exponentials}), but we never justified it---it is about time we do so.
\begin{exr}{$0^0\coloneqq 1$}{00}
Show that, for every $\varepsilon >0$, there is some open neighborhood $U$ of $\coord{0,0}\in \R _0^+\times \R _0^+$ such that
\begin{equation}
\frac{\meas \left( \left\{ \coord{x,y}\in U:\abs{x^y-1}\geq \varepsilon \right\} \right)}{\meas (U)}<\varepsilon .
\end{equation}
\begin{rmk}
In words, for every $\varepsilon >0$, there is an open neighborhood of $\coord{0,0}$ on which the fraction of the measure of the set on which $x^y$ is more than $\varepsilon$ away from $1$ is less than $\varepsilon$.
\end{rmk}
\begin{rmk}
I totally understand if people wish to leave the symbol $0^0$ undefined, but I think this result makes it clear that, \emph{if} $0^0$ is to represent any real number at all, then that real number should be $1$.
\end{rmk}
\begin{rmk}
Perhaps it's still worth keeping in mind that, for any real number $a\in [0,1]$, there is a net $\lambda \mapsto \coord{x_\lambda ,y_\lambda}\in \R _0^+\times \R _0^+$ that converges the origin, but for which $\lambda \mapsto (x_\lambda )^{y_\lambda}$ converges to $a$.
\end{rmk}
\end{exr}

\section{The integral}


\subsection[Char., simple, Borel, and integrable functions]{Characteristic, simple, Borel, and integrable functions}

An important type of function in measure theory are the \emph{simple functions}, of which the \emph{characteristic functions} are an important special case.
\begin{dfn}{Characteristic function}{CharacteristicFunctions}
Let $X$ be a set and let $S\subseteq X$.  Then, the \term{characteristic function}\index{Characteristic function} of $S$, $\chi _S:X\rightarrow \{ 0,1\}$\index[notation]{$\chi _S$}, is defined by
\begin{equation}
\chi _S(x)\coloneqq \begin{cases}1 & \text{if }x\in S \\ 0 & \text{if }x\notin S.\end{cases}
\end{equation}
\begin{rmk}
One reason that characteristic functions are important in measure theory is because the integral of $\chi _S$ will turn out to simply be $\meas (S)$.  In particular, \emph{if you know the integral, you know the measure}.
\end{rmk}
\begin{rmk}
We have seen a characteristic function before:  the \nameref{DirichletFunction} is the characteristic function of $\Q \subseteq \R$.
\end{rmk}
\begin{rmk}
You might say that the Lebesgue integral is to characteristic functions as the Riemann integral is to rectangles---one can define the Lebesgue integral (though we will not) by approximating functions with characteristic functions,\footnote{Well, actually finite linear combinations of characteristic functions, the so-called \emph{simple functions}\index{Simple function}.} just as one defines the Riemann integral by approximating functions with rectangles (aka (scalar multiples of) characteristic functions of intervals).\footnote{I actually find this approach a bit messy, but it's just yet more evidence that the Lebesgue integral is hardly more difficult than the Riemann integral---it's the same thing with arbitrary (measurable) sets instead of intervals.}
\end{rmk}
\end{dfn}
\begin{dfn}{Simple function}{SimpleFunctions}
Let $f\colon X\rightarrow Y$ be a function.  Then, $f$ is \term{simple}\index{Simple function} iff $\Ima (f)$ is finite.
\begin{rmk}
That is, iff $\Ima (f)$ has a finite number of elements.
\end{rmk}
\begin{rmk}
We will only be concerned with the case $Y=[-\infty ,\infty ]$, in which case this is equivalent to the statement that $f$ a finite nonnegative linear combination of characteristic functions of (disjoint) sets.
\end{rmk}
\end{dfn}
In principle, we can integrate \emph{any} function.\footnote{Literally.  You don't need the function to be measurable or Borel or whatever---you can always talk about ``measure of the `area' under the curve'', it's just that, unless you make some assumptions, this thing will be poorly behaved.}  The `problem' with this, of course, is that the integral will not satisfy certain properties we would like it to (e.g.~$\int \dif x\, [f(x)+g(x)]=\int \dif x\, f(x)+\int \dif x\, g(x)$) if we don't restrict the functions we integrate.  This is exactly analogous to how we do \emph{not} have $\meas (S\cup T)=\meas (S)+\meas (T)$ for $S$ and $T$ disjoint---in general we have to assume that $S$ and $T$ are measurable.  We thus seek a condition for functions that is analogous to the condition of measurability for sets.  The condition we are looking for is what is called \emph{Borel}.
\begin{dfn}{Borel functions}{IntegrableFunction}
Let $X$ be a measure space and let $f\colon X\rightarrow [-\infty ,\infty ]$ be a function.  Then, $f$ is \term{Borel}\index{Borel (function)} iff
\begin{equation}\label{5.2.60}
\{ \coord{x,y}\in X\times [-\infty ,\infty ]:0\leq y<f(x)\}
\end{equation}
and
\begin{equation}
\{ \coord{x,y}\in X\times [-\infty ,\infty ]:f(x)\leq y<0\}
\end{equation}
are measurable.
\begin{rmk}
The collection of all Borel functions $X\rightarrow [-\infty ,\infty]$ is denoted $\Bor (X)$\index[notation]{$\Bor (X)$}.  The collection of all nonnegative Borel functions on $X$ is denoted $\Bor _0^+(X)$\index[notation]{$\Bor _0^+(X)$}. 
\end{rmk}
\begin{rmk}
The two sets in \eqref{5.2.60} are the ``area `under' the curve'' and the ``area `below' the curve'' respectively.  The integral is going to be defined to be (\cref{Integral}) the measure of the first set minus the measure of the second set.  Thus, the condition of ``Borel'' is going to be a condition commonly imposed on are functions so that the integral satisfies the properties you would expect it to.\footnote{For example, just as we need not have $\meas (S\cup T)=\meas (S)+\meas (T)$ for $S$ and $T$ disjoint in general unless $S$ and $T$ are measurable, so to we need not have that $\int _X\dif \meas (x)\, [f(x)+g(x)]=\int _X\dif \meas (x)\, f(x)+\int _X\dif \meas (x)\, g(x)$ in general unless $f$ and $g$ are Borel.}
\end{rmk}
\begin{rmk}
The reason for the difference of the inequalities (i.e.\ ``$<$'' vs.\ ``$\leq $'') is essentially the same as the reason working with closed-open intervals is convenient:  the union of the two of these sets is a disjoint union.
\end{rmk}
\begin{rmk}
We explain the reason this is called ``Borel'' in \cref{BorelIsBorel} below.
\end{rmk}
\end{dfn}
\begin{exr}{}{exr5.2.139}
Let $\coord{X,\meas}$ be a topological measure space and let $f,g\colon X\rightarrow [-\infty ,\infty ]$.  Then, if $f(x)=g(x)$ almost-everywhere, then $f$ is Borel iff $g$ is Borel.
\end{exr}

It is most convenient to work with
\begin{equation}
\left\{ \coord{x,y}\in X\times [-\infty ,\infty ]:0\leq y<f(x)\right\} ,
\end{equation}
for similar reasons as it is more convenient to work with $[a,b)$ than it is $[a,b]$, but it actually makes no difference, as we now check.
\begin{prp}{}{}
Let $\coord{X,\meas}$ be a topological measure space and let $f\colon X\rightarrow [0,\infty ]$ be a function.  Then,
\begin{equation}
\left\{ \coord{x,y}\in X\times [0,\infty ]:y<f(x)\right\}
\end{equation}
is measurable iff
\begin{equation}
\left\{ \coord{x,y}\in X\times [0,\infty ]:y\leq f(x)\right\} 
\end{equation}
is.  Furthermore, they both have the same measure.
\begin{proof}
We leave this as an exercise.
\begin{exr}[breakable=false]{}{}
Prove the result yourself.
\begin{rmk}
Hint:  For the $\Rightarrow$ direction, consider the sequence $m\mapsto \left( 1+\frac{1}{m}\right) f$.  For the $\Leftarrow$ direction, consider the sequence $m\mapsto \left( 1-\frac{1}{m}\right) f$.
\end{rmk}
\end{exr}
\end{proof}
\end{prp}
There is a very important characterization of Borel functions.  In fact, this characterization is the reason Borel functions are called  ``\emph{Borel} functions''.
\begin{prp}{}{BorelIsBorel}
Let $\coord{X,\meas}$ be a topological measure space.  Then, the following are equivalent.
\begin{enumerate}
\item \label{BorelIsBorel.i}$f\in \Bor (X)$.
\item \label{BorelIsBorel.ii}$f^{-1}(U)$ is measurable for every $U\subseteq [-\infty ,\infty ]$ open.
\item \label{BorelIsBorel.iii}$f^{-1}(C)$ is measurable for every $C\subseteq [-\infty ,\infty ]$ closed.
\end{enumerate}
\begin{rmk}
As preimages preserves unions, intersections, and complements, it follows that the preimage of any Borel set (\cref{BorelMeasure}) will likewise be measurable.  This is why we call Borel functions ``\emph{Borel} functions''.
\end{rmk}
\begin{rmk}
A lot of authors call these \emph{measurable functions}.  This conflicts with other standard terminology (which we do make use of---see \cref{MeasurableFunction}), and so I recommend you not use it.  There is a way to make this not inconsistent, but it's awkward:  when you have a function $f\colon \R \rightarrow \R$, for example, you simply declare that the ``measurable'' sets in the codomain are the Borel sets but the ``measurable'' sets in the domain are all (Lebesgue) measurable sets.  Ew.
\end{rmk}
\begin{proof}\footnote{Proof adapted from \cite[pg.~384]{Pugh}.}
$(\cref{BorelIsBorel.i}\Rightarrow \cref{BorelIsBorel.ii})$ Suppose that $f\in \Bor (X)$.  Let us write
\begin{equation}
\Gamma _f\coloneqq \{ \coord{x,y}\in X\times [0,\infty ]:y<f(x)\} .
\end{equation}

We first do the case where $f$ is bounded, say by $C\in [0,\infty )$, and is $0$ outside a set of finite measure $M\subseteq X$.  In this case, $[\meas \times \meas _{\mrm{L}}](\Gamma _f)$ is finite,\footnote{$\meas _{\mrm{L}}$ denotes Lebesgue measure on $[0,\infty ]$, to distinguish it from the measure on $X$.} and so by inner-regularity on measurable subsets of finite measure, there is an increasing sequence of compact sets $K_0\subseteq K_1\subseteq K_2\subseteq \cdots \subseteq \Gamma _f$ such that $[\meas \times \meas _{\mrm{L}}](\Gamma _f\setminus C)=0$, where $C\coloneqq \bigcup _{m\in \N}K_m$ (\cref{InnerRegularFinite}).  Now, define $g_m,h_m:X\rightarrow [0,\infty ]$ by
{\scriptsize
\begin{equation}
g_m(x)\coloneqq \begin{cases}\sup _{\coord{x,y}\in K_m}\{ y\} & \text{if }K_m\cap (\{ x\} \times [0,\infty ])\neq \emptyset \\ 0 & \text{if }K_m\cap (\{ x\} \times [0,\infty ])=\emptyset \end{cases}
\end{equation}
}
and
{\scriptsize
\begin{equation}
h_m(x)\coloneqq \begin{cases}\inf _{\coord{x,y}\in (X\times \R )\setminus K_m}\{ y\} & \text{if }K_m\cap (\{ x\} \times [0,\infty ])\neq \emptyset \\ 0 & \text{if }K_m\cap (\{ x\} \times [0,\infty ])=\emptyset .\end{cases}
\end{equation}
}
Note that, as $K_m\subseteq \Gamma _f$, $g_m(x)\leq f(x)$ and $f(x)\leq h_m(x)$. Furthermore, for each fixed $x\in X$, the sequence $m\mapsto g_m(x)$ is nondecreasing, and so by the \namerefpcref{MonotoneConvergenceTheorem}, we may define $g(x)\coloneqq \lim _mg_m(x)$.  Similarly, we may define $h(x)\coloneqq \lim _mh_m(x)$.
\begin{exr}[breakable=false]{}{}
Show that the preimage under $g_m$ of every open set is measurable by showing that the preimage of $(-\infty ,b)$ is open for all $b\in \R$.  Show that the preimage under $h_m$ of every open set is measurable by showing that the preimage of $(a,\infty )$ is open for all $a\in \R$.
\end{exr}
\begin{exr}[breakable=false]{}{}
Use this to deduce that the preimage of every open set under both $g$ and $h$ measurable.
\end{exr}
\begin{exr}[breakable=false]{}{}
Show that $[\meas \times \meas _{\mrm{L}}](\Gamma _{h-g})=0$.
\end{exr}
\begin{exr}[breakable=false]{}{}
Use this, together with the fact that the preimage of every open set under both $g$ and $h$ is measurable, to show that $g(x)=h(x)$ for almost-every $x$.
\end{exr}
As $g\leq f\leq h$, it follows that $f$ itself is equal almost-everywhere to both $g$ and $h$.
\begin{exr}[breakable=false]{}{}
Use the fact that $f(x)=g(x)$ almost-everywhere and the fact that the preimage of every open set under $g$ is measurable to show that the preimage of every open set under $f$ is measurable.
\end{exr}
This completes the proof in the case that $f$ is bounded and vanishes outside a set of finite measure.

We now do the proof for $f\in \Bor _0^+(X)$ be arbitrary.  Write $X=\bigcup _{m\in \N}M_m$ as the increasing union of measurable sets of finite measure, so that
\begin{equation}
X\times [0,\infty ]=\bigcup _{m\in \N}M_m\times [0,m)\cup \bigcup _{m\in \N}M_m\times \{ \infty \} .
\end{equation}
Thus,
\begin{equation}
\begin{multlined}
\Gamma _f=\bigcup _{m\in \N}\Gamma _f\cap (M_m\times [0,m)) \\ \cup \bigcup _{m\in \N}\Gamma _f\cap (M_m\times \{ \infty \} ).
\end{multlined}
\end{equation}
Note that
\begin{equation}
\begin{split}
\MoveEqLeft
\Gamma _f\cap (M_m\times [n,n+1)) \\
& =\left\{ \coord{x,y}\in X\times [-\infty ,\infty ]:\right. \\ & \qquad \left. y<f(x),\ x\in M_m ,\ n\leq y<n+1\right\} \\
& =\Gamma _{f_m},
\end{split}
\end{equation}
where $f_m:X\rightarrow \R$ is defined by
\begin{equation}
f_m(x)\coloneqq \begin{cases}f(x) & \text{if }x\in M_m\text{ and }f(x)<m \\ m & \text{if }x\in M_m\text{ and }f(x)\geq m \\ 0 & \text{otherwise.}\end{cases}
\end{equation}
$\Gamma _{f_m}$ is measurable because $\Gamma _f$ is.  Furthermore, $f_m$ is bounded and vanishes outside a set of finite measure, and so the previous case applies, and we have that the preimage of open sets under $f_m$ are measurable.
\begin{exr}[breakable=false]{}{}
Use this to show that the preimage of open sets under $f$ are measurable.
\end{exr}

Finally, for $f\in \Bor (X)$ arbitrary (not necessarily nonnegative), each of the nonnegative and nonpositive parts of $f$ are Borel (by definition), and so the previous case shows that the preimage under either of these functions of open sets is measurable.  It then follows that the preimage under $f$ of open sets are measurable.

\blankline
\noindent
$(\cref{BorelIsBorel.ii}\Rightarrow \cref{BorelIsBorel.iii})$ Suppose that $f^{-1}(U)$ is measurable for every $U\subseteq [-\infty ,\infty ]$ open.  Let $C\subseteq [-\infty ,\infty ]$ be closed.  Then, $C^{\comp}$ is open, and so $f^{-1}(C^{\comp})=f^{-1}(C)^{\comp}$ is measurable, and so the complement of this, $f^{-1}(C)$, is measurable.

\blankline
\noindent
$(\cref{BorelIsBorel.iii}\Rightarrow \cref{BorelIsBorel.i})$ Suppose that $f^{-1}(C)$ is measurable for every $C\subseteq [-\infty ,\infty ]$ closed.  It follows that the preimage of half-open rectangles are measurable,\footnote{Why?} and so
\begin{equation}
\begin{multlined}
\left\{ \coord{x,y}\in X\times [0,\infty ]:0\leq y<f(x)\right\} \\ =\bigcup _{r\in \Q}f^{-1}([r,\infty ))\times [0,r)
\end{multlined}
\end{equation}
is measurable.  Similarly the `negative' version of this set is measurable too, and so $f$ is Borel.
\end{proof}
\end{prp}

\begin{prp}{}{SupInfBorelIsBorel}
Let $\coord{X,\meas}$ be a topological measure space and let $\{ f_m:m\in \N \} \subseteq \Bor (X)$ be a countable collection of measurable functions.  Then, $x\mapsto \sup _{m\in \N}f_m(x)$ and $x\mapsto \inf _{m\in \N}f_m(x)$ are Borel.
\begin{proof}
We leave this as an exercise.
\begin{exr}[breakable=false]{}{}
Prove the result yourself.
\end{exr}
\end{proof}
\end{prp}
\begin{prp}{}{LimitBorelIsBorel}
Let $\coord{X,\meas}$ be a topological measure space and let $m\mapsto f_m\in \Bor (X)$ be a sequence converging to $f_{\infty}\colon X\rightarrow [-\infty ,\infty ]$ in $\Mor _{\Set}(X,[-\infty ,\infty ])/\sim _{\AlE}$.  Then, $f_\infty \in \Bor (X)$.
\begin{rmk}
Saying that it converges
\begin{equation}
\text{``in }\Mor _{\Set}(X,[-\infty ,\infty ])/\sim _{\AlE}\text{''}
\end{equation}
is a fancy way of saying that it converges to $f_{\infty}(x)$ almost-everywhere.
\end{rmk}
\begin{wrn}
Warning:  This fails for nets in general---see the following counter-example (\cref{exm5.2.169}).
\end{wrn}
\begin{proof}
We leave this as an exercise.
\begin{exr}[breakable=false]{}{}
Prove the result yourself.
\begin{rmk}
Hint:  Use the previous result.
\end{rmk}
\end{exr}
\end{proof}
\end{prp}
\begin{exm}{A limit of a Borel function which is not Borel}{exm5.2.169}
Let $N\subseteq \R$ be a set that is not measurable and define
\begin{equation}
\collection{S}\ceqq \left\{ S\subseteq N:S\text{ is measurable.}\right\} .
\end{equation}
This is a directed set with respect to inclusion.  Certainly each $\chi _S$ for $S\in \collection{S}$ is measurable.  I claim that $\lim _{S\in \collection{S}}\chi _S=\chi _N$ point-wise, which will show that $S\mapsto \chi _S$ is a net of Borel functions converging to a function which is not Borel.\footnote{In fact, it is a nondecreasing sequence and converges \emph{everywhere} (not just almost-everywhere).}

So, let $x\in \R$.  If $x\notin N$, then $\chi _S(x)=0$ for all $S\in \collection{S}$, and so certainly $S\mapsto \chi _S(x)$ converges to $\chi _N(x)=0$.  On the other hand, if $x\in N$, then, whenever $S\supseteq \{ x\}$,\footnote{Note that $\{ x\} \in \collection{S}$.} it follows that $x\in S$ and hence $\chi _S(x)=1$.  Thus, in this case we will have $S\mapsto \chi _S(x)$ converges to $\chi _N(x)=1$ as well.
\end{exm}

In the course of showing uniqueness of the integral, it will be important to know that we can approximate Borel functions by finite linear combinations of characteristic functions.
\begin{prp}{}{SimpleFunctionApproximation}
Let $X$ be a topological measure space and let $f\colon X\rightarrow [-
\infty ,\infty ]$.  Then, the following are equivalent.
\begin{enumerate}
\item \label{SimpleFunctionApproximation.i}$f$ is nonnegative (Borel).
\item \label{SimpleFunctionApproximation.ii}There exists a nondecreasing sequence $m\mapsto s_m\in \Mor _{\Set}(x,[0,\infty ])$ of (Borel) simple functions such that $\lim _ms_m=f$ in $\Mor _{\Set}(X,[0,\infty ])/\sim _{\AlE}$.
\item \label{SimpleFunctionApproximation.iii}There are (measurable) sets $M_m\subseteq X$ and positive real numbers $c_m>0$ such that $f=\sum _{m\in \N}c_m\chi _{M_m}$.
\end{enumerate}
\begin{rmk}
To clarify, there are really \emph{two} sets of equivalent conditions here, one including the words in parentheses and one without.
\end{rmk}
\begin{proof}\footnote{Proof adapted from \cite[pg.~31]{Stein}.}
$(\cref{SimpleFunctionApproximation.i}\Rightarrow \cref{SimpleFunctionApproximation.ii})$ Suppose that $f$ is nonnegative (Borel).  Write $X=\bigcup _{m\in \N}K_m$ for $K_m\subseteq X$ compact.  Define $f_m:X\rightarrow [0,\infty ]$ by
\begin{equation}
f_m(x)\coloneqq \begin{cases}f(x) & \text{if }x\in K_m\text{ and }f(x)\leq m \\ m & \text{if }x\in K_m\text{ and }f(x)>m \\ 0 & \text{otherwise.}\end{cases}
\end{equation}
For $n\in \N$ and $0\leq o<mn$, define
\begin{equation}
S_{m,n,o}\coloneqq \left\{ x\in K_m:\tfrac{o}{m}<f_n(x)\leq \tfrac{o+1}{m}\right\} .
\end{equation}
Finally, define $s_m:X\rightarrow [0,\infty ]$ by
\begin{equation}
s_m\coloneqq \sum _{o=0}^{m^2}\tfrac{o}{m}\chi _{S_{m,m,o}}.
\end{equation}
\begin{exr}[breakable=false]{}{}
Show that $m\mapsto s_m$ is a nondecreasing sequence of (Borel) simple functions converging pointwise to $f$
\end{exr}

\blankline
\noindent
$(\cref{SimpleFunctionApproximation.ii}\Rightarrow \cref{SimpleFunctionApproximation.iii})$ Suppose that there exists a nondecreasing sequence $m\mapsto s_m\in \Mor _{\Set}(X,[0,\infty ])$ of (Borel) simple functions such that $\lim _ms_m=f$ in $\Mor _{\Set}(X,[0,\infty ])/\AlE$.  Let $c\chi _M$ be a term appearing in $s_m$, with $M\subseteq X$ (measurable) and $c>0$.  As $s_{m+1}\geq s_m$, there must be some term $c'\chi _{M'}$ appearing in $s_{m+1}$ with $M'\subseteq X$ (measurable), $c'>0$, $M\subseteq M'$, and $c\leq c'$.  Then note that $c'\chi _{M'}-c\chi _M=(c'-c)\chi _M+c'\chi _{M'\setminus M}$.  It follows that $s_{m+1}-s_m$ is again a simple (Borel) function.\footnote{Recall that (\cref{SimpleFunctions}) a simple function is a \emph{nonnegative} linear combination of characteristic functions, so we needed to check that we could rewrite $s_{m+1}-s_m$ in such a way so that no negative coefficients appeared.}  It follows in turn that
\begin{equation*}
\begin{split}
s_m & =(s_m-s_{m-1})+(s_{m-1}-s_{m-2})+\cdots +(s_1-s_0)+s_0 \\
& =\sum _{k=0}^{n_m}c_k\chi _{M_k}
\end{split}
\end{equation*}
for $M_k\subseteq X$ (measurable), $c_k>0$, and some $n_m\in \N$.  We thus have that $f=\sum _{m\in \N}c_m\chi _{M_m}$, as desired.

\blankline
\noindent
$(\cref{SimpleFunctionApproximation.iii}\Rightarrow \cref{SimpleFunctionApproximation.i})$ Suppose that there are (measurable) sets $M_m\subseteq X$ and positive real numbers $c_m>0$ such that $f=\sum _{m\in \N}c_m\chi _{M_m}$.  Then, $f$ is nonnegative (Borel) because sums of nonnegative (Borel) functions are nonnegative (Borel (\cref{exr5.2.167})) and limits of sequences of nonnegative (Borel) functions are nonnegative (Borel (\cref{LimitBorelIsBorel})).
\end{proof}
\end{prp}

One technique that will prove invaluable to us is the ability to decompose a function into its nonnegative and nonpositive parts.  A common strategy will be to prove results for nonnegative functions, and then use the below decomposition to deduce the result in the general case.
\begin{dfn}{}{dfn5.2.38}
Let $X$ be a set and let $f\colon X\rightarrow [-\infty ,\infty ]$ be a function.  Then, we write
\begin{equation}
f_+\coloneqq \max \{ f,0\} \text{ and }f_-\coloneqq -\min \{ f,0\} .
\end{equation}\index[notation]{$f_+$}\index[notation]{$f_-$}
\begin{rmk}
That is, $f_+$ is equal to $f$ if $f$ is positive and $0$ otherwise; likewise, $f_-$ is equal to $f$ if $-f$ is negative and $0$ otherwise.
\end{rmk}
\begin{rmk}
Note that \emph{always} $f_+,f_-\geq 0$.
\end{rmk}
\begin{rmk}
Also note that $f=f_+-f_-$ and $\abs{f}=f_++f_-$.
\end{rmk}
\end{dfn}
\begin{exr}{}{}
Let $X$ be a measure space and let $f\colon X\rightarrow [-\infty ,\infty ]$ be a function.  Show that $f$ is Borel iff $f_+$ and $f_-$ are Borel.
\end{exr}

Of course, we certainly want it to be the case case that sums and products of Borel functions are Borel.
\begin{exr}{}{exr5.2.167}
Let $X$ be a topological measure space and let $f,g\in \Bor (X)$.
\begin{enumerate}
\item Show that $f+g\in \Bor (X)$.
\item Show that $fg\in \Bor (X)$.
\end{enumerate}
\begin{rmk}
Hint:  First prove it for nonnegative Borel functions, and then use the decomposition $f=f_+-f_-$ to prove the result for arbitrary Borel functions.
\end{rmk}
\end{exr}

\subsection{The integral itself}\label{TheIntegralItself}

And finally, I present unto thee, the \emph{integral}.
\begin{dfn}{Integral}{Integral}
Let $\coord{X,\meas}$ be a topological measure space, let $f\colon X\rightarrow [-\infty ,\infty ]$, and write
{\scriptsize
\begin{equation}\label{eqn5.2.173}
\mrm{I}_+(f)\ceqq [\meas \times \meas _{\mrm{L}}]\left( \left\{ \coord{x,y}\in X\times [-\infty ,\infty ]:0\leq y<f(x)\right\} \right)
\end{equation}
}
and
{\scriptsize
\begin{equation}\label{eqn5.2.174}
\mrm{I}_-(f)\ceqq [\meas \times \meas _{\mrm{L}}]\left( \left\{ \coord{x,y}\in X\times [-\infty ,\infty ]:f(x)\leq y<0\right\} \right) ,
\end{equation}
}
where $\meas _{\mrm{L}}$ is Lebesgue measure.

Then, $f$ is \term{$\infty$-integrable}\index{$\infty$-integrable} iff at least one of $\mrm{I}_+$ and $\mrm{I}_-$ is finite, in which case the \term{integral}\index{Integral} of $f$, $\int _X\dif \meas (x)\, f(x)$, is defined by
\begin{equation}\label{eqn5.2.175}
\int _X\dif \meas (x)\, f(x)\ceqq \mrm{I}_+(f)-\mrm{I}_-(f).
\end{equation}\index[notation]{$\int _X\dif \meas (x)\, f(x)$}
Furthermore, if both $\mrm{I}_+$ and $\mrm{I}_-$ are finite, then $f$ is \term{integrable}\index{Integrable}.
\begin{rmk}
We say that $\int _X\dif \meas (x)\, f(x)$ \term{converges}\index{Convergence (of an integral)} iff $f$ is integrable.
\end{rmk}
\begin{rmk}
For $S\subseteq X$, we define
\begin{equation}
\int _S\dif \meas (x)\, f(x)\coloneqq \int _X\dif \meas (x)\, \chi _S(x)f(x).
\end{equation}\index[notation]{$\int _X\dif \meas (x)\, f(x)$}
If $X=\R$ and $S=[a,b]$, we define
\begin{equation}
\int _a^b\dif x\, f(x)\coloneqq \int _{[a,b]}\dif \meas _{\mrm{L}}(x)\, f(x)
\end{equation}\index[notation]{$\int _a^b\dif x\, f(x)$}
\noindent Warning:  Many results require the integrand to be Borel, which means that, if you want to apply the result to $\int _S\dif \meas (x)\, f(x)$, you need $f$ to be Borel \emph{and} $S$ to be measurable.  Don't forget about $S$!
\end{rmk}
\begin{rmk}
The two sets in \eqref{eqn5.2.173} and \eqref{eqn5.2.174} are the ``area `under' the curve'' and the ``area `below' the curve'' respectively.  Thus, the condition ``$\infty$-integrable'' is exactly the condition needed for this difference, that is, $\int _X\dif \meas (x)\, f(x)$, to \emph{make sense}, and the condition ``integrable'' is exactly the condition needed for this difference to make sense and be \emph{finite} (that is, if you have the prefix ``$\infty$'', you allow for the integral to be infinite).
\end{rmk}
\begin{rmk}
Note that we do \emph{not} require ($\infty$-)integrable functions be Borel.  Indeed, the assumption that $f$ be $\infty$-integrable is essentially the bare-minimum assumption we need to make in order for this to make sense, and the measure of these sets makes sense irrespective of whether or not they are measurable (i.e.~whether or not $f$ is Borel).  If you're wondering ``But are there any functions that are not Borel that I actually want to be able to integrate?'', the answer is ``Almost certainly not.''.  Despite this, however, it is convenient to have the integral make sense for not-necessarily-Borel functions because, while in practice the functions you're working with are Borel anyways, it is nice that you don't have to check this before writing down the integral, for example, see the remark in \namerefpcref{FubinisTheorem}.
\end{rmk}
\begin{rmk}
We should probably mention that this convention is nonstandard---for most authors, the term ``integrable'' would be equivalent to ``Borel and integrable'' in our terminology.  Furthermore, most authors just don't use a term analogous to $\infty$-integrable.\footnote{By this point, it probably goes without saying that I prefer the terminology I do because it allows me to be more precise.  For example, I now have terms for four classes of functions:  integrable, $\infty$-integrable, Borel integrable, and Borel $\infty$-integrable; whereas with the other convention we would only have the terminology to speak of just one class of functions.}
\end{rmk}
\begin{rmk}
If the measure is clear from context, we may just simply write $\dif x$ instead of $\dif \meas (x)$.\index[notation]{$\int _X\dif x\, f(x)$}  We may even write $\int _X\dif \meas \, f$ or just $\int _Xf$ if `the variable of integration' is irrelevant.
\end{rmk}
\begin{rmk}
I happen to prefer to put the $\dif x$ in front of the integral,\footnote{For two reasons:  (i)~it tells you immediately what the variable you are integrating with respect to is (just a slight convenience, especially when the integrand is complicated) and (ii)~it is more analogous to how we write the derivative---no one writes $\dif f(x)/\dif x$---people write $\frac{\dif}{\dif x}f(x)$.} but don't let that confuse you---the meaning is just the same as $\int f(x)\dif x$.
\end{rmk}
\begin{rmk}
In earlier versions of the notes, I wanted to extend the definition of the integral to other functions whose integral `should' exist, but does not exist in this sense.  One classic example of this is the function $\R \ni x\mapsto \frac{\sin (x)}{x}$.  On one hand, this is not even$\infty$-integrable; however, $\lim _{a\to \infty}\int _{-a}^a\dif x\, \frac{\sin (x)}{x}$ \emph{does} converge.  The idea was then to define the integral of such functions to be $\lim _{K\in \collection{K}}\int _K\dif \meas (x)\, f(x)$, where $\collection{K}$ is the collection of quasicompact subsets of $X$.  Unfortunately, however, this just doesn't work:  if $f$ is $\infty$-integrable (and Borel), then it converges to the usual integral, but otherwise the limit does not exist.  However we will have to wait awhile to see this---see \cref{prp5.2.229}.

Needless to say, the Riemann integral has this `defect' as well.  In fact, it's \emph{much} worse:  the Riemann integral is not defined (at least not without taking limits) over \emph{any} unbounded set for \emph{any} function.  On the other hand, there are definitions of the integral that agree with the Lebesgue integral for all Lebesgue integrable functions, and furthermore, will assign the `correct' value to the integral of functions like $x\mapsto \frac{\sin (x)}{x}$.\footnote{The so-called \emph{Henstock integral}\index{Henstock integral} is such an example.}  Unfortunately, however, all such definitions I am aware of are $\R ^d$-specific, and this is nowhere even close to being powerful enough to cover all cases of interest.  It is a very tiny inconvenience to have to use $\lim _{a\to \infty}\int _{-a}^a\dif x\, f(x)$ instead of just $\int _{\R}\dif x\, f(x)$; on the other hand, it is a \emph{huge} inconvenience to not be able to integrate over anything besides Euclidean space.
\end{rmk}
\begin{rmk}
Note that
\begin{equation*}
\int _X\dif \meas (x)\, f(x)=\int _X\dif \meas (x)\, f_+(x)-\int _X\dif \meas (x)\, f_-(x)
\end{equation*}
\emph{even} if $f$ is not Borel.\footnote{We mention this because in general you \emph{do} need $f$ and $g$ to be Borel to have $\int _X\dif \meas (x)\, [f(x)+g(x)]=\int _X\dif \meas (x)\, f(x)+\int _X\dif \meas (x)\, g(x)$---see \cref{FundamentalTheoremOfTheIntegral}.}  This is important because it means a lot of properties you can prove about the integral by first proving the result for nonnegative functions, and then using the decomposition $f=f_+-f_-$ to extend the result for \emph{all} functions.
\end{rmk}
\begin{rmk}
If one needs to specify, this is the \term{Lebesgue integral}\index{Lebesgue integral}.  You may have heard of the \term{Lebesgue-Stieltjes integral}\index{Lebesgue-Stieltjes integral}.  This is in fact no more general than the Lebesgue integral as the Lebesgue-Stieltjes integral with respect to a function $h$ (that has to satisfy some hypotheses in order for things to make sense) of $f$ is just the (Lebesgue) integral of $f$ with respect to a measure that depends on $h$.  We could study this measure if we wanted to, but we have no need.  The point is to reassure you that you're not `missing out' on something more general.\footnote{In fact, I know of no definition of the integral which strictly generalizes the Lebesgue integral (in a way that makes no direct reference to the Lebesgue integral).  The parenthetical comment refers to the fact that you can define integrals with values in more general spaces (e.g.~the \term{Gelfand-Pettis integral}\index{Gelfand-Pettis integral} for topological vector spaces), but in order to define such integrals you ultimately reduce their definitions to the Lebesgue integral.}
\end{rmk}
\end{dfn}
I am not sure how to uniquely characterize the definition of the integral on \emph{all} ($\infty$-integrable) functions, \emph{but}, for Borel functions, there are a couple of relatively easy properties that uniquely characterize the integral.
\begin{thm}{Fundamental Theorem of the Integral}{FundamentalTheoremOfTheIntegral}
Let $\coord{X,\meas}$ be a topological measure space.  Then, $f\mapsto \int _X\dif \meas (x)\, f(x)$ is the unique function $\mrm{I}\colon \Bor _0^+(X)\rightarrow [0,\infty ]$ such that
\begin{enumerate}
\item (Normalization)\label{Integral.Normalization} $\mrm{I}(\chi _S)=\meas (S)$ for $S\subseteq X$ measurable;\footnote{$\chi _S$ won't be Borel unless $S$ is measurable, though certainly this result is true just the same.}
\item (Additivity)\label{Integral.Additivity} $\mrm{I}(f+g)=\mrm{I}(f)+\mrm{I}(g)$ for $f,g\in \Bor _0^+(X)$;
\item (Nonnegative-homogeneity)\label{Integral.NonnegativeHomogeneity} $\mrm{I}(af)=a\mrm{I}(f)$ for $f\in \Bor _0^+(X)$ and $a\geq 0$;\footnote{Of course, this is actually true for \emph{all} $a\in \R$, but for the time being, as we are currently regarding $\mrm{I}$ as a function of \emph{nonnegative} Borel functions, we need $a\geq 0$ in order that $af$ itself is nonnegative.} and
\item (Lebesgue's Monotone Convergence Theorem)\label{Integral.LebesguesMonotoneConvergenceTheorem}\index{Lebesgue's Monotone Convergence Theorem} whenever $m\mapsto f_m$ is a nondecreasing sequence of Borel functions then
\begin{equation}
\lim _m\mrm{I}(f_\lambda )=\mrm{I}(\lim _mf_{m}).\footnote{Note that $\lim _mf(x)$ \emph{always} exists, by the usual \nameref{MonotoneConvergenceTheorem}---either it is bounded, in which case it converges in $\R$, or it is unbounded, in which case it converges to $\infty$.  (I lied a bit\textellipsis as it is only nondecreasing almost-everywhere, then the limit exists only almost-everywhere.)  Also note that $f_\infty \in \Bor _0^+(X)$ by \cref{LimitBorelIsBorel}.}
\end{equation}
\end{enumerate}
\begin{rmk}
Furthermore, we always have $\mrm{I}(f+g)\leq \mrm{I}(f)+\mrm{I}(g)$ and $\lim _{\lambda}\mrm{I}(f_{\lambda})\leq \mrm{I}(\lim _{\lambda}f_{\lambda})$, even if the functions involved are not necessarily Borel and even if the net is not a sequence, though, oddly, we have to wait to prove this---see \cref{prp5.2.287}.
\end{rmk}
\begin{wrn}
Warning:  $\lim _m\int _Xf_m=\int _X\lim _m f_m$ can fail if the convergence is not monotone, or even if it is monotone \emph{decreasing}---see the \cref{MonotoneCounterexample}.
\end{wrn}
\begin{rmk}
First of all, we are characterizing the integral uniquely via certain properties it satisfies, and then furthermore, it turns out that this unique such function is simply given by \emph{the `area' under the curve}.  What do you think of that, Riemann?\footnote{Seriously, people actually claim that the Lebesgue integral is not ``geometric''.  Da fuq?  Can someone please explain to me how limits of partitions of sums of areas of rectangles is more geometric than the area under the curve?}
\end{rmk}
\begin{proof}\footnote{Proof adapted from \cite[pg.~377]{Pugh}.}
\Step{Introduce notation}
As we will be making use of the set quite a bit, it will be useful to introduce the notation
\begin{equation}
\Gamma _f\coloneqq \left\{ \coord{x,y}\in X\times [0,\infty ]:y<f(x)\right\} .
\end{equation}

\Step{Define $\mrm{I}$}
Let $f\colon X\rightarrow \R _0^+$ and define
\begin{equation}
\mrm{I}(f)\coloneqq \int _X\dif \meas (x)\, f(x)=[\meas \times \meas _{\mrm{L}}](\Gamma _f) .
\end{equation}

\Step{Show that $\mrm{I}(\chi _S)=\meas (S)$}
\begin{equation}
\Gamma _{\chi _S}\coloneqq \left\{ \coord{x,y}\in X\times \R :y<\chi _S(x)\right\} =S\times [0,1)
\end{equation}
and so
\begin{equation}
\mrm{I}(\chi _S)\coloneqq [\meas \times \meas _{\mrm{L}}](\Gamma _{\chi _S})=\meas (S)\cdot 1=\meas (S).
\end{equation}

\Step{Show that $\mrm{I}$ is additive}
We wish to show that $\mrm{I}(f+g)=\mrm{I}(f)+\mrm{I}(g)$.  If either $\mrm{I}(f)$ or $\mrm{I}(g)$ is infinite, then this is trivially satisfied as this equation then reads $\infty =\infty$.  Thus, without loss of generality, we may assume that $\mrm{I}(f),\mrm{I}(g)<\infty$.

For $f\colon X\rightarrow [0,\infty ]$, define $\tau _f:X\times [0,\infty ]\rightarrow X\times [0,\infty ]$ by
\begin{equation}
\tau _f(\coord{x,y})\coloneqq \coord{x,f(x)+y}.
\end{equation}
This definition was made so that we have
\begin{equation}\label{Pugh3}
\Gamma _{f+g}=\Gamma _f\cup \tau _f(\Gamma _g)
\end{equation}
is a disjoint union.\footnote{This requires the use of the \emph{strict} inequality in the definition of $\Gamma _f$ and $\Gamma _g$.}  Thus, it suffices to show that (i)~$\tau _f(M)$ is measurable if $M$ is and (ii)~that $[\meas \times \meas _{\mrm{L}}](\tau _f(M))=[\meas \times \meas _{\mrm{L}}](M)$ for $M\subseteq X\times [0,\infty ]$ measurable.

First of all note that, while we don't immediately know that $\tau _f(M)$ is measurable for \emph{all} measurable $M$, we do know that
\begin{equation}
\tau _f(\Gamma _g)=\Gamma _{f+g}\setminus \Gamma _f,
\end{equation}
is measurable, because sums of Borel functions are Borel (\cref{exr5.2.167}).

We now show (ii), that is, that $[\meas \times \meas _{\mrm{L}}](\tau _f(M))=[\meas \times \meas _{\mrm{L}}](M)$ for $M\subseteq X\times [0,\infty ]$ measurable.  To show this, we apply \cref{prp5.1.39} (sets in topological measure spaces are measurable iff you can approximate them with open and closed sets---it is thus very important that the product measure is regular and Borel).

We prove this in stages.  First of all, take $M=M_X\times [0,b)$ for $M_X\subseteq X$ measurable.  Then,
\begin{equation}
M=\Gamma _g\text{ for }g\coloneqq b\chi _{M_X},
\end{equation}
and so we have that (by \eqref{Pugh3})
\begin{equation}\label{eqn5.2.198}
\Gamma _f\cup \tau _f(M)=\Gamma _{f+g}=\Gamma _{g+f}=M\cup \tau _g(\Gamma _f).
\end{equation}
Furthermore,
\begin{equation}
\begin{split}
\tau _g(\Gamma _f) & =\tau _g\left( \{ \coord{x,y}\in \Gamma _f:x\in M_X\} \right. \\ & \qquad \left. \cup \{ \coord{x,y}\in \Gamma _f:x\notin M_X\} \right) \\
& =\{ \coord{x,y+b}\in \Gamma _f:x\in M_X\} \\ & \qquad \cup \{ \coord{x,y}\in \Gamma _f:x\notin M_X\} ,
\end{split}
\end{equation}
and so by translation invariance (and the measurability of $M_X$), we have
\begin{equation}
\begin{split}
\MoveEqLeft {}
[\meas \times \meas _{\mrm{L}}](\tau _g(\Gamma _f)) \\
& =[\meas \times \meas _{\mrm{L}}]\left( \{ \coord{x,y+b}\in \Gamma _f:x\in M_X\} \right. \\ & \qquad \left. \cup \{ \coord{x,y}\in \Gamma _f:x\notin M_X\} \right) \\
& =[\meas \times \meas _{\mrm{L}}]\left( \{ \coord{x,y}\in \Gamma _f:x\in M_X\} \right. \\ & \qquad \left. \cup \{ \coord{x,y}\in \Gamma _f:x\notin M_X\} \right) \\
& =[\meas \times \meas _{\mrm{L}}](\Gamma _f),
\end{split}
\end{equation}
and hence (by \eqref{eqn5.2.198} and the fact that these sets (in particular, $\tau _f(M)$) are measurable)
\begin{equation}
\begin{split}
\MoveEqLeft {}
[\meas \times \meas _{\mrm{L}}](\Gamma _f)+[\meas \times \meas _{\mrm{L}}](\tau _f(M)) \\
& =[\meas \times \meas _{\mrm{L}}](M)+[\meas \times \meas _{\mrm{L}}](\tau _g(\Gamma _f)) \\
& =[\meas \times \meas _{\mrm{L}}](M)+[\meas \times \meas _{\mrm{L}}](\Gamma _f),
\end{split}
\end{equation}
which yields (because $[\meas \times \meas _{\mrm{L}}](\Gamma _f)<\infty$ by assumption)
\begin{equation}
[\meas \times \meas _{\mrm{L}}](\tau _f(M))=[\meas \times \meas _{\mrm{L}}](M).
\end{equation}

Now take $M=M_X\times [a,b)$.  Then,
\begin{equation}
M=\Gamma _g\setminus \Gamma _h\text{ for }g\coloneqq b\chi _{M_X}\text{ and }h\coloneqq a\chi _{M_X}.
\end{equation}
Now the fact that $[\meas \times \meas _{\mrm{L}}](\tau _f(M))=[\meas \times \meas _{\mrm{L}}](M)$ follows from the previous case:
\begin{equation}
\begin{split}
\MoveEqLeft \relax
[\meas \times \meas _{\mrm{L}}](\tau _f(M)) \\
& =[\meas \times \meas _{\mrm{L}}](\tau _f(\Gamma _g))-[\meas \times \meas _{\mrm{L}}](\tau _f(\Gamma _h)) \\
& =[\meas \times \meas _{\mrm{L}}](\Gamma _g)-[\meas \times \meas _{\mrm{L}}](\Gamma _h) \\
& =[\meas \times \meas _{\mrm{L}}](\Gamma _g\setminus \Gamma _h)=[\meas \times \meas _{\mrm{L}}](M).
\end{split}
\end{equation}

For the general case, let $M\subseteq X\times [0,\infty ]$ be arbitrary measurable.  Write $X=\bigcup _{m\in \N}R_m$ as the disjoint union of measurable sets of finite measure (this is an application of \cref{SigmaFinite}), and define $M_{m,n}\coloneqq M\cap (R_m\times [n,n+1))$ (as well as $M_{m,\infty}\coloneqq M\cap (R_m\times \{ \infty \} )$ if you really like, though this will not matter as everything here has measure $0$).  Then, $M$ is the disjoint union of the $M_{m,n}$s, and so it suffices to prove the result for subsets of $R_m\times [n,n+1)$.

So, now, let us change notation, let $M_X\subseteq X$ be measurable of finite measure and take $M$ to be a measurable subset of $M_X\times [n,n+1)$.  Let $\varepsilon >0$.  By `rectangle outer-regularity' (see \eqref{RectangleOuterRegularity}), there is a countable cover of $M$ by open rectangles $\bigcup _{m\in \N}U_{1,m}\times U_{2,m}$ such that
\begin{equation}
\sum _{m\in \N}\meas (U_{1,m})\meas _{\mrm{L}}(U_{2,m})-\varepsilon <[\meas \times \meas _{\mrm{L}}](M).
\end{equation}
Without loss of generality,\footnote{Because every open set can be written as the countable disjoint union of closed-open intervals---see \cref{exr5.2.77}.} we may assume that each $U_{2,m}=[a_m,b_m)$, in which case the previous case applies, so that
\begin{equation*}
\begin{split}
[\meas \times \meas _{\mrm{L}}](\tau _f(M))& \leq \sum _{m\in \N}[\meas \times \meas _{\mrm{L}}](\tau _f(U_{1,m}\times U_{2,m})) \\
& =\sum _{m\in \N}[\meas \times \meas _{\mrm{L}}](U_{1,m}\times U_{2,m}) \\
& <[\meas \times \meas _{\mrm{L}}](M)+\varepsilon ,
\end{split}
\end{equation*}
and so as $\varepsilon >0$ is arbitrary
\begin{equation}\label{eqn5.2.206}
[\meas \times \meas _{\mrm{L}}](\tau _f(M))\leq [\meas \times \meas _{\mrm{L}}](M).
\end{equation}
Similarly,
\begin{equation}
\begin{multlined}
[\meas \times \meas _{\mrm{L}}]\left( \tau _f\left( (M_X\times [n,n+1))\setminus M\right) \right) \\ \leq [\meas \times \meas _{\mrm{L}}]((M_X\times [n,n+1))\setminus M).
\end{multlined}
\end{equation}
Hence,
{\small
\begin{equation*}
\begin{split}
\MoveEqLeft \relax
[\meas \times \meas _{\mrm{L}}](\tau _f(M))+[\meas \times \meas _{\mrm{L}}]\left( \tau _f\left( (M_X\times [n,n+1))\setminus M\right) \right) \\
& \leq [\meas \times \meas _{\mrm{L}}](M)+[\meas \times \meas _{\mrm{L}}]((M_X\times [n,n+1))\setminus M) \\
& =[\meas \times \meas _{\mrm{L}}](M_X\times [n,n+1)) \\
& =[\meas \times \meas _{\mrm{L}}](\tau _f(M_X\times [n,n+1))) \\
& \leq [\meas \times \meas _{\mrm{L}}](\tau _f(M)) \\ & \qquad +[\meas \times \meas _{\mrm{L}}]\left( \tau _f(M_X\times [n,n+1))\setminus \tau _f(M)\right) \\
& =[\meas \times \meas _{\mrm{L}}](\tau _f(M)) \\ & \qquad +[\meas \times \meas _{\mrm{L}}]\left( \tau _f\left( (M_X\times [n,n+1))\setminus M\right) \right) .
\end{split}
\end{equation*}
}
Thus, all of the inequalities must be inequalities, which in particular implies that
\begin{equation}
\begin{multlined}
[\meas \times \meas _{\mrm{L}}](\tau _f(M_X\times [n,n+1))) \\ =[\meas \times \meas _{\mrm{L}}](\tau _f(M)) \\ +[\meas \times \meas _{\mrm{L}}]\left( \tau _f(M_X\times [n,n+1))\setminus \tau _f(M)\right) .
\end{multlined}
\end{equation}
\begin{exr}[breakable=false]{}{}
Use this to conclude that $\tau _f(M)$ is measurable.
\begin{rmk}
Hint:  See \cite[Theorem 6.4.13]{Pugh}.
\end{rmk}
\end{exr}
Hence,
{\scriptsize
\begin{equation*}
\begin{split}
0 & =[\meas \times \meas _{\mrm{L}}](M_X\times [n,n+1))-[\meas \times \meas _{\mrm{L}}](\tau _f(M_X\times [n,n+1))) \\
& =\left( [\meas \times \meas _{\mrm{L}}](M)+[\meas \times \meas _{\mrm{L}}]((M_X\times [n,n+1))\setminus M)\right) \\
& \qquad -\left( [\meas \times \meas _{\mrm{L}}](\tau _f(M))+[\meas \times \meas _{\mrm{L}}](\tau _f(M_X\times [n,n+1))\setminus \tau _f(M))\right) \\
& =\left( [\meas \times \meas _{\mrm{L}}](M)-[\meas \times \meas _{\mrm{L}}](\tau _f(M))\right) \\
& \qquad +\left( [\meas \times \meas _{\mrm{L}}]((M_X\times [n,n+1))\setminus M) \right. \\ & \qquad \left. -[\meas \times \meas _{\mrm{L}}]\left( \tau _f\left( (M_X\times [n,n+1))\setminus M\right) \right) \right) .
\end{split}
\end{equation*}
}
We already showed in \eqref{eqn5.2.206} that $[\meas \times \meas _{\mrm{L}}](M)\leq [\meas \times \meas _{\mrm{L}}](\tau _f(M))$ (and similarly for the complement), so that both of these terms are nonnegative, and hence $0$.  This gives us $[\meas \times \meas _{\mrm{L}}](\tau _f(M))=[\meas \times \meas _{\mrm{L}}](M)$, as desired.

\Step{Show that $\mrm{I}$ is nonnegative homogeneous}
Note that
\begin{equation}
\begin{split}
\Gamma _{af} & \coloneqq \left\{ \coord{x,y}\in X\times \R :y<af(x)\right\} \\
& =\left\{ \coord{x,ay}\in X\times \R :y<f(x)\right\} =a\Gamma _f.
\end{split}
\end{equation}
Then, it follows from \cref{exr5.2.28} that
\begin{equation}
\mrm{I}(af)\coloneqq \meas (\Gamma _{af})=\meas (a\Gamma _f)=a\meas (\Gamma _f)\eqqc a\mrm{I}(f).
\end{equation}

\Step{Prove Lebesgue's Monotone Convergence Theorem}
Let $m\mapsto f_{m}$ be a nondecreasing sequence of functions (it is implicit that the it is nondecreasing \emph{almost-everywhere}).  Let us define $f_\infty :X\rightarrow [0,\infty ]$ by $f_\infty (x)\coloneqq \lim _mf(x)$.  For almost every $x\in X$, the sequence $m\mapsto f_m(x)$ is nondecreasing, and so by the usual \nameref{MonotoneConvergenceTheorem}, this limit $f_{\infty}(x)$ exists in $[0,\infty ]$\footnote{If the sequence is bounded, the \nameref{MonotoneConvergenceTheorem} tells us it converges in $\R$.  Otherwise, it converges to $\infty$.} for almost-every $x$.  For all the points where we do not have monotonicity, we may without loss of generality take $f_\infty$ to be infinite there.  

We wish to show that
\begin{equation}
\lim _m\mrm{I}(f_m)=\mrm{I}(f_\infty ).
\end{equation}
However, note that
\begin{equation}
\Gamma _{f_m}\subseteq \Gamma _{f_{m+1}}\text{ and }\bigcup _{m\in \Z ^+}\Gamma _{f_m}=\Gamma _{f_\infty}
\end{equation}
for all $m$, and hence by \cref{exr5.1.27} (``continuity from below'')
\begin{equation}
\begin{split}
\lim _m\mrm{I}(f_m) & \eqqc \lim _m[\meas \times \meas _{\mrm{L}}](\Gamma _{f_m}) \\
& =\lim _m[\meas \times \meas _{\mrm{L}}]\bigg( \bigcup _{k=1}^m\Gamma _{f_k}\bigg) \\
& =[\meas \times \meas _{\mrm{L}}]\bigg( \bigcup _{m\in \Z ^+}\Gamma _{f_m}\bigg) \\
& =[\meas \times \meas _{\mrm{L}}](\Gamma _{f_{\infty}})\eqqc \mrm{I}(f_{\infty}).\footnote{I find it fascinating how relatively trivial this is compared to additivity, despite this having a name attached to it and additivity practically taken for granted.  On the contrary:  Lebesgue's Monotone Convergence Theorem requires just basic theory of measures---additivity on the other hand requires a reasonably well-developed theory of topological measure spaces!}
\end{split}
\end{equation}

\Step{Show uniqueness}
Now, let $\mrm{I}:\Bor _0^+(X)\rightarrow [0,\infty ]$ be an additive nonnegative-homogeneous map that satisfies Lebesgue's Monotone Convergence Theorem and $\mrm{I}(\chi _S)=\meas (S)$.  It follows immediately that they agree on Borel simple functions.  However, we already know (\cref{SimpleFunctionApproximation}) that \emph{any} Borel function can be written as the limit monotone limit of a sequence of simple Borel functions, so that, by Lebesgue's Monotone Convergence Theorem, we actually have equality of $\mrm{I}$ with the integral on \emph{all} Borel functions.
\end{proof}
\end{thm}
\begin{exr}{}{}
Let $\coord{X,\meas}$ be a topological measure space, let $f\colon X\rightarrow [-\infty ,\infty]$ be integrable and Borel, and let $a\in [-\infty ,\infty ]$.  Show that
\begin{equation}
\int _X\dif \meas (x)\, af(x)=a\int _X\dif \meas (x)\, f(x).
\end{equation}
\begin{rmk}
Of course, this is just the same as the ``Nonnegative-homogeneity'' listed above, except without the requirement that $a$ be nonnegative.
\end{rmk}
\end{exr}

We now turn to the issue mentioned in a remark of the definition of the integral (\cref{Integral}) regarding the extension of the definition of the integral to functions whose integral `should' exist, but doesn't.
\begin{thm}{}{prp5.2.229}
Let $\coord{X,\meas}$ be a topological measure space, let $f\colon X\rightarrow [-\infty ,\infty ]$ be Borel, and let $\collection{K}$ denote the collection of quasicompact subsets of $X$.  Then, if $\restr{f}{K}$ is $\infty$-integrable for all $K\in \collection{K}$, then $f$ is $\infty$-integrable iff $\lim _{K\in \collection{K}}\int _K\dif \meas (x)\, f(x)$ exists, and furthermore, in this case, we have
\begin{equation}
\int _X\dif \meas (x)\, f(x)=\lim _{K\in \collection{K}}\int _K\dif \meas (x)\, f(x).
\end{equation}
\begin{rmk}
The condition that $\restr{f}{K}$ be $\infty$-integrable for all quasicompact $K\subseteq X$ is imposed just so that the terms appearing in the limit $\int _K\dif \meas (x)\, f(x)$ make sense.  It is the other hypothesis, that $\int _X\dif \meas (x)\, f_+(x)=\infty =\int _X\dif \meas (x)\, f_-(x)$, that is the significant one.
\end{rmk}
\begin{rmk}
Incidentally, in earlier versions of these notes, I called this condition\footnote{That is, the condition that $\restr{f}{K}$ be $\infty$-integrable for all quasicompact $K\subseteq X$.} ``quasicompactly $\infty$-integrable'', though as I don't really use this condition anywhere else in the current version of the notes, it didn't make sense to give it a name.

In the non-$\infty$ case this is just the condition that the integral of $f$ over every quasicompact set be finite.  One significance of this condition is that, while even something like the identity function on $\R$ need not be integrable, \emph{every} continuous function is necessarily quasicompactly integrable (by the \namerefpcref{ExtremeValueTheorem} and the fact that quasicompact sets have finite measure).  This is more often called \term{locally integrable}\index{Locally integrable}, but I find this to be misleading terminology---to mean, ``locally integrable'' \emph{should} mean ``Every point has a neighborhood base consisting of measurable sets on which the function is integrable.''.
\end{rmk}
\begin{rmk}
If at least one of $\int _X\dif \meas (x)\, f_{\pm}(x)$ is finite, then the integral of $f$ is just $\int _X\dif \meas (x)\, f_+(x)-\int _X\dif \meas (x)\, f_-(x)$, which \emph{does} make sense, that is, there is no reason to extend the definition of the integral to take into account such functions.  The only functions which require any extra work at all are those which have $\int _X\dif \meas (x)\, f_{\pm}(x)=\infty$.  However, this result shows that, in this case, $\lim _{K\in \collection{K}}\int _K\dif \meas (x)\, f(x)$ just doesn't make sense, and so you cannot use this to extend the definition of the integral for functions like $x\mapsto \frac{\sin (x)}{x}$.
\end{rmk}
\begin{proof}
Suppose that $\restr{f}{K}$ is $\infty$-integrable for all $K\in \collection{K}$.

\blankline
\noindent
$(\Rightarrow )$ Suppose that $f$ is $\infty$-integrable.

Let us reduce it to the case where $f$ is nonnegative.  So, suppose we have proven the result for $f$ nonnegative, Borel, and $\infty$-integrable.  Then, for $f$ arbitrary Borel and $\infty$-integrable, $f_+$ and $f_-$ are nonnegative Borel, and so we have
\begin{equation*}
\begin{split}
\MoveEqLeft
\int _X\dif \meas (x)\, f(x)=\int _X\dif \meas (x)\, f_+(x)-\int _X\dif \meas (x)\, f_-(x) \\
& =\lim _{K\in \collection{K}}\int _K\dif \meas (x)\, f_+(x)-\lim _{K\in \collection{K}}\int _K\dif \meas (x)\, f_-(x) \\
& =\lim _{K\in \collection{K}}\int _K\dif \meas (x)\, f(x),
\end{split}
\end{equation*}
as desired.

It thus remains to prove the result in the nonnegative case.  So, let $f\colon X\rightarrow [-\infty ,\infty ]$ be nonnegative and Borel.  We wish to show that
\begin{equation}
\int _X\dif \meas (x)\, f(x)=\lim _{K\in \collection{K}}\int _K\dif \meas (x)\, f(x).
\end{equation}
To show this, we show that every cofinal subnet $\lambda \mapsto \int _{K_{\lambda}}\dif \meas (x)\, f(x)$ has in turn a subnet that converges to $\int _X\dif \meas (x)\, f(x)$.  So, let $\lambda \mapsto \int _{K_{\lambda}}\dif \meas (x)\, f(x)$ be a cofinal subnet of $K\mapsto \int _X\dif \meas (x)\, f(x)$.  Write $X=\bigcup_{m\in \N}K_m$ with $L_m\subseteq X$ compact and $L_m\subseteq L_{m+1}$.\footnote{We can do this because, writing $X=\bigcup _{m\in \N}L_m'$, if we replace $L_m'$ with $K_m\ceqq \bigcup _{k=0}^mL_k'$, we can ensure that this is a nondecreasing union.}  As the subnet is strict, the `indices' $\{ K_{\lambda}:\lambda \}$ are cofinal in $\collection{K}$, which in particular implies that, for every $m\in \N$, there is some $K_{\lambda _m}\supseteq K_m$.  Then, the sequence $m\mapsto \chi _{K_{\lambda _m}}f$ is nondecreasing and converges to $f$, and so by Lebesgue's Monotone Convergence Theorem, we have that
\begin{equation}
\lim _m\int _{K_{\lambda _m}}\dif \meas (x)\, f(x)=\int _X,
\end{equation}
as desired.

\blankline
\noindent
$(\Leftarrow )$ Suppose that $\lim _{K\in \collection{K}}\int _K\dif \meas (x)\, f(x)$ exists.  We proceed by contradiction:  suppose that $f$ is not $\infty$-integrable.  By definition, this means that $\int _X\dif \meas (x)\, f_+(x)=\infty =\int _X\dif \meas (x)\, f_-(x)$.

Write $X=\bigcup _{m\in M}F_m$ as a countable union of disjoint measurable sets of finite measure.  Define $P\ceqq f^{-1}((0,\infty ])$ and $N\ceqq f^{-1}([-\infty ,0))$, as well as $P_m\ceqq P\cap F_m$ and $N_m\ceqq N\cap F_m$.  Each $P_m$ and $N_m$, as well as $P$ and $N$ themselves, are measurable because $f$ is Borel.

Also note that
\begin{equation}
\sum _{m\in \N}\int _{P_m}\dif \meas (x)\, f_+(x)=\footnote{This is an implicit application of Lebesgue's Monotone Convergence Theorem.}\int _P\dif \meas (x)\, f_+(x)=\infty .
\end{equation}
Similarly,
\begin{equation}
\sum _{m\in \N}\int _{N_m}\dif \meas (x)\, f_-(x)=\infty .
\end{equation}
Define
\begin{equation}
a_m\ceqq \int _{P_m}\dif \meas (x)\, f_+(x)-\int _{N_m}\dif \meas (x)\, f_-(x).
\end{equation}
It follows from \cref{thm2.4.160} that there is some rearrangement which converges to any real number we like (as well as $\pm \infty$).  Given any such rearrangement, we will find nondecreasing collection of compact sets $K_0\subseteq K_1\subseteq \cdots$ with $X=\bigcup _{m\in \N}K_m$ that has the property that $m\mapsto \int _{K_m}\dif \meas (x)\, f(x)$ converges to the value of this rearrangement.  This will in particular imply that $\collection{K}\ni K\mapsto \int _K\dif \meas (x)\, f(x)$ has subnets converging to distinct values, whence it follows that the net $\collection{K}\ni K\mapsto \int _K\dif \meas (x)\, f(x)$ itself cannot converge.  This will give us our contradiction, thereby completing the proof.

So, let $\alpha \in \R$, and pick some rearrangement which converges to $\alpha$.  Thus, after reindexing if necessary, we have that $\sum _{m\in \N}a_m=\alpha$.  As just explained, we seek to find a nondecreasing collection of compact sets $K_0\subseteq K_1\subseteq \cdots$ with $X=\bigcup _{m\in \N}K_m$ and $\lim _m\int _{K_m}\dif \meas (x)\, f(x)=\alpha$. 

By \cref{exr5.1.84}, we can write
\begin{equation}
P_m=\bigcup _{n\in \N}K_{m,n}\cup Y_m\text{ and }N_m=\bigcup _{n\in \N}L_{m,n}\cup Z_m,
\end{equation}
where each $K_{m,0}\subseteq K_{m,1}\subseteq \cdots$ and $L_{m,0}\subseteq L_{m,1}\subseteq \cdots$ is a nondecreasing countable collection of compact sets, and each $Y_m$ and $Z_m$ has measure $0$.  By Lebesgue's Monotone Convergence Theorem (and the fact that each $Y_m$ and $Z_m$ has measure $0$), we have
\begin{equation}
\int _{P_m}\dif \meas (x)\, f_+(x)=\lim _n\int _{K_{m,n}}\dif \meas (x)\, f_+(x)
\end{equation}
and
\begin{equation}
\int _{N_m}\dif \meas (x)\, f_-(x)=\lim _n\int _{L_{m,n}}\dif \meas (x)\, f_-(x).
\end{equation}

Let $\varepsilon >0$.  For $m\in \N$, let $n_m\in \N$ be sufficiently large so that
\begin{equation}
\abs*{\int _{P_k}\dif \meas (x)\, f_+(x)-\int _{K_{k,n_m}}\dif \meas (x)\, f_+(x)}<\tfrac{\varepsilon}{2^m}
\end{equation}
and
\begin{equation}
\abs*{\int _{N_k}\dif \meas (x)\, f_-(x)-\int _{L_{k,n_m}}\dif \meas (x)\, f_-(x)}<\tfrac{\varepsilon}{2^m}
\end{equation}
for all $0\leq k\leq m$, and define
\begin{equation}
M_m\ceqq \bigcup _{k=0}^mK_{k,n_m}\cup \bigcup _{k=0}^mL_{k,n_m}.
\end{equation}
Certainly $M_0\subseteq M_1\subseteq \cdots$ is a nondecreasing countable collection of compact sets with $X=\bigcup _{m\in \N}M_m$.  It thus only remains to show that $\lim _m\int _{M_m}\dif \meas (x)\, f(x)=\alpha$.

So, let $\varepsilon >0$, and choose $m_0\in \N$ such that, whenever $m\geq m_0$, it follows that $\abs*{\sum _{k=0}^ma_k-\alpha}<\varepsilon$.  Then, whenever $m\geq m_0$, it follows that
\begin{equation*}
\begin{split}
\MoveEqLeft
\abs*{\int _{M_m}\dif \meas (x)\, f(x)-\alpha}=\footnote{Note that $\{ K_{k,n_m}:0\leq k\leq m\}$ is a disjoint collection because $\{ P_k:0\leq k\leq m\}$ is.  Similarly for $\{ L_{k,n_m}:0\leq k\leq m\}$.  Also note that $f_-$ vanishes on each $K_{k,n_m}$, and likewise $f_+$ vanishes on each $L_{k,n_m}$.}\left| \sum _{k=0}^m\left[ \int _{K_{k,n_m}}\dif \meas (x)\, f_+(x)\right. \right. \\ & \qquad \qquad \left. \left. -\int _{L_{k,n_m}}\dif \meas (x)\, f_-(x)\right] -\alpha \right| \\
& \leq \sum _{k=0}^m\abs*{\int _{K_{k,n_m}}\dif \meas (x)\, f_+(x)-\int _{P_k}\dif \meas (x)\, f_+(x)} \\ & \qquad +\sum _{k=0}^m\abs*{\int _{N_k}\dif \meas (x)\, f_-(x)-\int _{L_{k,n_m}}\dif \meas (x)\, f_-(x)} \\ & \qquad +\abs*{\sum _{k=0}^m\left[ \int _{P_k}\dif \meas (x)\, f_+(x)-\int _{N_k}\dif \meas (x)\, f_-(x)\right] -\alpha} \\
& <\sum _{k=0}^m\tfrac{\varepsilon}{2^m}+\sum _{k=0}^m\tfrac{\varepsilon}{2^m}+\abs*{\sum _{k=0}^ma_k-\alpha}\leq 2\varepsilon +2\varepsilon +\varepsilon =5\varepsilon .
\end{split}
\end{equation*}
Thus, $m\mapsto \int _{M_m}\dif \meas (x)\, f(x)$ converges to $\alpha$, as desired.
\end{proof}
\end{thm}

One crucial fact is that the integral `doesn't care' about sets of measure $0$.
\begin{prp}{}{prp5.2.227}
Let $\coord{X,\meas}$ be a topological measure space and let $f,g\colon X\rightarrow [-\infty ,\infty ]$ be $\infty$-integrable and Borel.  Then, if $f(x)=g(x)$ almost-everywhere, then $\int _X\dif \meas (x)\, f(x)=\int _X\dif \meas (x)\, g(x)$.
\begin{rmk}
In particular, if $f(x)=0$ almost-everywhere, then $\int _X\dif \meas (x)\, f(x)=0$.
\end{rmk}
\begin{rmk}
For example, the \emph{integral of the \nameref{DirichletFunction} is $0$}!\footnote{Another way to see this is from the fact that $\int _X\dif \meas (x)\, \chi _S(x)=\meas (S)$ for $S$ measurable and the fact that $\meas _{\mrm{L}}(\Q )=0$ because $\Q$ is countable---see \cref{exr5.2.33}.}
\end{rmk}
\begin{rmk}
Note that, by \cref{exr5.2.139}, if $f$ is Borel then so is $g$ (and vice versa), so in fact you only need to assume a priori that one of $f$ and $g$ is Borel.
\end{rmk}
\begin{proof}
\Step{Reduce to the case in which $f$ and $g$ are nonnegative}
Suppose we have proven the result when $f$ and $g$ are nonnegative.  If $f(x)=g(x)$ almost-everywhere, then $f_{\pm}(x)=g_{\pm}(x)$ almost-everywhere, and so
\begin{equation}
\begin{split}
\MoveEqLeft
\int _X\dif \meas (x)\, f(x) \\
& =\int _X\dif \meas (x)\, f_+(x)-\int _X\dif \meas (x)\, f_-(x) \\
& =\int _X\dif \meas (x)\, g_+(x)-\int _X\dif \meas (x)\, g_-(x) \\
& =\int _X\dif \meas (x)\, g(x).
\end{split}
\end{equation}

\Step{Prove the case where $f$ and $g$ are nonnegative}
Suppose that $f$ and $g$ are nonnegative, $\infty$-integrable, and Borel.  Then, by definition
\begin{equation}
\int _X\dif \meas (x)\, f(x)=[\meas \times \meas _{\mrm{L}}](\Gamma _f),
\end{equation}
where
\begin{equation}
\Gamma _f\ceqq \left\{ \coord{x,y}\in X\times [0,\infty ]:0\leq y<f(x)\right\} ,
\end{equation}
and similarly for $g$.

Define
\begin{equation}
S\coloneqq \{ x\in X:f(x)\neq g(x)\} .
\end{equation}
By hypothesis, we have that $\meas (S)=0$.  We also have that
{\small
\begin{equation*}
\begin{split}
\Gamma _f\cap \Gamma _g^{\comp} & =\{ \coord{x,y}\in X\times [0,\infty ]:y<f(x)\text{ and }y\geq g(x)\} \\
& \subseteq S\times [0,\infty ],
\end{split}
\end{equation*}
}
and so
\begin{equation}
\begin{split}
[\meas \times \meas _{\mrm{L}}](\Gamma _f\cap \Gamma _g^{\comp}) & \leq [\meas \times \meas _{\mrm{L}}](S\times [0,\infty ]) \\
& =0\cdot \infty \coloneqq 0.
\end{split}
\end{equation}
Hence,
\begin{equation*}
\begin{split}
[\meas \times \meas _{\mrm{L}}](\Gamma _f) & =\footnote{Because $\Gamma _g$ is measurable, which is precisely the definition of a borel function.}[\meas \times \meas _{\mrm{L}}](\Gamma _f\cap \Gamma _g)+[\meas \times \meas _{\mrm{L}}](\Gamma _f\cap \Gamma _g^{\comp}) \\
& =[\meas \times \meas _{\mrm{L}}](\Gamma _f\cap \Gamma _g).
\end{split}
\end{equation*}
By $f\leftrightarrow g$ symmetry, we likewise have that $[\meas \times \meas _{\mrm{L}}](\Gamma _g)=[\meas \times \meas _{\mrm{L}}](\Gamma _f\cap \Gamma _g)$.  Combining this with the previous equality gives us that $\int _X\dif \meas (x)\, f(x)=\int _X\dif \meas (x)\, g(x)$, as desired.
\end{proof}
\end{prp}
\begin{exr}{}{}
What happens if we don't assume $f$ and $g$ to be Borel?
\end{exr}

\subsection{Properties of the integral}

So obviously we have already discussed some properties of the integral, and admittedly there isn't really any hard and fast rules used to distinguish properties that came in the previous section \nameref{TheIntegralItself} and this one, but loosely speaking, results in the previous section are fundamental enough that they conceivably could have been integrated (no pun intended) into the definition itself (for example, you could have defined the integral as a function on $\sim _{\AlE}$-equivalence classes of functions, which would have been well-defined by \cref{prp5.2.227}).  In any case, this distinction doesn't really matter:  they're all true, and you should know them.

\begin{prp}{Triangle Inequality}{TriangleInequality}\index{Triangle Inequality (integral)}
Let $\coord{X,\meas}$ be a topological measure space and let $f\colon X\rightarrow [-\infty ,\infty ]$ be $\infty$-integrable.  Then,
\begin{equation}
\left| \int _X\dif \meas (x)\, f(x)\right| \leq \int _X\dif \meas (x)\, \left| f(x)\right| .
\end{equation}
\begin{proof}
Using the fact that $f=f_+-f_-$ and $\abs{f}=f_++f_-$, we have
\begin{equation*}
\begin{split}
\left| \int _X\dif \meas (x)\, f(x)\right| & =\footnote{This requires that $f$ be borel.}\left| \int _X\dif \meas (x)\, f_+(x)-\int _X\dif \meas (x)\, f_-(x)\right| \\
& \leq \int _X\dif \meas (x)\, f_+(x)+\int _X\dif \meas (x)\, f_-(x) \\
& =\int _X\dif \meas (x)\, \left| f(x)\right| .
\end{split}
\end{equation*}
\end{proof}
\end{prp}

\begin{exr}{Fatou's Lemma}{FatousLemma}\index{Fatou's Lemma}
Let $\coord{X,\meas}$ be a topological measure space and let $m\mapsto f_{m}\in \Bor _0^+(X)$ be a sequence.  Show that
\begin{equation*}
\begin{multlined}
\int _X\dif \meas (x)\, \liminf _{m}f_{m}(x)\leq \liminf _{m}\int _X\dif \meas (x)\, f_{m}(x) \\ \leq \limsup _{m}\int _X\dif \meas (x)\, f_{m}(x)\leq \int _X\dif \meas (x)\, \limsup _{m}f_{m}(x).
\end{multlined}
\end{equation*}
\end{exr}
\begin{exr}{}{}
Find an example of a topological measure space $\coord{X,\meas}$ and a sequence of nonnegative Borel functions $m\mapsto f_{m}\in \Bor _0^+(X)$ for which the inequalities in \nameref{FatousLemma} are strict.
\end{exr}

One of the most important results regarding the integral is the \term{Dominated Convergence Theorem}.
\begin{thm}{Dominated Convergence Theorem}{DominatedConvergenceTheorem}
Let $\coord{X,\meas}$ be a topological measure space and let $m\mapsto f_m\in \Bor (X)$ be a sequence converging to $f_\infty \in \Mor _{\Set}(X,[-\infty ,\infty ])/\sim _{\AlE}$.\footnote{Of course, $f_{\infty}$ is likewise Borel by \cref{LimitBorelIsBorel}.}  Then, if there is some integrable $g\in \Mor _{\Set}(X,[0,\infty ])$ such that eventually $\abs{f_m}\leq g$, then $f_\infty$ is integrable and
\begin{equation}
\lim _\lambda \int _X\dif \meas (x)\, \left| f_\lambda (x)-f_\infty (x)\right| =0.
\end{equation}
\begin{rmk}
In particular, by the \nameref{TriangleInequality}, it follows that
\intomargin
\begin{equation}
\lim _\lambda \int _X\dif \meas (x)\, f_\lambda (x)=\int _X\dif \meas (x)\, \lim _\lambda f_\lambda (x).
\end{equation}
\end{rmk}
\begin{wrn}
Warning:  It is not enough that $f_\infty$ itself be integrable\footnote{This is part of the conclusion, not part of the hypothesis}---see \cref{MonotoneCounterexample}.  But it's worth than this:  \emph{even if your space has finite measure, and every $f_m$ is integrable, and $f_\infty$ is integrable, this can still fail}---see \cref{exm5.2.236}.
\end{wrn}
\begin{rmk}
It's worth noting that if you want to pull this shit (i.e.\ commute limits with integrals) with the Riemann integral you need \emph{uniform convergence}.  Holy Jesus is this an inconveniently strong condition.
\end{rmk}
\begin{proof}
We leave the proof as a series of exercises.
\begin{exr}{}{}
First of all check that, \emph{if} the result is true, then $f_\infty$ is indeed integrable.
\end{exr}
\begin{exr}{}{}
Reduce the proof to the case where each $f_m\geq 0$.
\end{exr}
\begin{exr}[breakable=false]{}{}
Define $g_m\coloneqq \inf _{k\geq m}f_k$ and $h_m\coloneqq \sup _{k\geq m}f_k$.\footnote{These are all Borel functions by \cref{SupInfBorelIsBorel}.}  Show that
\begin{equation}
\begin{split}
\MoveEqLeft
\lim _m\int _X\dif \meas (x)\, \left| g_m(x)-f_\infty (x)\right| =0 \\
& =\lim _m\int _X\dif \meas (x)\, \left| h_m(x)-f_\infty (x)\right| .
\end{split}
\end{equation}
\end{exr}
\begin{exr}{}{}
Use this to finish the proof.
\end{exr}
\end{proof}
\end{thm}

\begin{exr}{}{Norm}
Let $\coord{X,\meas}$ be a topological measure space and let $f\in \Bor _0^+(X)$.  Show that if $\int _X\dif \meas (x)\, f(x)=0$, then $f(x)=0$ almost-everywhere.
\end{exr}
\begin{exr}{}{exr5.2.210}
Let $\coord{X,\meas}$ be a topological measure space and let $f,g\in \Bor (X)$ be $\infty$-integrable.  Show that if $f\leq g$, then $\int _X\dif \meas (x)\, f(x)\leq \int _X\dif \meas (x)\, g(x)$.
\end{exr}
\begin{prp}{Integral Test}{IntegralTest}\index{Integral Test}
Let $m_0\in \N$ and let $f\colon [m_0,\infty )\rightarrow  [0,\infty )$ be nonincreasing.  Then, $\sum _{m=m_0}^{\infty}f(m)$ converges iff $\int _{m_0}^{\infty}\dif x\, f(x)$ converges.
\begin{proof}
Define $g\colon [m_0,\infty )\rightarrow [0,\infty )$ by $g(x)\coloneqq f(\floor{x})$.  Similarly define $h\colon [m_0,\infty )\rightarrow [0,\infty )$ by $h(x)\coloneqq f(\ceil{x})$.  As $f$ is nonincreasing and $\floor{x}\leq x$, we have that $g(x)\coloneqq f(\floor{x})\geq f(x)$.  Similarly, we have $h(x)\leq f(x)$.  By the previous exercise, this gives
\begin{equation}
\begin{split}
\MoveEqLeft
f(m_0)+\sum _{m=m_0+1}^{\infty}f(m)=\int _{m_0}^{\infty}\dif x\, g(x) \\
& \geq \int _{m_0}^{\infty}\dif x\, f(x)\geq \int _{m_0}^{\infty}\dif x\, h(x) \\
& =\sum _{m=m_0}^{\infty}f(m).
\end{split}
\end{equation}
This implies the desired result.
\end{proof}
\end{prp}

\begin{exm}{A monotone sequence of functions which converges to $0$ whose integrals converge to $\infty$}{MonotoneCounterexample}
This is actually quite easy.  Not even close to being exotic compared to some of the counter-examples we've see.  For example, define $f_m:\R \rightarrow [0,\infty ]$ by $f_m\coloneqq \chi _{[m,\infty )}$.  Then, $\lim _mf_m(x)=0$ for all $x\in \R$, and so
\begin{equation}
\int _{\R}\dif x\, \lim _mf_m(x)=0.
\end{equation}
On the other hand, $\int _{\R}\dif x\, f_m(x)=\infty$ for all $m\in \N$, and so
\begin{equation}
\lim _m\int _{\R}\dif x\, \lim _mf_m(x)=\infty .
\end{equation}
\end{exm}
Fortunately, however, we do have `commutation' of limits with integrals in cases more than just a nondecreasing sequence of functions, because of, for example, the \namerefpcref{DominatedConvergenceTheorem}.  As mentioned in a remark of this theorem, integrability of the limit and each function in the sequence is \emph{not} enough---you really need them \emph{all} to be bounded by a \emph{single} integrable function.
\begin{exm}{A sequence of integrable functions converging to an integrable function on a space of finite measure, but for which the limit of the integral is not the integral of the limit}{exm5.2.236}
For $m\in \N$, define $f_m:[0,1]\rightarrow \R$ by $f_m(x)\coloneqq (m+1)x^m$.  Then, $\lim _mf_m(x)=0$ almost-everywhere, and so we have
\begin{equation}
\int _0^1\dif x\, \lim _mf_m(x)=0.
\end{equation}
On the other hand, we have that $\int _0^1\dif x\, f_m(x)=1$ for all $m\in \N$, and so
\begin{equation}
\lim _m\int _0^1\dif x\, f_m(x)=1.
\end{equation}
\end{exm}

We mentioned awhile back that sums are really just integrals over the counting measure.
\begin{prp}{}{prp5.2.228}
Let $\meas$ denote the counting measure on $\N$ and let $f\in \Bor _0^+(\coord{\N ,\meas})$.  Then,
\begin{equation}
\int _{\N}\dif \meas (m)\, f(m)=\sum _{m\in \N}f(m).
\end{equation}
\begin{rmk}
If you're ever teaching a child how to add two numbers, say $3$ and $5$, don't tell them to integrate the function $f\colon \{ x_1,x_2\} \rightarrow \N$, $f(x_1)=3$ and $f(x_2)=5$, over the two point space equipped with the counting measure.  As elucidating as that may seem, it's circular---we needed to know how to add natural numbers to get to this point.  It's probably best to teach them that it is the cardinality of the disjoint union of the set $\{ 0,1,2\}$ and $\{ 0,1,2,3,4\}$.\footnote{Don't forget to explain that a finite cardinal is an equivalence class of sets, all of which have the property that there is no bijection onto a proper subset, with respect to the equivalence relation that is isomorphism in the category sets.  If you omit this, I fear you might risk confusing them.}
\end{rmk}
\begin{proof}
For $m\in \N$, define
\begin{equation}
f_m\coloneqq \sum _{k=0}^m\chi _{\{ k\}}f(k).
\end{equation}
Note that $m\mapsto f_m$ is nondecreasing and converges to $f$.  Hence, by Lebesgue's Monotone Convergence Theorem,
\begin{equation}
\begin{split}
\MoveEqLeft
\int _{\N}\dif \meas (m)\, f(m) \\
& =\int _{\N}\dif \meas (m)\, \lim _n\sum _{k=0}^n\chi _{\{ k\}}(m)f(k) \\
& =\lim _n\int _{\N}\dif \meas (m)\, \sum _{k=0}^n\chi _{\{ k\}}(m)f(k) \\
& =\footnote{By additivity and nonnegative-homogeneity.}\lim _n\sum _{k=0}^nf(k)\int _{\N}\chi _{\{ k\}}(n) \\
& =\lim _n\sum _{k=0}^nf(k)\meas (\{ k\} )=\lim _n\sum _{k=0}^nf(k) \\
& =\sum _{m\in \N}f(m).
\end{split}
\end{equation}
\end{proof}
\end{prp}

\subsubsection{Fubini's Theorem and its consequences}

We next present Fubini's Theorem, which is absolutely crucial in the calculation of integrals in multivariable calculus.  It essentially reduces integral multivariable calculus to single-variable calculus.
\begin{thm}{Fubini's Theorem}{FubinisTheorem}\index{FubinisTheorem}
Let $\coord{X_1,\meas _1}$ and $\coord{X_2,\meas _2}$ be topological measure spaces, let $M_1\subseteq X_1$ and let $M_2\subseteq X_2$ be measurable, and let $f\colon X_1\times X_2\rightarrow [-\infty ,\infty ]$ be $\infty$-integrable.  Then, if $f$ is Borel or characteristic, then
{\tiny
\begin{equation}\label{eqn5.2.239}
\int _{M_1\times M_2}\dif [\meas _1\times \meas _2](x)\, f(x)=\int _{M_1}\dif \meas _1(x_1)\, \int _{M_2}\dif \meas _2(x_2)\, f(x_1,x_2)
\end{equation}
}
and
{\tiny
\begin{equation}\label{eqn5.2.240}
\int _{M_1\times M_2}\dif [\meas _1\times \meas _2](x)\, f(x)=\int _{M_2}\dif \meas _2(x_2)\, \int _{M_1}\dif \meas _1(x_1)\, f(x_1,x_2)
\end{equation}
}
\begin{rmk}
That is to say, the result holds if either $f$ is Borel or if $f=\chi _S$, for $S$ \emph{not necessarily measurable}.
\end{rmk}
\begin{rmk}
In particular, it is implicit that for every $x_1\in X_1$, $x_2\mapsto f(x_1,x_2)$ is $\infty$-integrable, and that $x_1\mapsto \int _{M_2}\dif \meas _2(x_2)\, f(x_1,x_2)$ is $\infty$-integrable, and similarly for the $1\leftrightarrow 2$ case.  (That is to say, the minimum requirements needed in order to ensure that the right-hand sides of \eqref{eqn5.2.239} and \eqref{eqn5.2.240} are defined are in fact true.)
\end{rmk}
\begin{rmk}
In other words, the ``double integral'' is equal to both ``iterated integrals''.
\end{rmk}
\begin{rmk}
The case where $f$ is nonnegative is sometimes called \term{Tonelli's Theorem}\index{Tonelli's Theorem}, in which case ``Fubini's Theorem'' is probably being used to refer to the case where $f$ is integrable (in the nonnegative case, $f$ would be allowed to be $\infty$-integrable in our terminology).  By stating and proving the result for $f$ $\infty$-integrable, we subsume both special cases (and more) into a single statement.
\end{rmk}
\begin{rmk}
This theorem is one motivation for us defining the integral for \emph{all} ($\infty$-integrable) functions, instead of just the Borel ($\infty$-integrable) functions.  While it \emph{is} true that $x_1\mapsto f(x_1,x_2)$ and $x_2\mapsto f(x_1,x_2)$ are Borel if $f$ is Borel (\cref{prp5.2.281}), we don't know this a priori, and so it is convenient that we can still write down these integrals and have them make sense without first having to prove that the integrands are Borel.
\end{rmk}
\begin{proof}
\Step{Make hypotheses}
Suppose that $f$ is Borel or characteristic.  First, consider the case where $f$ is Borel.  Ultimately, (\cref{stpFubinisTheorem.11}) we reduce this case to proving the case for $f$ characteristic as well, and so in that step we will finish the case $f$ characteristic as well.

\Step{Reduce to the case of \eqref{eqn5.2.239}}
By $1\leftrightarrow 2$ symmetry, it suffices to prove \eqref{eqn5.2.239}.

\Step{Reduce to the case in which $M_1=X_1$ and $M_2=X_2$}
Suppose that we have proven the result in the case $M_1=X_1$ and $M_2=X_2$, and let $S_1\subseteq X_1$ and $M_2\subseteq X_2$ be measurable.  Then, $\chi _{M_1\times M_2}(x_1,x_2)=\chi _{M_1}(x_1)\chi _{M_2}(x_2)$ is Borel, and so we have that.
\begin{equation*}
\begin{split}
\MoveEqLeft
\int _{M_1\times M_2}\dif [\meas _1\times \meas _2](x)\, f(x) \\
& \coloneqq \int _{X_1\times X_2}\dif [\meas _1\times \meas _2](x)\, \chi _{M_1\times M_2}(x)f(x) \\
& =\int _{X_1}\dif \meas _1(x_1)\, \int _{X_2}\dif \meas _2(x_2)\, \chi _{M_1\times M_2}(x_1,x_2)f(x_1,x_2) \\
& =\int _{X_1}\dif \meas _1(x_1)\, \chi _{M_1}(x_1)\int _{X_2}\dif \meas _2(x_2)\, \chi _{M_2}(x_2)f(x_1,x_2) \\
& \eqqcolon \int _{M_1}\dif \meas _1(x_1)\, \int _{M_2}\dif \meas _2(x_2)\, f(x_1,x_2).
\end{split}
\end{equation*}

\Step{Reduce to the case in which $f$ is nonnegative}
Suppose we have proven the result in which $f$ is nonnegative.  If $f$ is $\infty$-integrable, we have
\begin{equation}
\begin{split}
\MoveEqLeft
\int _{X_1\times X_2}\dif [\meas _1\times \meas _2](x)\, f(x) \\
& \coloneqq \int _{X_1\times X_2}\dif [\meas _1\times \meas _2](x)\, f_+(x) \\ & \quad -\int _{X_1\times X_2}\dif [\meas _1\times \meas _2](x\, f_-(x) \\
& =\int _{X_1}\dif \meas _1(x_1)\, \int _{X_2}\dif \meas _2(x_2)\, f_+(x_1,x_2) \\
& \quad -\int _{X_1}\dif \meas _1(x_1)\, \int _{X_2}\dif \meas _2(x_2)\, f_-(x_1,x_2) \\
& \eqqcolon \int _{X_1}\dif \meas _1(x_1)\, \int _{X_2}\dif \meas _2(x_2)\, f(x_1,x_2),
\end{split}
\end{equation}
as desired.

\Step{Reduce to the case in which $f$ is a characteristic function}
Suppose that we have proven the result for characteristic functions.  Then, by \cref{SimpleFunctionApproximation} (every nonnegative Borel function is a monotonic limit of simple functions), Lebesgue's Monotone Convergence Theorem, and linearity of the integral, the result follows for arbitrary $f\in \Bor _0^+(X_1\times X_2)$.

\Step{Define a measure $\meas$ on $X_1\times X_2$}
Define $\meas \colon 2^{X_1\times X_2}\rightarrow [0,\infty ]$ by
\begin{equation}
\meas (S)\coloneqq \int _{X_1}\dif \meas _1(x_1)\, \int _{X_2}\dif \meas _2(x_2)\, \chi _{S_{x_1}}(x_2),
\end{equation}
where
\begin{equation}
S_{x_1}\coloneqq \left\{ x_2\in X_2:\coord{x_1,x_2}\in S\right\} .
\end{equation}
We wish to show that this is a regular Borel measure on $X_1\times X_2$ such that $\meas (K_1\times K_2)=\meas _1(K_1)\meas _2(K_2)$ for $K_i\subseteq X_i$ quasicompact.

\Step{Show that $\meas [K_1\times K_2]=\meas _1(K_1)\meas _2(K_2)$}[stpTonellisTheorem.5]
As
\begin{equation}
[K_1\times K_2]_{x_1}=\begin{cases}K_2 & \text{if }x_1\in K_2 \\ \emptyset & \text{if }x\in K_2^{\comp},\end{cases}
\end{equation}
we have $\chi _{[K_1\times K_2]_{x_1}}(x_2)=\ chi _{K_1}(x_1)\chi _{K_2}(x_2)$, and hence indeed $\meas (K_1\times K_2)\coloneqq \int _{X_1}$.  Thus, it suffices to show that this defines a regular Borel measure.

\Step{Show that $\meas$ is a measure}
It is immediate that $\meas (\emptyset )=0$.

$\meas$ is nondecreasing by \cref{exr5.2.210}.

Let $M_m\subseteq X_1\times X_2$ for $m\in \N$.  Define $S_m\coloneqq \bigcup _{k=0}^mM_k$ for $m\in \N$.  Then
\begin{equation}
\chi _{\bigcup _{m\in \N}M_m}(x)=\lim _n\chi _{S_n}(x)
\end{equation}
for all $x\in X_1\times X_2$, and so, by Lebesgue's Monotone Convergence Theorem, we have that $\meas$ is subadditive.

Thus, $\meas$ is a measure.

\Step{Show that $\meas$ is regular}
We first check outer-regularity.  Let $S\subseteq X_1\times X_2$.  If $\meas (S)=\infty$, then
\begin{equation}
\inf \left\{ \meas (U):S\subseteq U,\ U\text{ open.}\right\} =\infty
\end{equation}
because $\meas$ is nondecreasing, so we may without loss of generality assume that $\meas (S)<\infty$.  Let $\varepsilon >0$.  By the defining theorem of the product measure (\cref{ProductMeasure}),
\begin{equation*}
\begin{split}
\MoveEqLeft
[\meas _1\times \meas _2](S)=\inf \left\{ \sum _{m\in \N}\meas _1(U_{1,m})\meas _2(U_{2,m}):\right. \\ & \qquad \qquad \left. U_{i,m}\subseteq X_i\text{ open, }S\subseteq \bigcup _{m\in \N}U_{1,m}\times U_{2,m}\right\} ,
\end{split}
\end{equation*}
and so there are open $U_{i,m}\subseteq X_i$ with $S\subseteq \bigcup _{m\in \N}U_{1,m}\times U_{2,m}$ such that
\begin{equation}\label{eqn5.2.244}
\begin{split}
[\meas _1\times \meas _2](S)-\varepsilon & <\sum _{m\in \N}\meas _1(U_{1,m})\meas _2(U_{2,m}) \\
& \leq [\meas _1\times \meas _2](S).
\end{split}
\end{equation}
Without loss of generality,\footnote{Why can we do this?} assume that the $U_{1,m}$s and the $U_{2,m}$ are disjoint.  Then,
{\footnotesize
\begin{equation}
\begin{split}
\MoveEqLeft
\meas \bigg( \bigcup _{m\in \N}U_{1,m}\times U_{2,m}\bigg) \\
& \coloneqq \int _{X_1}\dif \meas _1(x_1)\int _{X_2}\dif \meas _2(x_2)\, \chi _{\left[ \bigcup _{m\in \N}U_{1,m}\times U_{2,m}\right] _{x_1}}(x_2) \\
& =\footnote{Because $\left[ \bigcup _{m\in \N}U_{1,m}\times U_{2,m}\right] _{x_1}$ is $\bigcup _{\substack{k\in \N \\ x_1\in U_{1,k}}}U_{2,k}$, the characteristic function of which is, by disjointness, $\sum _{\substack{k\in \N \\ x_1\in U_{1,k}}}\chi _{2,k}$, which itself is equal to $\sum _{m\in \N}\chi _{U_{1,m}}(x_1)\chi _{U_{2,m}}$.}\int _{X_1}\dif \meas _1(x_1)\, \int _{X_2}\dif \meas _2(x_2)\, \sum _{m\in \N}\chi _{U_{1,m}}(x_1)\chi _{U_{2,m}}(x_2) \\
& =\footnote{By Lebesgue's Monotone Convergence Theorem.}\sum _{m\in \N}\int _{X_1}\dif \meas _1(x_1)\, \chi _{U_{1,m}}(x_1)\int _{X_2}\dif \meas _2(x_2)\, \chi _{U_{2,m}}(x_2) \\
& =\sum _{m\in \N}\meas _1(U_{1,m})\meas _2(U_{2,m}).
\end{split}
\end{equation}
}
This, together with \cref{eqn5.2.244} shows that $\meas$ is outer-regular.

A very similar argument using
\begin{equation}\label{eqn5.2.246}
\begin{split}
\MoveEqLeft
[\meas _1\times \meas _2](U) \\
& \coloneqq \sup \left\{ \sum _{k=0}^m\meas _1(K_{1,k})\meas _2(K_{2,k}):\right. \\ & \quad \left. m\in \N ,\ K_{i,k}\subseteq X_i\text{ quasicompact},\right. \\
& \quad \left. \left\{ K_{1,k}\times K_{2,k}:0\leq k\leq m\right\} \text{ is disjoint},\right. \\
& \quad \left. \bigcup _{k=0}^mK_{1,k}\times K_{2,k}\subseteq U\right\}
\end{split}
\end{equation}
for $U\subseteq X_1\times X_2$ open and
\begin{equation}\label{eqn5.2.246x}
\begin{multlined}
[\meas _1\times \meas _2](S) \\ \coloneqq \inf \left\{ [\meas _1\times \meas _2](U):S\subseteq U,\ U\text{ open}\right\} ;
\end{multlined}
\end{equation}
for $S\subseteq X_1\times X_2$ arbitrary, shows that $\meas$ is inner-regular on open sets.

By \cref{stpTonellisTheorem.5}, $\meas$ is finite on quasicompact subsets.

Thus, $\meas$ is regular.

\Step{Show that $\meas$ is Borel}
Let $U\subseteq X_1\times X_2$ be open.  The argument for inner-regularity on open sets using \eqref{eqn5.2.246} and \eqref{eqn5.2.246x} will show that the same equation holds with $\meas _1\times \meas _2$ on the left-hand side replaced with $\meas$.  Thus, modulo a set of measurable $0$ (which is measurable), $U$ is a countable disjoint union of quasicompact rectangles.  Thus, it suffices to show that $\meas (K_1\times K_2)$ is measurable for $K_i\subseteq X_i$ quasicompact.

So, let $S\subseteq X_1\times X_2$ be arbitrary.  We wish to show that
\begin{equation}
\meas (S)\geq \meas (S\cap (K_1\times K_2))+\meas (S\cap (K_1\times K_2)^{\comp}).
\end{equation}
If $\meas (S)=\infty$, we are done, so we may as well assume that $\meas (S)<\infty$.  Let $\varepsilon >0$.  Let $U\subseteq X_1\times X_2$ be open and such that (i)~$S\subseteq U$ and (ii)~$\meas (S)>\meas (U)-\varepsilon$.  Because the space is $T_2$, $K_1\times K_2$ is closed, and so $U\cap (K_1\times K_2)^{\comp}$ is open.  Therefore, there is a disjoint union
\begin{equation}
\bigcup _{k=0}^mL_{1,k}\times L_{2,k}\subseteq U\cap (K_1\times K_2)^{\comp}
\end{equation}
for $L_{i,k}\subseteq X_i$ quasicompact such that
\begin{equation}
\meas (U\cap (K_1\times K_2)^{\comp})-\varepsilon <\sum _{k=0}^m\meas (L_{1,k}\times L_{2,k}).
\end{equation}
Similarly, there is a disjoint union
\begin{equation}
\bigcup _{k=0}^nM_{1,k}\times M_{2,k}\subseteq U\cap \bigg( \bigcup _{k=0}^mL_{1,k}\times L_{2,k}\bigg) ^{\comp}
\end{equation}
for $M_{i,k}\subseteq X_i$ quasicompact such that
\begin{equation}
\begin{multlined}
\meas \bigg( U\cap \bigg( \bigcup _{k=0}^mL_{1,k}\times L_{2,k}\bigg) ^{\comp}\bigg) -\varepsilon \\ <\sum _{k=0}^n[\meas _1\times \meas _2](M_{1,k}\times M_{2,k}).
\end{multlined}
\end{equation}
Hence,
\begin{equation*}
\begin{split}
\meas (S) & >\meas (U)-\varepsilon \\
& \geq \meas \bigg( \bigcup _{k=0}^mL_{1,k}\times L_{2,k}\cup \bigcup _{k=0}^nM_{1,k}\times M_{2,k}\bigg) -\varepsilon \\
& =\footnote{Because $\meas$ agrees with $\meas _1\times \meas _2$ on disjoint unions of quasicompact rectangles, and $\meas _1\times \meas _2$ is additive on quasicompact rectangles.}\sum _{k=0}^m\meas (L_{1,k}\times L_{2,k}) \\ & \qquad +\sum _{k=0}^n\meas (M_{1,k}\times M_{2,k})-\varepsilon \\
& >\meas \left( U\cap (K_1\times K_2)^{\comp}\right) \\ & \qquad +\meas \bigg( U\cap \bigg( \bigcup _{k=0}^mL_{1,k}\times L_{2,k}\bigg) ^{\comp}\bigg) -3\varepsilon \\
& \geq \meas (U\cap (K_1\times K_2)^{\comp}) \\ & \qquad +\meas (U\cap (K_1\times K_2))-3\varepsilon .
\end{split}
\end{equation*}
As $\varepsilon >0$ was arbitrary, we obtain the desired result.

\Step{Deduce the result for $f$ a characteristic function}\label{stpFubinisTheorem.11}
Then, by uniqueness of the product measure, we will have
\begin{equation}
\begin{split}
\MoveEqLeft
\int _{X_1}\dif \meas _1(x_1)\, \int _{X_2}\dif \meas _2(x_2)\, \chi _{S_{x_1}}(x_2) \\
& \eqqcolon \meas (S)=[\meas _1\times \meas _2](S) \\
& =\int _{X_1\times X_2}\dif [\meas _1\times \meas _2](x)\, \chi _S(x),
\end{split}
\end{equation}
as desired.
\end{proof}
\begin{rmk}
While the proof here is quite long and the details are not a particularly high priority the first time through the subject, the technique used here is quite important, and is worth taking note of.  The technique I have in mind is proving the result in steps:  first for characteristic functions of measurable sets, then for nonnegative Borel functions, and finally for Borel $\infty$-integrable functions.
\end{rmk}
\end{thm}
While certainly very useful for computation, \nameref{FubinisTheorem} can usually not be used alone.  To really make use of this result, we must first meet the Fundamental Theorem of Calculus (\cref{FTCI,FTCII})---see \cref{exm6.4.48}.  That said, we can use it to prove one result that you are almost certainly familiar with, though perhaps not by name:  \term{Cavalieri's Principle}.
\begin{crl}{Cavalieri's Principle}{CavalierisPrinciple}\index{Cavalieri's Principle}
Let $\coord{X_1,\meas _1}$ and $\coord{X_2,\meas _2}$ be topological measure spaces and let $S\subseteq X_1\times X_2$.  Then,
{\tiny
\begin{equation}\label{eqn5.2.279}
\int \dif \meas _1(x_1)\, \meas _2(S_{x_1})=[\meas _1\times \meas _2](S)=\int \dif \meas _2(x_2)\, \meas _1(S_{x_2}),
\end{equation}
}
where
\begin{equation}\label{eqn5.2.291}
S_{x_1}\ceqq \left\{ x_2\in X_2:\coord{x_1,x_2}\in S\right\}
\end{equation}
and
\begin{equation}
S_{x_2}\ceqq \left\{ x_1\in X_1:\coord{x_1,x_2}\in S\right\}
\end{equation}
for $x_i\in X_i$.
\begin{rmk}
Cavalieri's Principle is the statement the `volume' of an object can be computed by integrating the `area' of the cross-section as a function of the `height'.  For example, this is most likely what you would use to show that volume of a cone with radius $r$ and height $h$ is $\frac{1}{3}\uppi r^2h$.\footnote{Though god (and perhaps readers who have consulted \cref{Tau}) only knows what this wacky symbol ``$\uppi$'' is.}
\end{rmk}
\begin{rmk}
Cavalieri's Principle is one good reason why we wanted to state \nameref{FubinisTheorem} for $f$ the characteristic function of a not-necessarily-measurable.  The proof of this result essentially amounts to plugging in $f=\chi _S$ into \nameref{FubinisTheorem}, and if we needed $f$ to be Borel, we would need $S$ to be measurable.
\end{rmk}
\begin{proof}
By \nameref{FubinisTheorem}, we have
\begin{equation}
\begin{split}
\MoveEqLeft \relax
[\meas _1\times \meas _2](S) \\
& =\int _{X_1\times X_2}\dif [\meas _1\times \meas _2](x)\, \chi _S(x) \\
& =\int _{X_1}\dif \meas _1(x_1)\, \int _{X_2}\dif \meas _2(x_2)\, \chi _S(x_1,x_2) \\
& =\int _{X_1}\dif \meas _1(x_1)\, \int _{X_2}\dif \meas _2(x_2)\, \chi _{S_{x_1}}(x_2) \\
& =\int _{X_1}\dif \meas _1(x_1)\, \meas _2(S_{x_1}).
\end{split}
\end{equation}
\end{proof}
\end{crl}

\begin{exr}{}{exr5.2.298}
Let $\coord{X_1,\meas _1}$ and $\coord{X_2,\meas _2}$ be topological measure spaces and let $M\subseteq X_1\times X_2$.  Show that if $M$ is measurable that
\begin{equation}
M_{x_1}\text{ and }M_{x_2}
\end{equation}
are measurable for almost-every $x_i\in X_i$, where
\begin{equation}
M_{x_1}\ceqq \left\{ x_2\in X_2:\coord{x_1,x_2}\in M\right\}
\end{equation}
\begin{equation}
M_{x_2}\ceqq \left\{ x_1\in X_1:\coord{x_1,x_2}\in M\right\} .
\end{equation}
\begin{wrn}
Warning:  This is \emph{not} true for \emph{all} $x_1$ and $x_2$, only \emph{almost}-all $x_1$ and $x_2$---see \cref{exr5.2.298}.
\end{wrn}
\end{exr}

This allows us to prove the following well-known result about product measures.
\begin{prp}{}{prp5.2.281}
Let $\coord{X_1,\meas _1}$ and $\coord{X_1,\meas _2}$ be topological measure spaces, and let $f\colon X_1\times X_2\rightarrow [-\infty ,\infty ]$.  Then, if $f$ is Borel and $\infty$-integrable, then
\begin{equation}
X_1\ni x_1\mapsto \int _{X_2}\dif \meas (x_2)\, f(x_1,x_2)\in [-\infty ,\infty]
\end{equation}
and
\begin{equation}
X_2\ni x_2\mapsto \int _{X_1}\dif \meas _1(x_1)\, f(x_1,x_2)\in [-\infty ,\infty ]
\end{equation}
are Borel.
\begin{proof}
\Step{Make hypotheses}
Suppose that $f$ is Borel and $\infty$-integrable.

\Step{Reduce to the case of $x_1\mapsto f(x_1,x_2)$}
By $1\leftrightarrow 2$ symmetry, it suffices to prove that $X\ni x_1\mapsto \int _{X_2}\dif \meas _2(x_2)\, f(x_1,x_2)\in [-\infty ,\infty ]$ is Borel.

\Step{Reduce to the case where $f$ is nonnegative}
Suppose we have proven the result for nonnegative functions and write $f=f_+-f_-$.  Then, $x_1\mapsto \int _{X_2}\dif \meas _2(x_2)\, f_{\pm}(x_1,x_2)$ is Borel, and so the difference of these two functions, namely $x_1\mapsto \int _{X_2}\dif \meas _2(x_2)\, f(x_1,x_2)$ is Borel.

\Step{Reduce to the case where $f=\chi _M$ for $M\subseteq X_1\times X_2$ measurable}
Suppose that we have proven the result for characteristic functions of measurable sets, and let $f\colon X_1\times X_2\rightarrow [0,\infty ]$ be Borel.  By \cref{SimpleFunctionApproximation}, $f$ is the almost-everywhere pointwise limit of a nondecreasing sequence of simple Borel functions.  Hence, as sums and limits of Borel functions are Borel, it follows that the $x_1\mapsto f(x_1,x_2)$ is Borel because this is true for all characteristic functions.

\Step{Reduce to the case where $f=\chi _M$ for $M\subseteq X_1\times X_2$ measurable of finite measure}
Suppose we have proven the result characteristic functions of measurable sets with finite measure, and let $M\subseteq X_1\times X_2$ be measurable (not necessarily of finite measure).  Write $X_i=\bigcup _{m\in \N}K_{i,m}$ for $K_{i,m}\subseteq X_i$ compact and define
\begin{equation}
L_m\ceqq \bigcup _{k=0=l}^mK_{1,k}\times K_{2,l},
\end{equation}
so that each $L_m$ is compact, $L_m\subseteq L_{m+1}$, and $X_1\times X_2=\bigcup _{m\in \N}L_m$.  Now define $M_m\ceqq M\cap L_m$.  It follows that $M=\bigcup _{m\in \N}M_m$, and hence $M_{x_1}=\bigcup _{m\in \N}[M_m]_{x_1}$, and hence $\chi _{M_{x_1}}=\lim _m\chi _{[M_m]_{x_1}}$, and hence $\meas _2(M_{x_1})=\lim _m\meas _2([M_m]_{x_1})$.  As $M_m$ is measurable with finite measure, by assumption, $x_1\mapsto \int _{X_2}\dif \meas (x_2)\, \chi _{[M_m]_{x_1}}(x_2)=\footnote{Note that $[M_m]_{x_1}$ need not be measurable (\cref{exm5.2.153}), and so this is not true for \emph{all} $x_1$.  Instead, it is only true for \emph{almost}-all $x_1$ by the previous exercise, which, fortunately, is all we need by \cref{exr5.2.139} (functions equal to a borel function almost-everywhere are borel).}\meas _2([M_m]_{x_1})$ is Borel, and so $x_1\mapsto \meas _2(M_{x_1})$ is Borel by \cref{LimitBorelIsBorel} (limits of Borel functions are Borel).

\Step{Prove the result for $f=\chi _M$ for $M\subseteq X_1\times X_2$ measurable of finite measure}
Let $M\subseteq X_1\times X_2$ be measurable of finite measure.  Let $\varepsilon >0$, and , using ``rectangle outer regularity'' (\eqref{RectangleOuterRegularity}), choose $U_{i,m}^{\varepsilon}\subseteq X_i$ open such that $M\subseteq \bigcup _{m\in \N}U_{1,m}^{\varepsilon}\times U_{2,m}^{\varepsilon}\eqqc U^{\varepsilon}$ and
\begin{equation}
\sum _{m\in \N}\meas _1(U_{1,m}^{\varepsilon})\meas _2(U_{2,m}^{\varepsilon})-\varepsilon <[\meas _1\times \meas _2](M).
\end{equation}
Using the usual trick,\footnote{The one you weren't supposed to get---see \eqref{5.1.13}.} we may without loss of generality assume that $\{ U_{i,m}^{\varepsilon}:m\in \N \}$ are disjoint for $i=1,2$ (though they will likely not be open anymore).  Thus,
\begin{equation}
U_{x_1}^{\varepsilon}=\begin{cases}U_{2,m}^{\varepsilon} & \text{if }x_1\in U_{1,m}^{\varepsilon} \\ \emptyset & \text{otherwise,}\end{cases}
\end{equation}
and so
\begin{equation}
\meas _2(U_{x_1}^{\varepsilon})=\sum _{m\in \N}\meas _2(U_{2,m}^{\varepsilon})\chi _{U_{1,m}}^{\varepsilon}(x_1),
\end{equation}
and so, being a limit of Borel functions, $x_1\mapsto \meas _2(U_{x_1}^{\varepsilon})$ is Borel.

Now, by \nameref{CavalierisPrinciple}, we have
\begin{equation}
\begin{split}
\varepsilon & >[\meas _1\times \meas _2](M\setminus U^{\varepsilon}) \\
& =\int _{X_1}\dif \meas _1(x_1)\, \abs*{\meas _2(M_{x_1})-\meas _2(U_{x_1}^{\varepsilon})}
\end{split}
\end{equation}
Thus, this integral must vanish as $\varepsilon \to 0^+$, which implies that $\lim _{\varepsilon \to 0^+}\meas _2(U_{x_1}^{\varepsilon})=\meas _2(M_{x_1})$ almost-everywhere (\cref{Norm}), and hence $x_1\mapsto \meas _2(M_{x_1})$ is Borel, once again, because pointwise almost-everywhere limits of Borel functions are Borel.
\end{proof}
\end{prp}
Once again, the special case of $f=\chi _M$ is worth stating separately.
\begin{crl}{}{}
Let $\coord{X_1,\meas _1}$ and $\coord{X_2,\meas _2}$ be topological measure spaces and let $M\subseteq X_1\times X_2$.  Then, if $M$ is measurable, then
\begin{equation}
X_1\ni x_1\mapsto \meas _2(M_{x_1})\in [0,\infty ]
\end{equation}
and
\begin{equation}
X_2\ni x_2\mapsto \meas _1(M_{x_2})\in [0,\infty ]
\end{equation}
are Borel, where
\begin{equation}
M_{x_1}\ceqq \left\{ x_2\in X_2:\coord{x_1,x_2}\in M\right\}
\end{equation}
\begin{equation}
M_{x_2}\ceqq \left\{ x_1\in X_1:\coord{x_1,x_2}\in M\right\} .
\end{equation}
\begin{proof}
Suppose that $M$ is measurable.  By $1\leftrightarrow 2$ symmetry, it suffices to prove that $x_1\mapsto \meas _2(M_{x_1})$ is Borel.  As $M$ is measurable, $\chi _M$ is Borel and $\infty$-integrable, and so by the previous result,
\begin{equation*}
\begin{split}
x_1\mapsto \int _{X_2}\dif \meas (x_2)\, \chi _M(x_1,x_2) & =\int _{X_2}\dif \meas (x_2)\, \chi _{M_{x_1}}(x_2) \\
& =\meas _2(M_{x_1})
\end{split}
\end{equation*}
is Borel.
\end{proof}
\end{crl}

In \cref{FundamentalTheoremOfTheIntegral}, we gave a list of properties that uniquely characterized the integral on nonnegative Borel functions.  We mentioned in a remark there that in fact weaker versions of the statements held even when the functions in question are not necessarily Borel.  \nameref{CavalierisPrinciple} finally allows to to prove all of these.
\begin{thm}{}{prp5.2.287}
Let $\coord{X,\meas}$ be a topological measure space.  Then,
\begin{enumerate}
\item \label{prp5.2.287.ii}
\begin{equation*}
\int _X\dif \meas (x)\, [f(x)+g(x)]\leq \int _X\dif \meas (x)\, f(x)+\int _X\dif \meas (x)\, g(x)
\end{equation*}
for $f,g\colon X\rightarrow [0,\infty ]$ $\infty$-integrable; and
\item \label{prp5.2.287.iv}whenever $\lambda \mapsto f_{\lambda}$ is a nondecreasing net of nonnegative $\infty$-integrable functions, then
\begin{equation}
\lim _{\lambda}\int _X\dif \meas (x)\, f_{\lambda}(x)\leq \int _X\dif \meas (x)\, \lim _{\lambda}f_{\lambda}(x).
\end{equation}
\end{enumerate}
\begin{rmk}
The point is that we do not need to require the functions be Borel or the net to be a sequence for these statements to hold.  Of course, the inequalities become equalities if all functions involved are Borel---see \cref{FundamentalTheoremOfTheIntegral}.
\end{rmk}
\begin{proof}
\cref{prp5.2.287.ii} By definition, we have
\begin{equation}
\begin{split}
\MoveEqLeft
\int _X\dif \meas (x)\, f(x) \\
& \ceqq [\meas \times \meas _{\mrm{L}}]\left( \left\{ \coord{x,y}\in X\times [-\infty ,\infty ]:\right. \right. \\ & \qquad \qquad \left. \left. 0\leq y<f(x)\right\} \right) ,
\end{split}
\end{equation}
where $\meas _{\mrm{L}}$ is Lebesgue measure.  Writing
\begin{equation}
\Gamma _f\ceqq \left\{ \coord{x,y}\in X\times [0,\infty ]:0\leq y<f(x)\right\} ,
\end{equation}
it thus suffices to show that
\begin{equation}
[\meas \times \meas _{\mrm{L}}](\Gamma _{f+g})\leq [\meas \times \meas _{\mrm{L}}](\Gamma _f)+[\meas \times \meas _{\mrm{L}}](\Gamma _g).
\end{equation}

Define $\tau _f:X\times [0,\infty ]\rightarrow X\times [0,\infty ]$ by
\begin{equation}
\tau _f(\coord{x,y})\coloneqq \coord{x,y+f(x)}.
\end{equation}
This definition was made so that we have
\begin{equation}
\Gamma _{f+g}=\Gamma _f\cup \tau _f(\Gamma _g)
\end{equation}
is a disjoint\footnote{This is true, but actually we don't need disjointness for this proof.} union.  From this, it follows that
\begin{equation}
\begin{multlined}
[\meas \times \meas _{\mrm{L}}](\Gamma _{f+g}) \\ \leq [\meas \times \meas _{\mrm{L}}](\Gamma _f)+[\meas \times \meas _{\mrm{L}}](\tau _f(\Gamma _g)),
\end{multlined}
\end{equation}
and so it suffices to show that
\begin{equation}
[\meas \times \meas _{\mrm{L}}](\tau _f(\Gamma _g))=[\meas \times \meas _{\mrm{L}}](\Gamma _g).
\end{equation}
However, by \nameref{CavalierisPrinciple}, we have
\begin{equation*}
\begin{split}
\MoveEqLeft \relax
[\meas \times \meas _{\mrm{L}}](\tau _f(\Gamma _g)) \\
& =\int _X\dif \meas (x) \\ & \qquad \left[ \meas _{\mrm{L}}\left( \left\{ y\in [0,\infty ]:f(x)\leq y<g(x)+f(x)\right\} \right) \right] \\
& =\footnote{By translation invariance.}\int _X\dif \meas (x) \\ & \qquad \left[ \meas _{\mrm{L}}\left( \left\{ y\in [0,\infty ]:0\leq y<g(x)\right\} \right) \right] \\
& =[\meas \times \meas _{\mrm{L}}](\Gamma _g),
\end{split}
\end{equation*}
as desired.

\blankline
\noindent
\cref{prp5.2.287.iv} For each $\lambda$, we have that
\begin{equation*}
\begin{split}
\MoveEqLeft
\int _X\dif \meas (x)\, f_{\lambda}(x) \\
& \ceqq [\meas \times \meas _{\mrm{L}}]\left( \left\{ \coord{x,y}\in X\times [0,\infty ]:0\leq y<f_{\lambda}(x)\right\} \right) \\
& \leq [\meas \times \meas _{\mrm{L}}]\left( \left\{ \coord{x,y}\in X\times [0,\infty ]:0\leq y<f_{\infty}(x)\right\} \right) \\
& \eqqc \int _X\dif \meas (x)\, f_{\infty}(x),
\end{split}
\end{equation*}
where we have written $f_{\infty}(x)\ceqq \lim _{\lambda}f_{\lambda}(x)$.  Of course, as the net is nondecreasing, we have $f_{\lambda}\leq f_{\infty}$ for all $\lambda$.  Taking the limit of this inequality gives the desired result
\begin{equation}
\lim _{\lambda}\int _X\dif \meas (x)\, f_{\lambda}(x)\leq \int _X\dif \meas (x)\, \lim _{\lambda}f_{\lambda}(x).
\end{equation}
\end{proof}
\end{thm}
This in turn allows us to prove the following important `continuity' result.
\begin{prp}{}{}
Let $\coord{X,\meas}$ be a topological measure space and let $f\colon X\rightarrow [-\infty ,\infty ]$ be integrable and Borel.  Then, for every $\varepsilon >0$, there is some $\delta >0$ such that, whenever $\meas (S)<\delta$, it follows that $\int _S\dif \meas (x)\, \left| f(x)\right| <\varepsilon$.
\begin{proof}
For convenience, write $g\coloneqq \abs{f}$.  Let $\varepsilon >0$.  For $m\in \N$, define
\begin{equation}
M_m\coloneqq g^{-1}([0,m])
\end{equation}
Note that this is measurable because $g$ is Borel (see \cref{BorelIsBorel}).  It follows that $g_m\coloneqq g\chi _{M_m}$ is Borel.  As $m\mapsto g_m$ is nondecreasing and converges to $g$ pointwise, it follows from Lebesgue's Monotone Convergence Theorem that
\begin{equation}
\lim _m\int _X\dif \meas (x)\, g_m(x)=\int _X\dif \meas (x)\, g(x).
\end{equation}
Thus, there is some $m_0\in \Z ^+$ such that, whenever $m\geq m_0$, it follows that\footnote{This is where we use the fact that $f$ is integrable, so that $\int _X\dif \meas (x)\, g(x)$ is finite.}
\begin{equation}
\int _X\dif \meas (x)\, \left[ g(x)-g_m(x)\right] <\varepsilon .
\end{equation}
Define $\delta \coloneqq \frac{\varepsilon}{m_0}$.  Now, let $S\subseteq X$ be such that $\meas (S)<\delta$.  Then,
\begin{equation}
\begin{split}
\MoveEqLeft
\int _S\dif \meas (x)\, \abs{f(x)}
\\ & \eqqc \int _S\dif \meas (x)\, g(x) \\
& \leq \footnote{Note that we may not have equality here if $S$ is not measurable; however, the inequality still holds by virtue of \cref{prp5.2.287}.}\int _S\dif \meas (x)\, \left[ g(x)-g_{m_0}(x)\right] \\ & \quad +\int _S\dif \meas (x)\, g_{m_0}(x) \\
& \leq \footnote{Because $g(x)-g_{m_0}(x)\geq 0$ and $g_{m_0}$ is bounded above by $m_0$.}\int _X\dif \meas (x)\, \left[ g(x)-g_{m_0}(x)\right] +m_0\meas (S) \\
& <\varepsilon +m_0\delta =2\varepsilon .
\end{split}
\end{equation}
\end{proof}
\end{prp}
\begin{exr}{}{}
Let $I\subseteq \R$ be an interval, let $f\in \Bor (I)$ be integrable, and let $a\in I$.  Show that the map $x\mapsto \int _a^x\dif \, tf(t)$ is uniformly-continuous.
\end{exr}

In a similar spirit, we have the following result.
\begin{prp}{}{}
Let $\coord{X,\meas}$ be a topological measure space and let $f\colon X\rightarrow [-\infty ,\infty ]$ be integrable and Borel.  Then, for every $\varepsilon >0$, there is some compact $K\subseteq X$ such that $\abs*{\int _{K^{\comp}}\dif \meas (x)\, f(x)}<\varepsilon$.
\begin{proof}
Let $\varepsilon >0$.  As $f$ is integrable, we have
\begin{equation}
\int _X\dif \meas (x)\, f(x)=\lim _{K\in \collection{K}}\int _K\dif \meas (x)\, f(x),
\end{equation}
where $\collection{K}$ is the collection of quasicompact subsets of $X$.  Write $X=\bigcup _{m\in \N}K_m$ as a nondecreasing collection of compact sets.  Then, as subnets of convergent nets converge to the same thing, we have
\begin{equation}
\lim _X\dif \meas (x)\, f(x)=\lim _m\int _{K_m}\dif \meas (x)\, f(x).
\end{equation}
So, let $m_0\in \N$ be such that, whenever $m\geq m_0$, it follows that
\begin{equation}
\begin{split}
\MoveEqLeft
\abs*{\int _{K_m^{\comp}}\dif \meas (x)\, f(x)} \\
& =\footnote{This requires that $f$ is borel.}\abs*{\int _{K_m}\dif \meas (x)\, f(x)-\int _X\dif \meas (x)\, f(x)} \\
& <\varepsilon .
\end{split}
\end{equation}
\end{proof}
\end{prp}
\begin{exr}{}{}
Show that we cannot replace
\begin{equation}
\text{``}\abs*{\int _{K^{\comp}}\dif \meas (x)\, f(x)}<\varepsilon \text{''}
\end{equation}
in the previous result with
\begin{equation}
\text{``}\int _{K^{\comp}}\dif \meas (x)\, \abs{f(x)}\text{''}.
\end{equation}
More precisely, find an example of a topological measure space $\coord{X,\meas}$ and an integrable Borel function $f\colon X\rightarrow [-\varepsilon ,\varepsilon ]$ for which it is \emph{not} the case that for every $\varepsilon >0$ there is some compact $K\subseteq X$ such that $\int _{K^{\comp}}\abs*{f(x)}<\varepsilon$.
\end{exr}

\horizontalrule

We next present a result that is quite important in its own right, but can additionally be viewed as justification for the `naturality' of the condition \emph{regular}.
\begin{thm}{Riesz-Markov Theorem}{RieszMarkovTheorem}\index{Riesz-Markov Theorem}
Let $X$ be a $\sigma$-compact topological space and let $\mrm{I}\colon \Mor _{\Top}(X,\R )\rightarrow [-\infty ,\infty ]$ be linear.\footnote{That is, we require that $\mrm{I}(f+g)=\mrm{I}(f)+\mrm{I}(g)$ for all $f,g\in \Mor _{\Top}(X,\R )$ except when one of these is $+\infty$ and the other $-\infty$.}  Then, if $X$ is locally quasicompact and $\mrm{I}$ is nondecreasing, then there exists a unique topological measure $\meas$ on $X$ such that
\begin{equation}
\mrm{I}(f)=\int _X\dif \meas (x)\, f(x)
\end{equation}
for all $f\in \Mor _{\Top}(X,\R )$.
\begin{rmk}
``Nondecreasing'' of course means that $f\leq g$ implies that $\mrm{I}(f)\leq \mrm{I}(g)$, where $f\leq g$ means that $f(x)\leq g(x)$ for all $x\in X$.  In particular, $\mrm{I}(f)\geq 0$ if $f\geq 0$.
\end{rmk}
\begin{rmk}
The point is that, if you start with something that a priori has nothing to do with measure theory at all, you can in fact obtain a measure, and not just any measure, but a \emph{regular Borel} measure.  Thus, you might view the condition of ``regular'' (and ``Borel'') as being a natural condition to impose on measures in the sense that these conditions arise from the theory naturally \`{a} la Riesz-Markov---you don't have to put them in ``by hand''.
\end{rmk}
\begin{proof}
We leave this as an exercise.
\begin{exr}[breakable=false]{}{}
Prove this yourself.
\begin{rmk}
Hint:  This is one of those exercises that really shouldn't be an exercise, but it is anyways because I ran out of time.  In particular, as the phrasing of the theorem is slightly nonstandard, there's a possibility it's not correct verbatim.  Thus, you should view it as part of the exercise to `tweak' the statement to make it correct if it turns out that this is not completely correct as stated.  See \cite[Theorem 2.14]{BigRudin} to guide you.
\end{rmk}
\end{exr}
\end{proof}
\end{thm}

The following is an important result that we will need to prove the Fundamental Theorem of Calculus in the next chapter.  To state it, we will need to introduce a couple of definitions.
\begin{dfn}{Upper-semicontinuity and lo\-wer-se\-mi\-con\-tin\-u\-ity}{Semicontinuity}
Let $f\colon X\rightarrow [-\infty ,\infty ]$ be a function on a topological space $X$ and let $x_0\in X$.  Then, $f$ is \term{upper-semicontinuous}\index{Upper-semicontinuous} at $x_0$ iff for every $\varepsilon >0$, there is an open neighborhood $U$ of $x_0$ such that, whenever $x\in U$, it follows that $f(x)-f(x_0)<\varepsilon$.  $f$ is \term{lower-semicontinuous}\index{Lower-semicontinuous} at $x_0$ iff for every $\varepsilon >0$, there is an open neighborhood $U$ of $x_0$ such that, whenever $x\in U$, it follows that $f(x_0)-f(x)<\varepsilon $.
\begin{rmk}
Upper-semicontinuity means that you can force $f(x)$ to be as close as you like to $f(x_0)$ \emph{from above} (by moving sufficiently close to $x_0$), but you have no control over what happens below---$f(x)$ can be arbitrarily negative in a neighborhood of $x_0$ and it can still be upper-semicontinuous.  Similarly, for lower-semicontinuous.
\end{rmk}
\end{dfn}
\begin{thm}{Carathéodory-Vitali Theorem}{CaratheodoryVitaliTheorem}\index{Carathéodory-Vitali Theorem}
Let $\coord{X,\meas}$ be a topological measure space and let $f\colon X\rightarrow [-\infty ,\infty ]$ be integrable and Borel.  Then, for every $\varepsilon >0$, there are $u,v\in \Bor (X)$ such that
\begin{enumerate}
\item $u\leq f\leq v$;
\item $u$ is upper-semicontinuous and bounded above;
\item $v$ is lower-semicontinuous and bounded below; and
\item
\begin{equation}
\int _X\dif \meas (x)\, [v(x)-u(x)]<\varepsilon .
\end{equation}
\end{enumerate}
\begin{proof}\footnote{Proof adapted from \cite[pg.~56]{BigRudin}.}
We first prove the case for $f$ nonnegative.  In this case, by \cref{SimpleFunctionApproximation}, we can write $f$ as
\begin{equation}
f=\sum _{m\in \N}c_m\chi _{M_m},
\end{equation}
for $c_m>0$ and $M_m\subseteq X$ measurable.  Let $\varepsilon >0$, and by inner and outer-regularity let $K_m\subseteq M_m\subseteq U_m$ be such that $K_m$ is compact, $U_m$ is open, and
\begin{equation}
c_m\meas (U_m\setminus K_m)<\tfrac{\varepsilon}{2^m}.
\end{equation}
As $f$ was assumed to be integrable, it must be the case that
\begin{equation}
\sum _{m\in \N}c_m\meas (M_m)=\int _X\dif \meas (x)\, f(x)
\end{equation}
converges, and so there is some $m_0\in \N$ such that
\begin{equation}
\sum _{m=m_0+1}^\infty c_m\meas (M_m)<\varepsilon .
\end{equation}
Finally, define
\begin{equation}
v\coloneqq \sum _{m=0}^\infty c_m\chi _{U_m}\text{ and }u\ceqq \sum _{m=0}^{m_0}c_m\chi _{K_m}.
\end{equation}
\begin{exr}[breakable=false]{}{}
Show that $u$ and $v$ satisfy the desired properties.
\end{exr}
\begin{exr}[breakable=false]{}{}
Finish the proof by doing the general case by writing $f=f_+-f_-$.
\end{exr}
\end{proof}
\end{thm}

\subsection{Riemann integrability}

Finally, as I am required to teach you the Riemann integral, I begrudgingly include a cop-out version just to say I taught it.
\begin{dfn}{Riemann integrable}{RiemannIntegrable}
Let $f\colon [a,b]\rightarrow \R$.  Then, $f$ is \term{Riemann integrable}\index{Riemann integrable} iff $f$ is bounded and continuous almost-everywhere.  In this case, its \term{Riemann integral}\index{Riemann integral} is defined to be $\int _a^b\dif x\, f(x)$.
\begin{rmk}
This is usually a theorem, but because I think it's a waste of time to develop the Riemann integral when the Lebesgue integral does everything the Riemann integral does (and much more), I include it as a definition just to say I did it.  Congratulations!  You now know the Riemann integral!
\end{rmk}
\end{dfn}

\section{\texorpdfstring{$L^p$}{Lp} spaces}\label{Lp}

For $X$ a topological measure space, $L^p(X)$ will wind up being a certain collection of Borel functions defined on $X$.  While there are many reasons one might be interested in such objects, one possible motivation is that, if you care about topological measure spaces, then you can obtain information about $X$ by a study of $L^p(X)$.\footnote{In fact, the idea of studying spaces by instead studying functions defined on those spaces is quite a pervasive theme in all of mathematics.} In this case, the space is a topological measure space, and so the relevant functions will be the Borel functions.  In a perfect world, I suppose we could use the integral to put (semi)norms on $\Bor (X)$ itself, but unlike in the continuous case where we had theorems (namely the \nameref{ExtremeValueTheorem}) to guarantee the finiteness of the supremum seminorms on quasicompact sets, there are no such theorems in this case---there's no getting around the fact that some Borel function will have infinite integral on any set of positive measure.  And so, instead of studying all of the Borel functions at once, we study certain subsets of them.  The $L^p$ spaces are certain nice spaces of Borel functions, namely the ones whose $p^{\text{th}}$ power is integrable.

Before we do anything else, we define the so-called \emph{$L^p$ norms}.
\begin{dfn}{$L^p$-norm}{LpNorm}
Let $\coord{X,\meas}$ be a topological measure space and let $p\in [1,\infty ]$.  Then, the \term{$L^p$ norm}\index{$L^p$ norm}, $\norm _p:\Bor (X)\rightarrow [0,\infty ]$, is defined by
\begin{equation*}
\norm{f}_p\coloneqq \begin{cases}\left( \int \dif _X\meas (x)\, \abs{f(x)}^p\right) ^{\tfrac{1}{p}} & p\neq \infty \\ \sup \left\{ y\in \R :\meas \left( f^{-1}([y,\infty ])\right) >0\right\} & p=\infty .\end{cases}
\end{equation*}\index[notation]{$\norm{f}_p$}
\begin{rmk}
If you take $X\coloneqq \{1,\ldots ,d\}$ to be a $d$ point set with the counting measure, this reads
\begin{equation}
\norm{v}_p=\left( \abs{v_1}^p+\cdots +\abs{v_d}^p\right) ^{\tfrac{1}{p}},
\end{equation}
where we have suggestively written $v_k\coloneqq v(k)$.  In particular, $L^2(\{ 1,\ldots ,d\} )\cong _{\Hil _{\R}}\R ^d$.\footnote{$\Hil _{\R}$ is the category of \emph{(real) Hilbert spaces}.  If you know what this is, great---here's another cookie \emph{*gives yet another cookie to the reader*}.  If not, don't worry about it for now---as long as you intuitively see why this is the same as the Euclidean norm, you're fine.}
\end{rmk}
\begin{rmk}
$\norm{f}_\infty$ is the \term{essential supremum}\index{Essential supremum}.  Intuitively, you're taking the supremum over all $y$-values that have the property that the measure of the set on which $f$ is at least $y$ is \emph{strictly} positive.  Intuitively speaking, it is the supremum norm `modulo sets of measure $0$'.  For example, the function that is $0$ on $\Q ^{\comp}$ and $\infty$ on $\Q$ has essential supremum $0$---that it is infinite on an infinite set does not matter, as this infinite set has measure $0$.  Note that in the case that $X$ is quasicompact, the supremum norm $\norm{f}_\infty$ as defined in \cref{exm4.3.60} agrees with the essential supremum norm $\norm{f}_\infty$, so there is no ambiguity.  The reason this is called the $p=\infty$ norm is because $\lim _{p\to \infty}\norm{f}_p=\norm{f}_\infty$ (at least when $\norm{f}_{p_0}<\infty$ for some $p_0<\infty$)---see \cref{exr5.3.4}.
\end{rmk}
\begin{rmk}
The ``$L$'' is for ``Lebesgue''.  I have no idea what the $p$ is for.  In fact, I think it's a bad choice of letter, but we're pretty much stuck with it at this point.
\end{rmk}
\end{dfn}
\begin{exr}{}{exr5.3.4}
Let $\coord{X,\meas}$ be a topological measure space and let $f\in \Bor (X)$.  Show that if $\norm{f}_{p_0}<\infty$ is finite for some $1\leq p_0<\infty$, then
\begin{equation}
\lim _{p\to \infty}\norm{f}_p=\norm{f}_{\infty}.
\end{equation}
\end{exr}
\begin{dfn}{Hölder conjugate}{}
Let $1<p,q<\infty$.  Then, $p$ and $q$ are \term{Hölder conjugate}\index{Hölder conjugate} iff
\begin{equation}
\tfrac{1}{p}+\tfrac{1}{q}=1.
\end{equation}
$1$ and $\infty$ are Hölder conjugates.
\begin{rmk}
Note that $2$ is Hölder conjugate to itself.
\end{rmk}
\begin{rmk}
One way to see that we don't want to\footnote{Well, usually not.} investigate the case $0<p<1$ is because then it does not have a Hölder conjugate.\footnote{Or at least, its Hölder conjugate would be negative.}  This really breaks things because then, for example, \nameref{HoldersInequality} will not hold.
\end{rmk}
\end{dfn}
\begin{thm}{Hölder's Inequality}{HoldersInequality}\index{Hölder's Inequality}
Let $X$ be a topological measure space, let $1\leq p,q\leq \infty$ be Hölder conjugates, and let $f,g\in \Bor (X)$.  Then,
\begin{equation}
\norm{fg}_1\leq \norm{f}_p\norm{g}_q.
\end{equation}
\begin{rmk}
The case $p=2=q$ is called the \term{Cauchy-Schwarz Inequality}\index{Cauchy-Schwarz Inequality}.  In the case that $X$ is a $d$ point space with the counting measure, it is literally the statement that the `dot product' of two vectors is at most the product of their Euclidean norms.
\end{rmk}
\begin{proof}
If both $\norm{f}_p=\infty =\norm{g}_q$, then the inequality is trivially satisfied.  If one of $\norm{f}_p$ and $\norm{g}_q$ is $0$, without loss of generality, say $\norm{f}_p$, then we get that $\abs{f(x)}=0$ almost-everywhere by \cref{Norm}, and so $\abs{f(x)g(x)}=0$ almost-everywhere, and so $\norm{fg}_1=0$,\footnote{I feel as if this notation, while consistent, is a bit obtuse---$\abs{f}$ is the function defined by $x\mapsto \abs{f(x)}$, and $\norm{f}_1$ is the number defined by $\int _X\dif \meas (x)\, \left| f(x)\right|$.} and so in this case the inequality is satisfied as well.  Thus, we may as well assume that $0<\norm{f}_p,\norm{g}_q<\infty$.  Then, replacing $f$ by $\frac{f}{\norm{f}_p}$ and $\frac{g}{\norm{g}_q}$, we see that it suffices to show that $\norm{fg}_1\leq 1$ if $\norm{f}_p=1=\norm{g}_q$.

First take $1<p,q<\infty$.
\begin{exr}[breakable=false]{}{}
Show that there are $s,t\in \Bor (X)$ such that $2^s=\abs{f}^p$ and $2^t=\abs{g}^q$.
\end{exr}
\begin{exr}[breakable=false]{}{}
Show that that if $a+b=1$ with $a,b\geq 0$, then $2^{ax+by}\leq a2^x+b2^y$.
\begin{rmk}
Note that this does not necessarily work if $a$ or $b$ is negative.  Thus, is the point where the proof breaks down if $p<1$ (because then either $p$ is negative or $q$ is negative).
\end{rmk}
\begin{rmk}
This is called \term{convexity}.
\end{rmk}
\end{exr}
Using this, we obtain
\begin{equation}
\begin{split}
\abs{f}\abs{g} & =2^{s/p}2^{t/q}=2^{s/p+t/q}\leq \tfrac{2^s}{p}+\tfrac{2^t}{q} \\
& =\tfrac{\abs{f}^p}{p}+\tfrac{\abs{g}^q}{q}.
\end{split}
\end{equation}
Integrating this inequality gives
\begin{equation}
\abs{fg}_1\leq \tfrac{1}{p}\abs{f}_p^p+\tfrac{1}{q}\abs{f}_q^q=\footnote{Because we have assumed that $\abs{f}_p=1=\abs{f}_q$.}=\tfrac{1}{p}+\tfrac{1}{q}=1,
\end{equation}
as desired.

\begin{exr}[breakable=false]{}{}
Prove the result for $p=1$ and $q=\infty$.
\end{exr}
\end{proof}
\end{thm}
\begin{thm}{Minkowski's Inequality}{MinkowskisInequality}\index{Minkowski's Inequality}
	Let $\coord{X_1,\meas _1}$ and $\coord{X_2,\meas _2}$ be topological measure spaces, let $f\in \Bor (X_1\times X_2)$, and let $1\leq p\leq \infty$.  Then,
	\begin{equation}
		\norm{\int _{X_2}\dif \meas _2(x_2)\, f(\blankdot ,x_2)}_p\leq \int _{X_2}\dif \meas _2(x_2)\, \norm{f(\blankdot ,x_2)}_p.
	\end{equation}
	\begin{rmk}
		Explicitly, this reads
		\begin{equation*}
			\begin{multlined}
			\left[ \int _{X_1}\dif \meas _1(x_1)\, \abs*{\int _{X_2}\dif \meas _2(x_2)\, f(x_1,x_2)}^p\right] ^{1/p} \\ \leq \int \dif _{X_2}\dif \meas _2(x_2)\, \left[ \int _{X_1}\dif \meas _1(x_1)\, \abs{f(x_1,x_2)}^p\right] ^{1/p}.
			\end{multlined}
		\end{equation*}
	\end{rmk}
	\begin{rmk}
		I remember this roughly as ``You can bring the exponents `in a level' if you switch the order of integration.''
	\end{rmk}
	\begin{proof}
		We leave this as an exercise.
		\begin{exr}[breakable=false]{}{}
			Prove the result.
			\begin{rmk}
				Hint:  See \cite[6.19]{Folland}.
			\end{rmk}
		\end{exr}
	\end{proof}
\end{thm}
\begin{prp}{Triangle Inequality}{TriangleInequalityLp}
Let $\coord{X,\meas}$ be a topological measure space, let $f,g\in \Bor (X)$, and let $1\leq p\leq \infty$.  Then,
\begin{equation}
\norm{f+g}_p\leq \norm{f}_p+\norm{g}_p.
\end{equation}
\begin{rmk}
Sometimes this itself is referred to as ``Minkowski's Inequality''.  I think that's a bit silly because (i) the more general inequality of the previous result is also called ``Minkowski's Inequality'' and (ii) this inequality already has a name---the Triangle Inequality!
\end{rmk}
\begin{proof}
Note that this equation is trivially satisfied if either one of $\norm{f}_p$ and $\norm{g}_p$ is infinite or if both of $\norm{f}_p$ and $\norm{g}_p$ are zero, so suppose that neither of these is the case.

First take $1<p<\infty$.  Then,
\begin{equation*}
\begin{split}
\norm{(f+g)^p}_1 & =\norm{f(f+g)^{p-1}+g(f+g)^{p-1}}_1 \\
& \leq \footnote{By the usual Triangle Inequality.}\norm{f(f+g)^{p-1}}_1+\norm{g(f+g)^{p-1}} \\
& \leq \footnote{By \nameref{HoldersInequality}.}\norm{f}_p\norm{(f+g)^{p-1}}_q+\norm{g}_p\norm{(f+g)^{p-1}}_q \\
& =\norm{(f+g)^{p-1}}_q\left( \norm{f}_p+\norm{g}_p\right) ,
\end{split}
\end{equation*}
where $q$ is the Hölder conjugate of $p$.  This implies that $(p-1)q=p$, and so
\begin{equation*}
\begin{split}
\norm{(f+g)^{p-1}}_q & \coloneqq \left( \int _X\dif \meas (x)\, \left| (f(x)+g(x))^{p-1}\right| ^q\right) ^{\tfrac{1}{q}} \\
& =\left( \int _X\dif \meas (x)\, \left| f(x)+g(x)\right| ^p\right) ^{\tfrac{1}{q}} \\
& \eqqc \norm{(f+g)^p}_1^{\tfrac{1}{q}}.
\end{split}
\end{equation*}
So in fact, the previous inequality reads
\begin{equation}
\norm{(f+g)^p}_1\leq \norm{(f+g)^p}_1^{\tfrac{1}{q}}\left( \norm{f}_p+\norm{g}_p\right),
\end{equation}
so that
\begin{equation}
\begin{split}
\norm{f+g}_p & \ceqq \norm{(f+g)^p}_1^{\tfrac{1}{p}}=\norm{(f+g)^p}_1^{1-\tfrac{1}{q}} \\
& \leq \norm{f}_p+\norm{g}_p,
\end{split}
\end{equation}
as desired.

\begin{exr}[breakable=false]{}{}
Do the case $p=1$ and $p=\infty$.
\end{exr}
\end{proof}
\end{prp}
This is just the statement that the `norm' $\norm _p$ is in fact a norm, the Triangle Inequality being the only nonobvious axiom that needs to be satisfied.  Now that we know this, we are free to defined the $L^p$ spaces.
\begin{dfn}{$L^p$ spaces}{LpSpaces}
Let $\coord{X,\meas}$ be a topological measure space and let $1\leq p\leq \infty$.  Then, $L^p(X)$ is the normed vector space defined by
\begin{equation}
L^p(X)\coloneqq \{ f\in \Bor (X)/\sim _{\AlE}:\norm{f}_p<\infty \} ,
\end{equation}
with the addition and scalar multiplication the same as that in $\Bor (X)$ equipped with the $L^p$ norm, $\norm _p$.
\begin{rmk}
I don't like this definition.  I think it is artificially cooked up because people are too afraid to work with things that aren't metric spaces.  A definition like this is analogous to studying only the \emph{bounded} continuous functions on a topological space, instead of \emph{all} continuous functions $\Mor _{\Top}(X,\R )$.  That is, you restrict your class of functions for no other reason that you want \emph{one} norm, instead of a \emph{family} of seminorms.\footnote{The supremum norm of a bounded function is always finite, even if your topological space is not necessarily quasicompact.}  I think it is much more natural to look at \emph{quasicompactly} integrable\footnote{See the remark in \cref{prp5.2.229}.  In brief, this just means integrable over every quasicompact set.} functions.  Then, just as in $\Mor _{\Top}(X,\R )$, you would obtain a seminorm for each quasicompact subset.  For one thing, continuous functions are all compactly integrable.  Obviously this fails for the `global' $L^p$ spaces.  For example, $x\mapsto x$ is not an element of $L^1(\R )$.  In fact, I prefer the local approach so much more that I think I may add it at a later date when I have more time.  For the moment, however, the `usual' $L^p$ spaces it is.
\end{rmk}
\end{dfn}
One hugely important property of the $L^p$ spaces is that they are \emph{complete}.
\begin{thm}{Riesz-Fischer Theorem}{RieszFischerTheorem}\index{Riesz-Fischer Theorem}
Let $\coord{X,\meas}$ be a topological measure space and let $1\leq p\leq \infty$.  Then, $L^p(X)$ is complete.
\begin{proof}
We leave this as an exercise.
\begin{exr}[breakable=false]{}{}
Prove this result yourself.
\begin{rmk}
Hint:  Yet another one that should not be an exercise.  This time, check-out \cite[pg.~70]{Stein}.
\end{rmk}
\end{exr}
\end{proof}
\end{thm}
Remember way back when we did Heine-Borel and Bolzano-Weierstrass?  We mentioned there briefly that these results fail to hold in general, that is, there are spaces in which closed bounded sets are not necessarily quasicompact, and there are spaces with bounded infinite sets that do not have accumulation points.  The $L^p$ spaces provide examples for which both of these phenomena can happen.
\begin{exm}{A closed bounded set in a normed vector space that is not quasicompact}{exm5.3.23}
We take our topological measure space to be $X\coloneqq \N$ equipped with the counting measure.  Note then that elements of $L^1(X)$ are just sequences $m\mapsto a_m$ for which
\begin{equation}
\sum _{m\in \N}\abs{a_m}<\infty .
\end{equation}
Consider the closed unit ball $D_1(0)\coloneqq \{ f\in L^1(X):\norm{f}_1\leq 1\}$.
\begin{exr}[breakable=false]{}{}
Check that $D_1(0)$ is closed.
\end{exr}
It is clearly bounded, by definition.  We show that $D_1(0)$ is not quasicompact.  To do so, we construct a sequence in $D_1(0)$ which has no convergent subnet.  For $m\in \N$, let $a^m$ be the sequence which sends $m$ to $1$ and everything else to $0$.  So, for example,
\begin{equation}
\begin{aligned}
a^0 & \coloneqq \coord{1,0,0,0,\ldots} \\
a^1 & \coloneqq \coord{0,1,0,0,\ldots} \\
a^2 & \coloneqq \coord{0,0,1,0,\ldots} \\
\text{etc.} &
\end{aligned}
\end{equation}
We then have that
\begin{equation}
\norm{a^m-a^n}_1=\begin{cases}2 & \text{if }m\neq n \\ 0 & \text{if }m=n.\end{cases}
\end{equation}
This shows that there is no Cauchy subnet, and hence certainly no convergent subnet.
\end{exm}
\begin{exm}{A bounded infinite set without an accumulation point}{exm5.3.28}
Let $m\mapsto a_m\in L^1(X)$ be as in the previous example.\footnote{Briefly, the `standard unit vectors' in $L^1(\N )$.}  Then set $\{ a_m:m\in \N \}$ is certainly bounded and infinite, so it suffices to show that it doesn't have an accumulation point.  As all the points in $\{ a_m:m\in \N \}$ are at least a distance of $2$ from each other, no net whose terms come from $\{ a_m:m\in \N \}$ can be Cauchy unless it is eventually constant.  It follows that this set has no limit points, and so hence of course no accumulation points.
\end{exm}

\begin{exr}{}{}
Let $\coord{X,\meas}$ be a topological measure space and let $1\leq p\leq q\leq \infty$.
\begin{enumerate}
\item Show that if $\meas (X)<\infty$, then $L^q(X)\subseteq L^p(X)$.
\item Find a counter-example to show that this is false if $\meas (X)=\infty$.
\end{enumerate}
\end{exr}