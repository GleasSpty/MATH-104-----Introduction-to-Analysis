\chapter{Uniform spaces}\label{chp5}

A uniform space is the most general context in which one can talk about concepts such as uniform continuity, uniform convergence, Cauchyness, completeness, etc..  To formalize this notion, we will equip a set with a distinguished set of covers, called \emph{uniform covers}.  The example you should always keep in the back of your mind is the collection of all $\varepsilon$-balls for a \emph{fixed} $\varepsilon$:  $\cover{U}_\varepsilon \coloneqq \left\{ B_\varepsilon (x):x\in \R \right\}$.  The idea is that, somehow, all of the sets in the same uniform cover are of the `same size'.

Having specified the uniform covers, we will then be able to say things like a net $\lambda \mapsto x_\lambda$ is Cauchy iff for every uniform cover $\cover{U}$, there is some $U\in \cover{U}$ such that $\lambda \mapsto x_\lambda$ is eventually contained in $U$.  In the case that the collection of uniform covers is $\left\{ \cover{U}_\varepsilon :\varepsilon >0\right\}$,\footnote{Disclaimer:  The collection of all the $\cover{U}_{\varepsilon}$ is not actually a uniformity but rather a \emph{uniform base}---see \cref{UniformBase}.} you can check that this is precisely the definition of Cauchyness we had in $\R$ (\cref{dfn3.3.26}).

Moreover, the generalization from $\R$ to uniform spaces is not a needless abstraction.  Indeed, I am \emph{required} to cover metric spaces (\cref{MetricSpace}) in this course, and this is just a very special type of uniform space.  Indeed, essentially every topological space we look at in these notes---besides ones cooked up for the express purpose of producing a counter-example---has a canonical uniformity.  On the other hand, it is certainly not the case that every topological space we encounter will be a metric space.  For example, something as simple as all continuous functions from $\R$ to $\R$ has no canonical metric,\footnote{For what it's worth, I believe the topology is metrizable (homeomorphic to a metric space), but certainly not with any metric you would like to work with, much less a canonical one.} but is trivially a uniform space (because it is a topological group---see \cref{dfnB.7}).

\section{Basic definitions and facts}

A uniform space will wind-up being a set equipped with a special set of covers, the \emph{uniform covers}.  Of course, however, as you should expect, we cannot just take \emph{any} collection of covers and declare them to be the uniform covers---the collection of uniform covers has to satisfy certain reasonable properties, analogous to the properties satisfied by the collections of all $\varepsilon$-balls.  The key requirement is that the collection of uniform covers has to be \emph{downward-directed} with respect to a relation called \emph{star-refinement}.  Thus, before getting to the definition of a uniform space itself, we must say what we mean by ``star-refinement'' (we will say what we mean by ``downward-directed'' in the definition of a uniform space itself).

\subsection{Star-refinements}

\begin{dfn}{Star}{Star}
Let $X$ be a set, let $S\subseteq X$, and let $\collection{U}$ be a collection of subsets of $X$.  Then, the \term{star}\index{Star} of $S$ with respect to $\cover{U}$, $\Star _{\collection{U}}(S)$,\index[notation]{$\Star _{\collection{U}}(S)$} is defined by
\begin{equation}
\Star _{\collection{U}}(S)\coloneqq \bigcup _{\substack{U\in \collection{U} \\ U\cap S\neq \emptyset}}U.
\end{equation}
The star of a point is the star of its singleton and denoted $\Star _{\collection{U}}(x)$\index[notation]{$\Star _{\collection{U}}(x)$}.
\begin{rmk}
In other words, the star of a set with respect to a cover is the union of all elements of the cover which intersect the set.
\end{rmk}
\end{dfn}
To help guide your intuition, stars play similar roles as $\varepsilon$-balls did in $\R ^d$.  For example, if $\cover{U}_{\varepsilon}\coloneqq \{ B_{\varepsilon}(x):x\in \R ^d\}$ is the cover of $\R$ ball all $\varepsilon$-balls, then the star `centered' at a point $x\in \R ^d$ is
\begin{equation}
\Star _{\cover{U}_{\varepsilon}}(x)=B_{2\varepsilon}(x).
\end{equation}

We will certainly be wanting to take the `image' and `preimage' of a cover.
\begin{dfn}{Image and preimage of a cover}{}
Let $f\colon X\rightarrow Y$ be a function, let $\cover{U}$ be a cover of $X$, and $\cover{V}$ be a cover of $Y$.  Then, the \term{image}\index{Image (of a cover)} of $\cover{U}$, $f(\cover{U})$\index[notation]{$f(\cover{U})$}, is defined by
\begin{equation}
f(\cover{U})\coloneqq \{ f(U):U\in \cover{U}\} .
\end{equation}
The \term{preimage}\index{Preimage (of a cover)} of $\cover{V}$, $f^{-1}(\cover{V})$\index[notation]{$f^{-1}(\cover{V})$}, is defined by
\begin{equation}
f^{-1}(\cover{V})\coloneqq \{ f^{-1}(V):V\in \cover{V}\} .
\end{equation}
\begin{rmk}
We needed to make these definitions because, technically speaking, we only defined the image an preimage of \emph{subsets} of $X$ and $Y$ respectively.  As $\cover{U}$ and $\cover{V}$ are subsets of $2^X$ and $2^Y$ respectively, not $X$ and $Y$, to talk about their `usual' image and preimage, we would need to have a function from $2^X$ to $2^Y$.\footnote{Actually, given a function $f\colon X\rightarrow Y$, we obtain a function from $2^X$ to $2^Y$ \emph{and} a function from $2^Y$ to $2^X$:  $f\colon 2^X\rightarrow 2^Y$ (the function that sends a set to its image) and $f^{-1}:2^Y\rightarrow 2^X$ (the function that sends a set to its preimage) respectively.  The preimage of a cover is the image of the cover with respect to the preimage function $f^{-1}:2^Y\rightarrow 2^X$.  Likewise, the image of a cover is the image of the cover with respect to the image function $f\colon 2^X\rightarrow 2^Y$.}  In particular, note that the definition of the preimage of a cover is \emph{not} $\{ U\in 2^X:f(U)\in \cover{V}\}$.
\end{rmk}
\end{dfn}

\begin{exm}{The star of the preimage is not the preimage of the star}{}
One might hope for the preimage of the star to be the star of the preimage, that is, for $f\colon X\rightarrow Y$, $V\subseteq Y$, and $\cover{V}$ a cover of $Y$, that
\begin{equation}
f^{-1}\left( \Star _{\cover{V}}(V)\right) =\Star _{f^{-1}(\cover{V})}(f^{-1}(V)).
\end{equation}
Unfortunately, this is not necessarily the case.  For example, take $f$ to be the inclusion $\R \hookrightarrow \R ^2$ (with image the $x$-axis), take $V$ to be any subset of $\R ^2$ which does not intersect the $x$-axis (e.g.~$V=\{ \coord{x,y}\in \R ^2:y=1\}$), and take $\cover{V}\coloneqq \{ \R ^2\}$, namely the cover of $\R ^2$ consisting of only $\R ^2$ itself.  Then,
\begin{equation}
\Star _{\cover{V}}(V)=\R ^2,
\end{equation}
and so
\begin{equation}
f^{-1}\left( \Star _{\cover{V}}(V)\right) =\R .
\end{equation}
On the other hand,
\begin{equation}
\Star _{f^{-1}(\cover{V})}(f^{-1}(V))=\emptyset 
\end{equation}
simply because $f^{-1}(V)=\emptyset$.
\end{exm}
As the image tends to be even more poorly behaved,\footnote{For example, preimages preserves intersections but images in general do not---see \cref{exrA.1.30}.} you might guess that we have a similar problem with the image.  You would be correct.
\begin{exm}{The star of the image is not the image of the star}{}
Take $f\colon \R ^2\rightarrow \R$ to be the projection onto the $x$-axis, define
\begin{equation}
\cover{U}\coloneqq \{ [m+y,m+1+y]\times \{ y\} :m\in \Z ,\ y\in \R \} ,
\end{equation}
and $U_0\coloneqq (0,1)\times \{ 0\}$.  Then,
\begin{equation}
\Star_{\cover{U}}(U_0)=[0,1]\times \{ 0\} ,\footnote{The only elements of $\cover{U}$ which have $y$-coordinate $0$ are of the form $[m,m+1]\times \{ 0\}$ for $m\in \Z$, and this intersects $(0,1)$ only for $m=0$.}
\end{equation}
and so
\begin{equation}
f\left( \Star _{\cover{U}}(U_0)\right) =[0,1].
\end{equation}
On the other hand,
\begin{equation}
f(\cover{U})=\{ [m+y,m+1+y]:m\in \Z ,\ y\in \R \} 
\end{equation}
and $f(U_0)=(0,1)$, and so $f(U_0)$ intersects $[m+y,m+1+y]$ for $m=0$ and $-1<y<1$, and so
\begin{equation}
\Star _{f(\cover{U})}f(U_0)\supseteq \bigcup _{-1<y<1}[y,y+1]=(-1,2),
\end{equation}
which is strictly larger than $f\left( \Star _{\cover{U}}(U_0)\right)$.
\end{exm}
On the other hand, we do have the following.
Despite this, we always have one inclusion.
\begin{prp}{}{prpC.2.3}
Let $f\colon X\rightarrow Y$ be a function, let $S\subseteq X$, let $T\subseteq Y$, let $\cover{U}$ be a cover of $X$, and let $\cover{V}$ be a cover of $Y$.  Then, we have the following.
\begin{enumerate}
\item \label{prpC.2.3.i}
\begin{equation}
\Star _{f^{-1}(\cover{V})}(f^{-1}(T))\subseteq f^{-1}\left( \Star _{\cover{V}}(T)\right) .
\end{equation}
Furthermore, if $f$ is surjective, then we have equality.
\item \label{prpC.2.3.ii}
\begin{equation}
\Star _{f(\cover{S})}(f(S))\supseteq f(\Star _{\cover{U}}(U)).
\end{equation}
Furthermore, if $f$ is injective, then we have equality.
\item \label{prpC.2.3.iii}
\begin{equation}
\Star _{f^{-1}(\cover{V})}(S)=f^{-1}\left( \Star _{\cover{V}}(f(S))\right) .
\end{equation}
\item \label{prpC.2.3.iv}
\begin{equation}
\Star _{f(\cover{U})}(T)=f\left( \Star _{\cover{U}}f^{-1}(T)\right) .
\end{equation}
\end{enumerate}
\begin{proof}
\cref{prpC.2.3.i} Note that we have that (\cref{exrA.1.30}.\cref{enmA.1.30.ii} and \cref{exrA.1.47}.\cref{enmA.1.47.i}) $V\cap T\supseteq f(f^{-1}(V)\cap f^{-1}(T))$.  Therefore, if $f^{-1}(V)$ intersects $f^{-1}(T)$, it must be the case that $V$ intersects $T$.  Furthermore, $f$ is surjective, so that $f(f^{-1}(T'))=T'$ for all $T'\subseteq Y$ (\cref{exrA.1.47}), then we would in fact that that $V\cap T=f(f^{-1}(V)\cap f^{-1}(T))$, so that in this case $V$ intersects $T$ iff $f^{-1}(V)$ intersects $f^{-1}(T)$.
\begin{equation}
\begin{split}
\MoveEqLeft
\Star _{f^{-1}(\cover{V})}(f^{-1}(T))\coloneqq \bigcup _{\substack{V\in \cover{V} \\ f^{-1}(V)\cap f^{-1}(T)\neq \emptyset}}f^{-1}(V) \\
& =\footnote{\cref{exrA.1.30}.\cref{enmA.1.30.i}}f^{-1}\biggg( \bigcup _{\substack{V\in \cover{V} \\ f^{-1}(V)\cap f^{-1}(T)\neq \emptyset}}V\biggg) \\
& \subseteq \footnote{Here we are using the fact that $f^{-1}(V)$ intersects $f^{-1}(T)$ implies that $V$ intersects $T$.  Also note that we have equality here if $f$ is surjective.}f^{-1}\biggg( \bigcup _{\substack{V\in \cover{V} \\ V\cap T\neq \emptyset}}V\biggg) \\
& \eqqcolon f^{-1}\left( \Star _{\cover{V}}(T)\right) .
\end{split}
\end{equation}

\blankline
\noindent
\cref{prpC.2.3.ii} Note that we have that (\cref{exrA.1.30}.\cref{enmA.1.30.iv} and \cref{exrA.1.47}.\cref{enmA.1.47.ii}) $U\cap S\subseteq f^{-1}(f(U)\cap f(S))$.  Therefore, if $U$ intersects $S$, it must be the case that $f(U)$ intersects $f(S)$.  Furthermore, note that as $f$ is injective, $f(S_1\cap S_2)=f(S_1)\cap f(S_2)$ (\cref{exrA.1.30}) and $f^{-1}(f(S'))=S'$ (\cref{exrA.1.47}) for all subsets $S_1,S_2,S'\subseteq X$.  Thus, if $f$ is injective,
\begin{equation}
\begin{split}
U\cap S & =f^{-1}(f(U))\cap f^{-1}(f(S)) \\
& =f^{-1}(f(U)\cap f(S)),
\end{split}
\end{equation}
and hence
\begin{equation}
f(U\cap S)=f(U)\cap f(S),
\end{equation}
so in this case $U$ intersects $S$ iff $f(U)$ intersects $f(S)$.  Hence,
\begin{equation}
\begin{split}
f(\Star _{\cover{U}}(S)) & \coloneqq f\biggg( \bigcup _{\substack{U\in \cover{U} \\ U\cap S\neq \emptyset}}U\biggg) \\
& =\footnote{\cref{exrA.1.30}.\cref{enmA.1.30.iii}}\bigcup _{\substack{U\in \cover{U} \\ U\cap S\neq \emptyset}}f(U) \\
& \subseteq \footnote{Here we are using the fact that $U$ intersects $S$ implies that $f(U)$ intersects $f(S)$.  Also note that we have equality here if $f$ is injective.}\bigcup _{\substack{U\in \cover{U} \\ f(U)\cap f(S)\neq \emptyset}}f(U) \\
& \eqqcolon \Star _{f(\cover{U})}f(S).
\end{split}
\end{equation}

\blankline
\noindent
\cref{prpC.2.3.iii} Note that $f^{-1}(T)$ intersects $S$ iff $T$ intersects $f(S)$.  Using this, we have
\begin{equation}
\begin{split}
\Star _{f^{-1}(\cover{V})}(S) & \coloneqq \bigcup _{\substack{V\in \cover{V} \\ f^{-1}(V)\cap S\neq \emptyset}}f^{-1}(V) \\
& =f^{-1}\bigg( \bigcup _{\substack{V\in \cover{V} \\ V\cap f(S)\neq \emptyset}}V\biggg) \\
& \eqqcolon f^{-1}\left( \Star _{\cover{V}}(f(S))\right) .
\end{split}
\end{equation}

\blankline
\noindent
\cref{prpC.2.3.iv} Similarly as in \cref{prpC.2.3.iii},
\begin{equation}
\begin{split}
\Star _{f(\cover{U})}(T) & \coloneqq \bigcup _{\substack{U\in \cover{U} \\ f(U)\cap T\neq \emptyset}}f(U)=f\left( \bigcup _{\substack{U\in \cover{U} \\ U\cap f^{-1}(T)\neq \emptyset}}U\right) \\
& \eqqcolon f\left( \Star _{\cover{U}}(f^{-1}(T))\right) .
\end{split}
\end{equation}
\end{proof}
\end{prp}

\begin{dfn}{Refinement and star-refinement}{dfnC.1}
Let $X$ be a set, and let $\cover{U}$ and $\cover{V}$ be covers on $X$.
\begin{enumerate}
\item $\cover{U}$ is a \term{refinement}\index{Refinement} of $\cover{V}$, written $\cover{U}\preceq \cover{V}$\index[notation]{$\cover{U}\preceq \cover{V}$} iff for every $U\in \cover{U}$ there is some $V\in \cover{V}$ such that $U\subseteq V$.
\item $\cover{U}$ is a \term{star-refinement}\index{star-refinement} of $\cover{V}$, written $\cover{U}\llcurly \cover{V}$\index[notation]{$\cover{U}\llcurly \cover{V}$} iff for every $U\in \cover{U}$ there is a $V\in \cover{V}$ such that $\Star _{\cover{U}}(U)\subseteq V$.
\end{enumerate}
\begin{rmk}
The intuition is that every element of $\cover{U}$ is small enough to be contained in some element of $\cover{V}$.
\end{rmk}
\begin{rmk}
In other words, $\cover{U}$ is a star-refinement of $\cover{V}$ iff for all $U\in \cover{U}$, there is some $V\in \cover{V}$ such that, whenever $U'\in \cover{U}$ intersects $U$, it follows that $U'\subseteq V$.  The intuition for star-refinements is the same as for refinements, except that a star-refinement is \emph{much} finer than a mere refinement.
\end{rmk}
\end{dfn}
\begin{prp}{}{}
Let $X$ be a set, let $\cover{U}$ and $\cover{V}$ be covers of $X$, and let $S\subseteq X$.  Then, if $\cover{U}$ refines $\cover{V}$, then
\begin{equation}\label{eqn4.1.31}
\Star _{\cover{U}}(S)\subseteq \Star _{\cover{V}}(S).
\end{equation}
\begin{rmk}
In particular, \eqref{eqn4.1.31} holds if $\cover{U}$ \emph{star}-refines $\cover{V}$.
\end{rmk}
\begin{proof}
Let $U\in \cover{U}$ intersect $S$.  From the definition of refinement, there is some $V\in \cover{V}$ such that $U\subseteq V$.  As $U$ intersects $S$, $V$ certainly intersects $S$, and so $V\subseteq \Star _{\cover{V}}(S)$, and hence $U\subseteq \Star _{\cover{U}}(S)$.  Taking the union over all elements of $\cover{U}$ which intersect $S$, we find $\Star _{\cover{U}}(S)\subseteq \Star _{\cover{V}}(S)$.
\end{proof}
\end{prp}
\begin{exr}{}{}
Show that $\preceq$ is a preorder, but not a partial-order.
\end{exr}
\begin{exr}{}{exr4.2.22}
Show that $\llcurly$ is transitive, but not even reflexive.
\end{exr}
Any two covers always have a common refinement.  In fact, they have a canonical (but not unique!) largest one.
\begin{dfn}{Meet of covers}{}
Let $X$ be a set, and let $\cover{U}$ and $\cover{V}$ be covers of $X$.  Then, the \term{meet}\index{Meet (of covers)} of $\cover{U}$ and $\cover{V}$, $\cover{U}\wedge \cover{V}$\index[notation]{$\cover{U}\wedge \cover{V}$}, is defined by
\begin{equation}
\cover{U}\wedge \cover{V}\coloneqq \{ U\cap V:U\in \cover{U}\text{ and }V\in \cover{V}\} .
\end{equation}
\begin{rmk}
The term ``meet'' and notation ``$\cover{U}\wedge \cover{V}$'' is notation taken from the theory of partially-ordered sets where $x\wedge y$ (the \emph{meet}) of $x$ and $y$ is defined to be $\inf \{ x,y\}$.  In our case, however, this is abuse of notation and terminology as $\preceq$ is not a partial-order (and so infima need not be unique---see \cref{exr1.4.4}).
\end{rmk}
\end{dfn}
\begin{exr}{}{}
Show that (i)~$\cover{U}\wedge \cover{V}\preceq \cover{U},\cover{V}$; and (ii)~if $\cover{W}$ refines both $\cover{U}$ and $\cover{V}$, then it refines $\cover{U}\wedge \cover{V}$.  Find an example to show that it is \emph{not} the unique such cover with these two properties.
\end{exr}
\begin{exr}{}{}
Let $X$ be a set and let $\cover{U}$ be a cover of $X$.  Show that there does not necessarily exist a cover $\cover{V}$ of $X$ such that (i)~$\cover{V}$ star-refines $\cover{U}$; and (ii)~if $\cover{W}$ star-refines $\cover{U}$, then it star-refines $\cover{V}$?
\begin{rmk}
In particular, there can't be any construction that does for star-refinements what the meet does for refinements.
\end{rmk}
\end{exr}
\begin{prp}{}{prp4.2.27}
Let $X$ be a set, let $x\in X$, and let $\cover{U}$ and $\cover{V}$ be covers of $X$.  Then,
\begin{equation}
\Star _{\cover{U}}(x)\cap \Star _{\cover{V}}(x)=\Star _{\cover{U}\wedge \cover{V}}(x).
\end{equation}
\begin{wrn}
Warning:  This fails if you replace $x$ with a general set $S$---see the following counter-example (though we should always have the $\supseteq$ inclusion).
\end{wrn}
\begin{proof}
We simply `compute'.
\begin{equation}
\begin{split}
\MoveEqLeft
\Star _{\cover{U}}(x)\cap \Star _{\cover{V}}(x)\coloneqq \biggg( \bigcup _{\substack{U\in \cover{U} \\ x\in U}}U\biggg) \cap \biggg( \bigcup _{\substack{V\in \cover{V} \\ x\in V}}V\biggg) \\
& =\bigcup[0]_{\substack{U\in \cover{U}\st x\in U \\ V\in \cover{V}\st x\in V}}U\cap V \\
& =\bigcup[0]_{\substack{U\in \cover{U},V\in \cover{V} \\ x\in U\cap V}}U\cap V\eqqcolon \Star _{\cover{U}\wedge \cover{V}}(x).
\end{split}
\end{equation}
\end{proof}
\end{prp}
\begin{exm}{$\Star _{\cover{U}}(S)\cap \Star _{\cover{V}}(S)\neq \Star _{\cover{U}\wedge \cover{V}}(S)$}{}
Define $X\coloneqq \{ 0,1,2,3\}$, and
\begin{equation*}
\cover{U}\coloneqq \left\{ \{ 0,1\} ,\{ 2\} ,\{ 3\} \right\} \text{ and }\cover{V}\coloneqq \left\{ \{ 0,2\} ,\{ 1,3\} \right\} .
\end{equation*}
Then,
\begin{equation}
\cover{U}\wedge \cover{V}=\left\{ \emptyset ,\{ 0\} ,\{ 1\} ,\{ 2\} ,\{ 3\} \right\} .
\end{equation}

Define $S\coloneqq \{ 1,2\}$.  Then, $\Star _{\cover{U}}(S)=\{ 0,1\} \cup \{ 2\} =\{ 0,1,2\}$ and $\Star _{\cover{V}}(S)=\{ 0,2\} \cup \{ 1,3\} =\{ 0,1,2,3\}$, and so
\begin{equation}
\Star _{\cover{U}}(S)\cap \Star _{\cover{V}}(S)=\{ 0,1,2\} .
\end{equation}
On the other hand,
\begin{equation}
\Star _{\cover{U}\wedge \cover{V}}(S)=\{ 1,2\} .
\end{equation}
\end{exm}
\begin{exr}{}{exr4.2.35}
Show that if $\cover{U}_1\preceq \cover{U}_2$ (resp.$\cover{U}_1\llcurly \cover{U}_2$) and $\cover{V}_1\preceq \cover{V}_2$ (resp.~$\cover{V}_1\llcurly \cover{V}_2$), then $\cover{U}_1\wedge \cover{V}_1\preceq \cover{U}_2\wedge \cover{V}_2$ (resp.~$\cover{U}_1\wedge \cover{V}_1\llcurly \cover{U}_2\wedge \cover{V}_2$).
\end{exr}
\begin{prp}{}{prpB.2.12}
Let $f\colon X\rightarrow Y$ be a function and let $\cover{U}$ and $\cover{V}$ be covers of $Y$ such that $\cover{U}\preceq \cover{V}$ (resp.~$\cover{U}\llcurly \cover{V}$).  Then, $f^{-1}(\cover{U})\preceq f^{-1}(\cover{V})$ (resp.~$f^{-1}(\cover{U})\llcurly f^{-1}(\cover{V})$).
\begin{proof}
We first do the case with $\cover{U}\preceq \cover{V}$.  Let $f^{-1}(U)\in f^{-1}(\cover{U})$ for $U\in \cover{U}$.  Then, as $\cover{U}\preceq \cover{V}$, there is some $V\in \cover{V}$ such that $U\subseteq V$.  Then, $f^{-1}(U)\subseteq f^{-1}(V)$, and so $f^{-1}(\cover{U})\preceq f^{-1}(\cover{V})$.

Now we do the case $\cover{U}\llcurly \cover{V}$.  Let $f^{-1}(U)\in f^{-1}(\cover{U})$ for $U\in \cover{U}$.  Then, as $\cover{U}\llcurly \cover{V}$, there is some $V\in \cover{V}$ such that $\Star _{\cover{U}}(U)\subseteq V$.  Hence,
\begin{equation}
\Star _{f^{-1}(\cover{U})}(f^{-1}(U))\subseteq f^{-1}\left( \Star _{\cover{U}}(U)\right) \subseteq f^{-1}(V),
\end{equation}
where we have applied \cref{prpC.2.3}, and so $f^{-1}(\cover{U})\llcurly f^{-1}(\cover{V})$.
\end{proof}
\end{prp}
\begin{exr}{}{}
Show that if $\cover{U}\preceq \cover{V}$, then $f(\cover{U})\preceq f(\cover{V})$.
\end{exr}
\begin{prp}{}{}
Let $f\colon X\rightarrow Y$ be a function and let $\cover{U}$ and $\cover{V}$ be covers of $Y$.  Then, $f^{-1}(\cover{U}\wedge \cover{V})=f^{-1}(\cover{U})\wedge f^{-1}(\cover{V})$.
\begin{proof}
We simply `compute'.
\begin{equation}
\begin{split}
\MoveEqLeft
f^{-1}(\cover{U})\wedge f^{-1}(\cover{V}) \\
& \coloneqq \left\{ f^{-1}(U)\cap f^{-1}(V):U\in \cover{U},V\in \cover{V}\right\} \\
& =\left\{ f^{-1}(U\cap V):U\in \cover{U},V\in \cover{V}\right\} \\
& \eqqcolon f^{-1}(\cover{U}\wedge \cover{V}).
\end{split}
\end{equation}
\end{proof}
\end{prp}
Unfortunately, however, in general, it will not be the case that the image preserves star-refinements.
\begin{exr}{}{}
Find an example of covers $\cover{U}$ and $\cover{V}$ with $\cover{U}\llcurly \cover{V}$, but $f(\cover{U})$ not a star-refinement of $f(\cover{V})$.
\end{exr}
However, in special cases, it will.
\begin{prp}{}{prpC.2.3x}
Let $f\colon X\rightarrow Y$ be a function and let $\cover{U}$ and $\cover{V}$ be covers of $X$ such that $\cover{U}\llcurly \cover{V}$.  Then, if $f$ is surjective and $f^{-1}(f(U))=U$ for all $U\in \cover{U}$, then $f(\cover{U})\llcurly f(\cover{V})$.
\begin{proof}
Suppose that $f^{-1}(f(U))=U$ for all $U\in \cover{U}$.  Let $f(U)\in f(\cover{U})$.  Then, there is some $V\in \cover{V}$ such that $\Star _{\cover{U}}(U)\subseteq V$.  As $f^{-1}(f(U))=U$ for all $U\in \cover{U}$, we have
\begin{equation}
\Star _{\cover{U}}(U)=\Star _{f^{-1}(f(\cover{U}))}(U).
\end{equation}
By \cref{prpC.2.3}, we have
\begin{equation}
\Star _{f^{-1}(f(\cover{U}))}(U)=f^{-1}\left( \Star _{f(\cover{U})}(f(U))\right) .
\end{equation}
Putting this together, we get
\begin{equation}
f^{-1}\left( \Star _{f(\cover{U})}(f(U))\right) =\Star _{\cover{U}}(U)\subseteq V,
\end{equation}
and hence, as $f$ is surjective
\begin{equation*}
\Star _{f(\cover{U})}(f(U))=f\left( f^{-1}\left( \Star _{f(\cover{U})}(f(U))\right) \right) \subseteq f(V)
\end{equation*}
Thus, $f(\cover{U})\llcurly f(\cover{V})$.
\end{proof}
\end{prp}

\subsection{Uniform spaces}

\begin{dfn}{Uniform space}{UniformSpace}
A \term{uniform space}\index{Uniform space} is a set $X$ equipped with a nonempty collection $\uniformity{U}$ of covers, the \term{uniformity}\index{Uniformity}, such that
\begin{enumerate}
\item \label{UniformSpace.UpwardClosed}(Upward-closed)\index{Upward-closed} if $\cover{U}\in \uniformity{U}$ and $\cover{U}\llcurly \cover{V}$, then $\cover{V}\in \uniformity{U}$; and
\item \label{UniformSpace.DownwardDirected}(Downward-directed)\index{Downward-directed} if $\cover{U},\cover{V}\in \uniformity{U}$, then there is some $\cover{W}\in \uniformity{U}$ such that $\cover{W}\llcurly \cover{U}$ and $\cover{W}\llcurly \cover{V}$.
\end{enumerate}
\begin{rmk}
The elements of $\uniformity{U}$ are \term{uniform covers}\index{Uniform covers}.
\end{rmk}
\begin{rmk}
The intuition is that, in a given uniform cover $\cover{U}$, every element of $\cover{U}$ is `of the same size' (think $\cover{U}\coloneqq \{ B_{\varepsilon}(x):x\in \R \}$ for a \emph{fixed} $\varepsilon >0$).
\end{rmk}
\begin{rmk}
Note that, by taking $\cover{U}=\cover{V}$ in \cref{UniformSpace.DownwardDirected}, we see that, in particular, every uniform cover is star-refined by some other uniform cover.
\end{rmk}
\begin{rmk}
Note that the cover $\{ X\}$ is an element of every uniformity.  This follows from the fact that any collection of uniform covers is required to be nonempty and the fact that collections of uniform covers are upward-closed with respect to star-refinement.  (We mention this because sometimes that $\{ X\}$ is a uniform cover is taken as an axiom, in place of the requirement that the collection of uniform covers simply be nonempty.)
\end{rmk}
\begin{rmk}
Another common way to define uniform spaces are to make use of what are called \emph{entourages}\index{Entourage}.  An \emph{entourage} is a collection of relations on $X$ that has to satisfy a list of axioms.  For example, in the case of $\R ^d$, for every $\varepsilon >0$, you would have the relation $x\sim _{\varepsilon}y$ iff $\abs{x-y}<\varepsilon$.  The translation between that definition and this one is as follows:  given a relation $\sim$ in an entourage, you obtain a corresponding uniform cover $\left\{ B_x:x\in X\right\}$, where $B_x\coloneqq \{ y\in X:y\sim x\}$; and given a uniform cover $\cover{U}$, you obtain a relation $\sim$ defined by $x\sim y$ iff there is some $U\in \cover{U}$ with $x,y\in U$.  We will not use this definition, but as it is pretty common, it is worth knowing how to understand it when you encounter it.
\end{rmk}
\end{dfn}
Of incredible importance is that uniformities \emph{define} a canonical topology.  Thus, we can think of uniform spaces as topological spaces with \emph{extra structure}.
\begin{prp}{Uniform topology}{UniformTopology}
Let $\coord{X,\uniformity{U}}$ be a uniform space.  Then, for $x\in X$,
\begin{equation}
\cover{B}_x\coloneqq \left\{ \Star _{\cover{U}}(x):\cover{U}\in \uniformity{U}\right\}
\end{equation}
is a neighborhood base at $x$.  The topology defined by this neighborhood base is the \term{uniform topology}\index{Uniform topology} on $X$ with respect to $\uniformity{U}$.
\begin{rmk}
Explicitly, this means that $U\subseteq X$ is open iff for every $x\in U$ there is some uniform cover $\cover{U}$ such that $\Star _{\cover{U}}(x)\subseteq U$.  Note how this is precisely the same as our definition in $\R ^d$ when you replace the start with $B_{\varepsilon}(x)$---see \cref{OpenSetInR}.
\end{rmk}
\begin{rmk}
Unless otherwise stated, uniform spaces are \emph{always} equipped with the uniform topology.
\end{rmk}
\begin{proof}
Let $\Star _{\cover{U}_1}(x),\Star _{\cover{U}_2}(x)\in \cover{B}_x$.  Let $\cover{U}_3$ be a common star-refinement of both $\cover{U}_1$ and $\cover{U}_2$.  Define
\begin{equation}
\begin{multlined}
U\ceqq \left\{ y\in \Star _{\cover{U}_3}(x):\text{there is some }\cover{V}\in \uniformity{U}\right. \\ \left. \text{such that }\Star _{\cover{V}}(y)\subseteq \Star _{\cover{U}_3}(x)\right\} .
\end{multlined}
\end{equation}
We wish to show that $U\subseteq \Star _{\cover{U}_1}(x)\cap \Star _{\cover{U}_2}(x)$ such that (i) $x\in U$ and (ii) for every $y\in U$ there is some $\cover{V}\in \uniformity{U}$ such that $\Star _{\cover{V}}(y)\subseteq U$---see \cref{prp4.1.8} (the result which tells us how to define a topology by defining a neighborhood base).

To show that $U\subseteq \Star _{\cover{U}_1}(x)\cap \Star _{\cover{U}_2}(x)$, we show that
\begin{equation}
\Star _{\cover{U}_3}(x)\subseteq \Star _{\cover{U}_1}(x)\cap \Star _{\cover{U}_2}(x).
\end{equation}
By $1\leftrightarrow 2$ symmetry, it suffices to just prove that $\Star _{\cover{U}_3}\subseteq \Star _{\cover{U}_1}$.  By definition, we have
\begin{equation}
\Star _{\cover{U}_3}(x)\coloneqq \bigcup _{\substack{U\in \cover{U}_3 \\ x\in U}}U.
\end{equation}
So, let $U\in \cover{U}_3$ contain $x$.  Because $\cover{U}_3$ star-refines $\cover{U}_1$, there is some $V\in \cover{U}_1$ such that
\begin{equation}
\Star _{\cover{U}_3}(U)\subseteq V.
\end{equation}
In particular, $U\subseteq V$.  Then, $V$ contains $x$, and so $V\subseteq \Star _{\cover{U}_1}(x)$, and so $U\subseteq \Star _{\cover{U}_1}(x)$.  It follows that
\begin{equation}
\Star _{\cover{U}_3}(x)\subseteq \Star _{\cover{U}_1}(x),
\end{equation}
as desired.

$x\in U$ as tautologically $\Star _{\cover{U}_3}(x)\subseteq \Star _{\cover{U}_3}(x)$.

Now, let $y\in U$.  We wish to find a cover $\cover{V}\in \uniformity{U}$ such that $\Star _{\cover{V}}(y)\subseteq U$.  By definition, there is a cover $\cover{W}\in \uniformity{U}$ such that $\Star _{\cover{W}}(y)\subseteq \Star _{\cover{U}_3}(x)$.  Let $\cover{V}\in \uniformity{U}$ be a star-refinement of $\cover{W}$.  We wish to show that $\Star _{\cover{V}}(y)\subseteq U$.  So, let $z\in \Star _{\cover{V}}(y)$.  To show that $z\in U$, it suffices to show that $\Star _{\cover{V}}(z)\subseteq \Star _{\cover{U}_3}(x)$.  As $\Star _{\cover{W}}(y)\subseteq \Star _{\cover{U}_3}(x)$, it in turn suffices to show that $\Star _{\cover{V}}(z)\subseteq \Star _{\cover{W}}(y)$.  So, let $V\in \cover{V}$ be such that $z\in V$.  We wish to show that $V\subseteq \Star _{\cover{W}}(y)$.  As $\cover{V}\llcurly \cover{W}$, there is some $W\in \cover{W}$ such that $\Star _{\cover{V}}(V)\subseteq W$.  As $V\subseteq \Star _{\cover{V}}(V)$, it suffices to show that $W\subseteq \Star _{\cover{W}}(y)$, that is, it suffices to show that $y\in W$.  As $\Star _{\cover{V}}(V)\subseteq W$, it suffices to show that $y\in \Star _{\cover{V}}(V)$.  However, as $z\in \Star _{\cover{V}}(y)$, $y\in \Star _{\cover{V}}(z)\subseteq \Star _{\cover{V}}(V)$, as desired.
\begin{rmk}
Note that this proof did not make use of the upward-closed axiom.  Thus, in fact, uniform bases (see below in \cref{UniformBase}) suffice to define the uniform topology as well.
\end{rmk}
\end{proof}
\end{prp}
\begin{exm}{Discrete and indiscrete uniform spaces}{}
Just as with topological spaces, we can always put the largest and the smallest uniformity on a set $X$.  The former case, in which every cover of $X$ is a uniform cover, is the \term{discrete uniformity}\index{Discrete uniformity}, and the latter, in which the only uniform cover is $\{ X\}$, is the \term{indiscrete uniformity}\index{Indiscrete uniformity}.
\begin{exr}[breakable=false]{}{}
Show that the uniform topology with respect to the discrete uniformity is the discrete topology and that the uniform topology with respect to the indiscrete uniformity is the indiscrete topology.
\end{exr}
\end{exm}

Just as we have continuous maps between topological spaces, we have \emph{uniformly}-continuous maps between uniform spaces.
\begin{dfn}{Uniformly-continuous function}{}
Let $f\colon X\rightarrow Y$ be a function between uniform spaces.  Then, $f$ is \term{uniformly-continuous}\index{Uniformly-continuous} iff the preimage of every uniform cover is a uniform cover.
\end{dfn}
\begin{exm}{The category of uniform spaces}{}\index{Category of uniform spaces}
The category of uniform spaces is the category $\Uni$\index[notation]{$\Uni$}
\begin{enumerate}
\item whose collection of objects $\Obj (\Uni )$ is the collection of all uniform spaces;
\item with morphism set $\Mor _{\Uni}(X,Y)$ precisely the set of all uniformly-continuous functions from $X$ to $Y$;
\item whose composition is given by ordinary function composition; and
\item whose identities are given by the identity functions.
\end{enumerate}
\begin{exr}[breakable=false]{}{}
Show that the composition of two uniformly-continuous functions is uniformly-continuous.
\begin{rmk}
Note that this is something you need to check in order for $\Uni$ to actually form a category $(\Mor _{\Uni}(X,Y)$ needs to be closed under composition).  You also need to verify the identity function is uniformly-continuous, but this is trivial (the preimage of a cover is itself, so\textellipsis ).
\end{rmk}
\end{exr}
\end{exm}
\begin{dfn}{Uniform-homeomorphism}{UniformHomeomorphism}
\\
Let $f\colon X\rightarrow Y$ be a function between uniform spaces.  Then, $f$ is a \term{uniform-homeomorphism}\index{Uniform-homeomorphism} iff it is an isomorphism in $\Uni$.
\end{dfn}
\begin{exr}{}{}
Show that a function is a uniform-homeomorphism iff (i)~it is bijective, (ii)~it is uniformly-continuous, and (iii)~its inverse is uniformly-continuous.
\end{exr}
\begin{exr}{}{}
Show that if a function is uniformly-continuous, then it is continuous.
\end{exr}
\begin{exr}{}{}
Find an example of a function that is bijective and uniformly-continuous, but not a uniform-homeomorphism.
\end{exr}

\subsection[Uniform bases and the initial/final uniformities]{Uniform bases, and the initial and final uniformities}

It is usually convenient to not specify every uniform cover explicitly, but rather, to specify a certain collection of uniform covers analogous to specifying a base for a topology.\footnote{Unfortunately, the way in which one might like to define a uniformity that is analogous to generating collections for topologies doesn't work---see \cref{exm4.1.80}.}
\begin{dfn}{Uniform base}{UniformBase}
Let $X$ be a uniform space and let $\uniformity{B}$ be a collection of uniform covers of $X$.  Then, $\uniformity{B}$ is a \term{uniform base}\index{Uniform base} for the uniformity on $X$ iff the statement that a cover $\cover{U}$ is a uniform cover is equivalent to the statement that there is some $\cover{B}\in \uniformity{B}$ such that $\cover{B}\llcurly \cover{U}$.
\begin{rmk}
You should compare this to the definition of a base for a topology (\cref{Base}).
\end{rmk}
\end{dfn}
And just like with bases, the real reason uniform bases are important is because they allow us to \emph{define} uniformities.  Thus, same as before, it is important to know when a collection of covers of a set form a uniform base for some uniformity.
\begin{prp}{}{prp4.3.2}
Let $X$ be a set and let $\uniformity{B}$ be a nonempty collection of covers of $X$.  Then, there exists a unique uniformity for which $\uniformity{B}$ is a uniform base iff $\uniformity{B}$ is downward-directed with respect to $\llcurly$.
\begin{rmk}
Just as we did for bases, if a set $X$ does not a priori come with a uniformity, we will still refer to any collection of covers that is downward-directed with respect to $\llcurly$ as a \emph{uniform base}.
\end{rmk}
\begin{proof}
$(\Rightarrow )$ Suppose that there exists a uniformity for which $\uniformity{B}$ is a uniform base.  Let $\cover{B},\cover{C}\in \uniformity{B}$.  Then, there is certainly some uniform cover $\cover{U}$ which star-refines both $\cover{B}$ and $\cover{C}$ (recall that covers in $\uniformity{B}$ are a priori taken to be uniform covers).  However, because $\cover{U}$ is a uniform cover and $\uniformity{B}$ is a uniform base, there is some $\cover{D}\in \uniformity{B}$ such that $\cover{D}\llcurly \cover{U}$.  As $\cover{U}$ star-refines both $\cover{B}$ and $\cover{C}$, it follows that $\cover{D}$ does as well.  Thus, $\uniformity{B}$ is indeed downward-directed with respect to star-refinement.

\blankline
\noindent
$(\Leftarrow )$ Suppose that $\uniformity{B}$ is downward-directed with respect to $\llcurly$.  Define $\uniformity{U}$ to be the collection of covers that are star-refined by some element of $\uniformity{B}$.  By the definition of uniform bases, this was the only possibility.  As $\uniformity{B}$ is nonempty and downward-directed with respect to $\llcurly$, it follows that $\uniformity{U}$ contains $\uniformity{B}$, and in particular is nonempty.  $\uniformity{U}$ is upward-closed with respect to $\llcurly$ because if $\cover{U}$ is a uniform cover and star-refines $\cover{V}$, then there is some $\cover{B}\in \uniformity{B}$ that star-refines $\cover{U}$, and hence in turn star-refines $\cover{V}$.  We now check that it is downward-directed with respect to $\llcurly$.  If $\cover{U}$ and $\cover{V}$ are covers, then there are $\cover{B},\cover{C}\in \uniformity{B}$ that star-refine $\cover{U}$ and $\cover{V}$ respectively.  Because $\uniformity{B}$ is downward-directed, there is then some $\cover{D}\in \uniformity{B}$ which star-refines both $\cover{B}$ and $\cover{C}$, and hence both $\cover{U}$ and $\cover{V}$.
\end{proof}
\end{prp}
As we mentioned above at the end of the proof of \cref{UniformTopology}, uniform bases define the uniform topology just as well as the entire uniformity.
\begin{crl}{}{}
Let $\uniformity{B}$ be a uniform base for the uniform space $X$.  Then, for $x\in X$,
\begin{equation}
\cover{B}_x\coloneqq \left\{ \Star _{\cover{B}}(x):\cover{B}\in \uniformity{B}\right\}
\end{equation}
is a neighborhood base at $x$ for the uniform topology.
\end{crl}
\begin{exr}{}{}
Show that $\uniformity{U}\coloneqq \left\{ \left\{ \{ x\} :x\in X\right\} \right\}$ is a uniform base for the discrete uniformity.
\begin{rmk}
That is, the collection consisting of just a \emph{single} open cover, which itself is just the collection of all singletons, forms a uniform base.  In other words, you need to check that $\cover{U}$ star-refines itself.\footnote{Of course, while $\llcurly$ in general is not reflexive, that doesn't mean we can't at least have $\cover{U}\llcurly \cover{U}$ \emph{some} of the time.}
\end{rmk}
\begin{rmk}
The `dual' result for the indiscrete uniformity is trivial---the indiscrete uniformity by definition only has a single cover to begin with (namely $\{ X\}$), and that single cover certainly forms a uniform base for itself.
\end{rmk}
\end{exr}
We will want to check that two collections of uniform covers are the same by just looking at uniform bases.  We did not present it because we did not need to make use of it, but of course there is an analogous result for bases of topological spaces.
\begin{prp}{}{prp1.6}
Let $X$ be a set, and let $\uniformity{B}$ and $\uniformity{C}$ be uniform bases on $X$.  Then, $\uniformity{B}$ and $\uniformity{C}$ determine the same uniformity iff for every $\cover{B}\in \uniformity{B}$, there is some $\cover{C}\in \uniformity{C}$ with $\cover{C}\llcurly \cover{B}$; and for every $\cover{C}\in \uniformity{C}$, there is some $\cover{B}\in \uniformity{B}$ with $\cover{B}\llcurly \cover{C}$.
\begin{proof}
$(\Rightarrow )$ Suppose that $\uniformity{B}$ and $\uniformity{C}$ determine the same uniformity.  Let $\cover{B}\in \uniformity{B}$.  Then, $\cover{B}$ is in particular in the uniformity generated by $\uniformity{B}$, and hence in the uniformity generated by $\uniformity{C}$.  Thus, there is some $\cover{C}\in \uniformity{C}$ such that $\cover{C}\llcurly \cover{B}$.  By $\uniformity{B}\leftrightarrow \uniformity{C}$ symmetry, the other result is true as well.

\blankline
\noindent
$(\Leftarrow )$ Suppose that for every $\cover{B}\in \uniformity{B}$, there is some $\cover{C}\in \uniformity{C}$ with $\cover{C}\llcurly \cover{B}$; and for every $\cover{C}\in \uniformity{C}$, there is some $\cover{B}\in \uniformity{B}$ with $\cover{B}\llcurly \cover{C}$.  Let $\cover{U}$ be a uniform cover in the uniformity determined by $\uniformity{B}$.  Then, there is some $\cover{B}\in \uniformity{B}$ such that $\cover{B}\llcurly \cover{U}$.  By the hypothesis, then, there is some $\cover{C}\in \uniformity{C}$ with $\cover{C}\llcurly \cover{B}\llcurly \cover{U}$.  Thus, $\cover{U}$ is in the uniformity determined by $\uniformity{C}$.  By $\uniformity{B}\leftrightarrow \uniformity{C}$ symmetry, the reverse inclusion is also true.
\end{proof}
\end{prp}
We will also want to check whether a function is uniformly-continuous by simply looking at a uniform base.
\begin{prp}{}{prpB.3.4}
Let $f\colon \coord{X,\uniformity{U}}\rightarrow \coord{Y,\uniformity{V}}$ be a function between uniform spaces and let $\uniformity{C}$ be a uniform base for $\uniformity{V}$.  Then, $f$ is uniformly-continuous iff $f^{-1}(\cover{C})\in \uniformity{U}$ for each $\cover{C}\in \uniformity{C}$.
\begin{proof}
$(\Rightarrow )$ There is nothing to check (because every open cover in a uniform base is itself a uniform cover).

\blankline
\noindent
$(\Leftarrow )$ Suppose that $f^{-1}(\cover{C})\in \uniformity{U}$ for each $\cover{C}\in \uniformity{C}$.  We need to show that the preimage of \emph{every} uniform cover is a uniform cover.  So, let $\cover{V}\in \uniformity{V}$.  Then, there is some $\cover{C}\in \uniformity{C}$ such that $\cover{C}\llcurly \cover{V}$.  Then, by \cref{prpB.2.12}, it follows that $f^{-1}(\cover{C})\llcurly f^{-1}(\cover{V})$.  As $f^{-1}(\cover{C})\in \uniformity{U}$ and $\uniformity{U}$ is upward-closed with respect to $\llcurly$, it follows that $f^{-1}(\cover{V})\in \uniformity{U}$, so that $f$ is uniformly-continuous.
\end{proof}
\end{prp}

A uniform space does not start its life as a topological space---instead, it obtains a canonical topology from its uniformity.  In particular, it does not make sense a priori to just restrict to open covers.  On the other hand, once we specify the uniform covers, a topology is determined, and then, it turns out (as the following proposition shows), that it suffices to just look at \emph{open} uniform covers, more precisely, those uniform covers obtained by taking the interior of the covers in your uniform base.
\begin{prp}{}{lma5.1.16}
Let $\uniformity{B}$ be a uniform base on a set $X$ and let $\cover{B}\in \uniformity{B}$.  Then, (i)~$\Int (\cover{B})\coloneqq \{ \Int (B):B\in \cover{B}\}$ is (still) a cover of $X$ and (ii)~$\Int (\uniformity{B})\coloneqq \left\{ \Int (\cover{B}):\cover{B}\in \uniformity{B}\right\}$ is (still) a uniform base on $X$ that generates the same uniformity as $\uniformity{B}$.
\begin{proof}
Let $\cover{B}\in \uniformity{B}$ and let $x\in X$.  Let $\cover{C}$ be a star-refinement of $\cover{B}$.  Let $C\in \cover{C}$ contain $x$ and let $B\in \cover{B}$ be such that $\Star _{\cover{C}}(C)\subseteq B$.  Then,
\begin{equation}
x\in \Star _{\cover{C}}(x)\subseteq \Star _{\cover{C}}(C)\subseteq B,
\end{equation}
and so $x\in \Int (B)$ (because $\Star _{\cover{C}}(x)$ is a neighborhood of $x$), and so indeed $\Int (\cover{B})$ is a cover of $X$.

Let $\cover{B},\cover{C}\in \uniformity{B}$ and let $\cover{D}$ be a common star-refinement of $\cover{B}$ and $\cover{C}$.  We show that $\Int (\cover{D})$ is a common star-refinement of $\Int (\cover{B})$ and $\Int (\cover{C})$.  Because of $\cover{B}\leftrightarrow \cover{C}$ symmetry, it suffices to show that it is a star-refinement of $\Int (\cover{B})$.  So, let $\Int (D)\in \Int (\cover{D})$.  Then, there is some $B\in \cover{B}$ such that $\Star _{\cover{D}}(D)\subseteq B$, and so because union of the interiors is contained in the interior of the union (\cref{exr3.4.53}.\cref{enm3.4.53.ii}), we have
\begin{equation}
\begin{split}
\Star _{\Int (\cover{D})}(\Int (D)) & \subseteq \Star _{\Int (\cover{D})}(D)\subseteq \Int \left( \Star _{\cover{D}}D\right) \\
& \subseteq \Int (B),
\end{split}
\end{equation}
and so indeed $\Int (\cover{D})$ star-refines $\Int (\cover{B})$.

It remains to show that $\uniformity{B}$ and $\Int (\uniformity{B})$ induce the same uniform structure.  To show this, we apply \cref{prp1.6}.  Let $\cover{B}\in \uniformity{B}$ and let $\Int (\cover{C})\in \Int (\uniformity{B})$.  If $\cover{D}\in \uniformity{B}$ is a star-refinement of $\cover{B}$, then $\Int (\cover{D})\in \Int (\uniformity{B})$ certainly star-refines $\cover{B}$ as well.  In the other direction, we wish to find a cover in $\uniformity{B}$ that star-refines $\Int (\cover{C})$.  So, let $\cover{D}\in \uniformity{B}$ be a star-refinement of $\cover{C}$ and let $\cover{E}\in \uniformity{B}$ be in turn a star-refinement of $\cover{D}$.  We wish to show that $\cover{E}\in \uniformity{B}$ star-refines $\Int (\cover{C})$.  So, let $E\in \cover{E}$.  Then, there is some $D\in \cover{D}$ such that $\Star _{\cover{E}}(E)\subseteq D$.  In turn, there is some $C\in \cover{C}$ such that $\Star _{\cover{D}}(D)\subseteq C$.  We wish to show that $\Star _{\cover{E}}(E)\subseteq \Int (C)\in \Int (\cover{C})$.  So, let $x\in \Star _{\cover{E}}(E)$.  Then, $x\in D$, and so $\Star _{\cover{D}}(x)\subseteq \Star _{\cover{D}}(D)\subseteq C$.  As $\Star _{\cover{D}}(x)$ is a neighborhood base of $x$, this implies that $x\in \Int (C)$.  Thus, $\Star _{\cover{E}}(E)\subseteq \Int (C)$, and so $\cover{E}$ star-refines $\Int (\cover{C})$, as desired.
\end{proof}
\end{prp}

As the idea of a uniformity is inherently `global' in nature, there really isn't a way to define a uniformity that is analogous to the method of defining a topology by specifying neighborhood bases.\footnote{Of course one cannot hope to make a statement like this precise (What does it mean for a method of defining a uniformity to be ``analogous to'' a method of defining a topology?), but hopefully the intuition is clear.  All elements in a uniform cover, no matter where they are in the space, are supposed to be thought of as the same size.  How could one hope to encode the idea of two sets living `far away' are of the same size by the specification of local information alone?}  Furthermore, a bit more unexpected, and quite a bit more unfortunate, is that we cannot generate a uniformity by simply declaring any collection of covers to be uniform in a way analogous to specifying a generating collection of a topology.
\begin{exm}{A collection of covers for which there is no unique minimal uniformity containing the collection}{exm4.1.80}
Before we begin with the actual counter-example, let us elaborate on what we are trying to do.  If you look back to generating collections for topologies (\cref{GeneratingCollection}), you'll see that what we would like to be true is the following:  if $\uniformity{S}$ is a nonempty collection of covers on $X$, then there is a unique uniformity $\uniformity{U}$ such that (i)~every cover in $\uniformity{S}$ is uniform with respect to $\uniformity{U}$, and (ii)~that if $\uniformity{U}'$ is any other uniformity for which this is true then $\uniformity{U}'\supseteq \uniformity{U}$.  The objective then is to construct a collection $\uniformity{S}$ for which there exists no such uniformity.

Define $X\coloneqq \{ 1,2,3\}$ and
\begin{equation}
\cover{U}\coloneqq \{ \{ 1,2\} ,\{ 2,3\} \} .
\end{equation}
We construct incomparable uniformities $\uniformity{U}_1$ and $\uniformity{U}_2$ which contain $\{ \cover{U}\}$ and for which there is \emph{no} uniformity $\uniformity{U}$ containing $\{ \cover{U}\}$ and contained in both $\uniformity{U}_1$ and $\uniformity{U}_2$.  This is enough, because if there were a minimal uniformity containing $\{ \cover{U}\}$ minimality would dictate it be contained in both $\uniformity{U}_1$ and $\uniformity{U}_2$.

Define
\begin{equation}
\cover{U}_1\coloneqq \left\{ \{ 1\} ,\{ 2\} ,\{ 3\} ,\{ 1,2\} \right\}
\end{equation}
and
\begin{equation}
\cover{U}_2\coloneqq \left\{ \{ 1\} ,\{ 2\} ,\{ 3\} ,\{ 2,3\} \right\} .\footnote{Draw a picture.}
\end{equation}
These covers both star-refine themselves, and hence both $\{ \cover{U}_1\}$ and $\{ \cover{U}_2\}$ form uniform bases:  denote by $\uniformity{U}_1$ and $\uniformity{U}_2$ the uniformities generated by these respective uniform bases (so that $\uniformity{U}_k$ is the collection of covers which are star-refined by $\cover{U}_k$).  As $\cover{U}$ is star-refined by both $\cover{U}_1$ and $\cover{U}_2$, $\cover{U}$ is contained in both $\uniformity{U}_1$ and $\uniformity{U}_2$.  It remains to check that there is no uniformity contained in both which contains $\cover{U}$.

So, let $\uniformity{U}$ be a uniformity containing $\cover{U}$ and contained in both $\uniformity{U}_1$ and $\uniformity{U}_2$.  $\uniformity{U}$ must then contain a star-refinement $\cover{V}$ of $\cover{U}$.  As $\cover{V}$ is in both $\uniformity{U}_1$ and $\uniformity{U}_2$, $\cover{V}$ must then in turn be star-refined by both $\cover{U}_1$ and $\cover{U}_2$.  However, there is no cover of $X$ which star-refines $\cover{U}$ and is star-refined by both $\cover{U}_1$ and $\cover{U}_2$:\footnote{$\Star _{\cover{U}_1}(\{ 2\} )=\{ 1,2\}$ and $\Star _{\cover{U}_2}(\{ 2\} )=\{ 2,3\}$, and so $\cover{V}$ would have to contain a set which contains $\{ 1,2\}$ and a set which contains $\{ 2,3\}$.  This implies that the star of either of these sets with respect to $\cover{V}$ is $\{ 1,2,3\}$, which is not contained in any element of $\cover{U}$.}  a contradiction.  Therefore, there is no such $\uniformity{U}$.
\end{exm}

With topological spaces, we could define a topology by specifying the closure or interior, or defining a notion of convergence.  To the best of my knowledge, there are no analogous methods for defining uniformities.  As for the initial and final topologies, however, there are analogous constructions (which you might not have expected as we no longer can simply `generate' uniformities like we could with topologies).
\begin{prp}{Initial uniformity}{InitialUniformity}
Let $X$ be a set, let $\cover{Y}$ be an indexed collection of uniform spaces, and for each $Y\in \cover{Y}$ let $f_Y:X\rightarrow Y$ be a function.  Then, there exists a unique uniformity $\uniformity{U}$ on $X$, the \term{initial uniformity} with respect to $\{ f_Y:Y\in \cover{Y}\}$, such that
\begin{enumerate}
\item $f_Y:X\rightarrow Y$ is uniformly-continuous with respect to $\uniformity{U}$ for all $Y\in \cover{Y}$; and
\item if $\uniformity{U}'$ is another uniformity for which each $f_Y$ is uniformly-continuous, then $\uniformity{U}\subseteq \uniformity{U}'$.
\end{enumerate}

Furthermore,
\begin{enumerate}
\item if $\uniformity{C}_Y$ is a uniform base for $\cover{Y}$, then the collection of all finite meets of covers of the form $f_Y^{-1}(\cover{V})$ with $\cover{V}\in \uniformity{C}_Y$ is a uniform base for $\uniformity{U}$; and
\item the uniform topology of the initial uniformity is the same as the initial topology.
\end{enumerate}
\begin{rmk}
In other words, the initial uniformity is the smallest uniformity for which each $f_Y$ is uniformly-continuous.
\end{rmk}
\begin{rmk}
But what about the largest such uniformity?  Well, the largest such uniformity is always going to be the discrete uniformity, which is not very uninteresting.  This is how you remember whether the initial uniformity is the smallest or largest---it can't be the largest because the discrete uniformity always works.
\end{rmk}
\begin{proof}
Let $\uniformity{C}_Y$ be any uniform base for $Y$ and let $\uniformity{C}$ be the collection of all finite meets of covers of the form $f_Y^{-1}(\cover{V})$ for $\cover{V}\in \uniformity{C}_Y$ and $Y\in \collection{Y}$.  We wish to check that this is a uniform base.

So, let $f_{Y_1}^{-1}(\cover{V}_1)\wedge \cdots \wedge f_{Y_m}^{-1}(\cover{V}_m)$ and $f_{Y_{m+1}}^{-1}(\cover{V}_m)\wedge \cdots \wedge f_{Y_{m+n}}^{-1}(\cover{V}_{m+n})$ be two elements of $\uniformity{C}$.  Let $\cover{W}_k$ be a star-refinement of $\cover{V}_k$ in $\uniformity{C}_{Y_k}$.  Then, because preimage preserves star-refinement (\cref{prpB.2.12}) and the meet is `compatible' with the relation of star-refinement (\cref{exr4.2.35}), it follows that
\begin{equation}
f_{Y_1}^{-1}(\cover{W}_1)\wedge \cdots \wedge f_{Y_{m+n}}^{-1}(\cover{W}_{m+n})
\end{equation}
is a star-refinement of the two initial covers.

Thus, $\uniformity{B}$ is a uniform base.  Denote by $\uniformity{U}$ the uniformity generated by $\uniformity{U}$.

By construction, $f_Y^{-1}(\cover{V})$ is a uniform cover for all $\cover{V}\in \uniformity{C}_Y$, so each $f_Y$ is certainly uniformly-continuous.  On the other hand, if $\uniformity{U}'$ were a uniformity for which each $f_Y$ were uniformly-continuous, then $\uniformity{U}'$ would have to contain every cover of the form $f_Y^{-1}(\cover{V})$ for $\cover{V}\in \uniformity{C}_Y$ a uniform cover of $Y$.  As uniformities are closed under meets,\footnote{Why?} $\uniformity{U}'$ would have to contain $\uniformity{C}$ and hence $\uniformity{U}$, as desired.
\begin{exr}[breakable=false]{}{}
Show that the initial uniformity is unique.
\end{exr}

We finally check that the uniform topology of the initial uniformity agrees with the initial topology.  First of all, each $f_Y$ is uniformly-continuous, hence continuous, and so as the initial topology is the coarsest (i.e.~smallest) topology for which each $f_Y$ is continuous, the initial topology must be coarser than the uniform topology of the initial uniformity.

In the other direction, let $U\subseteq X$ be open with respect to the uniform topology of the initial uniformity.  By the defining result of the uniform topology (\cref{UniformTopology}), this means that for all $x\in U$, there is some $\cover{B}\in \uniformity{B}$ such that $x\in \Star _{\cover{B}}(x)\subseteq U$.  Write $\cover{B}=f_{Y_1}^{-1}(\cover{V}_1)\wedge \cdots \wedge f_{Y_m}^{-1}(\cover{V}_m)$.  As, for points, the star with respect to a wedge is the intersection of the stars (\cref{prp4.2.27}),
{\small
\begin{equation*}
\begin{split}
\MoveEqLeft
\Star _{\cover{B}}(x)=\Star _{f_{Y_1}^{-1}(\cover{V}_1)}(x)\cap \cdots \cap \Star _{f_{Y_m}^{-1}(\cover{V}_m)}(x) \\
& =f_{Y_1}^{-1}\left( \Star _{\cover{V}_1}(f_{Y_1}(x))\right) \cap \cdots \cap f_{Y_m}^{-1}\left( \Star _{\cover{V}_m}(f_{Y_m}(x))\right) .
\end{split}
\end{equation*}
}
where we have applied \cref{prpC.2.3}.  As each $\Star _{\cover{V}_k}(f_{Y_k}(x))$ is a neighborhood of $f_{Y_k}(x)\in Y_k$, $f_{Y_k}^{-1}\left( \Star _{\cover{V}_k}(f_{Y_k}(x))\right)$ is a neighborhood of $x\in X$ for the initial topology, and so, by the above equality, $\Star _{\cover{B}}(x)$ is likewise a neighborhood of $x\in X$ for the initial topology.  This shows that every point in $U$ has a neighborhood for the initial topology contained in $U$, and hence $U$ is likewise open for the initial topology, as desired.
\end{proof}
\end{prp}
Of course, we have a result that is perfectly analogous to \cref{prp3.4.6} (a function is continuous iff its composition with each $f_Y$ is continuous).
\begin{prp}{}{prp4.2.54}
Let $X$ have the initial uniformity with respect to the collection $\{ f_Y:Y\in \cover{Y}\}$, let $Z$ be a uniform space, and let $f\colon Z\rightarrow X$ be a function.  Then, $f$ is uniformly-continuous iff $f_Y\circ f$ is uniformly-continuous for all $Y\in \cover{Y}$.  Furthermore, the initial uniformity is the unique uniformity with this property.
\begin{proof}
We leave the proof as an exercise.
\begin{exr}{}{}
Prove this result, using the proof of \cref{prp3.4.6} (the analogous result for the initial topology) as guidance.
\end{exr}
\end{proof}
\end{prp}
And just as we had with topological spaces, there is a `dual' version of the initial uniformity.
\begin{prp}{Final uniformity}{FinalUniformity}
Let $X$ be a set, let $\cover{Y}$ be an indexed collection of uniform spaces, and for each $Y\in \cover{Y}$ let $f_Y:Y\rightarrow X$ be a function.  Then, there exists a unique uniformity $\uniformity{U}$ on $X$, the \term{final uniformity}\index{Final uniformity} with respect to $\{ f_Y:Y\in \cover{Y}\}$, such that
\begin{enumerate}
\item $f_Y:Y\rightarrow X$ is uniformly-continuous with respect to $\uniformity{U}$; and
\item if $\uniformity{U}'$ is another uniformity for which each $f_Y$ is uniformly-continuous, then $\uniformity{U}\supseteq \uniformity{U}'$.
\end{enumerate}
Furthermore, the uniform topology of the final uniformity is coarser than the final topology.
\begin{rmk}
In other words, the final uniformity is the largest uniformity for which each $f_Y$ is uniformly-continuous.
\end{rmk}
\begin{rmk}
But what about the smallest such uniformity?  Well, the smallest such uniformity is always going to be the indiscrete uniformity, which is not very interest.  This is how you remember whether the final uniformity is the smallest or largest---it can't be the smallest because the indiscrete uniformity always works.
\end{rmk}
\begin{rmk}
For both the initial and final topologies, as well as the initial uniformity, we had a relatively concrete description of the induced structure in terms of the structure on the elements of $\cover{Y}$.  To the best of my knowledge, there is no such analogous description for the final uniformity.  For better insight as to why that is, see the proof.
\end{rmk}
\begin{wrn}
Warning:  The uniform topology of the final uniformity can be \emph{strictly} coarser than the final topology---see the following counter-example.
\end{wrn}
\begin{proof}\footnote{Proof adapted from \cite[Theorem 1.2.1.1]{Preuss}.}
Let $\cover{Z}$ be an indexed collection containing a copy of a uniform space $Z$ for all functions $g_Z\colon X\rightarrow Z$ that have the property that $g_Z\circ f_Y$ is uniformly-continuous for all $Y\in \cover{Y}$.  Let $\uniformity{U}$ be the initial uniformity with respect to $\{ g_Z:Z\in \cover{Z}\}$.

By the previous result \cref{prp4.2.54}, each $f_Y$ is uniformly-continuous $\uniformity{U}$.

Let $\uniformity{U}'$ be another uniformity on $X$ for which each $f_Y$ is uniformly-continuous.  Let $\cover{U}'\in \uniformity{U}'$.  We wish to show that $\cover{U}'\in \uniformity{U}$.  To shows this, it suffices to show that $\id _X\colon \coord{X,\uniformity{U}}\rightarrow \coord{X,\uniformity{U}'}$ is uniformly-continuous.  However, $\id _X\circ f_Y=f_Y$ is uniformly-continuous by assumption for all $Y\in \cover{Y}$, and so $\id _X$ is among the $g_Z$s, and in particular, is uniformly-continuous, as desired.
\begin{exr}[breakable=false]{}{}
Show that the final uniformity is unique.
\end{exr}

We finally check that the uniform topology of the final uniformity is coarser than the final topology.  First of all, each $f_Y$ is uniformly-continuous, hence continuous, and so as the final topology is the finest topology for which each $f_Y$ is continuous, the final topology must be finer than the uniform topology of the final uniformity.
\end{proof}
\end{prp}
\begin{exm}{The final topology need not agree with the uniform topology of the final uniformity}{}\footnote{Adapted from \cite[Exercise 2.1.10]{Kunzi}.}
Let $\q \colon \R \rightarrow \{ (-\infty ,0],(0,\infty )\}\eqqcolon X$ denote the quotient map.  To simplify notation, let us write $x_1\coloneqq (-\infty ,0]$ and $x_2\coloneqq (0,\infty )$.  A subset of $X$ is open in the final topology iff its preimage under $\q$ is open.  It follows that the final topology on $X$ is $\{ \emptyset ,\{ x_2\} ,X\}$.

Obviously, every cover of $X$ either contains $X$ itself or it does not.  If it does not, then it must contain separately $x_1$ and $x_2$, and so must be $\{ \{ x_1\} ,\{ x_2\} \}$ (or this together with the empty-set).  The preimage of this cover under $\q$ is the cover of $\q$ $\{ (-\infty ,0],(0,\infty )\}$, which is never a uniform cover of $\R$.\footnote{To be a uniform cover of $\R$, it must be star-refined by a cover of $\varepsilon$-balls, and the star of an $\varepsilon$-ball centered at $0\in \R$ is contained in neither $(-\infty ,0]$ nor $(0,\infty )$.}  Thus, every uniform cover of $X$ (in the final uniformity) must contain $X$.  It follows that the uniform topology of the final uniformity is the indiscrete topology, which is strictly coarser than the final topology.
\end{exm}
And the result `dual' to \cref{prp4.2.54}:
\begin{prp}{}{}
Let $X$ have the final uniformity with respect to the collection $\{ f_Y:Y\in \cover{Y}\}$, let $Z$ be a uniform space, and let $f\colon X\rightarrow Z$.  Then, $f$ is uniformly-continuous iff $f_Y\circ f$ is uniformly-continuous for all $Y\in \cover{Y}$.  Furthermore, the final uniformity is the unique uniformity with this property.
\begin{proof}
We leave the proof as an exercise.
\begin{exr}[breakable=false]{}{}
Prove this result, using the proof of \cref{prp3.4.34x} (the analogous result for the final topology) as guidance 
\end{exr}
\end{proof}
\end{prp}
Of course, just as with topological spaces, a key application of the initial and final uniformities is that they provide canonical uniformities on subsets, quotients, products, and disjoint-unions.  The definitions and results are completely analogous to the case of topological spaces, and so we omit stating them explicitly.

After having discussed the real numbers themselves as a uniform space, we show below (\cref{exm4.2.85}) that functions even as nice as polynomials are not uniformly continuous.  On the other hand, when restricted to \emph{quasicompact} sets, all continuous functions are uniformly-continuous.
\begin{prp}{Cantor-Heine Theorem}{prp4.2.73}\index{Cantor-Heine Theorem}
\\
Let $f\colon X\rightarrow Y$ be a continuous function between uniform spaces and let $K\subseteq X$ be quasicompact.  Then, $\restr{f}{K}:K\rightarrow Y$ is uniformly-continuous.
\begin{rmk}
Ideally we would have presented this result shortly after giving the definition of uniformly-continuous functions, however, we do technically need the notion of the subspace uniformity to state this result.
\end{rmk}
\begin{proof}
Let $\cover{V}$ be a uniform cover of $Y$.  By \cref{lma5.1.16,prpB.3.4} (the the interior covers define the same uniformity and it suffices to check uniform-continuity on a uniform base), without loss of generality we can take $\cover{V}$ to be an open cover.  We would like to show that $\restr{f}{K}^{-1}(\cover{V})=f^{-1}(\cover{V})\wedge \{ K\}$ is a uniform-cover of $K$.  To do this, by upward-closedness, it suffices to find a uniform-cover of $K$ which star-refines $f^{-1}(\cover{V})\wedge \{ K\}$.

$f^{-1}(\cover{V})$, while not necessarily a uniform cover of $X$, will certainly be an open cover, and in particular will be an open cover of $K$.  So, for $x\in K$, let $V_x\in \cover{V}$ be such that $x\in f^{-1}(V_x)$.  Then, choose an open\footnote{Applying \cref{lma5.1.16} again.} uniform cover $\cover{U}_x$ of $X$ such that\footnote{This implicitly uses the fact that the subspace topology is the same as the uniform topology of the subspace uniformity---see \cref{InitialUniformity}.}
\begin{equation}
\Star _{\cover{U}_x\wedge \{ K\}}(x)\subseteq f^{-1}(V_x)\cap K.
\end{equation}
As
\begin{equation}
\left\{ \Star _{\cover{U}_x\wedge \{ K\}}(x):x\in K\right\}
\end{equation}
is an open cover of $K$, there is a finite subcover.  So, let $x_1,\ldots ,x_m\in K$ be such that
\begin{equation}
\left\{ \Star _{\cover{U}_{x_k}\wedge \{ K\}}(x_k):1\leq k\leq m\right\}
\end{equation}
is an open cover of $K$.  Let $\cover{U}$ be a common star-refinement of each $\cover{U}_k$\footnote{It is here that the finiteness given to us by quasicompactness is key.}.  Then, for $x\in K$, if $x\in \Star _{\cover{U}_k\wedge \{ K\}}(x_k)$, then
\begin{equation}\label{eqn4.1.98}
\Star _{\cover{U}_0\wedge \{ K\}}(x)\subseteq f^{-1}(V_{x_k})\cap K
\end{equation}
for all $x\in K$.

Let $\cover{U}_0$ be in turn a star-refinement of $\cover{U}_0$.  We show that $\cover{U}\wedge \{ K\}$ is a star-refinement of $f^{-1}(\cover{V})\wedge \{ K\}$.  So, let $U\in \cover{U}$.  Let $U_0\in \cover{U}_0$ be such that $\Star _{\cover{U}}(U)\subseteq U_0$.  Let $x\in U$ and pick $k$ so that \eqref{eqn4.1.98} holds.  Then,
\begin{equation}
\begin{split}
\Star _{\cover{U}\wedge \{ K\}}(U\cap K) & \subseteq U_0\cap K\subseteq \Star _{\cover{U}_0\wedge \{ K\}}(x) \\
& \subseteq f^{-1}(V_{x_k})\cap K.
\end{split}
\end{equation}
\end{proof}
\end{prp}

\horizontalrule

\begin{exm}{The real numbers}{}
The real numbers have a canonical uniformity (and in fact, we will see below that this is just a special base of a more general construction):  let $\varepsilon >0$ and define
\begin{equation}
\cover{U}_{\varepsilon}\coloneqq \left\{ B_{\varepsilon}(x):x\in \R \right\}
\end{equation}
\index[notation]{$\cover{U}_{\varepsilon}$} and
\begin{equation}
\uniformity{U}\coloneqq \left\{ \cover{U}_{\varepsilon}:\varepsilon >0\right\} .
\end{equation}
\begin{exr}[breakable=false]{}{}
Show that $\uniformity{U}$ is a uniform base on $\R$.
\end{exr}
\begin{exr}[breakable=false]{}{}
Show that $f\colon \R \rightarrow \R$ is uniformly-continuous iff for every $\varepsilon >0$ there is some $\delta >0$ such that $f(B_{\delta}(x))\subseteq B_{\varepsilon}(f(x))$ for every $x\in \R$.
\begin{rmk}
Compare this with the condition for $f\colon \R \rightarrow \R$ being \emph{continuous at $a\in \R$} given in \cref{exr3.4.5}.\cref{enm3.4.5.iii}.  We will spell-it-out here for convenience:
\begin{textequation}
$f\colon \R \rightarrow \R$ is continuous iff for every $x\in \R$ and for every $\varepsilon >0$ there is some $\delta >0$ such that $f(B_{\delta}(x))\subseteq B_{\varepsilon}(f(x))$.
\end{textequation}
The key difference between continuity and uniform-continuity is \emph{the location in which the quantification ``for every $x\in \R$'' appears}.  In the former (just continuous case), your choice of $\delta$ is \emph{allowed to depend on $x$}, whereas to be uniformly-continuous, \emph{a single $\delta$ has to `work' for every $x\in \R$}.
\end{rmk}
\begin{rmk}
The result you just proved characterizing uniform-continuity in $\R$ is often taken as the definition of uniform-continuity.  Had we studied uniform-continuity in the context of just the real numbers first (as opposed to in the context of uniform spaces), we would have done the same.  My personal feeling, however, is that uniform continuity is not that incredibly important, at least not to the point where it is worth going out of our way to discuss it just in the context of $\R$.  The real reason we discuss uniform spaces is for the purpose of discussing Cauchyness and completeness, not uniform continuity per se (and also of course because a huge collection of examples of topological spaces are canonically uniform spaces).
\end{rmk}
\end{exr}
\end{exm}
\begin{exm}{A uniformly-continuous function}{}
By \cref{prp4.2.73}, any continuous function restricted to a quasicompact set will be uniformly-continuous, so, for example the function $x\mapsto x^2$ is uniformly-continuous on $[0,1]$.  However, be careful:  it is not uniformly-continuous on all of $\R$.
\end{exm}
\begin{exm}{A continuous function that is not uniformly-continuous}{exm4.2.85}
Define $f\colon \R \rightarrow \R$ by $f(x)\coloneqq x^2$.  Of course $f$ is continuous (because it is the product of continuous functions---see \cref{exr3.4.12}).

On the other hand, we show that $f$ does not satisfy the condition given in the previous exercise.  Take $\varepsilon \coloneqq 1$.  Then, if $f$ were uniformly-continuous, there should be some $\delta >0$ such that
\begin{equation}
\begin{split}
\left\{ x^2:\abs{x-x_0}<\delta \right\} & \eqqcolon f(B_\delta (x_0))\subseteq B_{\varepsilon}(f(x_0))\\
& \coloneqq \left\{ x\in \R :\abs{x-x_0^2}<1\right\}
\end{split} 
\end{equation}
for all $x_0\in \R$.  However, $(x_0+\frac{1}{2}\delta )^2$ is an element of the left-hand side, but
\begin{equation}
\left( x_0+\tfrac{1}{2}\delta \right) ^2-x_0^2=\delta x_0+\tfrac{1}{4}\delta ^2
\end{equation}
is not less than $1$ in general (for example, for $x_0=\frac{1}{\delta}(1-\tfrac{1}{4}\delta ^2)$.

On the other hand, by \cref{prp4.2.73}, $f$ restricted to any closed interval is uniformly-continuous.
\begin{rmk}
In particular, \emph{the product of uniformly-continuous functions is not necessarily uniformly-continuous}.  This sucks, but alas, what's a mathematician to do?
\end{rmk}
\end{exm}

\section{Semimetric spaces and topological groups}

As was previously mentioned, one big motivation for studying uniform spaces is that a huge collection of very important examples of topological spaces admit a canonical uniformity.  Two such families of spaces that we will study are \emph{semimetric spaces} and \emph{topological groups}.

\subsection{Semimetric spaces}

Before we talk about any sort of uniformity, we had better first say what we mean by \emph{semimetric space}.
\begin{dfn}{Semimetric and metric}{Semimetric}
Let $X$ be a set.  Then, a \term{semimetric}\index{Semimetric} on $X$ is a function $\metric:X\times X\rightarrow \R _0^+$ such that
\begin{enumerate}
\item $\metric{x}{x}=0$;
\item (Symmetry) $\metric{x}{y}=\metric{y}{x}$;
\item (Triangle Inequality) $\metric{x}{z}\leq \metric{x}{y}+\metric{y}{z}$.
\end{enumerate}
$\metric$ is a \term{metric}\index{Metric} if furthermore (Definiteness) $\metric{x}{y}=0$ implies $x=y$.
\begin{rmk}
Semimetrics are also sometimes called \term{pseudometrics}\index{Pseudometric}.  However, the term seminorm (something we haven't discussed yet---see \cref{Seminorm}) is actually much more common than either of these terms, and as metrics are to norms (also something we haven't discussed yet---see \cref{Seminorm} again) as semimetrics/pseudometrics are to seminorms, I feel as if the terminology ``semimetric'' is more appropriate.

If fact, you should be warned that \emph{some authors use a different meaning than us for ``semimetric''}.
\end{rmk}
\begin{rmk}
It is much more common to denote (semi)metrics by ``$d(\blankdot ,\blankdot )$'', however, this conflicts with our conventions of reserving the letter ``$d$'' for dimension.
\end{rmk}
\end{dfn}
\begin{exm}{}{}
Let $X\coloneqq \R$ and define $\metric{x}{y}\coloneqq \abs{x-y}$.  Then, $\metric$ is in fact a metric.
\begin{rmk}
Of course, this is where the notation $\metric$ in general comes from.
\end{rmk}
\end{exm}
\begin{exm}{A semimetric that is not a metric}{exm4.4.3}
Let $X$ be a topological space and let $K\subseteq X$ be quasicompact.  For $f,g\in \Mor _{\Top}(X,\R )$, we define
\begin{equation}\label{4.4.4}
\metric{f}{g}_K\coloneqq \sup _{x\in K}\{ \abs{f(x)-g(x)}\} .\footnote{We require that $K$ be quasicompact so that $f-g$ is bounded on $K$ (by the \namerefpcref{ExtremeValueTheorem})}.
\end{equation}
In general, this will not be a metric.  For example, take $X\coloneqq \R$ and $K\coloneqq [0,1]$.  Then,  $\abs{f,0}_K=0$ iff $\restr{f}{[0,1]}=0$, but of course, there are many nonzero real-valued continuous functions on $\R$ that vanish on $[0,1]$ (by Urysohn's Lemma (\cref{UrysohnsLemma}), for example, if you want to use a sledgehammer (or maybe just a hammer?) to swat a fly).
\end{exm}
\begin{dfn}{Semimetric space}{Semimetric space}
A \term{semimetric space}\index{Semimetric space} is a set $X$ equipped with a collection $\collection{D}$ of semimetrics.
\begin{rmk}
For some reason, it seems that semimetric spaces are also referred to as \term{gauge spaces}\index{Gauge spaces}.  Off the top of my head, I can think of at least two other distinct ways in which the term ``gauge'' is used in mathematics, and so I would recommend not using this terminology
\end{rmk}
\end{dfn}
\begin{dfn}{Metric space}{MetricSpace}
A \term{metric space}\index{Metric space} $(X,\metric )$ is a semimetric space $\coord{X,\collection{D}}$ in which $\collection{D}$ is a singleton, $\cover{D}=\{ \metric \}$, and $\metric$ is a metric.
\begin{rmk}
The reason we take $\cover{D}$ to be a singleton instead of just an arbitrary collection of \emph{metrics} is to agree with standard terminology (metric spaces are almost always taken to be sets equipped with a (\emph{single}) metric).
\end{rmk}
\end{dfn}
\begin{exm}{A semimetric space that is not a metric space}{}
Let $X$ be a topological space, and for $K\subseteq X$ quasicompact nonempty, let $\metric _K$ be the semimetric on $\Mor _{\Top}(X,\R )$ in \eqref{4.4.4}, that is
\begin{equation}
\metric{f}{g}\coloneqq \sup _{x\in K}\{ \abs{f(x)-g(x)}\} .
\end{equation}
We already know from \cref{exm4.4.3} that each $\metric _K$ is a semimetric on $\Mor _{\Top}(X,\R )$.

As $\cover{D}$ clearly contains more than one element (at least so long as $X$ contains more than one point), you might think that this shows that this cannot be a metric space.  However, the real question is \emph{``Is it uniformly-homeomorphic} to a metric space?'',\footnote{Of course, this doesn't quite make sense yet as we have not put a uniformity on semimetric spaces.} and for general topological spaces the answer is \emph{no}.
\end{exm}
We noted at the very beginning of this chapter that the star of a point with respect to $\cover{B}_{\varepsilon}$ in $\R$ is just the ball of $2\varepsilon$.  You should be careful, however, as this does not hold in general.
\begin{exm}{A metric space for which \\ $\Star _{\cover{B}_{\varepsilon}}(x)\neq B_{2\varepsilon}(x)$}{}
Define $X\coloneqq \{ 0,1\}$ equipped with the metric $\metric$ for which $\metric{0}{1}=1$, $\varepsilon \coloneqq 1$, and $x_0\coloneqq 0$.  Then,
\begin{equation}
\cover{B}_{\varepsilon}=\left\{ \{ 0\} ,\{ 1\} \right\} ,
\end{equation}
and so
\begin{equation}
\Star _{\cover{B}_{\varepsilon}}(x_0)=\{ 0\} .
\end{equation}
On the other hand,
\begin{equation}
B_{2\varepsilon}(x_0)=\{ 0,1\} .
\end{equation}
\end{exm}
Despite this, we always have one inclusion.
\begin{exr}{}{}
Let $X$ be a metric space, let $x\in X$, and let $\varepsilon >0$.  Show that $\Star _{\cover{B}_{\varepsilon}}(x)\subseteq B_{2\varepsilon}(x)$.
\end{exr}
\begin{exr}{}{}
\begin{enumerate}
\item Show that $\Star _{\cover{B}_{\varepsilon}}(B_{\varepsilon}(x))=B_{3\varepsilon}(x)$ in $\R ^d$.
\item Find a metric space in which this fails.
\end{enumerate}
\end{exr}

It's worth noting that, in a metric space, for every closed subset, the distance (as defined below in \eqref{4.8.50}) from a point to the closed subset is a continuous function of that point.
\begin{prp}{}{prp4.8.49}
Let $\coord{X,\metric}$ be a metric space and let $C\subseteq X$.  Then, the function $\dist _C:X\rightarrow \R$ defined by
\begin{equation}\label{4.8.50}
\dist _C(x)\coloneqq \inf _{c\in C}\{ \metric{x}{c}\} 
\end{equation}\index[notation]{$\dist _C(x)$}
is uniformly-continuous and furthermore $\dist _C^{-1}(0)=C$.
\begin{rmk}
The statement that $\dist _C^{-1}(0)=C$ means that, not only is $\dist _C$ $0$ on $C$, but in fact, it isn't $0$ anywhere else as well.  Contrast this with a set that is not closed, e.g. $(0,1)\subseteq \R$, for which $1\in \R$ is a distance of $0$ from this set but not actually an element of the set.
\end{rmk}
\begin{proof}
Let $\varepsilon >0$.  Let $x_1,x_2\in X$ lie in some $\varepsilon$ ball.  Choose some $c\in C$ such that $\metric{x_1}{c}-\dist_C(x_1)<\varepsilon$.\footnote{Using \cref{prp1.4.11}.}  Then,
\begin{equation}
\begin{split}
\dist _C(x_2) & \leq \metric{x_2}{c}\leq \metric{x_2}{x_1}+\metric{x_1}{c} \\
& <2\varepsilon +(\dist _C(x_1)+\varepsilon ) \\
& =3\varepsilon +\dist _C(x_1),
\end{split}
\end{equation}
and so
\begin{equation}
\dist _C(x_2)-\dist _C(x_1)<3\varepsilon .
\end{equation}
By $1\leftrightarrow 2$ symmetry, we also have that
\begin{equation}
\dist _C(x_1)-\dist _C(x_2)<3\varepsilon ,
\end{equation}
and hence
\begin{equation}
\abs{\dist _C(x_1)-\dist _C(x_2)}<3\varepsilon .
\end{equation}
This shows that $\dist _C$ is uniformly-continuous (by \cref{prp4.2.27x}).

Of course $C\subseteq \dist _C^{-1}(0)$.  On the other hand, if $x\in \dist _C^{-1}(0)$, then $x$ is an accumulation point of $C$, and hence contained in $C$.  Thus, $C=\dist _C^{-1}(0)$.
\end{proof}
\end{prp}

Now that we've gotten that out of the way, we are ready to equip semimetric spaces with a topology and uniformity.
\begin{dfn}{Uniformity on a semimetric space}{dfnB.10}
Let $\coord{X,\cover{D}}$ be a semimetric space, and equip $X$ with the uniformity generated by the uniform base defined by
\begin{equation}\label{B.11}
\begin{multlined}
\uniformity{B}_{\cover{D}}\coloneqq \left\{ \cover{B}_{\varepsilon _1,\ldots ,\varepsilon _m}^{\metric _1,\ldots ,\metric _m}:m\in \Z ^+;\right. \\ \left. \metric _1,\ldots ,\metric _m\in \cover{D};\ \varepsilon _1,\ldots ,\varepsilon _m>0\right\} ,\index[notation]{$\uniformity{B}_{\cover{D}}$}
\end{multlined}
\end{equation}
where
\begin{equation}\label{1.12}
\cover{B}_{\varepsilon _1,\ldots ,\varepsilon _m}^{\metric _1,\ldots ,\metric _m}\coloneqq \left\{ B_{\varepsilon _1,\ldots ,\varepsilon _m}^{\metric _1,\ldots ,\metric _m}(x):x\in X\right\} \index[notation]{$\cover{B}_{\varepsilon _1,\ldots ,\varepsilon _m}^{\metric _1,\ldots ,\metric _m}$}
\end{equation}
and
\begin{equation}\label{1.13}
\begin{multlined}
B_{\varepsilon _1,\ldots ,\varepsilon _m}^{\metric _1,\ldots ,\metric _m}(x)\coloneqq \left\{ y\in X:\right. \\ \left. \metric{y}{x}_1<\varepsilon _1,\ldots ,\metric{y}{x}_m<\varepsilon _m\right\} .
\end{multlined}
\end{equation}\index[notation]{$B_{\varepsilon _1,\ldots ,\varepsilon _m}^{\metric _1,\ldots ,\metric _m}(x)$}
\begin{exr}{}{}
Show that $\uniformity{B}_{\cover{D}}$ is indeed a uniform base.
\end{exr}
\begin{rmk}
So the notation admittedly makes this look a bit atrocious.  Let's try to break it down.  $B_{\varepsilon _1,\ldots ,\varepsilon _m}^{\metric _1,\ldots ,\metric _m}$ is just like the $\varepsilon$-balls you know and love, but now, as we have more than one semimetric, we don't just have $\varepsilon$-balls, but instead we have $\coord{\varepsilon _1,\ldots ,\varepsilon _m}$-balls for all $m\in \Z ^+$.  Then, we just do the same as we did when there was just one metric:  for each choice of $\coord{\varepsilon _1,\ldots ,\varepsilon _m}$ and corresponding semimetric, there is a uniform cover consisting of $\coord{\varepsilon _1,\ldots ,\varepsilon _m}$-balls centered at each point, and the collection of all of these uniform covers is the uniform base.
\end{rmk}
\end{dfn}
Our first order of business is to make explicit what it means to be uniformly-continuous for semimetric spaces.
\begin{prp}{}{prp4.2.27x}
Let $f\colon \coord{X,\collection{D}}\rightarrow \coord{Y,\collection{E}}$ be a function between semimetric spaces.  Then, $f$ is uniformly continuous iff for every $\metric \in \collection{E}$ and $\varepsilon >0$, there are some $\metric _1,\ldots ,\metric _m\in \collection{D}$ and $\delta _1,\ldots ,\delta _m>0$ such that, whenever $\metric{x_1}{x_2}_k<\delta _k$ for $1\leq k\leq m$, it follows that $\metric{f(x_1)}{f(x_2)}<\varepsilon$.
\begin{rmk}
In particular, for metric spaces, $f$ is uniformly-continuous iff for every $\varepsilon >0$ there is some $\delta >0$ such that, whenever $\metric{x_1}{x_2}<\delta$, it follows that $\metric{f(x_1)}{f(x_2)}<\varepsilon$.
\end{rmk}
\begin{proof}
$(\Rightarrow )$ Suppose that $f$ is uniformly-continuous.  Let $\metric _1\in \collection{E}$ and let $\varepsilon >0$.  Then, as $f$ is uniformly-continuous, $f^{_1}\left( \cover{B}_{\varepsilon}^{\metric}\right)$ is a uniform cover, and so, by the definition of a uniform base (\cref{UniformBase}), there are $\metric _1,\ldots ,\metric _m\in \collection{D}$ and $\delta _1,\ldots ,\delta _m>0$ such that $\cover{B}_{\delta _1,\ldots ,\delta _m}^{\metric _1,\ldots ,\metric _m}\llcurly f^{-1}\left( \cover{B}_{\varepsilon}^{\metric}\right)$.  Thus, for every $B_{\delta _1,\ldots ,\delta _m}^{\metric _1,\ldots ,\metric _m}(x)\in \cover{B}_{\delta _1,\ldots ,\delta _m}^{\metric _1,\ldots ,\metric _m}$ there is some $f^{-1}(B_{\varepsilon}^{\metric}(y_x))\in f^{-1}(\cover{B}_{\varepsilon}^{\metric})$ such that
\begin{equation*}
\Star _{\cover{B}_{\delta _1,\ldots ,\delta _m}^{\metric _1,\ldots ,\metric _m}}\left( B_{\delta _1,\ldots ,\delta _m}^{\metric _1,\ldots ,\metric _m}(x)\right) \subseteq f^{-1}(B_{\varepsilon}^{\metric}(y_x)).
\end{equation*}

So, let $x_1,x_2\in X$ and suppose that $\metric{x_1}{x_2}_k<\delta _k$ for $1\leq k\leq m$.  This means that $x_1,x_2\in B_{\delta _1,\ldots ,\delta _m}^{\metric _1,\ldots ,\metric _m}(x_2)$, and so, by the above equation, $x_1,x_2\in f^{-1}(B_{\varepsilon}^{\metric}(y_x))$.  Hence,
\begin{equation}
\begin{split}
\metric{f(x_1)}{f(x_2)} & \leq \metric{f(x_1)}{y_{x_1}}+\metric{y_{x_1}}{f(x_2)} \\
& <\varepsilon +\varepsilon =2\varepsilon .
\end{split}
\end{equation}

\blankline
\noindent
$(\Leftarrow )$ Suppose that for every $\metric \in \collection{E}$ and $\varepsilon >0$, there are some $\metric _1,\ldots ,\metric _m\in \collection{D}$ and $\delta _1,\ldots ,\delta _m>0$ such that, whenever $\metric{x_1}{x_2}_k<\delta _k$ for $1\leq k\leq m$, it follows that $\metric{f(x_1)}{f(x_2)}<\varepsilon$.  Let $\metric _1,\ldots ,\metric _m\in \collection{E}$\footnote{Note that these semimetrics are elements of $\collection{E}$, whereas the same symbols in the previous sentence were used to represent elements of $\collection{D}$.} and let $\varepsilon _1,\ldots ,\varepsilon _m>0$.  We wish to show that $f^{-1}\left( \cover{B}_{\varepsilon _1,\ldots ,\varepsilon _m}^{\metric _1,\ldots ,\metric _m}\right)$ is a uniform cover.

By hypothesis, for each $k$, there are finitely many $\metric _{k,1},\ldots ,\metric _{k,m_k}\in \collection{D}$ and $\delta _{k_1},\ldots ,\delta _{k,m_k}>0$ such that, whenever $\metric{x_1}{x_2}_{k,l}<\delta _{k,l}$ for $1\leq l\leq m_k$, it follows that $\metric{f(x_1)}{f(x_2)}_k<\frac{1}{2}\varepsilon _k$.  We claim that $\cover{B}_{\delta _{1,1},\ldots ,\delta _{m,m_m}}^{\metric _{1,1},\ldots ,\metric _{m,m}}$ star-refines $f^{-1}\left( \cover{B}_{\varepsilon _1,\ldots ,\varepsilon _m}^{\metric _1,\ldots ,\metric _m}\right)$, which will complete the proof.

So, let $B_{\delta _{1,1},\ldots ,\delta _{m,m_m}}^{\metric _{1,1},\ldots ,\metric _{m,m_m}}(x)\in \cover{B}_{\delta _{1,1},\ldots ,\delta _{m,m_m}}^{\metric _{1,1},\ldots ,\metric _{m,m}}$.  We claim that
\begin{equation}
\begin{multlined}
\Star _{\cover{B}_{\delta _{1,1},\ldots ,\delta _{m,m_m}}^{\metric _{1,1},\ldots ,\metric _{m,m}}}\left( B_{\delta _{1,1},\ldots ,\delta _{m,m_m}}^{\metric _{1,1},\ldots ,\metric _{m,m_m}}(x)\right) \subseteq \\ f^{-1}\left( B_{\varepsilon _1,\ldots ,\varepsilon _m}^{\metric _1,\ldots ,\metric _m}(f(x))\right) ,
\end{multlined}
\end{equation}
which itself will complete the proof.

So, let $x'\in \Star _{\cover{B}_{\delta _{1,1},\ldots ,\delta _{m,m_m}}^{\metric _{1,1},\ldots ,\metric _{m,m}}}\left( B_{\delta _{1,1},\ldots ,\delta _{m,m_m}}^{\metric _{1,1},\ldots ,\metric _{m,m_m}}(x)\right)$.  We wish to show that $f(x')\in B_{\varepsilon _1,\ldots ,\varepsilon _m}^{\metric _1,\ldots ,\metric _m}(f(x))$.  That $x'\in \Star _{\cover{B}_{\delta _{1,1},\ldots ,\delta _{m,m_m}}^{\metric _{1,1},\ldots ,\metric _{m,m}}}\left( B_{\delta _{1,1},\ldots ,\delta _{m,m_m}}^{\metric _{1,1},\ldots ,\metric _{m,m_m}}(x)\right)$ means that there is some $x''\in X$ with $x''\in B_{\delta _{1,1},\ldots ,\delta _{m,m_m}}^{\metric _{1,1},\ldots ,\metric _{m,m_m}}(x')$ and $x''\in B_{\delta _{1,1},\ldots ,\delta _{m,m_m}}^{\metric _{1,1},\ldots ,\metric _{m,m_m}}(x)$.  As $\metric{x''}{x'}_{k,l}<\delta _k$ for $1\leq k\leq m_l$ (and similarly for $\metric{x''}{x}_{k,l}$), it follows from our hypothesis that $\metric{f(x'')}{f(x')}_k<\frac{1}{2}\varepsilon _k$ and $\metric{f(x'')}{f(x)}_k<\frac{1}{2}\varepsilon _k$ for $1\leq k\leq m$.  Hence,
\begin{equation}
\begin{split}
\metric{f(x')}{f(x)}_k & \leq \metric{f(x')}{f(x'')}_k+\metric{f(x'')}{f(x)}_k \\
& <\tfrac{1}{2}\varepsilon _k+\tfrac{1}{2}\varepsilon _k=\varepsilon _k.
\end{split}
\end{equation}
Thus, $f(x')\in B_{\varepsilon _1,\ldots ,\varepsilon _m}^{\metric _1,\ldots ,\metric _m}(f(x))$, as desired.
\end{proof}
\end{prp}
Similarly, it would be nice to have a more explicit description of the uniform topology of the uniformity of a semimetric space.
\begin{exr}{}{exr4.2.31x}
Let $\coord{X,\collection{D}}$ be a semimetric space.  Show that
\begin{equation}
\begin{multlined}
\left\{ B_{\varepsilon _1,\ldots ,\varepsilon _m}^{\metric _1,\ldots ,\metric _m}(x):m\in \Z ^+;\right. \\ \left. \metric _1,\ldots,\metric _m\in \collection{D};\ \varepsilon _1,\ldots ,\varepsilon _m>0\right\}
\end{multlined}
\end{equation}
is a neighborhood base at $x$ consisting of open sets for the uniform topology of the uniformity associated to the semimetric space $\coord{X,\collection{D}}$ defined in \cref{dfnB.10}.
\begin{rmk}
In particular, for a metric space $\coord{X,\metric}$, a subset $U\subseteq X$ is open iff for every $x\in U$ there is some $\varepsilon >0$ such that $B_{\varepsilon}(x)\subseteq U$.
\end{rmk}
\end{exr}

\begin{exm}{Discrete metric}{}\index{Discrete metric}
Not only does the discrete topology come from a uniformity, but so to does the discrete uniformity in turn come from a metric.

Let $X$ be a set and for $x,y\in X$, define
\begin{equation}
\metric{x}{y}\coloneqq \begin{cases}0 & \text{if }x=y \\ 1 & \text{otherwise.}\end{cases}
\end{equation}
\begin{exr}[breakable=false]{}{}
Show that $\metric$ is indeed a metric on $X$.
\end{exr}
\begin{exr}[breakable=false]{}{}
Show that the uniformity defined by $\metric$ is the discrete uniformity.
\end{exr}
\end{exm}
\begin{exr}{}{exr4.2.31}
Let $\coord{X,\cover{D}}$ be a semimetric space, let $\lambda \mapsto x_\lambda \in X$ be a net, and let $x_{\infty}\in X$.  Show that $\lambda \mapsto x_\lambda$ converges to $x_{\infty}$ iff for every $\metric \in \cover{D}$ and for every $\varepsilon >0$, $\lambda \mapsto x_\lambda$ is eventually contained in $B_{\varepsilon}^{\metric}(x_{\infty})$.
\begin{rmk}
In other words, a net converges to a point in a semimetric space iff it converges to that point with respect to each semimetric.
\end{rmk}
\end{exr}
\begin{exr}{}{}
Why does the indiscrete uniformity (on a set with at least two elements) not come from a metric?
\end{exr}

But before we head onto topological groups, what about the morphisms in the category of semimetric spaces, you ask?  Good question.
\begin{dfn}{Bounded map (of semimetric \\ spaces)}{BoundedMap}
Let $f\colon \coord{X,\cover{D}}\rightarrow \coord{Y,\cover{E}}$ be a function between semimetric spaces.  Then, $f$ is \term{bounded}\index{Bounded map (of semimetric spaces)} iff for every $\metric \in \cover{E}$, there are \emph{finitely-many} $\metric _1,\ldots ,\metric _m\in \cover{D}$ and constants $K_1,\ldots ,K_m\geq 0$ such that
\begin{equation}
\metric{f(x_1)}{f(x_2)}\leq K_1\metric{x_1}{x_2}_1+\cdots +K_m\metric{x_1}{x_2}_m
\end{equation}
for all $x_1,x_2\in X$.
\begin{rmk}
If $X$ and $Y$ are metric spaces with metric $\metric _X$ and $\metric _Y$ respectively, this condition reads just
\begin{equation}
\metric{f(x_1)}{f(x_2)}_Y\leq K\metric{x_1}{x_2}_X.
\end{equation}
In this case, $f$ is called \term{Lipschitz-continuous}\index{Lipschitz-continuous}.
\end{rmk}
\begin{rmk}
Of all the categories we've come across, that the bounded maps are the `right' notion of morphism between semimetric spaces is probably the least obvious.\footnote{Of course, we can declare any collection of morphisms we like.  It's just that, taking the morphisms to be \emph{all} functions when the objects are groups (for example) is not particularly useful---the category won't be able to tell that the groups are groups!}  The motivation for the definition is that, for seminormed vector spaces, this definition is equivalent to continuity---see \cref{exr4.3.57}.  Perhaps a simpler explanation is that it guarantees uniform-continuity.
\end{rmk}
\end{dfn}
\begin{exr}{}{}
Show that bounded maps between semimetric spaces are uniformly-continuous.
\end{exr}
\begin{exm}{The category of semimetric spaces}{}\index{Category of semimetric spaces}
The category of semimetric spaces is the category $\Semi \Met$\index[notation]{$\Semi \Met$}
\begin{enumerate}
\item whose collection of objects $\Obj (\Semi \Met )$ is the collection of all semimetric spaces;
\item with morphism set $\Mor _{\Semi \Met}(X,Y)$ is precisely the set of all bounded maps from $X$ to $Y$;
\item whose composition is given by ordinary function composition; and
\item whose identities are given by the identity functions.
\end{enumerate}
\begin{rmk}
Every semimetric space is canonically a uniform space---see \cref{dfnB.10}.  By the previous exercise, every bounded map is likewise uniformly-continuous.  Therefore, in fact, the category $\Semi \Met$ \emph{embeds}\footnote{We have not defined what precisely this means for categories, but with a little mathematical maturity, you can probably figure it out.  In any case, it's okay if you don't know the precise definition.} in $\Uni$:  the thing to take note of is that \emph{both} the objects \emph{and} the morphisms have to be contained in $\Uni$.
\end{rmk}
\end{exm}
\begin{exm}{A Lipschitz-continuous function}{}
The function $x\mapsto x$ from $\R$ to $\R$.
\end{exm}
\begin{exm}{A uniformly-continuous function that is not Lipschitz-continuous}{exm4.3.34}
Define $f\colon [0,1]\rightarrow \R$ by $f(x)\coloneqq \sqrt{x}$.  This function is continuous on $[0,1]$, and hence uniformly-continuous by the \namerefpcref{prp4.2.73} ($[0,1]$ is quasicompact by the \namerefpcref{HeineBorelTheorem}).  On the other hand, to show that it is \emph{not} Lipschitz-continuous, we need to show that
\begin{equation}
\frac{\sqrt{x}-\sqrt{y}}{x-y}
\end{equation}
is \emph{not} bounded for $x,y\in [0,1]$ distinct.  However, simply take $x=0$.  Then, we need to show that
\begin{equation}
\frac{\sqrt{y}}{y}=\frac{1}{\sqrt{y}}
\end{equation}
is not bounded on $[0,1]$.  Equivalently, you can show that $\lim _{y\to 0^+}\sqrt{y}=0$.\footnote{We have technically not defined one-sided limits.  If this bothers you, it's not a bad exercise to try to come up with the definition yourself.}
\end{exm}

\subsection{Topological groups}

Before we talk about any sort of uniformity, we had better first define what we mean by a topological group.
\begin{dfn}{Topological group}{TopologicalGroup}
A \term{topological group} is a group $\coord{G,\cdot ,1,\blank ^{-1}}$ equipped with a topology such that
\begin{enumerate}
\item $\cdot :G\times G\rightarrow G$ is continuous; and
\item $\blank ^{-1}:G\rightarrow G$ is continuous.
\end{enumerate}
\begin{rmk}
That is to say, a topological group is a thing that is both a group and a topological space, subject to a couple of `compatibility' axioms that demand that the two structures `work together'.  This is very analogous to our definition of preordered rgs (\cref{dfn1.1.38})---a preordered rg is both a preordered set and a rg subject to a couple of `'compatibility'' conditions.  This idea is not uncommon throughout all of mathematics, and, as you might have expected by this point, can be unified (to an extent anyways) with the use of categories.
\end{rmk}
\begin{rmk}
Recall that in a remark of the definition of a group (\cref{Group}), we made a slight deal about ``having inverses'' not being stated as an \emph{extra property} but rather as \emph{extra structure}.  This is one reason why.  When we go to define a topological group, the operation of taking inverses should be thought of as just that---an operation, on the \emph{same footing as the product}.  If we think of the operation of taking inverses as on the same footing as the product, then we almost \emph{have} to also assume that the inverse operation is likewise continuous, whereas if it were thought of just as an existence property, it would not make as much sense to do this.
\end{rmk}
\begin{wrn}
Warning:  You do need to check both:  it is possible that multiplication be continuous but not inversion, or vice-versa---see the following counter-examples.
\end{wrn}
\end{dfn}
\begin{exm}{A topology on a group for which multiplication is continuous but inversion is not}{}\footnote{This is adapted from an answer on \href{http://math.stackexchange.com/questions/1393303/group-of-units-in-a-topological-ring}{math.stackexchange}.}
Define $G\coloneqq \Q ^{\times}=\left\{ q\in \Q :q\neq 0\right\}$ equipped with the ordinary multiplication in $\Q$.

We define a topology on $G$ by specifying a neighborhood base for the topology.  For $q\in G$, define
\begin{equation}
\cover{B}_q\coloneqq \left\{ (q+n\Z )\cap G:n\in \Z ,\ n\neq 0\right\} .
\end{equation}
We claim that this satisfies the hypotheses of \cref{prp4.1.8}, so that there will be a unique topology on $G$ for which $\{ \cover{B}_q:q\in G\}$ is a neighborhood base.  So, let $(q+n_1\Z )\cap G,(q+n_2\Z )\cap G\in \cover{B}_q$.  Then, $(q+\lcm (n_1,n_2)\Z )\cap G\subseteq( (q+n_1\Z )\cap G)\cap ((q+n_2\Z )\cap G)$, where here $\lcm$ least common multiple (\cref{LeastCommonMultiple}).  Furthermore, for $r\in (q+\lcm (n_1,n_2)\Z )\cap G$, we have that $(r+\lcm (n_1,n_2)\Z )\cap G\subseteq (q+\lcm (n_1,n_2)\Z )\cap G$.  Thus, there is a unique topology on $G$ for which $\{ \cover{B}_q:q\in G\}$ is a neighborhood base.

We now wish to show that all the elements in this neighborhood base are in fact open neighborhoods.  So, fix an element in the neighborhood base $(q_0+n_0\Z )\cap G$ and let $q\in (q_0+n_0\Z )\cap G$.  We need to show that an element of $\cover{B}_q$ is contained in $(q_0+n_0\Z )\cap G$.  We claim that in fact $q+n_0\Z =q_0+n_0\Z$.  As $q\in q_0+n_0\Z$, we can write $q=q_0+n_0k$ for some $k\in \Z$.  Hence, $q+n_0\Z =q_0+n_0k+n_0\Z =q_0+n_0\Z$.  Thus, all elements of the neighborhood base are in fact open.  It follows (\cref{prp4.1.8} again) that
\begin{equation}
\left\{ q+n\Z :q\in G,\ n\in \Z ,\ n\neq 0\right\}
\end{equation}
is a base for this topology.

The odd integers, $(1+2\Z)\cap G=1+2\Z$, thus themselves form an open set in this topology (take $q=1$ and $n=2$), and its preimage under inversion is $\left\{ \frac{1}{m}:m\in 2\Z +1\right\}$.  If this were open, then in particular, we would be able to find an element of $\cover{B}_1$ contained it in, which is impossible as the only integers in this set are $\pm 1$.

We now show that multiplication on the other hand \emph{is} continuous.  So, let $\lambda \mapsto \coord{q_{\lambda},r_{\lambda}}\in G\times G$ be a net converging to $\coord{q,r}\in G\times G$.  We wish to show that $\lambda \mapsto q_{\lambda}r_{\lambda}$ converges to $qr$.  To do that, it suffices to show that $\lambda \mapsto q_{\lambda}r_{\lambda}$ is eventually contained in $qr+n\Z$ for all nonzero $n\in \Z$.  So, fix $n\in \Z$ nonzero arbitrary.  That $\lambda \mapsto \coord{q_{\lambda},r_{\lambda}}$ converges to $\coord{q,r}$ implies that $\lambda \mapsto q_{\lambda}$ converge to $q$ and $\lambda \mapsto r_{\lambda}$ converges to $r$ (\cref{crl4.5.15}).  Thus, there is some $\lambda _0$ such that $q_{\lambda}\in q+n\Z$ and $r_{\lambda}\in r+n\Z$ whenever $\lambda \geq \lambda _0$.  Of course, however, if $q_{\lambda}\in q+n\Z$ and $r_{\lambda}\in r+n\Z$, then we have that $q_{\lambda}r_{\lambda}\in qr+n\Z$, as desired.
\end{exm}
\begin{exm}{A topology on a group for which inversion is continuous but multiplication is not}{}
Define $G\coloneqq \R ^{\times}$ and equip $G$ with the cocountable topology.  Inversion is bijective, and so the preimage of a countable set is countable, and hence the preimage of every closed set under inversion is closed.  Thus, inversion is continuous.

We now check that multiplication is not continuous.  The set $\{ x\in G:x\neq 1\}$ is cocountable, and hence open.  Therefore, its preimage under multiplication,
\begin{equation}
S\coloneqq \left\{ \coord{x,y}\in G\times G:xy\neq 1\right\} ,
\end{equation}
should likewise be open.  Thus, as $\coord{1,2}\in S$, if $S$ were open, we should be able to find $U,V\subseteq G$ open (i.e.~cocountable) such that $\coord{1,2}\in U\times V\subseteq S$.  If this were true, then in particular we would have that
\begin{equation}
\begin{split}
\left\{ \coord{x,\tfrac{1}{x}}:x\in \R ^{\times}\right\} & =S^{\comp}\subseteq (U\times V)^{\comp} \\
& =(U^{\comp}\times \R ^{\times})\cup (\R ^{\times}\times V^{\comp}).
\end{split}
\end{equation}
There are uncountably many $x\in \R ^{\times}$ not contained in $U^{\comp}$, and as $V^{\comp}$ is countable, at least one of them (in fact, uncountably many of them) must have the property that $\frac{1}{x}$ is likewise not in $V^{\comp}$.  This, however, yields an element of $S^{\comp}$ that cannot be contained in $(U\times V)^{\comp}$:  a contradiction.
\end{exm}
\begin{exm}{The category of topological \\ groups}{}\index{Category of topological groups}
The category of topological groups is the category $\Top \Grp$\index[notation]{$\Top \Grp$}
\begin{enumerate}
\item whose collection of objects $\Obj (\Top \Grp )$ is the collection of all topological groups;
\item with morphism set $\Mor _{\Top \Grp}(G,H)$ is precisely the set of all continuous group homomorphisms from $G$ to $H$;
\item whose composition is given by ordinary function composition; and
\item whose identities are given by the identity functions.
\end{enumerate}.
\end{exm}
\begin{exm}{A topological group that is not $T_0$}{}
Define $G\coloneqq \R /\Q$.\footnote{This is the quotient group construction---see \cref{IdealsAndQuotientGroups}.}  Let $\q :\R \rightarrow G$ be the quotient map (i.e.~the map that sends an element to its equivalence class) and equip $\R /\Q$ with the quotient topology.
\begin{exr}[breakable=false]{}{}
Show that $+:G\times G\rightarrow G$ and $\blank ^{-1}:G\rightarrow G$ are continuous.
\end{exr}
We now check that the quotient topology on $G$ is not $T_0$.  So, let $x\in \R$ be irrational.  We wish to show every open neighborhood of $x+\Q$ contains $0+\Q$ and conversely.  So, let $U\ni x+\Q$ be open.  Then, by definition, $\q ^{-1}(U)$ is an open neighborhood of $x$, and so by density, must contain some rational number $r\in \q ^{-1}(U)$.  But then, $0+\Q =r+\Q \in \q \left( \q ^{-1}(U)\right) \subseteq U$.  On the other hand, if $U$ is an open neighborhood of $0+\Q$, $\q ^{-1}(U)$ is an open neighborhood of $0$, and so by density again, must contain some $\varepsilon >0$ so that $x-\varepsilon$ is rational.  But if $x-\varepsilon \in \Q$, then $x+\Q =\varepsilon +\Q \in \q \left( \q ^{-1}(U)\right) \subseteq U$.
\begin{rmk}
We show in the next section that every $T_0$ uniform space is in fact (uniformly-)completely-$T_3$.  Thus, this serves as an example of a uniform space which is not (uniformly-)completely-$T_3$.
\end{rmk}
\end{exm}
A large number of examples of topological groups arise from totally-ordered rngs (or more generally, totally-ordered commutative groups).
\begin{exr}{}{exr4.8.58}
Let $G$ be a totally-ordered commutative group.  Show that $G$ is a topological group with respect to the order topology.
\end{exr}

Before we put a uniformity on topological groups, it will be useful to know at least one basic fact about them.
\begin{prp}{}{prp4.8.59}
Let $G$ be a topological group and let $U$ be a neighborhood of the identity.  Then, there exists an open neighborhood $V$ of the identity such that (i)~$VV\subseteq U$ and (ii)~$V^{-1}=V$.
\begin{rmk}
The notation means what you think it means:  $VV\coloneqq \{ v_1v_2:v_1\in V,\ v_2\in V\}$ (note how this is not the same as $V^2\ceqq \{ v^2:v\in V\}$) and $V^{-1}\coloneqq \{ v^{-1}:v\in V\}$.
\end{rmk}
\begin{proof}
Regarding the group operation $\cdot$ as a function from $G\times G$ to $G$, we know that $\cdot ^{-1}(U)$ is an open neighborhood of $\coord{1,1}$ in $G\times G$, and therefore we have that $V\times W\subseteq \cdot ^{-1}(U)$ for some $V,W\subseteq G$ open neighborhoods of the identity (by the definition of the product topology \cref{ProductTopology}).   Replace $V$ with $V\cap W$, another open neighborhood of the identity, so that $V\times V\subseteq \cdot ^{-1}(U)$.  In other words, $VV\subseteq U$.  Now do this exact same construction again and find another open neighborhood of the identity $W$ (replacing our `old' $W$) with $WW\subseteq V$.  Now define
\begin{equation}
W'\coloneqq W\cap [\blank ^{-1}]^{-1}(W),
\end{equation}
that is, the intersection of $W$ with the preimage of $W$ under the inverse function $\blank ^{-1}:G\rightarrow G$.  This will be yet another open neighborhood of the identity, with both $W',(W')^{-1}\subseteq W$.  Finally, define
\begin{equation}
W''\coloneqq (W')(W')^{-1}.
\end{equation}
This certainly satisfies $(W'')^{-1}=W''$, and furthermore,
\begin{equation}
\begin{split}
W''W'' & \coloneqq (W')(W')^{-1}(W')(W')^{-1}\subseteq WWWW \\
& \subseteq VV\subseteq U.
\end{split}
\end{equation}
\end{proof}
\end{prp}

\begin{prp}{Uniformities on a topological group}{dfnB.7}
Let $G$ be a topological group and define
\begin{equation}\label{B.8}
\uniformity{B}_{G,L}\coloneqq \left\{ \cover{B}_{U,L}:U\ni 1\text{ is open.}\right\}
\end{equation}\index[notation]{$\uniformity{B}_{G,L}$}
and
\begin{equation}
\uniformity{B}_{G,R}\coloneqq \left\{ \cover{B}_{U,R}:U\ni 1\text{ is open.}\right\} 
\end{equation}\index[notation]{$\uniformity{B}_{G,R}$}
where
\begin{subequations}\label{1.9}
\begin{align}
\cover{B}_{U,L} & \coloneqq \left\{ gU:g\in G\right\} \\
\cover{B}_{U,R} & \coloneqq \left\{ Ug:g\in G\right\} .
\end{align}
\end{subequations}\index[notation]{$\cover{B}_{U,L}$}\index[notation]{$\cover{B}_{U,R}$}
Then, $\uniformity{B}_{G,L}$ and $\uniformity{B}_{R,L}$ are uniform bases, which generate respectively the \term{left uniformity}\index{Left uniformity (of a topological group)} $\uniformity{U}_{G,L}$\index[notation]{$\uniformity{U}_{G,L}$} and \term{right uniformity}\index{Right uniformity (of a topological group)} $\uniformity{U}_{G,R}$\index[notation]{$\uniformity{G}_{G,R}$}.

Furthermore,
\begin{enumerate}
\item \label{dfnB.7(i)}for every $g\in G$, the maps $G\ni x\mapsto gx\in G$ and $G\ni x\mapsto xg\in G$ are uniformly continuous with respect to the left and right uniformities respectively;
\item \label{dfnB.7(ii)}$\coord{G,\uniformity{U}_{G,L}}\ni x\mapsto x^{-1}\in \coord{G,\uniformity{U}_{G,R}}$ is a uniform-homeomorphism and vice-versa;
\item \label{dfnB.7(iii)}the uniform topologies of both the left and right uniformities are the same as the original topology.
\end{enumerate}
\begin{rmk}
Note that we could have equally well taken the covers $\cover{B}_{U,L}$ and $\cover{B}_{U,R}$ for only $U\in \topology{N}$, $\topology{N}$ a fixed neighborhood base of the identity (by \cref{prp1.6}).
\end{rmk}
\begin{rmk}
It also turns out that $\coord{G,\uniformity{U}_{G,L}}$ is complete iff $\coord{G,\uniformity{U}_{G,R}}$ is complete (indeed, this follows immediately from \cref{dfnB.7(ii)})---see \cref{Completeness} for the definition of completeness.
\end{rmk}
\begin{rmk}
On the other hand, it is \emph{not} necessarily the case that a net is Cauchy with respect to the left uniformity iff it is Cauchy with respect to the right uniformity--see \cref{exr4.4.4}.
\end{rmk}
\begin{rmk}
For almost all purposes, whether we use the left or right uniformity is a matter of convention.  Moreover, a lot of the important examples, especially in functional analysis, are \emph{commutative} groups, in which case the two uniformities are the same.  In any case, for the sake of definitiveness, unless otherwise stated, assume we are using the \emph{left} uniformity.
\end{rmk}
\begin{proof}
We leave this as an exercise.
\begin{exr}[breakable=false]{}{}
Prove this yourself.
\end{exr}
\end{proof}
\end{prp}
We had a potential problem here---$G$ started its life as a topological group, and in particular, as a topological space.  We then defined on it two uniform structures, from each of which it obtains the uniform topology.  The question arises:  ``Are these topologies the same, and if not, which one should we use?''.  Fortunately, as stated above, they are the same.  However, we have have yet another potential problem to deal with here---$\R$ is both a metric space and a topological group (with respect to $+$), so we could use either the (semi)metric uniformity or the topological group uniformity.\footnote{They both agree of course because $\R$ is commutative.}  Fortunately, we needn't worry about this, because the two uniformities are the same.
\begin{exr}{}{}
Show that $\uniformity{B}_{\coord{\R ,\metric}}$ and $\uniformity{B}_{\coord{\R ,+}}$ define the same uniformity on $\R$.
\end{exr}

You might say that functional analysis is the study of topological vector spaces (the term ``functional'' a result of the fact that many `spaces' of functions are topological vector spaces).  As vector spaces are in particular a group (just forget about the scalars), everything we say regarding the uniformities of topological groups also applies to uniformities of topological vector spaces.  Thus, a knowledge of uniform spaces is very useful when studying functional analysis.  In particular, the following result is used ubiquitously (to the point where it is so common that it is not really even mentioned).
\begin{prp}{}{prpB.10}
Let $f\colon G\rightarrow H$ be a group homomorphism between topological groups and let $x_0\in G$.  Then, if $f$ is continuous at $x_0$, then $f$ is uniformly-continuous.
\begin{rmk}
This shows that the category $\Top \Grp$ \emph{embeds} into the category $\Uni$.  We already knew that the \emph{objects} `embedded' (from the canonical uniformity on topological groups given in \cref{dfnB.7})---this result tells us furthermore that the morphisms `embed' as well.
\end{rmk}
\begin{proof}
\Step{Make hypotheses}
Suppose that $f$ is continuous at $x_0$.

\Step{Show that $f$ is continuous.}
We first show that $f$ is continuous (as opposed to just continuous at $x_0$).  To show that, we show that $f$ is continuous at $x\in G$ for arbitrary $x$.  Let $V$ be a neighborhood of $f(x)\in H$.  Then, $f(x_0)f(x)^{-1}V$ is a neighborhood of $f(x_0)\in H$.  As $f$ is continuous at $x_0$, it follows that $f^{-1}\left( f(x_0)f(x)^{-1}V\right)$ is a neighborhood of $x_0$, and so $xx_0^{-1}f^{-1}\left( f(x_0)f(x)^{-1}V\right)$ is a neighborhood of $x$.\footnote{The juxtaposition here is being used to denote multiplication in the group.  Be careful not to confuse preimages with inverse elements (even though the same symbol is used, the context makes the notation unambiguous).}  However,
\begin{equation}
\begin{split}
\MoveEqLeft
f\left( xx_0^{-1}f^{-1}\left( f(x_0)f(x)^{-1}V\right) \right) \\
& =f(x)f(x_0)^{-1}f\left( f^{-1}\left( f(x_0)f(x)^{-1}V\right) \right) \\
& \subseteq f(x)f(x_0)^{-1}f(x_0)f(x)^{-1}V=V,
\end{split}
\end{equation}
so that
\begin{equation}
xx_0^{-1}f^{-1}\left( f(x_0)f(x)^{-1}V\right) \subseteq f^{-1}(V),
\end{equation}
so that $f^{-1}(V)$ is a neighborhood of $x$, so that $f$ is continuous at $x$.

\Step{Show that $f$ is uniformly-continuous.}
To show that $f$ is uniformly-continuous, we apply \cref{prpB.3.4} (it suffices to check uniform-continuity on a uniform base).  So, let $V\subseteq H$ be an open neighborhood of the identity and consider the uniform cover $\cover{U}_V\coloneqq \left\{ hV:h\in H\right\}$.  To show that $f^{-1}(\cover{U}_V)$ is a uniform cover, it suffices to find an open neighborhood $U\subseteq G$ of the identity such that $\cover{U}_U\llcurly f^{-1}(\cover{U}_V)$.  Take $U'$ to be an open neighborhood of the identity such that $U'U'\subseteq f^{-1}(V)$, and then in turn take $U$ to be an open neighborhood of the identity such that (i)~$UU\subseteq U'$ and (ii)~$U=U^{-1}$ (which we may do by \cref{prp4.8.59}).  We wish to show that
\begin{equation}\label{B.4.7}
\Star _{\cover{U}_U}(U)\coloneqq \bigcup _{\substack{x\in G \\ xU\cap U\neq \emptyset}}xU\subseteq f^{-1}(V).
\end{equation}
It will follow from this (see \eqref{4.8.72}) that $\cover{U}_U\llcurly f^{-1}(\cover{U}_V)$.  So, let $x\in G$ be such that $xU\cap U\neq \emptyset$.  Then, there are $u_1,u_2\in U$ such that $xu_1=u_2$, so that $x=u_2u_1^{-1}\in UU^{-1}=UU\subseteq U'$.  Thus, $xU\subseteq U'U'\subseteq f^{-1}(V)$.  \eqref{B.4.7} follows from this.

We then have
\begin{equation}\label{4.8.72}
\begin{split}
\Star _{\cover{U}_U}(x_0U) & =\bigcup _{\substack{x\in G \\ xU\cap x_0U\neq \emptyset}}xU=\bigcup _{\substack{x\in G \\ xU\cap U\neq \emptyset}}x_0xU \\
& =x_0\Star _{\cover{U}_U}(U)\subseteq x_0f^{-1}(V) \\
& \subseteq f^{-1}(f(x_0)V)\in f^{-1}(\cover{U}_V),
\end{split}
\end{equation}
so that $\cover{U}_U\llcurly f^{-1}(\cover{U}_V)$.
\end{proof}
\end{prp}

\subsection{Topological vector spaces and algebras}

An \emph{incredibly} family of examples of topological groups are the topological vector spaces.
\begin{dfn}{Topological vector space}{}
A \term{topological vector space}\index{Topological vector space} is real vector space $\coord{V,+,0,\R ,\cdot}$ such that
\begin{enumerate}
\item $\coord{V,+,0}$ is a topological group; and
\item $\cdot :\R \times V\rightarrow V$ is continuous.
\end{enumerate}
\begin{rmk}
Of course, this definition makes sense if we were to replace $\R$ with any topological field\footnote{What do you think the definition of a topological field should be?}, but for our purposes, restricting ourselves to working over the reals will be sufficient.  The other case of most interest is over $\C$, but we have not even defined the complex numbers.
\end{rmk}
\end{dfn}
One big reason why we are interested in topological vector spaces is because almost all of the examples of semimetrics we counter actually come \emph{seminorms}.
\begin{dfn}{Seminorm and norm}{Seminorm}
Let $V$ be a real vector space.  Then, a \term{seminorm}\index{Seminorm} on $V$ is a function $\norm :V\rightarrow \R _0^+$ such that
\begin{enumerate}
\item (Homogeneity) $\norm{\alpha v}=\norm{\alpha }\norm{v}$ for $\alpha \in \R$ and $v\in V$;
\item (Triangle Inequality) $\norm{v_1+v_2}\leq \norm{v_1}+\norm{v_2}$.
\end{enumerate}
$\norm$ is a \term{norm} if furthermore (Definiteness) $\norm{x}=0$ implies $x=0$.
\end{dfn}
\begin{dfn}{Semimetric induced by a seminorm}{}
Let $V$ be a real vector space, let $\norm$ be a seminorm on $V$, and let $v_1,v_2\in V$.  Then, the \term{semimetric induced by $\norm$}, $\metric$, is defined by
\begin{equation}
\metric{v_1}{v_2}\coloneqq \norm{v_1-v_2}.
\end{equation}
\begin{exr}[breakable=false]{}{}
Check that $\metric$ is indeed a semimetric.  Show that if $\norm$ is a norm then $\metric$ is a metric.
\end{exr}
\begin{rmk}
Intuitively, the seminorm of something is like its `size' and semimetric is like `distance'---the `distance' between two vectors is the `size' of their difference.
\end{rmk}
\end{dfn}
\begin{dfn}{Seminormed vector space and normed vector space}{SeminormedVectorSpace}
A \term{seminormed vector space}\index{Seminormed vector space} is a real vector space $V$ together with a collection of seminorms $\cover{D}$ such that $\coord{V,\cover{D}}$ is a semimetric space.  $\coord{V,\cover{D}}$ is a \term{normed vector space}\index{Normed vector space} iff it is furthermore a metric space.
\end{dfn}
As by now you should have expected, the morphisms of seminormed vector spaces are the bounded (\cref{BoundedMap}) linear maps.
\begin{exm}{The category of seminormed vector spaces}{}\index{Category of seminormed vector spaces}
The category of topological spaces is the category $\Semi \Vect$\index[notation]{$\Semi \Vect$}
\begin{enumerate}
\item whose collection of objects $\Obj (\Semi \Vect )$ is the collection of all topological spaces;
\item with morphism set $\Mor _{\Semi \Vect}(V,W)$ precisely the set of all bounded linear maps from $V$ to $W$;
\item whose composition is given by ordinary function composition; and
\item whose identities are given by the identity functions.
\end{enumerate}
\end{exm}
\begin{exr}{}{}
Let $f\colon \coord{V,\cover{D}}\rightarrow \coord{W,\cover{E}}$ be a linear map between seminormed vector spaces.  Show that $f$ is bounded iff for every $\norm _0\in \cover{E}$ there are \emph{finitely-many} $\norm _1,\ldots ,\norm _m\in \cover{D}$ and constants $K_1,\ldots ,K_m\geq 0$ such that
\begin{equation}
\norm{f(v)}_0\leq K_1\norm{v}_1+\cdots +K_m\norm{v}_m
\end{equation}
for all $v\in V$.
\begin{rmk}
Compare this with the definition of bounded maps of semimetric spaces (\cref{BoundedMap}).  Essentially this boils down to the statement that, to check that $f$ is bounded, it suffices to check for $x_2=0$ and $x_1\eqqcolon v$ arbitrary (in the notation of \cref{BoundedMap}).
\end{rmk}
\end{exr}
Note that a priori a seminormed vector space is not a topological vector space.  However, being a semimetric space, it is in fact a uniform space, and so in turn is equipped with its uniform topology.  The question is then whether it is a topological vector space with respect to the uniform topology.  Of course, the answer is in the affirmative.  Once we know that the seminormed vector space $V$ is likewise a topological vector space, we know in turn that its underlying topological group $\coord{V,+,0,-}$ induces in turn yet another uniform structure, and so a new question arises as to whether or not this uniform structures agrees with the one induced from the semimetric space structure.  Of course, the answer to this is likewise in the affirmative.
\begin{exr}{}{}
Let $V$ be a seminormed vector space.  Show that $V$ is a topological vector space with respect to the uniform topology induced by the semimetric uniformity.
\end{exr}
\begin{exr}{}{}
Let $V$ be a seminormed vector space.  Show that the uniformity induced by the topological group structure $\coord{V,+,0,-}$ is the same as the semimetric uniformity.
\end{exr}
As a mater of fact, the morphisms don't care whether you're thinking of things as a semimetric space or as a topological group either.
\begin{exr}{}{exr4.3.57}
Let $f\colon \coord{V,\cover{D}}\rightarrow \coord{W,\cover{E}}$ be a \emph{linear} map between two seminormed vector spaces.  Show that it is continuous iff it is bounded.
\end{exr}
\begin{important}
Unless otherwise stated, seminormed vector spaces are always equipped with the uniformity induced by the semimetric space structure (or, equivalently, the uniformity induced by the topological group structure).
\end{important}

In fact, a lot of examples of seminormed vector spaces have \emph{even more} structure, namely, the structure of an algebra.
\begin{dfn}{Associative-algebra}{Algebra}
\\
An \term{associative-algebra}\index{Associative-algebra} is a set $A$ equipped with the structure of a vector space over a field $F$ $\coord{A,+,0,-,F}$ and the structure of a ring $\coord{A,+,0,-,\cdot ,1}$ such that
\begin{enumerate}
\item $(\alpha _1\alpha _2)\cdot a=\alpha _1\cdot (\alpha _2\cdot a)$ for $\alpha _1,\alpha _2\in F$ and $a\in A$; and
\item $\alpha \cdot (a_1a_2)=(\alpha \cdot a_1)a_2$ for $\alpha \in F$ and $a_1,a_2\in A$.
\end{enumerate}
\begin{rmk}
That is, an associative-algebra is both a vector space and a ring subject to a couple of compatibility axioms.
\end{rmk}
\begin{rmk}
The prefix ``associative'' here is to distinguish this concept from other things which go by the term ``algebra'', e.g.~``lie algebra'' or as used in general algebra.  For us, there is no just plain ``algebra''\footnote{Though you could certainly define such a thing, simply by dropping the assumption that the multiplication is associative, but as we don't a name for not-necessarily-associative rgs, we would either have to create such a term or explicitly list the axioms here, both of which are messy options.  As we have no need for not-necessarily-associative algebras, we simply do not bother.}
\end{rmk}
\end{dfn}
\begin{dfn}{Homomorphism (of associative-algebras)}{HomomorphismOfAlgebras}
Let $A$ and $B$ be associative-algebras and let $f\colon A\rightarrow B$ be a function.  Then, $f$ is a \term{homomorphism}\index{Homomorphism (of associative-algebras)} iff $f$ is both a linear map of the underlying vectors spaces and a ring homomorphism of the underlying vector spaces.
\end{dfn}
\begin{exm}{The category of associative-algebras over a field $F$}{}\index{Category of associative-algebras}
The category of associative-algebras over a field $F$ is the category $\Alg _F$\index[notation]{$\Alg _F$}
\begin{enumerate}
\item whose collection of objects $\Obj (\Alg _F )$ is the collection of all associative-algebras;
\item with morphism set $\Mor _{\Alg _F}(A,B)$ is precisely the set of all homomorphisms from $A$ to $B$;
\item whose composition is given by ordinary function composition; and
\item whose identities are given by the identity functions.
\end{enumerate}
\end{exm}
\begin{dfn}{Seminormed algebra}{SeminormedAlgebra}
A \term{seminormed algebra}\index{Seminormed algebra} is an associative-algebra whose underlying vector space is a seminormed vector space such that $\norm{a_1a_2}\leq \norm{a_1}\norm{a_2}$ for $a_1,a_2\in A$.
\begin{rmk}
Note that we don't mention the field we are working over because, by definition, seminormed vector spaces are always over $\R$---see \cref{SeminormedVectorSpace}.
\end{rmk}
\begin{rmk}
Note that seminormed algebras are \emph{associative}-algebras.  The term ``seminormed associative-algebras'' is unnecessarily verbose.
\end{rmk}
\end{dfn}
\begin{exm}{The category of seminormed algebras}{}\index{Category of seminormed algebras}
The category of seminormed algebras over a field $F$ is the category $\Semi \Alg{F}$\index[notation]{$\Semi \Alg{F}$}
\begin{enumerate}
\item whose collection of objects $\Obj (\Semi \Alg{F})$ is the collection of all seminormed algebras;
\item with morphism set $\Mor _{\Semi \Alg{F}}(A,B)$ is precisely the set of all bounded homomorphisms from $A$ to $B$;
\item whose composition is given by ordinary function composition; and
\item whose identities are given by the identity functions.
\end{enumerate}
\end{exm}
We're just about to put all these definitions to use for the purpose of introducing, among other things, the notion of \emph{uniform convergence}.  It will be important to be able to contrast this with the weaker notion of \emph{pointwise convergence}.
\begin{dfn}{Pointwise convergence}{PointwiseConvergence}
Let $X$ be a set,let $Y$ be a topological space, let $\lambda \mapsto f_{\lambda}\in \Mor _{\Set}(X,Y)$ be a net, and let $f_{\infty}\in \Mor _{\Set}(X,Y)$.  Then, $\lambda \mapsto f_{\lambda}$ \term{converges pointwise}\index{Pointwise convergence} to $f_{\infty}$ iff for every $x\in X$, $\lambda \mapsto f_{\lambda}(x)$ converges to $f_{\infty}(x)$.
\begin{rmk}
Thus, $\lambda \mapsto f_{\lambda}$ converges pointwise to $f_{\infty}$ iff if you plug in $x$ and take the limit, you get $f_{\infty}(x)$.
\end{rmk}
\end{dfn}
\begin{exm}{}{exm4.5.12x}
For $m\in \N$, define $f\colon [0,1]\rightarrow \R$ by $f(x)\coloneqq x^m$.  Then, $m\mapsto f_m$ converges pointwise to the function
\begin{equation}
f_{\infty}(x)\coloneqq \begin{cases}0 & \text{if }x\in [0,1) \\ 1 & \text{if }x=1.\end{cases}
\end{equation}
It turns out that this does \emph{not} converge uniformly to $f_{\infty}$ because, if it did, $f_{\infty}$ would need to be continuous---see \cref{thm4.5.6}.
\end{exm}
\begin{exr}{}{exr4.2.84}
Let $X$ be a set and let $Y$ be a topological space.  Show that the relation of pointwise convergence between nets in $\Mor _{\Set}(X,Y)$ and elements of $\Mor _{\Set}(X,Y)$ satisfies the axioms of \namerefpcref{KelleysConvergenceTheorem}, so as to define a unique topology on $\Mor _{\Set}(X,Y)$ for which the notion of convergence is precisely pointwise convergence, the \term{topology of pointwise convergence}.
\end{exr}

With these new definitions in hand, we now present an incredibly important example of a seminormed algebra, an example that was a large part of the motivation for introducing seminormed algebras at all.
\begin{exm}{Uniform convergence on quasicompact subsets}{exm4.3.60}
Let $X$ be a topological space and define
\begin{equation}
A\coloneqq \Mor _{\Top}(X,\R ).
\end{equation}
Pointwise addition and pointwise scalar multiplication gives $A$ the structure of a real vector space.  Pointwise multiplication gives $A$ in turn the structure of a real associative-algebra.  The collection $\{ \norm _K:K\subseteq X\text{ quasicompact}\}$, where
\begin{equation}
\norm{f}_K\coloneqq \sup _{x\in K}\{ \abs{f(x)}\} ,
\end{equation}\index[notation]{$\norm{f}_K$}
the \term{supremum seminorm}\index{Supremum seminorm} on $K$, then gives $A$ the structure of a seminormed algebra.  It is thus canonically a uniform space (and in turn a topological space).  If $X$ itself is quasicompact, convergence in $\Mor _{\Top}(X,\R )$ is called \term{uniform convergence}\index{Uniform convergence}.\footnote{Despite the name and the context in which we're presenting it, uniform convergence actually has nothing to do with uniform spaces per se (in contrast to uniform continuity, for example).  $\Mor _{\Top}(X,\R )$ has a topology, and hence a notion of convergence, which we happen to call ``uniform convergence''.  In particular, we only needed to equip $\Mor _{\Top}(X,\R )$ with a topology to define uniform convergence.  The reason we waited until the chapter on uniform spaces, of course, is because $\Mor _{\Top}(X,\R )$ obtains its topology from a family of semimetrics (or in the case $X$ is quasicompact, just a single metric), not because of any direct connection with uniform convergence and uniform spaces.}  Thus, in the general case, people refer to convergence in the topological space $\Mor _{\Top}(X,\R )$ as \term{uniform convergence on quasicompact subsets}, though, as this is never ambiguous, you can probably just say ``uniform convergence'' all the time for short\footnote{This is further justified in the exercise below.}.  This is important enough to spell out in detail.
\begin{important}
$\lambda \mapsto f_{\lambda}\in \Mor _{\Top}(X,\R )$ converges (uniformly) to $f_{\infty}\in \Mor _{\Top}(X,\R )$ iff for every quasicompact subset $K\subseteq X$ $\lambda \mapsto x_{\lambda}$ is eventually contained in $B_{\norm _K}(f_{\infty})\coloneqq \left\{ f\in \Mor _{\Top}(X,\R ):\norm{f-f_{\infty}}_K\right\} $.  
\end{important}
Even more explicitly:
\begin{important}
$\lambda \mapsto f_{\lambda}\in \Mor _{\Top}(X,\R )$ converges (uniformly) to $f_{\infty}\in \Mor _{\Top}(X,\R )$ iff for every quasicompact subset $K\subseteq X$ and for every $\varepsilon >0$ there is some $\lambda _0$ such that, whenever $\lambda \geq \lambda _0$, it follows that $\abs{f_{\lambda}(x)-f_{\infty}(x)}<\varepsilon$ for all $x\in K$.
\end{important}
This is in contrast to \emph{pointwise convergence} (\cref{PointwiseConvergence}):  here, we can choose a single $\lambda _0$ that `works' for all $x\in K$, whereas, in the pointwise case, your $\lambda _0$ will have to depend on $x$.  That is to say, we can choose $\lambda _0$ ``uniformly'' to work for \emph{all} $x\in K$.

The reason the case $X$ quasicompact is special is because, in this case, it is actually isomorphic (in the category of seminormed algebras) to a normed algebra.
\begin{exr}[breakable=false]{}{exr4.2.86}
Let $X$ be quasicompact.  Show that
\begin{equation}
\begin{multlined}
\id _X\colon \coord{X,\{ \norm _X\}}\rightarrow \\ \coord{X,\{ \norm _K:K\subseteq X\text{ quasicompact}\}} .
\end{multlined}
\end{equation}
is an isomorphism in the category of seminormed algebras.
\begin{rmk}
The point is that, if all we care about is the seminormed algebra structure, we may always assume without loss of generality that $\Mor _{\Top}(X,\R )$ is in fact a \emph{normed} algebra with the single norm being given by $\norm{f}_X\coloneqq \sup _{X\in X}\{ \abs{f(x)}\}$, the \term{supremum norm}\index{Supremum norm}.  In this case, we do not write $\norm{f}_X$ but rather $\norm{f}_\infty$\index[notation]{$\norm{f}_\infty$}.  The reason for this notation will become clear when we study $L^p$ spaces---see \crefnameref{Lp}.
\end{rmk}
\end{exr}
\begin{important}
Unless otherwise stated, $\Mor _{\Top}(X,\R )$ is a always given the structure of a seminormed algebra, the associative-algebra structure defined pointwise and the seminorms being $\norm _K$ for $K\subseteq X$ quasicompact.
\end{important}

Of \emph{incredible} importance is that this space is in fact complete (at least for so-called \emph{quasicompactly-generated} spaces).  Of course, we need to first actually define what we mean by complete, and so we postpone this result---see \cref{thm4.5.6}.
\end{exm}
Another reason to believe that the topology of uniform convergence on quasicompact subsets is a `natural' topology to use on $\Mor _{\Top}(X,\R )$ is that, in many cases, it agrees with another frequently-encountered topology, namely the quasicompact-open topology (\cref{XYZOpenTopology}).
\begin{prp}{}{prp4.2.99}
	The topology on $\Mor _{\Top}(X,\R )$ is the same as the quasicompact-open topology iff $X$ is completely-$T_2$.
	\begin{rmk}
		Recall that, as stated in the previous example, the topology on $\Mor _{\Top}(X,\R )$ is defined by the seminorms $\norm _K$, for $K\subseteq X$ quasicompact.
	\end{rmk}
	\begin{proof}
		By \cref{exr4.2.31x,prp4.1.8},\footnote{These results give respectively a neighborhood base for semimetric spaces and state how neighborhood bases define topologies.} a base for usual topology on $\Mor _{\Top}(X,\R )$ is given by the collection of sets of the form
		\begin{equation}\label{2.20}
			\begin{multlined}
			f+B_{\varepsilon _1,\ldots ,\varepsilon _n}^{K_1,\ldots ,K_m} \\ \ceqq f+\left\{ g\in \Mor _{\Top}(X,\R ):\left\| g\right\| _{K_k}<\varepsilon _k\right\}
			\end{multlined}
		\end{equation}
		for $m\in \Z ^+$, $K_k\subseteq X$ quasi-compact, and $\varepsilon _k>0$, where for convenience we have abbreviated $B_{\varepsilon _1,\ldots ,\varepsilon _m}^{K_1,\ldots ,K_m}\ceqq B_{\varepsilon _1,\ldots ,\varepsilon _m}^{\metric _{K_1},\ldots ,\metric _{K_m}}$.  On the other hand, a base for the quasicompact-open topology is given by
		\begin{equation}\label{2.21}
			\begin{multlined}
			O_{K_1,a_1,\varepsilon _1;\ldots ;K_n,a_n,\varepsilon _n} \\ \ceqq \left\{ f\in C(X;F):f(K_k)\subseteq B_{\varepsilon _k}(a_k)\right\}
			\end{multlined}
		\end{equation}
		for $K_k\subseteq X$ quasicompact, $a_k\in \R$, and $\varepsilon _k>0$.
		
		\blni
		$(\Rightarrow )$ Suppose that the topology on $\Mor _{\Top}(X,\R )$ is the same as the quasicompact-open topology.  Let $x,y\in X$.  Then, for every $\varepsilon >0$, the set $O_{\{ x\} ,1,\varepsilon ;\{ y\} ,0,\varepsilon}$ is open in the locally convex topology, and hence contains a set of the form
		\begin{equation}
		f+B_{\varepsilon _1,\ldots ,\varepsilon _n}^{K_1,\ldots ,K_n},
		\end{equation}
		and hence, in particular, contains $f$.  By taking $\varepsilon$ sufficiently small, we can guarantee that $f(y)\neq f(x)$, so that
		\begin{equation}
		\frac{f-f(y)}{f(x)-f(y)}
		\end{equation}
		is continuous, is $0$ at $y$, and $1$ at $x$.  Thus, $X$ is completely $T_2$.
		
		\blni
		$(\Leftarrow )$ Suppose that $X$ is completely $T_2$.  Let $K_1,\ldots ,K_n\subseteq X$ be arbitrary quasicompact subsets, let $a_1,\ldots ,a_n\in \R$ , and let $\varepsilon _1,\ldots ,\varepsilon _m>0$.  These arbitrary choices determine an arbitrary `$O$-set' of the form \eqref{2.21}.  Let $h$ be an element of this set.  We wish to show that it contains some set of the form \eqref{2.20} that contains $h$.  If $K_k$ meets $K_l$, then, in order that this ``$O$-set'' be nonempty, it had better be the case that $B_{\varepsilon _k}(a_k)$ meet $B_{\varepsilon _l}(a_l)$, in which case the intersection contains some $B_{\varepsilon}(a)\subseteq B_{\varepsilon _k}(a_k)\cap B_{\varepsilon _l}(a_l)$.  Now, remove $K_j$ and $K_k$ from this list and replace them respectively with just $\coord{K_j\cup K_k,a,\varepsilon}$.   After making these replacements, we obtain a new `$O$-set' smaller than the original, and furthermore, with the quasicompact sets disjoint.  As $X$ is completely-$T_2$, there exists a continuous function $f\colon X\rightarrow \R$ such that $f(K_k)=\{ a_k\}$ for $1\leq k\leq m$.  Then, $f+B_{\varepsilon _1,\ldots ,\varepsilon _n}^{K_1,\ldots ,K_m}$ is contained in this new `$O$-set', and hence is contained in the original `$O$-set' as well.  Furthermore
		\begin{equation*}
		\begin{split}
		\MoveEqLeft
		\sup \left\{ \abs{h(x)-f(x)}:x\in K_k\right\} \\
		& =\sup \left\{ \abs{h(x)-a_k}:x\in K_k\right\} <\varepsilon _k,
		\end{split}
		\end{equation*}
		and hence contains $h$, as desired. 
		
		For the other direction, let $f\in \Mor _{\Top}(X,\R )$, $K_1,\ldots ,K_n\subseteq X$ quasicompact, and $\varepsilon _1,\ldots ,\varepsilon _m>0$ all be arbitrary.  Then,
		{\small
		\begin{equation}\label{eqn4.2.105}
			\begin{split}
				\MoveEqLeft
				f+B_{\varepsilon _1,\ldots ,\varepsilon _n}^{K_1,\ldots ,K_n} \\
				& \ceqq f+\left\{ g\in \Mor _{\Top}(X,\R ):\left\| g\right\| _{K_k}<\varepsilon _k\right\} \\
				& =f+\left\{ g\in \Mor _{\Top}(X,\R ):g(K_k)\subseteq B_{\varepsilon _k}(0)\right\} 
			\end{split}
		\end{equation}
		}
		is an arbitrary set of the form \eqref{2.20}.  Finally, let $h$ be an arbitrary element of this set.   The function $x\mapsto \abs{h(x)-f(x)}$ achieves a maximum $\delta _k$ on $K_k$.  By hypothesis, we have $\delta _k<\varepsilon _k$.  Replace $\delta _k$ with $\frac{\delta _k+\varepsilon _k}{2}$, so that we still have $\delta _k<\varepsilon _k$, but we now also have the strict inequality $\abs{h(x)-f(x)}<\delta _k$ for all $x\in K_k$.
		
		For each $K_i$ and $x\in K_k$, let $U_x\subseteq K_k$ be open and such that $f$ is strictly within $\tfrac{\varepsilon _k-\delta _k}{2}$ of $f(x)$ on $U_x$.  Then, there is a finite cover---denote the \emph{closure} (in $K_k$) of each element of of this finite cover by $L_{k,1},\ldots ,L_{k,n_k}$ with $x_{k,l}\in L_{k,l}$ being the $x$ of $U_x$.  Thus, $K_k$ is covered by $L_{k,1},\ldots ,L_{k,n_k}$ and $f$ is strictly within $\varepsilon _k-\delta _k$ of $f(x_{k,l})$ on $L_{k,l}$  Do this for each $K_k$ to get compact sets $L_{k,l}$ for $1\leq k\leq n$ and $1\leq l\leq m_n$.  As $h$ is within $\delta _k$ of $f$ on all of $K_k$, it will certainly be within $\delta _k$ of $f(x_{k,l})$ for all $l$ as $x_{k,l}\in K_k$.  Thus,
		\begin{equation}\label{2.4.11}
			h\in O_{L_{1,1},f(x_{1,1}),\delta _1;\ldots ,L_{m,n_m},f(x_n,x_{n_m}),\delta _m}.
		\end{equation}
		Similarly, if $g$ is an element of this ``$O$-set'', we have that
		\begin{equation}
			\begin{split}
				\MoveEqLeft
				\sup \left\{ \abs{g(x)-f(x)}:x\in L_{k,l}\right\} \\
				& \leq \sup \left\{ \abs{g(x)-f(x_{k,l})}:x\in L_{k,l}\right\} \\ & \qquad +\sup \left\{ \abs{f(x_{k,l})-f(x)}:x\in L_{k,l}\right\} \\
				& <\delta _k+(\varepsilon _k-\delta _k)<\varepsilon _k.
			\end{split}
		\end{equation}
		Thus, $g$ is contained in the set in \eqref{eqn4.2.105}, and hence we have
		\begin{equation*}
			h\in O_{L_{1,1},f(x_{1,1}),\delta _1;\ldots ,L_{m,n_m},f(x_n,x_{n_m}),\delta _m}\subseteq f+B_{\varepsilon _1,\ldots ,\varepsilon _n}^{K_1,\ldots ,K_n},
		\end{equation*}
		as desired.
	\end{proof}
\end{prp}

\section[\texorpdfstring{$T_0$}{T0} uni.~spaces are uni.-completely-\texorpdfstring{$T_3$}{T3}]{\texorpdfstring{$T_0$}{T0} uniform spaces are uniformly-completely-\texorpdfstring{$T_3$}{T3}}

Our goal in this subsection is to show that all $T_0$ uniform spaces are uniformly-$T_3$.  Of course, to prove this, we had better say what uniformly-completely-$T_3$ means.

\subsection{Separation axioms in uniform spaces}

Throughout this subsection, let $S_1,S_2\subseteq X$ be \emph{disjoint} subsets of a uniform space $X$.
\begin{dfn}{Uniformly-distinguishable}{UniformlyDistinguishable}
$S_1$ and $S_2$ are \term{uniformly-distinguishable}\index{Uniformly-distinguishable} iff there is some uniform cover $\cover{U}$ for which $\Star _{\cover{U}}(S_1)$ does not intersect $S_2$ or there is a uniform cover $\cover{U}$ for which $\Star _{\cover{U}}(S_2)$ does not intersect $S_1$.
\end{dfn}
\begin{dfn}{Uniformly-separated}{UniformlySeparatedx}
$S_1$ and $S_2$ are \term{uniformly-separated} iff there is some uniform cover $\cover{U}$ for which $\Star _{\cover{U}}(S_1)$ does not intersect $S_2$ and there is a uniform cover $\cover{U}$ for which $\Star _{\cover{U}}(S_2)$ does not intersect $S_1$.
\end{dfn}
\begin{dfn}{Uniformly-separated}{UniformlySeparated}
$S_1$ and $S_2$ are \term{uniformly-separated by neighborhoods}\index{Uniformly-separated by neighborhoods} iff there is some uniform cover $\cover{U}$ for which $\Star _{\cover{U}}(S_1)$ is disjoint from $\Star _{\cover{U}}(S_2)$.
\end{dfn}
\begin{dfn}{Uniformly-completely-separated}{UniformlyCompletelySeparated}
\\ $S_1$ and $S_2$ are \term{uniformly-completely-separated}\index{Uniformly-completely-separated} iff there is a uniformly-continuous function $f\colon X\rightarrow [0,1]$ such that $\restr{f}{S_1}=0$ and $\restr{f}{S_2}=1$.
\end{dfn}
\begin{exr}{}{}
Show that if $S_1$ and $S_2$ are uniformly-completely-separated, then they are uniformly-separated.
\end{exr}
\begin{dfn}{Uniformly-perfectly-separated}{UniformlyPerfectlySeparated}
$S_1$ and $S_2$ are \term{uniformly-perfectly-separated}\index{Uniformly-perfectly-separated} iff there is a uniformly-continuous function $f\colon X\rightarrow [0,1]$ such that $S_1=f^{-1}(0)$ and $S_2=f^{-1}(1)$.
\end{dfn}
\begin{dfn}{Uniformly-$T_0$}{UniformlyT0}
$X$ is \term{uniformly-$T_0$}\index{Uniformly-$T_0$} iff any two distinct points are uniformly-distinguishable.
\end{dfn}
\begin{dfn}{Uniformly-$T_1$}{UniformlyT1}
$X$ is \term{uniformly-$T_1$}\index{Uniformly-$T_1$} iff any two distinct points are uniformly-separated.
\end{dfn}
\begin{dfn}{Uniformly-$T_2$}{UniformlyT2}
$X$ is \term{uniformly-$T_2$}\index{Uniformly-$T_2$} iff any two distinct points can be uniformly-separated by neighborhoods.
\end{dfn}
\begin{dfn}{Uniformly-completely-$T_2$\hfill}{UniformlyCompletelyT2}
$X$ is \term{uniformly-completely-$T_2$}\index{uniformly-completely-$T_2$} iff any two distinct points can be uniformly-completely-separated.
\end{dfn}
\begin{dfn}{Uniformly-perfectly-$T_2$\hfill}{UniformlyPerfectlyT2}
$X$ is \term{uniformly-perfectly-$T_2$}\index{uniformly-perfectly-$T_2$} iff any two distinct points can be uniformly-perfectly-separated.
\end{dfn}
\begin{dfn}{Uniformly-$T_3$}{UniformlyT3}
$X$ is \term{uniformly-$T_3$}\index{Uniformly-$T_3$} iff it is $T_1$ and any closed set and a point not contained in it can be uniformly-separated by neighborhoods.
\end{dfn}
\begin{dfn}{Uniformly-completely-$T_3$\hfill}{UniformlyCompletelyT3}
$X$ is \term{uniformly-completely-$T_3$}\index{Uniformly-completely-$T_3$} iff it is $T_1$ and any closed set and a point not contained in it can be uniformly-completely-separated.
\end{dfn}
\begin{dfn}{Uniformly-perfectly-$T_3$\hfill}{UniformlyPerfectlyT3}
$X$ is \term{uniformly-perfectly-$T_3$}\index{Uniformly-perfectly-$T_3$} iff it is $T_1$ and any closed set and a point not contained in it can be uniformly-perfectly-separated.
\end{dfn}
\begin{dfn}{Uniformly-$T_4$}{UniformlyT4}
$X$ is \term{uniformly-$T_4$}\index{Uniformly-$T_4$} iff it is $T_1$ and any two disjoint closed subsets can be uniformly-separated by neighborhoods.
\end{dfn}
\begin{dfn}{Uniformly-completely-$T_4$\hfill}{UniformlyCompletelyT4}
$X$ is \term{uniformly-completely-$T_4$}\index{Uniformly-completely-$T_4$} iff it is $T_1$ and any two disjoint closed subsets can be uniformly-completely-separated.
\end{dfn}
\begin{dfn}{Uniformly-perfectly-$T_4$\hfill}{UniformlyPerfectlyT4}
$X$ is \term{uniformly-perfectly-$T_4$}\index{Uniformly-perfectly-$T_4$} iff it is $T_1$ and any closed set and any two disjoint closed subsets can be can be uniformly-perfectly-separated.
\end{dfn}
The goal of this section is to prove that all of these separation axioms from uniformly-$T_0$ to uniformly-completely-$T_3$ (that is, $T_0$ implies uniformly-completely-$T_3$---see \cref{crl4.4.16}).  We also present counter-examples to show that all these equivalent axioms are strictly weaker than both uniformly-$T_4$ (see \cref{exm4.4.20}) and uniformly-perfectly-$T_3$ (see \cref{exm4.4.23}).  Before we begin our proof of that result, however, we present a smaller, but still quite useful result, which says that a relatively broad collection of spaces (metric spaces) satisfies the strongest separation axiom one could possibly hope for.
\begin{prp}{}{prp5.4.13}
Metric spaces are uniform\-ly-perfectly-$T_4$.
\begin{proof}
Let $X$ be a metric space and let $C_1,C_2\subseteq X$ be closed and disjoint.
\begin{exr}[breakable=false]{}{}
Show that there is some $f_1:X\rightarrow [0,\frac{1}{2}]$ uniformly-continuous, equal to $0$ precisely on $C_1$, and equal to $\frac{1}{2}$ on $C_2$.  Similarly, show that there is some $f_2:X\rightarrow [0,\frac{1}{2}]$ uniformly-continuous, equal to $\frac{1}{2}$ precisely on $C_2$, and equal to $0$ on $C_1$.
\end{exr}
Define $f\coloneqq f_1+f_2$.  Then, this is certainly $0$ on $C_1$ and $1$ on $C_2$.  Conversely, suppose that $f(x)=0$.  Then, in particular, $f_1(x)=0=f_2(x)=0$, and so in particular $x\in C_1$.  On the other hand, suppose that $f(x)=1$.  Then, we must have in particular that $f_2(x)=\frac{1}{2}$, which implies that $x\in C_2$.  Thus, $f^{-1}(0)=C_1$ and $f^{-1}(1)=C_2$, and hence $X$ is uniformly-perfectly-$T_4$.
\end{proof}
\end{prp}

\subsection{The key result}

We actually prove a stronger result which requires the notion of the \emph{diameter} of a set (in a metric space).
\begin{dfn}{Diameter}{Diameter}
Let $\coord{X,\metric}$ be a metric space and let $S\subseteq X$.  Then, the \term{diameter}\index{Diameter} of $S$, $\diam (S)$, is defined by
\begin{equation}
\diam (S)\coloneqq \sup _{x,y\in S}\{ \metric{x}{y}\} .
\end{equation}
\begin{rmk}
Of course, it may be the case that $\diam (S)=\infty$.
\end{rmk}
\end{dfn}
And now we are ready to state our key result.
\begin{thm}{}{}
Let $\cover{U}$ be a uniform cover of a $T_0$ uniform space $X$.  Then, there exists a metric space $Y$ and a uniformly-continuous surjective function $\q :X\rightarrow Y$ such that, if $\diam (S)<1$ for $S\subseteq Y$, then $\q ^{-1}(S)$ will be contained in some element of $\cover{U}$.
\begin{proof}\footnote{Proof adapted from \cite[pg.~8]{Isbell}.}
To construct $Y$, we shall put a semimetric on $X$ and then take the quotient set with respect to the equivalence relation of `being infinitely close to each other'.

\blankline
\Step{Construct a sequence of star-refinements of $\cover{U}$}
Let us write $\cover{U}_0\coloneqq \cover{U}$.  Then, we take a star-refinement $\cover{U}_1$ of $\cover{U}_0$, in turn another star-refinement $\cover{U}_2$ of $\cover{U}_1$, and so on.

\Step{Define $\ell (x_1,x_2)$ for $x_1,x_2\in X$}[stp4.8.76.2]
Define
{\scriptsize
\begin{equation}\label{eqn4.3.21}
\ell (x_1,x_2)\coloneqq \begin{cases}2 & \text{if }x_2\notin \Star _{\cover{U}_0}(x_1) \\ 2^{1-\max \{ m\in \N :x_2\in \Star _{\cover{U}_m}(x_1)\}} & \text{otherwise.}\end{cases}
\end{equation}
}
Note that this in particular implies that $\ell (x_1,x_2)=0$ if $x_2\in \Star _{\cover{U}_m}(x_1)$ for all $m\in \N$.  (We do need to make the other extreme case explicit as the maximum of the empty-set is $-\infty$.)  Thus, the statement that $X$ is $T_0$ (together with the fact that stars form a neighborhood base for the topology (\cref{UniformTopology})) implies that either $\ell (x_1,x_2)>0$ or $\ell (x_2,x_1)>0$.

\Step{Define $\ell (\cover{P})$ for paths $\cover{P}$}
For the purposes of this proof, a \emph{path} from $x_1$ to $x_2$ will be a finite sequence of points $\coord{x^\infty,x^1,\ldots ,x^m}$ with $x^\infty=x_1$ and $x^m=x_2$.\footnote{The superscripts (as opposed to subscripts) are for the purpose of not conflicting with the subscripts on $x_1$ and $x_2$.}  If $\cover{P}=\coord{x^\infty,\ldots ,x^m}$, then we define $\ell (\cover{P})\coloneqq \ell (x^\infty,x^1)+\ell (x^1,x^2)+\cdots +\ell (x^{m-1},x^m)$.   We shall call this the \emph{length} of the path.

\Step{Define the semimetric}
Finally, we define
\begin{equation*}
\begin{multlined}
\metric{x_1}{x_2}\coloneqq \min \left\{ 1,\inf \left( \left\{ \ell (\cover{P}):\right. \right. \right. \\ \left. \left. \left. \cover{P}\text{ is a path from }x_1\text{ to }x_2\text{ or a path from }x_2\text{ to }x_1\text{.}\right\} \right) \right\} .
\end{multlined}
\end{equation*}

\Step{Show that this is in fact a semimetric}
From the definition, we have that $\metric$ is symmetric (this is the reason for putting the ``or'' in the definition---note that the definition of $\ell (x_1,x_2)$ is not manifestly symmetric).  The triangle inequality follows from the fact that a path from $x_1$ to $x_3$ and a path from $x_3$ to $x_2$ gives us a path from $x_1$ to $x_2$, with the length of this new path being the sum of the lengths of the other two.  Thus, $\metric$ is in fact a semimetric.

\Step{Construct $Y$}
Define $x_1\sim x_2$ iff $\metric{x_1}{x_2}=0$.  That this is an equivalence relation follows from the fact that $\metric$ is a semimetric.  Thus, we may define
\begin{equation}
Y\coloneqq X/\sim .
\end{equation}

\Step{Construct the metric on $Y$}
We abuse notation and write the induced metric on $Y$ with the same symbol $\metric$ as the semimetric on $X$:
\begin{equation}
\metric{[x_1]_{\sim}}{[x_2]_{\sim}}\coloneqq \metric{x_1}{x_2}.
\end{equation}
\begin{exr}[breakable=false]{}{}
Check that $\metric$ on $Y$ is well-defined.
\end{exr}

\Step{Show that this in fact a metric}
$\metric$ on $Y$ is automatically symmetric and satisfies the triangle inequality because $\metric$ on $X$ does.  Furthermore, if $\metric{[x_1]_{\sim}}{[x_2]_{\sim}}=0$, then $\metric{x_1}{x_2}=0$, and so $x_1\sim x_2$ by the definition of $\sim$.  Thus, $\metric$ is indeed a metric on $Y$.

\Step{Define $\q :X\rightarrow Y$}
We take $\q :X\rightarrow Y$ to be the quotient map:  $\q (x)\coloneqq [x]_{\sim}$.  Of course $\q$ is surjective (all quotient maps are).

\Step{Show that $\q$ is uniformly-continuous}
We apply \cref{prp4.2.27x} which characterizes uniform-continuity for functions whose codomain is a metric space.  So, let $\varepsilon >0$.  We must find a uniform cover $\cover{U}$ of $X$ such that for every $U\in \cover{U}$, whenever $x_1,x_2\in U$, it follows that $\metric{\q (x_1)}{\q (x_2)}<\varepsilon$.  It suffices to show this for $\varepsilon \coloneqq 2^{1-m}$.  We show that $\cover{U}_m$ is a uniform cover that `works'.  So, let $U\in \cover{U}_m$ and let $x_1,x_2\in U$.   Then, in particular, $x_2\in \Star _{\cover{U}_m}(x_1)$, and so
\begin{equation}
\metric{\q (x_1)}{\q (x_2)}\coloneqq \metric{x_1}{x_2}\leq 2^{1-m}\eqqcolon \varepsilon .
\end{equation}

\Step{Finish the proof by proving the desired property of $\q$}
Let $S\subseteq Y$ and suppose that $\diam (S)<1$.  We wish to show that there is some $U\in \cover{U}_0$ such that $S\subseteq U$.  It suffices to show that for $x_1,x_2\in S$, there is some $U_{x_1,x_2}\in \cover{U}_1$ such that $x_1,x_2\in U_{x_1,x_2}$.  This is because, if this is true, then $S\subseteq \Star _{\cover{U}_1}(x_1)$, which in turn is contained in some element of $\cover{U}_0$ because $\cover{U}_1$ star-refines $\cover{U}_0$.

To show this, it suffices to show that if $\metric{x_1}{x_2}\leq 2^{1-m}$ for $m\in \Z ^+$, then there is some $U\in \cover{U}_m$ such that $x_1,x_2\in U$.  So, let $x_1,x_2\in X$ be such that $\metric{x_1}{x_2}\leq 2^{1-m}$.  Then, without loss of generality, there is some path $\coord{x^\infty,\ldots ,x^n}$ from $x_1$ to $x_2$ with
\begin{equation}\label{4.8.53}
\ell (x^\infty,x^1)+\cdots +\ell (x^{n-1},x^n)\leq 2^{1-m}.
\end{equation}
It thus suffices to show that, whenever \eqref{4.8.53} holds, there is some $U\in \cover{U}_m$ with $x^\infty,x^n\in U$.  We prove this by induction on $n$.  For $n=1$, \eqref{4.8.53} implies that $\ell (x_1,x_2),\ell (x_2,x_1)\leq 2^{1-m}$.  As was mentioned above in \cref{stp4.8.76.2}, because $X$ is $T_0$, at least one of these is strictly positive---without loss of generality suppose that $\ell (x_1,x_2)>0$.  Then, the fact that $\ell (x_1,x_2)\leq 2^{1-m}$ implies that
\begin{equation}
2^{1-\max \{ o\in \N :x_2\in \Star _{\cover{U}_o}(x_1)\}}\leq 2^{1-m},
\end{equation}
so that
\begin{equation}
m\leq \max \{ o\in \N :x_2\in \Star _{\cover{U}_o}(x_1)\} 
\end{equation}
which implies that $x_2\in \Star _{\cover{U}_{o}}(x_1)$ for some $o\geq m$, which implies that there is some $U\in \cover{U}_{o}$ such that $x_1,x_2\in U$.  As $\cover{U}_o$ star-refines $\cover{U}_m$, there is in particular some $U\in \cover{U}_m$ such that $x_1,x_2\in U$.  Thus, this does the case for $n=1$.  (Note that in fact we can take $U\in \cover{U}_{o}$---this will be important later.)

Now assume the result is true for all $k\leq n$.  We wish to prove the result for $n+1$.

We must have that $\ell (x^\infty,x^1)<2^{1-m}$, because otherwise we would have to have that $\ell (x^k,x^{k+1})=0$ for $k\geq 1$, in which case $x^k$ and $x^{k+1}$ lie in some $U\in \cover{U}_o$ for $o$ arbitrarily large.  We can then guarantee that $x^k$ for $k\geq 1$ are obtained in some element of $\cover{U}_{o}$, and hence, as $x^\infty$ and $x^1$ are obtained in some element of $\cover{U}_{o}$, everything is contained in some element of $\cover{U}_m$.

Thus, without loss of generality assume that $\ell (x^\infty,x^1)<2^{1-m}$, so that in fact $\ell (x^{\infty},x^1)\leq 2^{-m}$ (because of the definition of $\ell$ \eqref{eqn4.3.21}).  Then, there is some $k_0$ such that
\begin{equation}
\ell (x^\infty,x^1)+\cdots +\ell (x^{k_0-1},x^{k_0})\leq 2^{-m}
\end{equation}
(This is the same inequality with $m$ one larger.)  Take $k_0$ to be the largest such positive integer.  Similarly, there is some (largest) $l_0$ such that
\begin{equation}
\ell (x^{k_0},x^{k_0+1})+\cdots +\ell (x^{l_0-1},x^{l_0})\leq 2^{-m}
\end{equation}
By choice of $k_0$ and $l_0$, we have
\begin{equation*}
\ell (x^{\infty},x^1)+\cdots +\ell (x^{k_0-1},x^{k_0})+\ell (x^{k_0},x^{k_0+1})>2^{-m}
\end{equation*}
and
\begin{equation*}
\ell (x^{k_0},x^{k_0+1})+\cdots +\ell (x^{l_0-1},x^{l_0})+\ell (x^{l_0},x^{l_0+1})>2^{-m}.
\end{equation*}
In order that \eqref{4.8.53} still be satisfied, it thus must be the case that
\begin{equation}
\ell (x^{l_0},x^{l_0+1})+\cdots +\ell (x^{n-1},x^n)\leq 2^{-m}.
\end{equation}
By the induction hypotheses, we then must have in particular that there are $U_1,U_2,U_3\in \cover{U}_{m}$ such that $x^\infty,x^{k_0}\in U_1$, $x^{l_0},x^n\in U_2$, and $x^{k_0},x^{l_0}\in U_3$.  There is some $U\in \cover{U}_m$ such that $\Star _{\cover{U}_{m+1}}(U_3)\subseteq U$.  However, as $U_1,U_2\subseteq \Star _{\cover{U}_{m+1}}(U_3)$, we have that $x^{\infty},x^n\in U$, and this completes the proof.
\end{proof}
\end{thm}
From this, that every $T_0$ uniform space is uniformly-completely-$T_3$ follows relatively easily.
\begin{crl}{}{crl4.4.16}
Let $X$ be a $T_0$ uniform space.  Then, $X$ is uniformly-completely-$T_3$.
\begin{proof}\footnote{Proof adapted from \cite[pg.~8]{Isbell}.}
Let $X$ be a uniform space.  We show that uniformly-continuous functions on $X$ can separate closed sets from points.  So, let $C\subseteq X$ be closed, and let $x_0\in C^{\comp}$.  As $C$ is closed, there must be some neighborhood of $x_0$ that does not intersect $C$ (otherwise, $x_0$ would be an accumulation point of $C$).  Then, because stars form a neighborhood base for the topology (\cref{UniformTopology}), there is a uniform cover $\cover{U}$ such that
\begin{equation}\label{4.8.90}
\Star _{\cover{U}}(x_0)\subseteq C^{\comp}.
\end{equation}

Now apply the previous theorem for the uniform cover $\cover{U}$, so that there is a metric space $\coord{Y,\metric}$ and a uniformly-continuous map $\q :X\rightarrow Y$ such that, if $\diam (S)<1$ for $S\subseteq Y$, it follows that $\q ^{-1}(S)$ is contained in some element of $\cover{U}$.  From \eqref{4.8.90}, it follows that $C\cup \{ x_0\}$ is not contained in any element of $\cover{U}$, and so
\begin{equation}\label{4.8.97}
\diam (\q (C)\cup \{ \q (x_0)\})\geq 1.
\end{equation}
Define $f\colon X\rightarrow [0,1]$ by
\begin{equation}
f(x)\coloneqq \footnote{See \eqref{4.8.50} for the definition of $\dist _C$.}\max \{ \dist _C(x) ,1\} .
\end{equation}
This is uniformly-continuous because $\dist _C$ is.  \eqref{4.8.97} implies that $\dist _C(x_0)\geq 1$, and so $f(x_0)=1$.  We showed in \cref{prp4.8.49} that $\dist _C(C)=0$.

Finally, we check that $X$ is $T_1$.  We know that $X$ is $T_0$ by hypothesis, and so by \cref{prp4.6.53} (regular $T_0$ spaces are $T_2$), $X$ is $T_2$, hence $T_1$ (that uniformly-continuous functions separate closed sets from points in particular implies regularity).
\end{proof}
\end{crl}

\subsection{The counter-examples}

We know from the diagram \eqref{4.6.105}, that if we are to `do any better' in terms of separation axioms, we would be able to prove that every uniform space is either perfectly-$T_3$ or completely-$T_4$ (which is equivalent to $T_4$ by \namerefpcref{UrysohnsLemma}).  Unfortunately, however, there exist counter-examples to both these separation axioms.
\begin{exm}{A uniform space that is not perfectly-$T_3$}{exm4.4.20}
The Uncountable Fort Space $X$ of \cref{UncountableFortSpace} will do just fine yet again.  We already know that this space is not perfectly-$T_3$ from \cref{exm4.6.80}.  Thus, all that remains to be done is to equip $X$ with a uniformity that generates the Uncountable Fort Space Topology.
\begin{exr}[breakable=false]{}{}
Show that the uniform topology of the initial uniformity with respect to $\left\{ f\colon X\rightarrow \R :f\in \Mor _{\Top}(X,\R )\right\}$ is the Uncountable Fort Space Topology.
\end{exr}
\end{exm}
\begin{exm}{A uniform space that is not $T_4$}{exm4.4.23}
Equip $\Mor _{\Set}(\R ,\R )$ with the topology of pointwise convergence---see \cref{exr4.2.84}.
\begin{exr}[breakable=false]{}{}
Show that this definition satisfies the axioms of \nameref{KelleysConvergenceTheorem}, and so defines a topology on $\Mor _{\Set}(\R ,\R )$.
\end{exr}
\begin{exr}[breakable=false]{}{}
Show that $\coord{\Mor _{\Set}(\R ,\R ),+}$ is a topological group, where $+$ is defined pointwise:
\begin{equation}
[f_1+f_2](x)\coloneqq f_1(x)+f_2(x).
\end{equation}
\end{exr}
Thus, $\Mor _{\Set}(\R ,\R )$ is canonically a uniform space.  Furthermore, as it is $T_0$,\footnote{Recall that (\cref{prp4.5.37}) a space is $T_2$ iff limits are unique.  As our definition of convergence obviously has this property, our space is automatically $T_2$, hence $T_0$.} it is uniformly-completely-$T_3$.

We now shows that $\Mor _{\Set}(\R ,\R )$ is not completely-$T_4$ with respect to this topology.  To do this, we first show that $\Mor _{\Set}(\R ,\Z )\subseteq \Mor _{\Set}(\R ,\R )$ is not completely-$T_4$.\footnote{The proof of this is adapted from \cite[pg.~206]{Munkres}.}  Note that it is \emph{not} immediate just from this that $\Mor _{\Set}(\R ,\R )$ is not $T_4$, as subspaces of $T_4$ spaces need not be $T_4$ in general (\cref{exm4.4.69}).

For $m\in \Z$, define
\begin{equation*}
P_m\coloneqq \left\{ f\in \Mor _{\Set}(\R ,\Z ):\restr{f}{[f^{-1}(m)]^{\comp}}\text{ is injective.}\right\} ,
\end{equation*}
that is, the set of all functions that are injective `modulo sending more than one point to $m$'.  We show that $P_0$ and $P_1$ are closed and disjoint, but cannot be separated by open neighborhoods.

We first check that $P_0$ is closed (the proof that $P_1$ is closed is nearly identical).  So, let $\lambda \mapsto f_\lambda \in P_0$ converge to $f_\infty$.  Suppose that $f_\infty (x_1)=f_\infty (x_2)$ is distinct from $0$.  Then, by our definition of convergence and the fact that our functions are taking values in the integers, it must be the case that $\lambda \mapsto f_\lambda (x_i)$ is eventually equal to $f_\infty (x_i)$ for $i=1,2$.  Then, in particular, we will have that $f_{\lambda _0}(x_1)=f_{\infty}(x_1)=f_{\infty}(x_2)=f_{\lambda _0}(x_2)$ for $\lambda _0$ sufficiently large, and hence $x_1=x_2$.  Thus, $f_\infty \in P_0$.

We now check that $P_0$ and $P_1$ are disjoint.  If $f$ is injective on the complement of $f^{-1}(0)$ (i.e.~if $f\in P_0$), then this complement must be countable (because $f$ will restrict to an injection from this complement into $\Z$, which is countable).  In particular, there must be at least two elements in $f^{-1}(0)$, and so $f(x_1)=0=f(x_2)$ for $x_1\neq x_2$.  But then $f$ cannot be injective on the complement of $f^{-1}(1)$, and so $f\notin P_1$.  Thus, $P_0$ is disjoint from $P_1$.

Let $A\coloneqq \{ \alpha _0,\alpha _1,\alpha _2,\ldots \}$ be a countably-infinite subset of $\R$, and for $S\subseteq A$ a finite and $f\in \Mor _{\Set}(\R ,\Z )$, let us define
\begin{equation}
U_{S,f}\coloneqq \{ g\in \Mor _{\Set}(\R ,\Z ):\restr{g}{S}=\restr{f}{S}\} ,
\end{equation}
that is, the set of functions which agree with $f$ on $S$.  The complement of this is the collection of all functions which differ from $f$ at at least one point of $S$.  Because the functions take their value in $\Z$, however, if you take a net of such functions, the limit (if it has one) must still disagree with $f$ at least one point.  Therefore, $U_{S,f}^{\comp}$ is closed, and hence $U_{S,f}$ is open.  Moreover, as
\begin{equation}
U_{S,f}\cap U_{T,f}=U_{S\cup T,f},
\end{equation}
it follows from \cref{prp4.1.8} that this is a neighborhood base for $f$.  If fact, if we restrict ourselves to only taking $S$ from a given infinite subset of $A$, we still get a neighborhood base.

Now, let $U$ and $V$ be open neighborhoods of $P_0$ and $P_1$ respectively.  We seek to show that $U$ and $V$ must intersect.  To do so, we construct a sequence of functions $f_k\in U$ and a countably-infinite collection of finite subsets $B_k=\{ \alpha _0,\ldots ,\alpha _{m_k}\}$ of $A$ such that $U_{B_k,f_k}\subseteq U$ and
\begin{equation}\label{5.5.29}
f_k(x)\coloneqq \begin{cases} k & \text{if }x=\alpha _k\in B_{k-1} \\ 0 & \text{otherwise.}\end{cases}
\end{equation}
We do so inductively.  (We take $B_0\coloneqq \emptyset$.)

Take $f_1\coloneqq 0$, so that of course $f_1\in P_0$.  Thus, because $\left\{ U_{S,f_1}:S\subseteq A\text{ finite.}\right\}$ is a neighborhood base at $f_1$, there must be some finite subset $B_1\subseteq A$ such that $U_{B_1,f_1}\subseteq U$.  In fact, we can enlarge $B_1$ so that it is of the form $B_1=\{ \alpha _0,\ldots ,\alpha _{m_1}\}$ (as making $B_1$ larger makes the neighborhood smaller).  Now define $f_2$ according to \eqref{5.5.29}, that is
\begin{equation}
f_2(x)\coloneqq \begin{cases}k & \text{if }x=\alpha _k\in B_1 \\ 0 & \text{otherwise.}\end{cases}
\end{equation}
Then, $f_2\in P_0$, so that $f_2\in U$, and so there must be some finite set $B_2\subseteq A$ such that $U_{B_2,f_2}\subseteq U$.  Again, by enlarging $B_2$ if necessary, we can guarantee it is of the form $B_2=\{ \alpha _0,\ldots ,\alpha _{m_2}\}$ with $m_2>m_1$.  Then, we may define
\begin{equation}
f_3(x)\coloneqq \begin{cases}k & \text{if }x=\alpha _k\in B_2 \\ 0 & \text{otherwise.}\end{cases}
\end{equation}
Then, for the same reason as before, $f_3\in U$, and so there is some finite set $B_3\subseteq A$ (that is without loss of generality of the form $B_3=\{ \alpha _0,\ldots ,\alpha _{m_3}\}$ with $m_3>m_2$) and $U_{B_3,f_3}\subseteq U$.  Continue this process inductively.

Now define $g\in \Mor _{\Set}(\R ,\Z )$ by
\begin{equation}
g(x)\coloneqq \begin{cases}k & \text{if }x=\alpha _k\in A \\ 1 & \text{otherwise.}\end{cases}
\end{equation}
Then, $g\in P_1$, and so there is some finite set $B\subseteq A$ such that $U_{B,g}\subseteq V$.  Let $m$ be sufficiently large so that $B\subseteq B_m$.  Then, $f_m\in U_{B,g}\subseteq V$ and $f_m\in U_{B_m,f_m}\subseteq U$, and so, in particular, lies in $U\cap V$.  Thus, $P_0$ and $P_1$ are disjoint closed sets which cannot be separated by neighborhoods.

This shows that $\Mor _{\Set}(\R ,\Z )$ is not $T_4$, but we still must show that $\Mor _{\Set}(\R ,\\R )$ itself is not $T_4$.  This will follow from the following lemma.
\begin{lma}[break at=8cm/25cm]{}{}
Let $X$ be $T_4$ and let $C\subseteq X$ be closed.  Then, $C$ is $T_4$.
\begin{rmk}
Note that this doesn't hold in general.  For example, this example shows that $\prod _{\R}\R \subseteq \prod _{\R}[-\infty ,\infty ]$ is not normal even though $\prod _{\R}[-\infty ,\infty ]$ is (it is compact by \cref{exr4.6.38} and \namerefpcref{TychonoffsTheorem}, and hence $T_4$ by \cref{prp4.6.83}).
\end{rmk}
\begin{proof}
Let $C_1,C_2\subseteq C$ be disjoint and closed.  Then, by definition of the subspace topology (\cref{SubspaceTopology}), $C_1=C_1'\cap C$ and $C_2=C_2'\cap C$ for $C_1',C_2'\subseteq X$ closed, and so $C_1$ and $C_2$ are themselves closed in $X$ because $C$ is closed.  Thus, because $X$ is $T_4$, $C_1$ and $C_2$ can be separated by neighborhoods in $X$, and hence can be separated by neighborhoods in $C$.
\end{proof}
\end{lma}
\end{exm}

Finally, we are ready to begin discussing Cauchyness and completeness in the general context of uniform spaces.

\section{Cauchyness and completeness}

\begin{dfn}{Cauchyness}{Cauchyness}
Let $\coord{X,\uniformity{U}}$ be a uniform space and let $\lambda \mapsto x_\lambda$ be a net.  Then, we say that $\lambda \mapsto x_\lambda$ is \term{Cauchy}\index{Cauchy (in uniform spaces)} iff for every $\cover{U}\in \uniformity{U}$, there is some $U\in \cover{U}$ such that $\lambda \mapsto x_\lambda$ is eventually contained in $U$.
\begin{rmk}
You should compare this with our definition of Cauchyness in $\R$, \cref{dfn3.3.26}.  As the collection of all $\varepsilon$-balls for all $\varepsilon >0$ forms a uniform base, then the definition we gave before in \cref{dfn3.3.26} will be a word-for-word special case of this definition (once we show that we can replace $\uniformity{U}$ with any uniform base in the above definition---see \cref{prpB.16}).  Indeed, part of the motivation for phrasing the definition in \cref{dfn3.3.26} the way we did was to make the transition to this higher level of generality as transparent as possible---see the paragraphs that follow \cref{dfn3.3.26} for a few more comments regarding this.
\end{rmk}
\end{dfn}
To check whether a net is Cauchy, it suffices to check on just a uniform base for the uniformity.
\begin{prp}{}{prpB.16}
Let $X$ be a set, let $\uniformity{B}$ be a uniform base on $X$, and let $\lambda \mapsto x_\lambda$ be a net.  Then, $\lambda \mapsto x_\lambda$ is Cauchy iff for every $\cover{B}\in \uniformity{B}$ there is some $B\in \cover{B}$ such that $\lambda \mapsto x_\lambda$ is eventually contained in $B$.
\begin{rmk}
With this equivalence, the definition we gave before for Cauchyness in $\R$ in \cref{dfn3.3.26} is literally verbatim equivalent to this definition upon replacement of $\uniformity{B}$ with $\{ \cover{B}_\varepsilon :\varepsilon >0\}$ and of $\cover{B}$ with $\cover{B}_\varepsilon \coloneqq \{ B_\varepsilon (x):x\in \R \}$.
\end{rmk}
\begin{proof}
$(\Rightarrow )$ There is nothing to check.

\blankline
\noindent
$(\Leftarrow )$ Suppose that for every $\cover{B}\in \uniformity{B}$ there is some $B\in \cover{B}$ such that $\lambda \mapsto x_\lambda$ is eventually contained in $B$.  Denote the uniformity on $X$ by $\uniformity{U}$.  Let $\cover{U}\in \uniformity{U}$.  Then, there is some $\cover{B}\in \uniformity{B}$ such that $\cover{B}\llcurly \cover{U}$.  Thus, there is some $B\in \cover{B}$ such that $\lambda \mapsto x_\lambda$ is eventually contained in $B$.  As $\cover{B}\llcurly \cover{U}$, there is some $U\in \cover{U}$ such that $\Star _{\cover{B}}(B)\subseteq U$.   In particular, $B\subseteq U$, and so if $\lambda \mapsto x_\lambda$ is eventually contained in $B$, it is certainly eventually contained in $U$.
\end{proof}
\end{prp}
From this, we obtain relatively nice description of what it means to be Cauchy in our two large families of examples, namely semimetric spaces and topological groups.
\begin{exr}{}{exr4.5.3x}
Let $\coord{X,\cover{D}}$ be a semimetric space and let $\lambda \mapsto x_\lambda \in X$ be a net.  Show that $\lambda \mapsto x_\lambda$ is Cauchy iff for every $\metric \in \cover{D}$ and for every $\varepsilon >0$, $\lambda \mapsto x_\lambda$ is eventually contained in $B_{\varepsilon}^{\metric}(x)$ for \emph{some} $x\in X$.
\begin{rmk}
In other words, a net in a semimetric space is Cauchy iff it is Cauchy with respect to each semimetric.
\end{rmk}
\end{exr}
\begin{exr}{}{exr4.4.4}
Let $G$ be a topological group and let $\lambda \mapsto g_\lambda \in G$ be a net.  Show that the following are equivalent.
\begin{enumerate}
\item $\lambda \mapsto g_\lambda$ is Cauchy with respect to the left uniformity.
\item For every open neighborhood $U$ of the identity there is \emph{some} $g\in G$ such that $\lambda \mapsto g_\lambda$ is eventually contained in $gU$.
\end{enumerate}
Similarly, show that the following are equivalent.
\begin{enumerate}
\item $\lambda \mapsto g_\lambda$ is Cauchy with respect to the left uniformity.

\item For every open neighborhood $U$ of the identity there is \emph{some} $g\in G$ such that $\lambda \mapsto g_\lambda$ is eventually contained in $Ug$.
\end{enumerate}
On the other hand, find an example of a topological group $G$ and a net $\lambda \mapsto g_{\lambda}\in G$ that is Cauchy with respect to the left uniformity but not the right.
\end{exr}

Just as continuous functions preserve convergence, so to do uniformly-continuous functions preserve Cauchyness.
\begin{exr}{}{exr4.5.3}
Let $f\colon X\rightarrow Y$ be uniformly-continuous and let $\lambda \mapsto x_\lambda \in X$ be Cauchy.  Show that $\lambda \mapsto f(x_\lambda)$ is Cauchy.
\begin{wrn}
Warning:  Continuous functions do \emph{not} necessarily preserve Cauchyness.
\end{wrn}
\end{exr}
\begin{exm}{A continuous image of a Cauchy net need not be Cauchy}{}
The net $\Z ^+\ni m\mapsto \frac{1}{m}\in \R ^+$ is Cauchy.  On the other hand, its image under the continuous map $\R ^+\ni x\mapsto \frac{1}{x}\in \R ^+$ is $\Z ^+\ni m\mapsto m\in \R ^+$, which is not even eventually bounded, much less Cauchy.
\end{exm}

Of course, if we know what it means for nets to be Cauchy, then we likewise have a notion of what it means for uniform spaces to be \emph{complete}.
\begin{dfn}{Completeness}{Completeness}
A uniform space is \term{complete}\index{Complete (uniform space)} iff every Cauchy net converges.
\begin{rmk}
In case there might be some confusion (e.g.~if the topology of the underlying uniform space comes from a totally-ordered set), then we shall say \term{Cauchy-complete} in contrast to \emph{Dedekind complete}.  We made a big deal about not merely saying ``complete'' when we meant Dedekind-complete.  It's more important in that case as most of the Dedekind-complete things we ran into had canonical uniformities (though we didn't know it at the time), and so there is an ambiguity of Dedekind-completeness vs.~Cauchy-completeness.  On the other hand, most uniform spaces don't have an order structure, in which case there is no potential for ambiguity.  In any case, besides being slightly more verbose, it doesn't hurt to clarify.
\end{rmk}
\end{dfn}

By now, you've probably got the impression that quasicompactness is kind-of a strong condition.  Indeed, as we're about to see, quasicompactness implies completeness for uniform spaces.  However, we can do one better, and get an ``iff'' statement by introducing another relatively natural condition, that of \emph{total-boundedness}.
\begin{dfn}{Totally-bounded}{TotallyBounded}
Let $X$ be a uniform space.  Then, $X$ is \term{totally-bounded}\index{Totally-bounded} iff every uniform cover of $X$ has a finite subcover.
\begin{rmk}
This is just like the definition of quasicompactness, except that we only require that \emph{uniform} covers have finite subcovers, instead of \emph{all} covers.
\end{rmk}
\end{dfn}
The first thing we will want to know is that, as usual, it suffices to check this property on a uniform base.
\begin{prp}{}{}
Let $X$ be a uniform space and let $\uniformity{B}$ be a uniform base for $X$.  Then, $X$ is totally-bounded iff every $\cover{B}\in \uniformity{B}$ has a finite subcover.
\begin{proof}
$(\Rightarrow )$ There is nothing to check.

\blankline
\noindent
$(\Leftarrow )$ Suppose that every $\cover{B}\in \uniformity{B}$ has a finite subcover.  Let $\cover{U}$ be a uniform cover of $X$.  Then, there is a cover $\cover{B}\in \uniformity{B}$ that star-refines $\cover{U}$.  By hypothesis, there are $B_1,\ldots ,B_m\in \cover{B}$ with $X=B_1\cup \cdots \cup B_m$.  From the definition of star-refinement, there is a $U_k\in \cover{U}$ such that $\Star _{\cover{B}}(B_k)\subseteq U_k$ for $1\leq k\leq m$.  In particular, $B_k\subseteq U_k$, and so $X=U_1\cup \cdots \cup U_m$.  Thus, $\{ U_1,\ldots ,U_m\}$ is a finite subcover of $\cover{U}$, and so every uniform cover of $X$ has a finite subcover, as desired.
\end{proof}
\end{prp}
And now we present the result that was the motivation for the introduction of the condition ``totally-bounded'' in the first place:  an equivalence between quasicompactness, and the two conditions of completeness and totally-bounded together.
\begin{thm}{}{thm4.4.8}
Let $X$ be a uniform space.  Then, $X$ is quasicompact iff $X$ is complete and totally-bounded.
\begin{rmk}
This can loosely be thought of as a generalization of the \namerefpcref{HeineBorelTheorem}, with ``closed'' being replaced by ``complete'' and ``bounded'' being replaced by ``totally-bounded''.
\end{rmk}
\begin{proof}
$(\Rightarrow )$ Suppose that $X$ is quasicompact.  $X$ is totally-bounded by definition, and so it suffices to show that $X$ is complete.  So, let $\lambda \mapsto x_{\lambda}\in X$ be Cauchy.  As $X$ is quasicompact, there is a subnet $\mu \mapsto x_{\lambda _{\mu}}\in X$ of $\lambda \mapsto x_{\lambda}$ converging to $x_{\infty}\in X$.  We wish to show that $\lambda \mapsto x_{\lambda}$ converges to $x_{\infty}$.

To do this, we apply \cref{exr3.2.49} (characterization of convergence in topologies defined by neighborhood bases).  So, let $\cover{U}$ be a uniform cover.  We wish to show that $\lambda \mapsto x_{\lambda}$ is eventually contained in $\Star _{\cover{U}}(x_{\infty})$.

Let $\cover{V}$ be a star-refinement of $\cover{U}$.  As $\mu \mapsto x_{\lambda _{\mu}}$ converges to $x_{\infty}$, $\mu \mapsto x_{\lambda _{\mu}}$ is eventually contained in $\Star _{\cover{V}}(x_{\infty})$.

Furthermore, as $\lambda \mapsto x_{\lambda}$ is Cauchy, there is some $V\in \cover{V}$ such that $\lambda \mapsto x_{\lambda}$ is eventually contained in $V$.  As $V$ eventually contains $\lambda \mapsto x_{\lambda}$, by the definition of a subnet (\cref{Subnet}), it eventually contains $\mu \mapsto x_{\lambda _{\mu}}$.  In particular, there is some $\lambda _{\mu _0}$ such that $x_{\lambda _{\mu _0}}\in V\cap V'$ for some $V'\in \cover{V}$ with $x_{\infty}\in V'$ (because $\mu \mapsto x_{\lambda _{\mu}}$ is also eventually contained in $\Star _{\cover{V}}(x_{\infty})$).

Let $U\in \cover{U}$ be such that $\Star _{\cover{V}}(V)\subseteq U$.  As $V'\in \cover{V}$ intersects $V$, we have that $V'\subseteq \Star _{\cover{V}}(V)\subseteq U$.  In particular, $x_{\infty}\in U$, so that $U\subseteq \Star _{\cover{U}}(x_{\infty})$.  On the other hand, $\lambda \mapsto x_{\lambda}$ is eventually contained in $V\subseteq \Star _{\cover{V}}(V)\subseteq U\subseteq \Star _{\cover{U}}(x_{\infty})$, and so $\lambda \mapsto x_{\lambda}$ is eventually contained in $\Star _{\cover{U}}(x_{\infty})$, as desired.

\blankline
\noindent
$(\Leftarrow)$ Suppose that $X$ is complete and totally-bounded.  Let $\lambda \mapsto x_{\lambda}\in X$ be an ultra-net (\cref{UltraNet}).  To show that $X$ is quasicompact, it suffices to show that $\lambda \mapsto x_{\lambda}$ converges (\cref{prp3.3.16}).  To do that, by completeness, it suffices to show that $\lambda \mapsto x_{\lambda}$ is Cauchy.  So, let $\cover{U}$ be a uniform cover.  To show that $\lambda \mapsto x_{\lambda}$ is Cauchy, it suffices to show that $\lambda \mapsto x_{\lambda}$ is eventually contained in some element of $\cover{U}$.  Now, as $X$ is totally bounded, there are finitely many $U_1,\ldots ,U_m\in \cover{U}$ such that $X=U_1\cup \cdots \cup U_m$.  As $\lambda \mapsto x_{\lambda}$ is an ultra-net, it is eventually contained in $U_1$ or $U_1^{\comp}$.  If it is eventually contained in $U_1$, we're done, so suppose that it is eventually contained in $U_1^{\comp}=U_1^{\comp}\cap (U_2\cup \cdots \cup U_m)$.  Once again, if it is eventually contained in $U_2$, we're done, so suppose that is not the case.  Proceed inductively, we eventually find that $\lambda \mapsto x_{\lambda}$ is eventually contained in $U_1^{\comp}\cap \cdots \cap U_{m-1}^{\comp}\cap U_m$, in which case it is eventually contained in $U_m$, as desired.
\end{proof}
\end{thm}

\subsection{Completeness of \texorpdfstring{$\Mor _{\Top}(X,\R )$}{MorTop(X,R)}}\label{sbs4.4.1}

We mentioned back above in \cref{exm4.3.60} that $\Mor _{\Top}(X,\R )$ is complete.  We now prove this.
\begin{thm}{}{thm4.5.6}
Let $X$ be a topological space that has the property that a subset is open iff its intersection with each quasicompact subset $K$ is open in $K$.  Then, $\Mor _{\Top}(X,\R )$ is complete.
\begin{rmk}
In particular, the limit of a uniformly convergent\footnote{Recall (\cref{exm4.3.60}) that ``uniform convergence'' is just the notion of convergence in the topological space $\Mor _{\Top}(X,\R )$ (or I suppose ``uniform convergence on quasicompact subsets'' if you want to be verbose about it).  As this is a semimetric space, concretely what this is means, $\lambda \mapsto f_{\lambda}\in \Mor _{\Top}(X,\R )$ converges to $f_{\infty}$ iff for every quasicompact subset $K\subseteq X$, $\lambda \mapsto \sup _{x\in K}\left\{ \norm{f_{\lambda}(x)-f_{\infty}(x)}\right\}$ converges to $0$.} net of continuous functions is continuous.  This need not be the case if the convergence is just pointwise---see \cref{exm4.5.12x}.
\end{rmk}
\begin{rmk}
This condition on $X$ is called \term{quasicompactly-generated}\index{Quasicompactly-generated}.  For example, the cocountable topology on $\R$ is \emph{not} quasicompactly-generated---see \cref{exm4.5.12}.
\end{rmk}
\begin{proof}
\Step{Prove the result for $X$ quasicompact}
We first take $X$ to be quasicompact.  In this case, the uniform structure on $\Mor _{\Top}(X,\R )$ is the same as that generated by the single norm $\norm _X$ (\cref{exr4.2.86}).  Therefore, by \cref{exr4.5.3x} (Cauchyness in semimetric spaces), to show that $\Mor _{\Top}(X,\R )$ is complete, it suffices to show that every net that is Cauchy with respect to $\norm _X$ converges.  So, suppose that $\lambda \mapsto f_\lambda \in \Mor _{\Top}(X,\R )$ is Cauchy.  As, for every $x\in X$,
\begin{equation}
\abs{f_{\lambda _1}(x)-f_{\lambda _2}(x)}\leq \norm{f_{\lambda _1}-f_{\lambda _2}},
\end{equation}
it follows that, for each $x\in X$, the net $\lambda \mapsto f_\lambda (x)\in \R$ is Cauchy.  As $\R$ is complete, each of these nets has a limit.  Call this limit $f_\infty (x)$.  We need to check two things:  (i)~that $x\mapsto f_\infty (x)$ is continuous (so that indeed $f_\infty \in \Mor _{\Top}(X,\R )$), and (ii)~that $\lambda \mapsto f_\lambda$ converges to $f_\infty$ in $\Mor _{\Top}(X,\R )$.

We first show that $f_\infty$ is continuous.  Let $\varepsilon >0$.  Let $\lambda _0$ be such that, whenever $\lambda _1,\lambda _2\geq \lambda _0$, it follows that $\norm{f_{\lambda _1}-f_{\lambda _2}}<\varepsilon$.  Let $U$ be an open neighborhood of $x_\infty$ such that $f_{\lambda _0}(U)\subseteq B_{\varepsilon}(f_{\lambda _0}(x_\infty ))$.  Let $x\in U$.  Let $\lambda _1,\lambda _2\geq \lambda _0$ be such that $\abs{f_{\lambda _1}(x)-f_\infty (x)},\abs{f_{\lambda _2}(x_\infty )-f_\infty (x_\infty )}<\varepsilon$.  Then,
\begin{equation}
\begin{split}
\MoveEqLeft
\abs{f_\infty (x)-f_\infty (x_\infty )} \\
& \leq \abs{f_\infty (x)-f_{\lambda _1}(x)}+\abs{f_{\lambda _1}(x)-f_{\lambda _0}(x)} \\ & \quad +\abs{f_{\lambda _0}(x)-f_{\lambda _0}(x_\infty )} \\ & \quad +\abs{f_{\lambda _0}(x_\infty )-f_{\lambda _2}(x_\infty )} \\ & \quad +\abs{f_{\lambda _2}(x_\infty )-f_\infty (x_\infty )} \\
& <5\varepsilon .
\end{split}
\end{equation}
Thus, $f_\infty$ is continuous.

We now check that $\lambda \mapsto f_\lambda$ converges to $f_\infty$ in $\Mor _{\Top}(X,\R )$.  Let $\varepsilon >0$.  Now that we know that $f_\infty$ is continuous, for each $x\in X$, there is some open neighborhood $U_x$ of $x$ such that $f_\infty (U_x)\subseteq B_{\varepsilon}(f_\infty (x))$.  Then, 
\begin{equation}
\left\{ U_x:x\in X\right\} 
\end{equation}
is an open cover of $X$.  Therefore, there is a finite subcover, $\{U_{x_1},\ldots ,U_{x_m}\}$.  Thus, we may choose $\lambda _0$ such that, whenever $\lambda \geq \lambda _0$, it follows that $\abs{f_\lambda (x_k)-f_\infty (x_k)}<\varepsilon$ for all $1\leq k\leq m$.  Now let $x\in X$ be arbitrary.  Without loss of generality, assume that $x\in U_1$.  Then, whenever $\lambda \geq \lambda _0$, it follows that
\begin{equation}
\begin{split}
\MoveEqLeft
\abs{f_\lambda (x)-f_\infty (x)} \\
& \leq \abs{f_\lambda (x)-f_\lambda (x_1)}+\abs{f_\lambda (x_1)+f_\infty (x_1)}<2\varepsilon .
\end{split}
\end{equation}
Taking the supremum over $x$, we find that
\begin{equation}
\norm{f_\lambda -f_\infty }<2\varepsilon ,
\end{equation}
so that indeed $\lambda \mapsto f_\lambda$ converges to $f_\infty$.

\Step{Prove the result in general}
We now do the general case, in which case $X$ is not necessarily quasicompact.  So, let $\lambda \mapsto f_\lambda \in \Mor _{\Top}(X,\R )$ be Cauchy.  Then, by \cref{exr4.5.3x} again, we must have that $\lambda \mapsto \restr{f_\lambda}{K}\Mor _{\Top}(K,\R )$ is Cauchy for each quasicompact subset $K\subseteq X$.  Therefore, by the quasicompact case, $\lambda \mapsto f_\lambda$ converges to its pointwise limit $f_\infty$ on each quasicompact subset.  As $f_\infty$ is continuous on each quasicompact subset, the intersection of the preimage of every open set with every quasicompact subset of $X$ is open, and hence, by hypothesis, is open.  Therefore, $f_\infty$ is continuous.  Furthermore, $\lambda \mapsto f_\lambda$ converges to $f_\infty$ in $\Mor _{\Top}(X,\R )$ because it converges to $f_\infty$ on each quasicompact subset.
\end{proof}
\end{thm}
Now that we've just proven what goes right, we turn to the more interesting side of things---what can go wrong.
\begin{exm}{A discontinuous function that is continuous on every quasicompact subset}{exm4.5.12}
Take $X\coloneqq \R$ and equip it with our good old-buddy, the cocountable topology.  Recall that (\cref{exm3.6.45}) a subset of $X$ is quasicompact iff it is finite.

Now let $f\colon X\rightarrow \R$ be any discontinuous function, for example, the \namerefpcref{DirichletFunction} is discontinuous with respect to the cocountable topology (because $\Q ^{\comp}$ is not closed).  On the other hand, quasicompact subsets of $X$, that is, finite subsets of $X$ are discrete, and so $f$ restricted to finite subsets must be continuous.
\end{exm}
We can use this trick to find an example of a space for which $\Mor _{\Top}(X,\R )$ is not complete.
\begin{exm}{A topological space for which $\Mor _{\Top}(X,\R )$ is not complete}{}
Take $X\coloneqq \R$ equipped with the cocountable extension topology.  Recall that this means that the only closed sets are (i)~$X$ itself, (ii)~countable subsets, and (iii)~subsets which are closed in the usual topology of $\R$.

The same proof in the previous example shows that the only quasicompact subsets of $X$ are the finite sets.\footnote{In fact, the more open sets you have, the fewer quasicompact sets you have.  Thus, as the cocountable extension topology is finer than the cocountable topology, there are fewer quasicompact sets in the cocountable extension topology in the sense that, if $K$ is quasicompact for the cocountable extension topology, $K$ is quasicompact for the cocountable topology, hence finite.}  Let $f\colon X\rightarrow \R$ be the Dirichlet function.  As $f^{-1}(0)=\Q ^{\comp}$ is not closed, $f$ is not continuous.  We construct a net of functions in $\Mor _{\Top}(X,\R )$ converging to $f$ uniformly on each quasicompact (i.e.~each finite) subset of $X$.

Our directed set $\Lambda$ is the collection of all finite subsets of $X$ ordered by inclusion.  For $S\in \Lambda$, let us write $S=\{ x_1,\ldots ,x_m\}$ with $x_k<x_{k+1}$ and define
\begin{equation*}
f_S(x)\coloneqq \begin{cases}f(x) & \text{if }x\in S \\ f(x_1) & \text{if }x<x_1 \\ \frac{f(x_{k+1}-f(x_k)}{x_{k+1}-x_k}(x-x_k) & \text{for }x_k<x<x_{k+1} \\ f(x_m) & \text{if }x_m<x,\end{cases}
\end{equation*}
That is, it is a constant $f(x_1)$ for all $x\leq x_1$, and similarly for $x\geq x_m$.  For $x$ between $x_k$ and $x_{k+1}$, is it a just the line segment going from $f(x_k)$ at $x=x_k$ to $f(x_{k+1})$ at $x=x_{k+1}$.  By construction, this is continuous with respect to the usual topology, and hence continuous with respect to the cocountable extension topology.

Now, for $K\subseteq X$ quasicompact, $K$ is finite, and hence $K\in \Lambda$.  Thus, whenever $S\geq K$, that is, whenever $S\supseteq K$, we have that $f_S(x)=f(x)$ for all $x\in S$, so that $\norm{f_S-f}=0$.  In particular, (i)~$S\mapsto f_S$ is Cauchy; and (ii)~if it converges in $\Mor _{\Top}(X,\R )$, it must converge to $f\in \Mor _{\Top}(X,\R )$.  However, as $f$ is not continuous, i.e.~not an element of $\Mor _{\Top}(X,\R )$, evidently $S\mapsto f_S$ does not converge, and so $\Mor _{\Top}(X,\R )$ is not complete.
\end{exm}
The key result of this subsection was that $\Mor _{\Top}(X,\R )$ is complete for $X$ quasicompactly-generated.  It turns out that the converse of this is \emph{false}.
\begin{exm}{A space for which $\Mor _{\Top}(X,\R )$ is complete yet is not quasicompactly-generated}{}
Take $X\coloneqq \R$ equipped with the cocountable topology.  We showed in \cref{exm4.5.12} that $X$ is not quasicompact-generated.  It remains to show that $\Mor _{\Top}(X,\R )$ is complete.  To do this, we show that every element of $\Mor _{\Top}(X,\R )$ is constant.

Let $f\colon X\rightarrow \R$ be continuous.  If $f$ is not constant, then $f^{-1}(a)$ is closed and proper for every $a\in \R$, and hence countable.  The image cannot be countable then, because if it were, $\R$ would be a countable union of countable sets,\footnote{Because $\R \eqqcolon X=\bigcup _{a\in f(X)}f^{-1}(a)$.} and hence countable.  As $\R =\bigcup _{n\in \Z}[n,n+1]$ and the image is uncountable, some interval $[n,n+1]$ must intersect the image at uncountable many points.  But then, either $[m,m+1]$ contains the image, or its preimage is a countable set, in which case we would have
\begin{equation*}
\text{countable set}=f^{-1}([n,n+1])=\bigcup _{a\in f(X)\cap [n,n+1]}f^{-1}(a)
\end{equation*}
is an uncountable union of nonempty disjoint sets, a contradiction.  Therefore, it must be that $f(X)\subseteq [m,m+1]$, so that $f^{-1}([n,n+1])=X$.  Now, writing $[n,n+1]=[n,n+\frac{1}{2}]\cup [n+\frac{1}{2},n+1]$ and applying the same logic again, we find, without loss of generality, that the image of $f$ is contained in $[n,n+\frac{1}{2}]$.  Applying this logic inductively, we deduce that the image of $f$ is contained in an interval of length $\frac{1}{2^m}$ for all $m\in \N$.  In particular, the image of $f$ cannot contain more than one element, because two distinct points cannot fit inside an interval sufficiently small, a contradiction of the fact that $f$ is not constant.  Therefore, every $f\colon X\rightarrow \R$ continuous must be a constant function.

A Cauchy net of constant functions amounts to a Cauchy net of real numbers, which converges, and so the original Cauchy net of constant functions converges to this constant function.  Therefore, $\Mor _{\Top}(X,\R )$ is complete.
\end{exm}
In conclusion:
\begin{important}
If a space is quasicompactly-generated, then $\Mor _{\Top}(X,\R )$ is complete.  Furthermore, if $\Mor _{\Top}(X,\R )$ is complete, then $X$ is a topological space.\footnote{Uhm, duh.  The content here is not in the implication, but rather in the counter-example.}  Both of these implications are strict:  the reals with the cocountable topology show that the first implication is strict, and the reals with the cocountable extension topology show that the second implication is strict.
\end{important}

There are a couple other important results about the space $\Mor _{\Top}(X,\R )$ that we present before moving to the next subsection.
\begin{prp}{Dini's Theorem}{DinisTheorem}\index{Dini's Theorem}
Let $X$ be a topological space, let $\lambda \mapsto f_{\lambda}\in \Mor _{\Top}(X,\R )$ be a net converging pointwise to $f_{\infty}\in \Mor _{\Top}(X,\R )$.  Then, if $\lambda \mapsto f_{\lambda}(x)$ is uniformly eventually monotone,\footnote{Here, ``uniformly eventually monotone'' means that there is some $\lambda _0$ such that, whenever $\lambda \geq \lambda _0$, it follows that $\lambda \mapsto f_{\lambda}(x)$ is monotone for all $x\in X$.  This is in contrast to requiring merely that $\lambda \mapsto f_{\lambda}(x)$ is monotone for every $x$---see \cref{exm4.4.23x}.} then $\lambda \mapsto f_{\lambda}$ converges to $f_{\infty}$.
\begin{rmk}
Of course, when we say ``converges'', we mean in $\Mor _{\Top}(X,\R )$, that is to say, ``uniformly (on quasicompact subsets)''.
\end{rmk}
\begin{rmk}
In particular, the conclusion is true if $\lambda \mapsto f_{\lambda}$ is eventually monotone in $\Mor _{\Top}(X,\R )$, where the order on $\Mor _{\Top}(X,\R )$ is defined by $f\leq g$ iff $f(x)\leq g(x)$ for all $x\in X$.
\end{rmk}
\begin{rmk}
Note that it \emph{is} allowed for $\lambda \mapsto f_{\lambda}(x)$ to be nondecreasing for some $x$ and nonincreasing for others.
\end{rmk}
\begin{rmk}
Note that you do need to assume a priori that $f_{\infty}$ is continuous.  For example, the sequence $m\mapsto x^m$ convergence pointwise and is monotonic at every point, but its pointwise limit is not continuous (and so the convergence cannot be uniform by \cref{thm4.5.6}).
\end{rmk}
\begin{rmk}
Also note that you do not need $X$ to be quasicompactly-generated for this result.  $X$ being quasicompactly-generated guarantees that $\Mor _{\Top}(X,\R )$ is complete, but as we have to assume that $f_{\infty}$ is continuous a priori anyways, we don't actually need to make this assumption.
\end{rmk}
\begin{rmk}
This is sort of like the \namerefpcref{MonotoneConvergenceTheorem} for functions.
\end{rmk}
\begin{proof}
\Step{Prove the result for $X$ quasicompact}
Suppose that $\lambda \mapsto f_{\lambda}(x)$ is uniformly eventually monotone.  By definition, this means that there is some $\lambda _0$ such that, whenever $\lambda \geq \lambda _0$, $\lambda \mapsto f_{\lambda}(x)$ is monotone for all $x\in X$.  Define $g_{\lambda}\coloneqq \abs{f_{\lambda}-f_{\infty}}$.  It follows that, whenever $\lambda \geq \lambda _0$, $\lambda \mapsto g_{\lambda}$ is nonincreasing.  Without loss of generality, suppose that $\lambda \mapsto g_{\lambda}$ is \emph{actually} nonincreasing.  Not that $\lambda \mapsto g_{\lambda}$ converges pointwise to $0$.
 
Denote the index set by $\Lambda$.  Let $\varepsilon >0$.  Then, $X=\bigcup _{\lambda \in \Lambda}g_{\lambda}^{-1}([0,\varepsilon ))$ because $\lambda \mapsto g_{\lambda}$ converges pointwise to $0$.  As $X$ is quasicompact, there are finitely many $\lambda _1,\ldots ,\lambda _m\in \Lambda$ such that $X=g_{\lambda _1}^{-1}([0,\varepsilon ])\cup \cdots \cup g_{\lambda _m}^{-1}$.  Let $\lambda _0\in \Lambda$ be at least as large as each $\lambda _k$.  As $\lambda \mapsto g_{\lambda}$ is nonincreasing, it follows that $g_{\lambda _0}^{-1}([0,\varepsilon ))\supseteq g_{\lambda _k}^{-1}([0,\varepsilon ))$ for each $1\leq k\leq m$, and hence $X=g_{\lambda _0}^{-1}([0,\varepsilon ))$.

Now suppose that $\lambda \geq \lambda _0$.  Then, as $\lambda \mapsto g_{\lambda}$ is nonincreasing, we have that $g_{\lambda}\leq g_{\lambda _0}$, so that $g_{\lambda}(x)\leq g_{\lambda _0}(x)<\varepsilon$ for all $x\in X$, that is, $\abs{f_{\lambda}(x)-f_{\infty}(x)}\eqqcolon g_{\lambda}(x)<\varepsilon$ for all $x\in X$, and so $\norm{f_{\lambda}-f_{\infty}}\coloneqq \sup _{x\in X}\abs{f_{\lambda}(x)-f_{\infty}(x)}<\varepsilon$.  Thus, $\lambda \mapsto f_{\lambda}$ converges to $f_{\infty}$ uniformly.

\Step{Prove the result in general}
We now do the general case, in which case $X$ is not necessarily quasicompact.  So, suppose that $\lambda \mapsto f_{\lambda}$ is uniformly eventually monotone.  Then certainly $\lambda \mapsto \restr{f_{\lambda}}{K}$ is uniformly eventually monotone for all quasicompact $K\subseteq X$, and so by the previous step, $\lambda \mapsto \restr{f_{\lambda}}{K}$ converges to $\restr{f_{\infty}}{K}$ in $\Mor _{\Top}(K,\R )$, and hence $\lambda \mapsto f_{\lambda}$ converges to $f_{\infty}$ in $\Mor _{\Top}(X,\R )$.
\end{proof}
\end{prp}
\begin{exm}{A continuous pointwise limit of continuous functions for which the convergence is pointwise eventually monotone, but the convergence is not uniform}{exm4.4.23x}
For $\lambda \in (0,1)$, define $f_{\lambda}\colon [0,1]$ by
\begin{equation*}
f_{\lambda}(x)\coloneqq \begin{cases}\frac{\lambda}{\lambda ^{\tfrac{1}{\lambda}}}x^{\tfrac{1}{\lambda}} & \text{if }x\leq \lambda  \\ \lambda & \text{if }\lambda \leq x\leq \tfrac{1}{2}(\lambda +\sqrt{\lambda}) \\ -\frac{\lambda}{\tfrac{1}{2}(\lambda +\sqrt{\lambda})-1}(x-1).\end{cases}
\end{equation*}
This converges pointwise (as $\lambda \to 1$) to the constant $0$ on $[0,1]$, which of course is continuous, and furthermore, the convergence is eventually nonincreasing at each point; however, the convergence is \emph{not} uniform (as we check next) because it is not \emph{uniformly} eventually monotone.

If the convergence were uniform, then, in particular, for $\varepsilon \coloneqq \frac{1}{2}$, there would be some $\lambda _0$ such that, whenever $\lambda \geq \lambda _0$, it follows that $f_{\lambda}(x)<\frac{1}{2}$ for all $x\in [0,1]$.  In particular, plugging in $x=\lambda$, we would have $\lambda _0\leq \lambda <\frac{1}{2}$, so that $\lambda _0<\frac{1}{2}$.  Then, as $\frac{1}{2}\geq \lambda _0$, it would follows that $f_{\frac{1}{2}}(x)<\frac{1}{2}$ for all $x\in [0,1]$, but this is not the case (e.g.~for $x=\frac{1}{2}$).
\end{exm}
\begin{thm}{Stone-Weierstrass Theorem}{StoneWeierstrassTheorem}\index{Stone-Weierstrass Theorem}
Let $X$ be a topological space and let $\alg{A}\subseteq \Mor _{\Top}(X,\R )$ be a subalgebra.\footnote{For $\alg{A}$ to be a subalgebra of $\Mor _{\Top}(X,\R )$ is equivalent to saying that it is a subspace that is closed under multiplication with $1\in \alg{A}$.}  Then, if for every $x_1,x_2\in X$ with $x_1\neq x_2$ there is some $f\in \alg{A}$ such that $f(x_1)\neq f(x_2)$, then either $\alg{A}$ is dense in $\Mor _{\Top}(X,\R )$ or there is some $x_0\in X$ such that $\alg{A}=\left\{ f\in \Mor _{\Top}(X,\R ):f(x_0)=0\right\}$.
\begin{rmk}
If $\alg{A}$ satisfies the hypotheses of this theorem, then $\alg{A}$ is said to \term{separate points}\index{Separate points}.  Intuitively, the algebra $\alg{A}$ alone can `tell the difference' between distinct points of $X$.
\end{rmk}
\begin{rmk}
Recall that (\cref{Dense}) for $\alg{A}$ to be dense in $\Mor _{\Top}(X,\R )$ means that $\Cls (\alg{A})=\Mor _{\Top}(X,\R )$.  Explicitly, for any continuous function $f\colon X\rightarrow \R$, there is a net $\lambda \mapsto f_{\lambda}\in \alg{A}$ converging (uniformly) to $f$.  That is to say, you can uniformly approximate \emph{any} real-valued continuous function on $X$ with elements of $\alg{A}$.
\end{rmk}
\begin{rmk}
Note that we do \emph{not} have to assume that $X$ is quasicompactly-generated.  This is for essentially the same reason as we did not have to assume this for \nameref{DinisTheorem}.
\end{rmk}
\begin{rmk}
This generalizes to $\C$ nearly verbatim---the only difference in this case is you would have to additionally require that if $f\in \alg{A}$, then the complex conjugate (whatever the hell that is) of $f$ is also in $\alg{A}$.
\end{rmk}
\begin{proof}
\Step{Make hypotheses}
Suppose that for every $x_1,x_2\in X$ with $x_1\neq x_2$ there is some $f\in \alg{A}$ such that $f(x_1)\neq f(x_2)$.

\Step{Reduce to the quasicompact case}
Suppose that we have proven the result for $X$ quasicompact.  Let $f\in \Mor _{\Top}(X,\R )$.  Let $K\subseteq X$ be quasicompact.  By the quasicompact case, there is a net $\Lambda _K\ni \lambda \mapsto f_{K,\lambda}\in \alg{A}$ converging to $f$ in $\Mor _{\Top}(K,\R )$.  Let $\collection{K}$ denote the collection of quasicompact subsets of $X$ and regard it as a directed set by inclusion.  We claim that
\begin{equation}
\collection{K}\times \prod _{K\in \collection{K}}\Lambda _K\ni \coord{K,\lambda}\mapsto f_{K,\lambda _K}
\end{equation}
converges to $f\in \Mor _{\Top}(X,\R )$.  This will show that $f$ is a limit point of $\alg{A}$, so that $f\in \Cls (\alg{A})$, as desired.

By \cref{exr4.2.31} (characterization of convergence in semimetric spaces), to show that this converges to $f$, we need to show that this converges to $f$ with respect to $\norm _K$ for all $K\in \collection{K}$.  So, let $K_0\in \collection{K}$.  Let $\varepsilon >0$.  As $\lambda \mapsto f_{K_0,\lambda}$ converges to $f$ in $\Mor _{\Top}(X,\R )$, there is some $[\lambda _0]_{K_0}$ such that, whenever $\lambda \geq [\lambda _0]_{K_0}$, it follows that $\norm{f_{K_0,\lambda}-f}_{K_0}<\varepsilon$.

Now, suppose that $\coord{K,\lambda}\geq \coord{K_0,\lambda _0}$.  This means that $K\supseteq K_0$ and $\lambda _L\geq [\lambda _0]_L$ for every $L\in \collection{K}$.  Hence, in this case,
\begin{equation}
\norm{f_{K,\lambda _K}-f}_{K_0}\leq \norm{f_{K,\lambda _K}-f}_K<\varepsilon .
\end{equation}
Thus, this net does indeed converge to $f$ with respect to each $\norm _{K_0}$, and hence this net converges to $f$, as desired.

Thus, let us assume hereafter that $X$ is quasicompact.

\Step{Reduce to the case where $\alg{A}$ vanishes at no point}
If there is some $x_0\in X$ such that $\alg{A}=\left\{ f\in \Mor _{\Top}(X,\R ):f(x_0)=0\right\}$, we are done, so suppose that for every $x\in X$, there is some $f\in \alg{A}$ such that $f(x)\neq 0$.\footnote{This is what we mean when we say ``$\alg{A}$ vanishes at no point''.}

\Step{Show that for every $x_1,x_2\in X$ with $x_1\neq x_2$ and $c_1,c_2\in \R$ there is some $f\in \alg{A}$ such that $f(x_i)=c_i$}[stpStoneWeierstrassTheorem.2x]\footnote{Proof adapted from \cite[Theorem 7.31]{Rudin}.}
By hypothesis, there is some $g\in \alg{A}$ such that $g(x_1)\neq g(x_2)$.  From the previous step, there are $h,k\in \alg{A}$ such that $h(x_1)\neq 0$ and $k(x_2)\neq 0$.  Define
\begin{equation}
\begin{multlined}
f\coloneqq \frac{c_1}{g(x_1)h(x_1)-g(x_2)h(x_1)}\left( gk-g(x_1)k\right) \\ +\frac{c_2}{g(x_2)k(x_2)-g(x_1)k(x_2)}\left( gk-g(x_1)k\right) .
\end{multlined}
\end{equation}
As $h(x_1)\neq 0$, $k(x_2)\neq 0$, and $g(x_1)\neq g(x_2)$, the denominators are not $0$, and so this does indeed give an element of $\alg{A}$ (as it is an algebra).  Furthermore, plugging in, we see that $f(x_1)=c_1$ and $f(x_2)=c_2$.

\Step{Show that $\Cls (\alg{A})$ is also a subalgebra}[stpStoneWeierstrassTheorem.2]
We leave this step as an exercise.
\begin{exr}[breakable=false]{}{}
Show that $\Cls (\alg{A})$ is a subalgebra of $\Mor _{\Top}(X,\R )$.
\end{exr}

\Step{Show that if $f\in \alg{A}$, then $\abs{f}\in \Cls (\alg{A})$}\footnote{Proof adapted from \cite[Theorem 3.3.8]{Sally}.}
Suppose that $f\in \alg{A}$.  As $X$ is quasicompact, by the \namerefpcref{ExtremeValueTheorem}, $f(X)$ is quasicompact, hence bounded by the \namerefpcref{HeineBorelTheorem}, so let $M>0$ be such that $\abs{f(x)}\leq M$ for all $x\in X$.  Define $g\coloneqq \frac{1}{M}f\in \alg{A}$.  Note that $\abs{g}\leq 1$.  It suffices to show that $\abs{g}\in \Cls (\alg{A})$.  To do this, we construct a sequence $m\mapsto g_m\in \alg{A}$ converging to $\abs{g}$ in $\Mor _{\Top}(X,\R )$.

Define $g_0\coloneqq 0$ and for $m\in \Z ^+$ define
\begin{equation}\label{eqn4.4.25}
h_m\coloneqq h_{m-1}+\tfrac{1}{2}(g^2-h_{m-1}^2).
\end{equation}
As $\alg{A}$ is an algebra, it follows that $h_m\in \alg{A}$ for all $m\in \N$, and so it suffices to show that $\lim _mh_m=\abs{g}$.

Note that, if we can prove that this sequence is Cauchy, so that, by completeness of $\Mor _{\Top}(X,\R )$, it converges, its limit $h_{\infty}$ must satisfy
\begin{equation}
h_{\infty}=h_{\infty}+\tfrac{1}{2}(g^2-h_{\infty}^2),
\end{equation}
and hence that $g^2=h_{\infty}^2$.  Then, if $h_{\infty}\geq 0$, it will follow that $h_{\infty}=\abs{g}$.  Thus, it suffices to show (i)~that $m\mapsto h_m$ is Cauchy and (ii)~that $m\mapsto h_m$ is eventually nonnegative.

We prove that $0\leq h_m\leq \abs{g}$ and $h_m\leq h_{m+1}$ for all $m\in \N$.  We do this by induction.  For $m=0$, we have by definition $0\leq 0\leq \abs{g}$ and
\begin{equation}
h_1\coloneqq 0+\tfrac{1}{2}(g^2-0)=g^2\geq 0\eqqcolon h_0.
\end{equation}
Thus, the result is true for $m=0$.  Now suppose the result is true for all $0\leq k\leq m$.  We first note that $h_{m+1}\geq h_m\geq 0$.  Furthermore, note that
\begin{equation}
h_{m+2}-h_{m+1}\coloneqq \tfrac{1}{2}(g^2-h_{m+1}^2)\geq \tfrac{1}{2}(g^2-\abs{g}^2)=0.
\end{equation}
Finally, we have that
\begin{equation}
\begin{split}
h_{m+1} & \coloneqq h_m+\tfrac{1}{2}(g^2-h_m^2) \\
& =h_m+\tfrac{1}{2}(\abs{g}+h_m)(\abs{g}-h_m) \\
& \leq h_m+\tfrac{1}{2}(\abs{g}+\abs{g})(\abs{g}-h_m) \\
& =h_m+\abs{g}(\abs{g}-h_m)\\
& \leq \footnote{Because $\abs{g}\leq 1$.}h_m+\abs{g}-h_m \\
& =\abs{g}.
\end{split}
\end{equation}

Thus, $m\mapsto h_m$ is a nondecreasing sequence in $\Mor _{\Top}(X,\R )$ with $0\leq h_m\leq \abs{g}$.  By \namerefpcref{DinisTheorem}, the sequence $m\mapsto h_m$ converges.  If the limit is denoted $h_{\infty}$, then as each $h_m\geq 0$, $h_{\infty}\geq 0$ as well, and so as we have $g^2=h_{\infty}^2$, we indeed have that $h_{\infty}=\abs{g}$, as desired.

\Step{Show that if $f\in \Cls (\alg{A})$, then $\abs{f}\in \Cls (\alg{A})$}
Suppose that $f\in \Cls (\alg{A})$.  Then, there is a net $\Lambda \ni \lambda \mapsto f_{\lambda}\in \alg{A}$ converging to $f$.  By the previous step, $\abs{f_{\lambda}}\in \Cls (\alg{A})$.  Furthermore, as
\begin{equation}
\abs{\abs{f}(x)-\abs{f_{\lambda}(x)}}\leq \abs{f(x)-f_{\lambda}(x)},
\end{equation}
it follows that $\lambda \mapsto \abs{f_{\lambda}}\in \Cls (\alg{A})$ converges to $\abs{f}$, and so $\abs{f}\in \Cls (\alg{A})$.

\Step{Show that if $f,g\in \Cls (\alg{A})$, then $\max (f,g),\min (f,g)\in \Cls (\alg{A})$}[stpStoneWeierstrassTheorem.5]
Suppose that $f,g\in \Cls (\alg{A})$.  Note that
\begin{equation}
\max (f,g)=\tfrac{1}{2}(f+g)+\tfrac{1}{2}\abs{f-g}.
\end{equation}
By the previous step and the fact that $\Cls (\alg{A})$ is itself an algebra (\cref{stpStoneWeierstrassTheorem.2}), it follows that $\max (f,g)\in \Cls (\alg{A})$.  Similarly, $\min (f,g)\in \Cls (\alg{A})$.

\Step{Show that for $f\in \Mor _{\Top}(X,\R )$, $x_0\in X$, and $\varepsilon >0$, there is some $f_{x_0}\in \Cls (\alg{A})$ such that $f_{x_0}(x_0)=f(x_0)$ and $f_{x_0}>f-\varepsilon$}\footnote{Remainder of proof adapted from \cite[Theorem 7.32]{Rudin}.}
By \cref{stpStoneWeierstrassTheorem.2x}, for every $x\in X$ there is some $g_x\in \alg{A}$ such that $g_x(x_0)=f(x_0)$ and $g_x(x)=f(x)$.  As $g_x$ is continuous, there is an open neighborhood $U_x$ of $x$ such that $g_x(u)>f(u)-\varepsilon$ for all $u\in U_x$.  As $X$ is quasicompact, there are finitely many $U_{x_1},\ldots ,U_{x_m}$ that cover $X$.  Define $f_{x_0}\coloneqq \max (g_{x_1},\ldots ,g_{x_m})$.  By the previous step, we have that $f_{x_0}\in \Cls (\alg{A})$.  Furthermore, as $g_x(x_0)=f(x_0)$ for all $x\in X$, we certainly have that $f_{x_0}(x_0)=f(x_0)$.  Finally, from the inequalities $g_{x_k}(u)>f(U)-\varepsilon$ for all $u\in U_{x_k}$ and the fact that $X=U_{x_1}\cup \cdots \cup U_{x_m}$, it follows that $f_{x_0}>f-\varepsilon$.

\Step{Show that $\Cls (\alg{A})=\Mor _{\Top}(X,\R )$}
Let $f\in \Mor _{\Top}(X,\R )$.  Let $\varepsilon >0$.  We wish to find $g\in \Cls (\alg{A})$ such that $\norm{f-g}<\varepsilon$.  This will show that $f$ is an accumulation point of $\Cls (\alg{A})$, and hence $f\in \Cls (\alg{A})$ as $\Cls (\alg{A})$ is closed.

By the previous step, for every $x\in X$, there is some $g_x\in \Cls (\alg{A})$ such that $g_x(x)=f(x)$ and $g_x>f-\varepsilon$.  As $g_x$ is continuous, there is an open neighborhood $U_x$ of $x$ such that $g_x(u)<f(u)+\varepsilon$ for all $u\in U_x$.  As $X$ is quasicompact, there are finitely many $U_{x_1},\ldots ,U_{x_m}$ that cover $X$.  Define $g\coloneqq \min (g_{x_1},\ldots ,g_{x_m})$.  By \cref{stpStoneWeierstrassTheorem.5}, $g\in \Cls (\alg{A})$.

As each $g_x>f-\varepsilon$, it follows that $g>f-\varepsilon$.  Furthermore, as $g_{x_k}(u)<f(u)+\varepsilon$ for all $u\in U_{x_k}$ and the fact that $X=U_{x_1}\cup \cdots \cup U_{x_m}$, it follows that $g<f+\varepsilon$.  These inequalities together give $\norm{f-g}<\varepsilon$, as desired.
\end{proof}
\end{thm}
We obtain as a near immediate corollary of this a result that is quite significant in its own right.
\begin{crl}{Weierstrass Approximation Theorem}{WeierstrassApproximationTheorem}\index{Weierstrass Approximation Theorem}
Let $D\subseteq \R ^d$.  Then, $\R [x_1,\ldots ,x_d]$ is dense in $\Mor _{\Top}(D,\R )$.
\begin{rmk}
Recall that (\cref{PolynomialCring}) $\R [x]$ is the cring of all polynomials with real coefficients.  It of course has additionally the canonical structure of an algebra (that is, you can scale polynomials).  Similarly, $\R [x_1,\ldots ,x_d]$ is the algebra of polynomials in the variables $x_1,\ldots ,x_d$.  For example, for $d=3$, a typical element of $\R [x_1,x_2,x_3]$ might look like $\sqrt{2}x_1^2x_3-\frac{2}{3}x_2^4x_3^5+x_1x_2^7x_3^2$.
\end{rmk}
\begin{rmk}
Explicitly, given any continuous function $f\colon D\rightarrow \R$, then there is a net of polynomials $\lambda \mapsto p_{\lambda}$ converging (uniformly (on quasicompact subsets)) to $f$.  That is, you can ``approximate'' continuous functions uniformly by polynomials.
\end{rmk}
\begin{rmk}
We're actually being a bit sloppy here:  strictly speaking, we should be making a distinction between polynomials and polynomial \emph{functions}.  Elements of $\R [x]$ are technically just formal symbols that can be added and multiplied in the way you expect, but they themselves are \emph{not} functions.  Rather, a polynomial $p\in \R [x]$ \emph{defines} a function, namely the function $\R \ni x\mapsto p(x)\in \R$.  To see the difference, instead consider $\Z /2\Z [x]$.\footnote{See \cref{exmA.1.117} for the definition of $\Z /m\Z$.}  $x(x+1)\in \Z /2\Z [x]$ is a polynomial with coefficients in $\Z /2\Z$ and it is \emph{not} the zero polynomial.  On the other hand, if you plug in both elements of $\Z /2\Z$ (namely $0$ and $1$), you get $0$ both times.  Thus, while the polynomial $x(x+1)$ is \emph{not} zero, it \emph{defines} the function that is identically $0$ $\Z /2\Z \ni x\mapsto x(x+1)\in \Z /2\Z$.

In short, strictly speaking $\R [x]$ is \emph{not} a subset of $\Mor _{\Top}(X,\R )$, but rather \emph{embeds} in $\Mor _{\Top}(X,\R )$.
\end{rmk}
\begin{proof}
By the \nameref{StoneWeierstrassTheorem}, it suffices to show that $\R [x_1,\ldots ,x_d]$ is an algebra that separates points.\footnote{Recall that (\cref{StoneWeierstrassTheorem}) this means that for every $x_1,x_2\in X$ distinct, there is some $p\in \R [x]$ such that $p(x_1)\neq p(x_2)$.}  That it is an algebra is immediate (sums of polynomials are polynomials, products of polynomials are polynomials, scalings of polynomials are polynomials, and both $0$ and $1$ are polynomials).
\begin{exr}[breakable=false]{}{}
Show that $\R [x_1,\ldots ,x_d]$ separates points in $\Mor _{\Top}(D,\R )$.
\end{exr}
\end{proof}
\end{crl}

Before moving on to the completion, we end with an exercise that will be needed later in \cref{prp3.6.73}.
\begin{exr}{}{exr4.4.22}
Let $X$ be a topological space, let $\mcal{B}(X,\R )$ denote the collection of all bounded continuous real-valued functions on $X$, and for $f\in \mcal{B}(X,\R )$ define $\norm{f}\coloneqq \sup _{x\in X}\abs{f(x)}$.  Show that $\coord{\mcal{B}(X,\R ),\norm}$ is a complete normed vector space.
\begin{rmk}
Note that, unlike in \cref{thm4.5.6}, we do \emph{not} need $X$ to be quasicompactly-generated.
\end{rmk}
\end{exr}

\subsection{The completion}

And now we show that every uniform space can be \emph{completed}.\footnote{Of course, we already know that $\Q$ is not complete (see \cref{prp3.3.59,prp3.3.68}), and so in general there will most certainly be some `completing' to be done.}
\begin{thm}{Completion}{Completion}
Let $X$ be a $T_0$ uniform space.  Then, there exists a unique complete uniform space $\Cmp (X)$, the \term{completion}\index{Cauchy completion}\index{Completion (of a uniform space)} of $X$, such that
\begin{enumerate}
\item $\Cmp (X)$ contains $X$; and
\item if $Y$ is any other complete uniform space which contains $X$, then $Y$ contains $\Cmp (X)$.
\end{enumerate}
Furthermore, $X$ is dense in $\Cmp (X)$.
\begin{rmk}
You will see in the proof that this is the ``Cauchy sequence construction'' we mentioned in a footnote right before our proof of existence of the real numbers (\cref{RealNumbers}).  It turns out that, in the case of $\R$, this will gives us the right answer, but that the passage from $\Q$ to $\R$ should really be thought of as the \emph{Dedekind-completion}, not the \emph{Cauchy-completion}.
\end{rmk}
\begin{rmk}
If $X$ is not $T_0$, then, if you like, you can define the completion of $X$ as $\Cmp (\TZero (X))$---see \cref{T0Quotient}.  Note that in this case $X$ is \emph{not} (in general) contained in its completion.
\end{rmk}
\begin{proof}
\Step{Define an equivalence relation $\sim$ on Cauchy nets}[stpB.5.6.1]
Define first as a set
\begin{equation}
X'\coloneqq \left\{ \lambda \mapsto x_\lambda \in X:\lambda \mapsto x_\lambda \text{ is Cauchy.}\right\} 
\end{equation}
For $\lambda \mapsto x_\lambda$ and $\mu \mapsto y_\mu$ Cauchy nets, define $(\lambda \mapsto x_\lambda )\sim (\mu \mapsto y_\mu )$ iff open subsets of $X$ eventually contain $\lambda \mapsto x_\lambda$ iff they eventually contain $\mu \mapsto y_\mu$.

\Step{Show that $\sim$ is an equivalence relation}
That $(\lambda \mapsto x_\lambda )\sim (\lambda \mapsto x_\lambda )$ is tautological.  The definition of $\sim$ is $\lambda \mapsto x_\lambda \leftrightarrow \mu \mapsto y_\mu$ symmetric, and so of course $(\lambda \mapsto x_\lambda )\sim (\mu \mapsto y_\mu )$ implies $(\mu \mapsto y_\mu )\sim (\lambda \mapsto x_\lambda )$.

As for transitivity, suppose that $(\lambda \mapsto x_\lambda )\sim (\mu \mapsto y_\mu )$ and $(\mu \mapsto y_\mu )\sim (\nu \mapsto z_\nu)$.  We wish to show that $(\lambda \mapsto x_\lambda )\sim (\nu \mapsto z_\nu )$.  Let $U\subseteq X$ be open.  We must show that $U$ eventually contains $\lambda \mapsto x_\lambda$ iff it eventually contains $\nu \mapsto z_\nu$.  By $\lambda \mapsto x_\lambda \leftrightarrow \nu \mapsto z_\nu$, it suffices to show only one of these directions.  So, suppose that $U$ eventually contains $\lambda \mapsto x_\lambda$.  Then, $U$ eventually contains $\mu \mapsto y_\mu$ (because $(\lambda \mapsto x_\lambda )\sim (\mu \mapsto y_\mu )$), and so $U$ eventually contains $\nu \mapsto z_\nu$ (because $(\mu \mapsto y_\mu )\sim (\nu \mapsto z_\nu )$).

\Step{Define $\Cmp (X)$ as a set}
We define
\begin{equation}
\Cmp (X)\coloneqq X'/\sim .
\end{equation}

\Step{Define a uniformity on $X'$.}
Denote the uniformity on $X$ by $\uniformity{U}$.  For every uniform cover $\cover{U}\in \uniformity{U}$, we define a corresponding cover $\cover{U}'$ of $X'$ as follows.  For $\cover{U}\in \uniformity{U}$ and $U\in \cover{U}$, define
\begin{equation}\label{B.26}
\begin{split}
U' & \coloneqq \left\{ (\lambda \mapsto x_{\lambda}\in X':\right. \\ & \qquad \left. \lambda \mapsto x_{\lambda}\text{ is eventually contained in }U\text{.}\right\} \\
\cover{U}' & \coloneqq \{ U':U\in \cover{U}\} \\
\uniformity{U}' & \coloneqq \{ \cover{U}':\cover{U}\in \uniformity{U}\} .
\end{split}
\end{equation}
We wish to show that $\uniformity{U}'$ is a uniform \emph{base} on $X'$.  To show this (\cref{prp4.3.2}), we must show (i)~that each element of $\uniformity{U}'$ is in fact a cover of $X'$ and (ii)~that $\uniformity{U}'$ is downward-directed with respect to star-refinement.

We first check that each $\cover{U}'$ is in fact a cover of $X'$.  So, let $\cover{U}'\in \uniformity{U}'$ be arbitrary and let $(\lambda \mapsto x_{\lambda})\in X'$.  As $\lambda \mapsto x_{\lambda}$ is Cauchy, there is some $U\in \cover{U}$ such that $\lambda \mapsto x_{\lambda}$ is eventually contained in $U$, so that $(\lambda \mapsto x_{\lambda})\in U'$, and hence $\cover{U}'$ covers $X'$.

We now check that $\uniformity{U}'$ is downward-directed with respect to star-refinement.  So, let $\cover{U}',\cover{V}'\in \uniformity{U}'$.  Let $\cover{W}$ be a common star-refinement of $\cover{U}$ and $\cover{V}$.  We wish to show that $\cover{W}'$ is a common star-refinement of $\cover{U}'$ and $V'$.  By $\cover{U}\leftrightarrow \cover{V}$ symmetry, it suffices to prove that $\cover{W}'$ is a star-refinement of $\cover{U}'$.  So, let $W_0'\in \cover{W}'$.  As $\cover{W}$ is a star-refinement of $\cover{U}$, it follows that there is some $U_0\in \cover{U}$ such that $\Star _{\cover{W}}(W_0)\subseteq U_0$.  We wish to show that $\Star _{\cover{W}'}(W_0')\subseteq U_0'$.  So, let $W'\in \cover{W}'$ intersect $W_0'$.  Then, there is a net that is eventually contained in both $W$ and $W_0$, so that, in particular, $W$ and $W_0$ intersect.  It follows that $W\subseteq \Star _{\cover{W}}(W_0)\subseteq U_0$, and hence in turn, that any net eventually contained in $W$ is eventually contained in $U_0$, so that $W'\subseteq U_0'$, and hence $\Star _{\cover{W}'}(W_0')\subseteq U_0'$ as desired.

This completes the proof that $\uniformity{U}'$ is a uniform base on $X'$.

\Step{Show that the quotient map $\q :X'\rightarrow \Cmp (X)$ satisfies $\q ^{-1}(\q (U'))=U'$}[Completion.5]
As it is always the case that $U'\subseteq \q ^{-1}(\q (U'))$, it suffices to show that $\q ^{-1}(\q (U'))\subseteq U'$.  So, let $(\lambda \mapsto x_{\lambda})\in \q ^{-1}(\q (U'))$.  Then, $\q (\lambda \mapsto x_{\lambda})\in \Q (U')$, that is, $\lambda \mapsto x_{\lambda}$ is a Cauchy net and there is another Cauchy net $(\mu \mapsto y_{\mu})\in U'$ such that $(\lambda \mapsto x_{\lambda})\sim (\mu \mapsto y_{\mu})$.  That is, open sets eventually contain $\lambda \mapsto x_{\lambda}$ iff they eventually contain $\mu \mapsto y_{\mu}$.  However, as $(\mu \mapsto y_{\mu})\in U'$, by definition of $U'$ (\eqref{B.26}), $\mu \mapsto y_{\mu}$ is eventually contained in $U$, and so $\lambda \mapsto x_{\lambda}$ is eventually contained in $U$, and so $(\lambda \mapsto x_{\lambda})\in U'$.  Hence, $\q ^{-1}(\q (U'))\subseteq U'$, and we are done.

\Step{Define a uniform base on $\Cmp (X)$}
For every uniform cover $\cover{U}\in \uniformity{U}$, we define a corresponding cover $\Cmp (\cover{U})$ of $\Cmp (X)$.  For $\cover{U}\in \uniformity{U}$ and $U\in \cover{U}$, define\footnote{As you might have guessed, this is just the quotient uniformity induced from the one on $X'$ written out explicitly.}
\begin{equation}
\begin{split}
\Cmp (U) & \coloneqq \q (U') \\
\Cmp (\cover{U}) & \coloneqq \q (\cover{U}')\coloneqq \{ \Cmp (U):U\in \cover{U}\} \\
\widetilde{\Cmp (\cover{U})} & \coloneqq \q (\uniformity{U}') \coloneqq \{ \Cmp (\cover{U}):\cover{U}\in \uniformity{U}\} .
\end{split}
\end{equation}
We claim that $\widetilde{\Cmp (\cover{U})}$ is a uniform base on $\Cmp (X)$.  Certainly each $\Cmp (\cover{U})$ is a cover of $\Cmp (X)$ because $\cover{U}'$ is a cover of $X'$ and $\q$ is surjective.

It follows from \cref{prpC.2.3x} that $\q$ preserves star-refinement (the purpose of the previous step was to verify the requisite hypotheses of \cref{prpC.2.3x}), and so that $\Cmp (\uniformity{U})$ is a uniform base follows from the fact that that $\uniformity{U}'$ was a uniform base on $X'$.

\Step{Show that $\Cmp (X)$ is complete.}
Let $\lambda \mapsto \q (x^\lambda )$ be Cauchy in $\Cmp (X)$.  By definition of Cauchyness and our uniformity on $\Cmp (X)$, this means that, for every uniform cover $\Cmp (\cover{U})\in \Cmp (\uniformity{U})$, there is some $\Cmp (U)\in \Cmp (\cover{U})$ such that $\lambda \mapsto \q (x^\lambda )$ is eventually contained in $\Cmp (U)\coloneqq \q (U')$.  Thus, for each $x^\lambda \in X'$ for $\lambda$ sufficiently large, there is some $y^\lambda \in U'$ with $x^\lambda \sim y^\lambda$.  However, by definition of $U'$, $U$ eventually contains $y^\lambda$, and hence, because $x^\lambda \sim y^\lambda$, eventually contains $x^\lambda$, and so in fact $x^\lambda \in U'$.  Thus, we have a net $\lambda \mapsto x^\lambda \in X'$ that has the property that, for every uniform cover $\cover{U}\in \uniformity{U}$, there is some $U\in \cover{U}$ such that $\lambda \mapsto x^\lambda$ is eventually contained in $U'$.

Let us denote the domain of $\lambda \mapsto x^\lambda$ by $\Lambda$.  Now, each $x^\lambda$ is itself a net, and so let us denote its domain by $M^\lambda$.  Define $x^\infty \in X'$ to be the net
\begin{equation}
\Lambda \times \prod _{\lambda \in \Lambda}M^\lambda \ni (\coord{\lambda ,\mu }\mapsto [x^\lambda ]_{\mu ^\lambda}\in X.
\end{equation}
We first must check that this is Cauchy, so that indeed $x^\infty \in X'$.

So, let $\cover{U}\in \uniformity{U}$ be a uniform cover.  Then, there is some $U\in \cover{U}$ such that $\lambda \mapsto x^\lambda$ is eventually contained in $U'$.  So, let $\lambda _0$ be such that, whenever $\lambda \geq \lambda _0$, it follows that $x^\lambda \in U'$.  For all such $\lambda$, the net $x^\lambda$ itself must be eventually contained in $U$, so let $\mu _0^\lambda$ be such that, whenever $\mu ^\lambda \geq \mu _0^\lambda$, it follows that $[x^\lambda ]_{\mu ^\lambda}\in U$.  (For $\lambda$ not at least $\lambda _0$, $\mu _0^\lambda$ may be anything).  Now, suppose that $\coord{\lambda ,\mu}\geq \coord{\lambda _0,\mu _0}$.  Then,
\begin{equation}\label{eqn4.4.29}
(x^\lambda )_{\mu ^\lambda}\in U
\end{equation}
Thus, $\coord{\lambda ,\mu }\mapsto (x^\lambda )_{\mu ^\lambda}$ is Cauchy.

We now show that $\lambda \mapsto \q (x^\lambda )$ converges to $\q (x^\infty )$.  To show this, as stars form neighborhood bases, it suffices to show that $\lambda \mapsto \q (x^\lambda )$ is eventually contained in $\Star _{\Cmp (\cover{U})}(\q (x^\infty ))$ for all $\cover{U}\in \uniformity{U}$.  As $\q$ is surjective, the preimage of a star is equal to the star of the preimage (\cref{prpC.2.3}), and so it suffices to show that $\lambda \mapsto x^\lambda$ is eventually contained in $\Star _{\cover{U}'}(x^\infty )$ for all $\cover{U}\in \uniformity{U}$.  So, let $\cover{U}\in \uniformity{U}$ be a uniform cover.  Then, there is some $U\in \cover{U}$ such that $\lambda \mapsto x^\lambda$ is eventually contained in $U'$.  Thus, we will be done if we can show that $x^\infty \in U'$ (so that then $U'\subseteq \Star _{\cover{U}'}(x^\infty )$).  To show this, we must show that $x^\infty$ is eventually contained in $U$.  However, the exact same argument (see \eqref{eqn4.4.29}) that was used above to show that $x^\infty$ was Cauchy shows precisely this.  Therefore, $\lambda \mapsto x^\lambda$ converges to $x^\infty$.

\Step{Show that $\Cmp (X)$ contains $X$}
Of course, when we say that $\Cmp (X)$ ``contains'' $X$, what we really means is that there is a subset of $\Cmp (X)$ which is uniformly-homeomorphic to $X$.  So, we define $\iota \colon X\rightarrow \Cmp (X)$, and show that it is a uniform-homeomorphism onto its image.\footnote{Its image will be equipped with the subspace uniformity, that is, the initial uniformity (see \cref{InitialUniformity}) with respect to the inclusion into $\Cmp (X)$.}

Define $\c \colon X\rightarrow X'$ by $\c (x)\coloneqq (\lambda \mapsto x_{\lambda}\coloneqq x)$, that is, $\c$ sends $x$ to the constant net with value $x$.  (Constant nets converge, and hence are in particular Cauchy.)  Then we define $\iota \coloneqq \q \circ \c$.  We first show that this is injective.  Suppose that $\iota (x_1)=\iota (x_2)$.  Then, every neighborhood that eventually contains the constant net $x_1$ eventually contains the constant net $x_2$.  In other words, open sets in $X$ contain $x_1$ iff they contain $x_2$, which implies that $x_1=x_2$ because $X$ is $T_0$.

We now check that $\iota$ is uniformly-continuous.  We claim that $\iota ^{-1}(\Cmp (\cover{U}))=\cover{U}$.  However, using result that $\q ^{-1}(\q (U'))=U'$ from \cref{Completion.5}, we have that
\begin{equation}
\begin{split}
\iota ^{-1}(\Cmp (\cover{U})) & \coloneqq \c ^{-1}\left( \q ^{-1}\left( \q (\cover{U}')\right) \right) =\c ^{-1}(\cover{U}') \\
& \coloneqq \{ \c ^{-1}(U'):U\in \cover{U}\} .
\end{split}
\end{equation}
However, the only constant nets which are eventually contained in $U$ are the elements of $U$ themselves, and so $\c ^{-1}(U')=U$, and so indeed
\begin{equation}
\iota ^{-1}(\Cmp (\cover{U}))=\cover{U}.
\end{equation}
The subspace uniformity induced on $\iota (X)$ is generated by
\begin{equation}\label{4.5.13}
\left\{ \Cmp (\cover{U})\wedge \{ \iota (X)\} :\cover{U}\in \uniformity{U}\right\} ,
\end{equation}
that is, a cover of $\iota (X)$ is uniform iff it is a cover that is obtained from a uniform cover of $\Cmp (X)$ by simply restricting that cover to $\iota (X)$.  Because the preimage of a cover with respect to the inverse of a function is the same as the image of that uniform cover, to show that the inverse of $\iota :X\rightarrow \iota (X)$ is uniformly-continuous, it suffices to show that $\iota (\cover{U})$ is a uniform cover on $\iota (X)$.  By \eqref{4.5.13}, it thus suffices to show that
\begin{equation}\label{4.5.14}
\iota (\cover{U})=\Cmp (\cover{U})\wedge \{ \iota (X)\} .
\end{equation}
To show this, we first check that $\q (U')\cap \q (\c (X))=\q (U'\cap \c (X))$.  We always have that $\supseteq$ inclusion (\cref{exrA.1.30}), and so it suffices to prove the $\subseteq$ inclusion.  So, let $\lambda \mapsto x_{\lambda}\in X$ be a Cauchy net that is equivalent to a Cauchy net $\mu \mapsto y_{\mu}$ that is eventually contained in $U$ and the constant Cauchy net $\nu \mapsto z_{\infty}\in X$.  This implies in particular that $\nu \mapsto z_{\infty}$ is eventually contained in $U$, so that $(\nu \mapsto z_{\infty})\in U'\cap \c (X)$, so that $\q (\lambda \mapsto x_{\lambda})\in \q (U'\cap \c (X))$.  Thus,
\begin{equation}
\begin{split}
\MoveEqLeft
\Cmp (\cover{U}) \wedge \{ \iota (X)\} \\
& \coloneqq \left\{ \q (U')\cap \q (\c (X)):U\in \cover{U}\right\}  \\
& =\left\{ \q (U'\cap \c (X)):U\in \cover{U}\right\} \\
& =\left\{ \iota (U):U\in \cover{U}\right\} ,
\end{split}
\end{equation}
which demonstrates the truth of \eqref{4.5.14}.

\Step{Show that any other complete uniform space that contains $X$ contains $\Cmp (X)$}
Let $Y$ be some other complete uniform space that contains $X$.  Let $(\lambda \mapsto x_{\lambda})\in X'$ be a Cauchy net in $X$ and let $x_\infty$ be its (unique) limit in $Y$.  Define $\kappa :\Cmp (X)\rightarrow Y$ by $\kappa (\q (\lambda \mapsto x_{\lambda})) \coloneqq x_\infty$.
\begin{exr}[breakable=false]{}{}
Show that $\kappa$ is well-defined and is a uniform-homeomorphism onto its image.
\end{exr}

\Step{Show that $\Cmp (X)$ is unique}
Let $Y$ be another complete uniform space which (i)~contains $X$ and (ii)~is contained in every other complete uniform space that contains $X$.  From this, we know that $Y$ is contained in $\Cmp (X)$.  On the other hand, we already knew that $\Cmp (X)$ was contained in $Y$.  Therefore, $\Cmp (X)=Y$.

\Step{Show that $X$ is dense in $\Cmp (X)$}
Let $\q (\lambda \mapsto x_{\lambda})\in \Cmp (X)$.
\begin{exr}[breakable=false]{}{}
Show that $\lambda \mapsto \iota (x_\lambda )$ converges to $\q (\lambda \mapsto x_{\lambda})$ in $\Cmp (X)$.
\end{exr}
It follows from this exercise that $\Cls (\iota (X))=\Cmp (X)$, and so that $X$ is dense in $\Cmp (X)$.
\end{proof}
\end{thm}
One thing that we will frequently want to do is extend a given function to its completion.  If the codomain is complete as well, we can do this, and in fact, we can do it whenever we have a continuous function defined on a dense subspace.
\begin{prp}{}{}
Let $S\subseteq X$ be a dense subset of a uniform space, let $Y$ be a complete $T_0$ uniform space, and let $f\colon S\rightarrow Y$ be uniformly-continuous.  Then, there exists a unique uniformly-continuous map $g\colon X\rightarrow Y$ such that $\restr{g}{S}=f$.
\begin{rmk}
Mere continuity does not suffice, even in the nicest of cases---see the counter-example below (\cref{exm4.5.20x}).
\end{rmk}
\begin{proof}
Let $x\in X$.  As $\Cls (S)=X$, there is a net $\lambda \mapsto x_\lambda \in X$ converging to $x$.  Pick any such net.  By \cref{exr4.5.3},  $\lambda \mapsto f(x_\lambda )$ is Cauchy in $Y$, so that we may simply take its limit (because $Y$ is complete and is $T_0$, and hence $T_2$, so that limits are unique---see \cref{prp4.5.37}).  So, let us define $g\colon X\rightarrow Y$ by
\begin{equation}
g(x)\coloneqq \lim _\lambda f(x_\lambda ).
\end{equation}
\begin{exr}[breakable=false]{}{}
Show that if $\mu \mapsto y_\mu$ also converges to $x$, then $\lim _\lambda f(x_\lambda )=\lim _\mu f(x_\mu )$.
\begin{rmk}
This shows that $g$ is well-defined, that is, then the definition did not depend on our choice of net converging to $x$.
\end{rmk}
\end{exr}
Thus, the choice of net does not matter, and so for $x\in S$, we may simply take the constant net $\lambda \mapsto x_\lambda \coloneqq x$, so that $g$ is indeed an extension of $f$.
\begin{exr}[breakable=false]{}{}
Show that $g$ is uniformly-continuous.
\end{exr}
\end{proof}
\end{prp}
\begin{exm}{A real-valued continuous function on a dense subspace that does \emph{not} extend}{exm4.5.20x}
In the notation of the previous proposition, take $S\coloneqq \R$, $X\coloneqq [-\infty ,+\infty]$, $Y\coloneqq \R$, and $f\coloneqq \id _{\R}$.  If this had an extension to all of $X$, then in particular the sequence $m\mapsto x_m\coloneqq m$ would have to converge in $\R$.
\begin{rmk}
In fact, if perhaps you thought you could make use of the fact that continuous functions restricted to quasicompact subsets are uniformly-continuous (\cref{prp4.2.73}) to prove the result in special cases, this even provides a counter-example in which every point of $X$ has a compact neighborhood.
\end{rmk}
\end{exm}
Among other things, the significance of this result is that group operations of topological groups extend uniquely to their completions.

Having shown that uniform spaces always have completions, finally, we may return to an unresolved issue all the way back from \cref{chp1}.
\begin{exm}{A nonzero totally-ordered Cau\-chy-complete field distinct from $\R$}{exm4.4.41}
Recall the field of rational functions with coefficients in the reals, $\R (x)$, from \cref{exm2.3.12}.  Being a totally-ordered field, it is in particular a topological group (\cref{exr4.8.58}) with respect to its underlying commutative group $\coord{\R (x),+,0,-}$, and so has a canonical uniform structure.  Thus, we may complete to form the complete topological field $\Cmp (\R (x))$.
\begin{exr}[breakable=false]{}{}
Extend the order on $\R (x)$ to $\Cmp (\R (x))$ so that $\Cmp (\R (x))$ is a totally-ordered field containing $\R (x)$.
\end{exr}
By construction then, $\Cmp (\R (x))$ is a nonzero totally-ordered Cauchy-complete field.  Not only is it distinct from $\R$, but it cannot even embed in $\R$ as, if it did, then so to $\R (x)$ would embed in $\R$ (as it embeds in $\Cmp (\R (x))$), and hence $\R (x)$ would be archimedean, a contradiction of \cref{exm2.3.12}.
\end{exm}

In terms of intuition, the completion is not unlike the closure.  This is made precise by the following result.
\begin{prp}{}{prp4.4.59}
Let $X$ be a complete $T_0$ uniform space and let $S\subseteq X$.  Then, $\Cmp (S)=\Cls (S)$.
\begin{proof}
We leave this as an exercise.
\begin{exr}{}{}
Prove the result yourself.
\end{exr}
\end{proof}
\end{prp}
This allows us to present another example of a completion.
\begin{exm}{$\Cmp (\R [x])=\Mor _{\Top}(\R ,\R )$}{}
Note that we are making the same abuse of notation here as described in a remark of the \namerefpcref{WeierstrassApproximationTheorem}.

By the previous result, $\Cmp (\R [x])=\Cls (\R [x])$.  However, by the \namerefpcref{WeierstrassApproximationTheorem}, $\Cls (\R [x])=\Mor _{\Top}(\R ,\R )$, and so we have $\Cmp (\R [x])=\Mor _{\Top}(\R ,\R )$.
\end{exm}
\begin{exm}{$\Cmp (\Q )=\R$}{}
We've known for awhile now (\cref{CompletenessOfR}) that $\R$ is Cauchy-complete; however, we never stopped to check that it is the `smallest' complete uniform space that contains $\Q$.  This, however now follows from \cref{prp4.4.59} as $\Cls (\Q )=\Cls (\R )$ (which itself was (hopefully) proven in \cref{exr4.2.38}).
\end{exm}

\subsection{Complete metric spaces}

We present here two important results that are specific to complete metric spaces.

\subsubsection{The Baire Category Theorem}

The Baire Category Theorem is an important result concerning complete \emph{metric} spaces.  It has many important applications, most of which we haven't the time to present.  One important application for us, however, is that it will allow us to do one of the remaining separation axiom counter-examples---see \cref{NiemytzkisTangentDiskTopology}.
\begin{thm}{Baire Category Theorem}{BaireCategoryTheorem}
Let $X$ be a complete metric space.  Then,
\begin{enumerate}
\item the countable intersection of open dense subsets of $X$ is dense; and
\item the countable union of closed sets with empty interior has empty interior.
\end{enumerate}
\begin{rmk}
These conclusions are equivalent, the equivalence being obtained by taking the complement of the conclusion.  The former is arguably a bit more intuitive to prove, while the latter the form that is probably more frequently used in concrete situations.
\end{rmk}
\begin{wrn}
Warning:  This is \emph{false} in general for complete uniform spaces---see \cref{exm4.5.2x}.
\end{wrn}
\begin{rmk}
This is often applied as follows.  Let $C_m\subseteq X$ be closed and suppose that $X=\bigcup _{m\in \N}C_m$.  Then, as $X$ certainly has nonempty interior, there must be some $m_0\in \N$ such that $\Int (C_{m_0})\neq \emptyset$.
\end{rmk}
\begin{rmk}
The word ``category'' in the name has nothing to do with categories---the terminology it refers to is archaic.
\end{rmk}
\begin{proof}
For $m\in \N$, let $U_m\subseteq X$ be an open dense subset.  We wish to show that
\begin{equation}
U\coloneqq \bigcap _{m\in \N}U_m
\end{equation}
is dense.  The definition of density is that the closure is equal to all of $X$, in other words, that every point of $X$ is an accumulation point, or in other words, that every open subset intersects the dense set.  So, let $V\subseteq X$ be open.  We wish to show that $V$ intersects $U$.

As $U_0$ is dense, $V$ intersects $U_0$, say at $x_0\in U_0\cap V$.  As $U_0\cap V$ is open, we can fit an $\varepsilon$-ball around $x_0$ inside $U_0\cap V$.  In fact, as $X$ is perfectly-$T_4$, and in particular $T_3$, we can find some $\varepsilon _0>0$ such that
\begin{equation}
\Cls \left( B_{\varepsilon _0}(x_0)\right) \subseteq U_0\cap V.\footnote{We have applied \cref{prp4.5.91}, which is what required we be $T_3$.}
\end{equation}
In fact, by making $\varepsilon _0$ smaller if necessary, we may without loss of generality assume that $\varepsilon _0<2^{-0}=1$.

Now, because $U_1$ is dense, there is some $x_1\in U_1\cap B_{\varepsilon _0}(x_0)$, and so, just the same as before, there is some $0<\varepsilon _1<2^{-1}$ such that
\begin{equation}
\Cls \left( B_{\varepsilon _1}(x_1)\right) \subseteq U_1\cap B_{\varepsilon _0}(x_0).
\end{equation}
Proceeding inductively, we can find $x_m\in X$ and $0<\varepsilon _m<2^{-m}$ such that
\begin{equation}
\Cls \left( B_{\varepsilon _m}(x_m)\right) \subseteq U_m\cap B_{\varepsilon _{m-1}}(x_{m-1}).
\end{equation}

We now check that $m\mapsto x_m$ is Cauchy.  Let $\varepsilon >0$.  Choose $m\in \N$ such that $\varepsilon _m<\varepsilon$.  Suppose that $n\geq m$.  Then, $x_n\in B_{\varepsilon _n}(x_n)\subseteq B_{\varepsilon _m}(x_m)\subseteq B_{\varepsilon}(x_m)$.  Thus, $m\mapsto x_m$ is eventually contained in some $\varepsilon$ ball, and is hence Cauchy.  As $X$ is complete, it converges (to a unique limit) and so we may define
\begin{equation}
x_\infty \coloneqq \lim _mx_m.
\end{equation}

We claim that $x\in U\cap V$.  As explained above, this will complete the proof.  $m\mapsto x_m$ is eventually contained in $\Cls (B_{\varepsilon _0}(x_0))\subseteq V$, and so $x_\infty \in \Cls (B_{\varepsilon _0}(x_0))\subseteq V$.  Similarly, for $n\in \N$, $m\mapsto x_m$ is eventually contained in $\Cls \left( B_{\varepsilon _n}(x_n)\right) \subseteq U_n$, and so, same as before, $x_\infty \in U_n$.  Hence, $x_\infty \in U$, and we are done.
\end{proof}
\end{thm}

\begin{exm}{A complete uniform space which is not a Baire space}{exm4.5.2x}\footnote{A \term{Baire space}\index{Baire space} is a topological space in which the conclusion of the \nameref{BaireCategoryTheorem} holds.  This example was inspired by \href{http://mathoverflow.net/questions/212308/baire-category-theorem-for-complete-uniform-spaces}{priel's answer} on mathoverflow.net.  Thanks to Nate Eldredge for nudging me towards the correct proof.}
Define
\begin{equation}
\begin{multlined}
X\coloneqq \{ f\colon \Z ^+ \rightarrow [0,1] : \\ f(m)=0\text{ for all but finitely many }m\in \N \text{.}\} .\footnote{I am thinking of this as the set of all sequences which eventually terminate in all $0$s.}
\end{multlined}
\end{equation}
We define a uniformity on $X$ as follows.  First of all, for $m\in \Z ^+$, define
\begin{equation}
X_m\coloneqq \underbrace{[0,1] \times \cdots \times [0,1]}_{m}
\end{equation}
equipped with the product uniformity.  Note that $X_m$ embeds in $X$ via $\iota _m:X_m\rightarrow X$ defined by
\begin{equation}
\iota _m(\coord{x_1,\ldots ,x_m})\coloneqq \left( k\mapsto \begin{cases}x_k & \text{if }k\leq m \\ 0 & \text{otherwise,}\end{cases}\right) 
\end{equation}
that is, $\coord{x_1,\ldots ,x_m}$ is sent to the function from $\Z ^+$ into $[0,1]$ which sends $k$ to $x_k$ for $k\leq m$ and $0$ otherwise.  We then equip $X$ with the final uniformity (\cref{FinalUniformity}) with respect to the collection $\{ \iota _m:m\in \Z ^+\}$.

Let $\Lambda \ni \lambda \mapsto x_{\lambda}\in X$ be Cauchy.  We first wish to show that $\lambda \mapsto x_\lambda$ is eventually contained in $X_{m_0}$ for some $m_0\in \Z ^+$.  To show this, we proceed by contradiction:  suppose that for every $m\in \N$ and every $\lambda$ there is some $\lambda _{m,\lambda}\geq \lambda$ such that $x_{\lambda _{m,\lambda}}\notin X_m$.  Define $\Lambda '\coloneqq \Z ^+\times \Lambda$.  Then, $\Lambda '\ni \coord{m,\lambda}\mapsto x_{\lambda _{m,\lambda}}$ is a subnet of $\lambda \mapsto x_\lambda$, and hence is in turn Cauchy.  Thus, for every $\varepsilon _1,\varepsilon _2,\varepsilon _3,\ldots >0$, there is some $\coord{m_0,\lambda _0}$ such that, whenever $\coord{m,\lambda},\coord{n,\mu}\geq \coord{m_0,\lambda _0}$, it follows that
\begin{equation}
\abs{x_{\lambda _{m,\lambda}}(k)-x_{\lambda _{n,\mu}}(k)}<\varepsilon _k
\end{equation}
for all $k\in \Z ^+$.  As $x_{\lambda _{m_0,\lambda _0}}(k)=0$ for all $k$ sufficiently large, we find that
\begin{equation}
x_{\lambda _{n,\mu}}(k)<\varepsilon _k
\end{equation}
for all $k$ sufficiently large, say for $k\geq k_0$, and all $\coord{n,\mu}\geq \coord{m_0,\lambda _0}$.  Take $n\coloneqq \max \{ k_0,m_0\}$.  As $x_{\lambda _{n,\mu}}\notin X_n$, there is some $l>n\geq k_0$ such that $x_{\lambda _{n,\mu}}(l)\neq 0$.  Then,
\begin{equation}
0<x_{\lambda _{n,\mu}}(l)<\varepsilon _l.
\end{equation}
As $\varepsilon _l$ is arbitrary, this is a contradiction.  Thus, there is some $m_0\in \Z ^+$ such that $\lambda \mapsto x_\lambda$ is eventually contained in $X_{m_0}$.

However, $X_{m_0}$, being a finite product of compact metric spaces, is a compact metric space, and hence complete, so that $\lambda \mapsto x_\lambda$ converges in $X_{m_0}$, and hence in $X$.  Therefore, $X$ is complete.

We now wish to check that $X$ is not a Baire space.  From the definition, we have that
\begin{equation}
X=\bigcup _{m\in \Z ^+}X_m.
\end{equation}
\begin{exr}[breakable=false]{}{}
Show that $X_m$ is closed in $X$.
\end{exr}
Thus, to show that $X$ is not a Baire space, it suffices to show that each $X_m$ has empty interior.  So, let $x\in X_m$.  We show that every open neighborhood around $x$ contains an element of $X_{m+1}$ that is not contained in $X_m$.  Let us write
\begin{equation}
x=\coord{x_1,x_2,\ldots ,x_m,0,0,0,\ldots}
\end{equation}
Then, for every neighborhood $U$ of $x$, there will be some $\varepsilon _0>0$ sufficiently small so that
\begin{equation}
x=\coord{x_1,x_2,\ldots ,x_m,\varepsilon _0,0,0,\ldots}\in U,
\end{equation}
so that $x$ is not in the interior of $X_m$, so that each $X_m$ has empty interior.
\end{exm}

\subsubsection{Banach Fixed-Point Theorem}

We finish the chapter with an application of the \nameref{BaireCategoryTheorem} that will prove useful to us later when we study differentiation.
\begin{thm}{Banach Fixed-Point Theorem}{BanachFixedPointTheorem}\index{Banach Fixed-Point Theorem}
Let $\coord{X,\metric}$ be a metric space and let $f\colon X\rightarrow X$ be such that
\begin{equation}\label{4.5.70}
\metric{f(x_1)}{f(x_2)}\leq M\metric{x_1}{x_2}
\end{equation}
for some $0\leq 1<M$.  Then, there is \emph{at most one} point $x_0\in X$ such that $f(x_0)=x_0$.  If $X$ is nonempty and complete, then there is \emph{exactly one} $x_0\in X$ such that $f(x_0)=x_0$.
\begin{rmk}
Such an $x_0$ is called a \term{fixed-point}, hence the name of the theorem.  Thus, the theorem tells us that (i)~fixed-points, if they exist, have to be unique; and (ii)~in the (nonempty) complete case, there has to be some fixed-point (and hence exactly one fixed-point).
\end{rmk}
\begin{rmk}
\eqref{4.5.70} is just the statement that $f$ is Lipschitz-continuous \emph{for a constant} $M<1$.  Such maps are called \term{contraction-mappings}, hence the alternative name for this theorem, the \term{Contraction-Mapping Theorem}\index{Contraction-Mapping Theorem}.
\end{rmk}
\begin{proof}
Let $x_1,x_2\in X$ be two fixed points of $f$.  Then,
\begin{equation}
\metric{x_1}{x_2}=\metric{f(x_1)}{f(x_2)}\leq M\metric{x_1}{x_2},
\end{equation}
and hence, if $\metric{x_1}{x_2}\neq 0$, we would have $1\leq M$:  a contradiction.  Therefore, $\metric{x_1}{x_2}=0$, and hence $x_1=x_2$.

Now take $X$ to be nonempty and complete.  We construct an actual fixed point of $f$.  Let $x_0\in X$ (there is such a point because $X$ is nonempty).  For $m\in \Z ^+$, define
\begin{equation}
x_m\coloneqq f(x_{m-1}).
\end{equation}
We wish to show that the sequence $m\mapsto x_m$ is Cauchy.  If we can do so, then its limit $x_\infty$ must exist, and so by taking the limit of the previous equation, we would find that $x_\infty =f(x_\infty )$ ($f$ is Lipschitz-continuous, hence uniformly-continuous, hence continuous).  Thus, it suffices to show that $m\mapsto x_m$ is Cauchy.

To see this, we first notice that\footnote{We're using $y$s instead of $x$s because those symbols are already used-up.}
\begin{equation*}
\begin{split}
\metric{y_1}{y_2} & \leq \metric{y_1}{f(y_1)}+\metric{f(y_1)}{f(y_2)}+\metric{f(y_2)}{y_2} \\
& \leq \metric{y_1}{f(y_1)}+M\metric{y_1}{y_2}+\metric{f(y_2)}{y_2},
\end{split}
\end{equation*}
and so
\begin{equation}\label{4.5.75}
\metric{y_1}{y_2}\leq \frac{1}{1-M}\left( \metric{y_1}{f(y_1)}+\metric{f(y_2)}{y_2}\right) 
\end{equation}
Also note that
\begin{equation}
\metric{f^m(y_1)}{f^m(y_2)}\leq M^m\metric{y_1}{y_2},
\end{equation}
which follows of course from just applying \eqref{4.5.70} inductively.  Taking $y_1\coloneqq f^m(x_0)$ and $y_2\coloneqq f^n(x_0)$ in \eqref{4.5.75}, we find
\begin{equation*}
\begin{split}
\metric{x_m}{x_n} & \leq \frac{1}{1-M}\left( \metric{x_m}{x_{m+1}}+\metric{x_{n+1}}{x_n}\right) \\
& \leq \frac{1}{1-M}\left( M^m\metric{x_0}{f(x_0)}+M^n\metric{f(x_0)}{x_0}\right) \\
& =\frac{M^m+M^n}{1-M}\metric{x_0}{f(x_0)}.
\end{split}
\end{equation*}
Because $M<1$, we can make $\frac{M^m+M^n}{1-M}$ arbitrarily small by taking $m$ and $n$ sufficiently large.\footnote{If you are not comfortable with this amount of detail, I suggest you fill in the gaps.  You will want to get to the point where you feel comfortable just asserting Cauchyness after obtaining an inequality like this.}  Hence, this sequence is Cauchy, and we are done.
\end{proof}
\end{thm}
\begin{exm}{A contraction mapping with no fixed-point}{}
Take $X\coloneqq \R ^+$ and define $f(x)\coloneqq \frac{1}{2}x$.  This is certainly a contraction mapping, but if $x\in X$ were a fixed point, that is, if $\frac{1}{2}x\eqqcolon f(x)=x$, then we would have $x=0\notin X$.  Therefore, there is no fixed point (in $X$).
\end{exm}