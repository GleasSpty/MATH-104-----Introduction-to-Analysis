\chapter{Topological spaces}\label{chp3}

We have mentioned topological spaces several times throughout the notes already, and finally we turn to studying general topological spaces themselves.  I said before that I think it is fair to say that the purpose of topological spaces is to introduce the most general context as possible in which one can talk about continuity.  The motivating result in this regard is \cref{thm3.4.16}, which says that a function is continuous iff the preimage of every open set is open.  The idea is then to axiomatize the notion of open set:  a topological space will be a set $X$ equipped with a collection of subsets $\topology{U}\subseteq 2^X$, called the \emph{open sets}.  Of course, if we want to be able to say anything of interest, we can't just take any old collection of subsets---we must require the collection to satisfy some conditions.  The conditions we require are those that come from \cref{thm3.4.34,thm3.4.36}, namely, that an arbitrary union and finite intersection of open sets is open.  So, without further ado, I present to you, the definition of a topological space.

\section{The definition of a topological space}

\begin{dfn}{Topological space}{TopologicalSpace}
A \term{topological space}\index{Topological space} is a set $X$ equipped with a collection of subsets $\topology{U}\subseteq 2^X$, the \term{topology}\index{Topology} on $X$, such that
\begin{enumerate}
\item \label{enmTopologicalSpace.i}$\emptyset ,X\in \topology{U}$;
\item \label{enmTopologicalSpace.ii}$\bigcup _{U\in \topology{V}}U\in \topology{U}$ if $\topology{V}\subseteq \topology{U}$; and
\item \label{enmTopologicalSpace.iii}$\bigcap _{k=1}^mU_k\in \topology{U}$ if $U_k\in \topology{U}$ for $1\leq k\leq m$.
\end{enumerate}
\begin{rmk}
The elements of $\topology{U}$ are the \term{open sets}\index{Open (in a topological space)}.
\end{rmk}
\begin{rmk}
In other words, a topological space is a set $X$ equipped with a collection of subsets (containing at least $\emptyset$ and $X$) closed under arbitrary union and finite intersection.
\end{rmk}
\begin{rmk}
A subset $C\subseteq X$ is \term{closed}\index{Closed (in a topological space)} iff $C^{\comp}$ is open.  (Recall that this is precisely the definition we gave in $\R$---see \cref{dfn3.4.17}.)
\end{rmk}
\begin{wrn}
Warning:  A common mistake beginners make is to think ``More open sets means fewer closed sets.''.  This is \emph{wrong}.  There are always the same number of open sets as closed sets:  $U\mapsto U^{\comp}$ gives a bijection between them.
\end{wrn}
\begin{rmk}
In order to exclude stupid things like the empty topology, we require that the empty-set and the entire set are open.  (Recall that these were open in $\R$---see \cref{exr3.4.13,exr3.4.14}.)
\end{rmk}
\end{dfn}
Of course, you can specify a topology just as well by saying what the closed sets are (the open sets are then just the complements of these sets).
\begin{exr}{}{exr4.1.2}
Let $X$ be a set and let $\collection{C}\subseteq 2^X$ be a collection of subsets of $X$ such that
\begin{enumerate}
\item $\emptyset ,X\in \collection{C}$;
\item $\bigcap _{C\in \collection{D}}C$ if $\collection{D}\subseteq \collection{C}$; and
\item $\bigcup _{k=1}^mC_k\in \collection{C}$ if $C_k\in \collection{C}$ for $1\leq k\leq m$.
\end{enumerate}
Show that there is a unique topology on $X$ whose collection of closed sets is precisely $\collection{C}$.
\begin{rmk}
In other words, your collection of closed sets must be nontrivial, closed under arbitrary intersection, and closed under finite union, just as was the case in $\R$---see \cref{exr3.4.38x,exr3.4.40}.
\end{rmk}
\end{exr}
\begin{dfn}{$G_\delta$ and $F_\sigma$ sets}{GDeltaFSigma}
Let $S\subseteq X$ be a subset of a topological space.  Then, $S$ is a \term{$G_\delta$ set}\index{$G_\delta$ set} iff $S$ is the countable intersection of open sets.  $S$ is an \term{$F_\sigma$ set}\index{$F_\sigma$ set} iff $S$ i the countable union of closed sets.
\begin{rmk}
For us, these concepts will not come up very often, so it is not imperative that you remember these terms---indeed, you can probably just skip this definition for now.  They do come up, however, so it would be incomplete to not include them, and if they're going to be included somewhere, this is not a bad place to do so.
\end{rmk}
\end{dfn}

\subsection{(Neighborhood) bases and generating collections}

It is often not necessary to define every single open set explicitly, but rather, only a special class of open sets that determine all the others.  For example, in the real numbers, to determine whether an arbitrary set was open, we made use of the `special' open sets $B_\varepsilon (x)$:  you could determine whether an arbitrary set was open simply by knowing the $\varepsilon$ balls---see \cref{OpenSetInR}.  The idea that generalizes this notion is that of a \emph{base} for a topology.
\begin{dfn}{Base}{Base}
Let $X$ be a topological space and let $\topology{B}$ be a collection of open sets of $X$.  Then, $\topology{B}$ is a \term{base}\index{Base} for the topology of $X$ iff the statement that a subset $U$ of $X$ is open is equivalent to the statement that, for every $x\in U$, there is some $B\in \topology{B}$ such that $x\in B\subseteq U$.
\end{dfn}
\begin{exm}{}{}
The collection $\left\{ B_\varepsilon (x):x\in \R ^d,\ \varepsilon >0\right\}$ is a base for the topology of $\R ^d$.
\end{exm}
The real reason bases are important is because they allow us to \emph{define} topologies, and so it is important to know when a collection of subsets of a set form a base for some topology.
\begin{prp}{}{prp4.1.5}
Let $X$ be a set and let $\topology{B}$ be a collection of subsets of $X$.  Then, there exists a unique topology for which $\topology{B}$ is a base iff
\begin{enumerate}
\item \label{enm4.1.4.i}$\topology{B}$ covers $X$; and
\item \label{enm4.1.4.ii}for every $x\in X$ and $B_1,B_2\in \topology{B}$ with $x\in B_1,B_2$, there is some $B_3\in \topology{B}$ such that $x\in B_3\subseteq B_1\cap B_2$.
\end{enumerate}
\begin{rmk}
If $X$ does not a priori come with a topology, we will still refer to any collection of sets that satisfy \cref{enm4.1.4.i}--\cref{enm4.1.4.ii} as a \emph{base}.
\end{rmk}
\begin{proof}
$(\Rightarrow )$ Suppose that there is a unique topology $\topology{U}$ for which $\topology{B}$ is a base.  We first show that $\topology{B}$ covers $X$.  We proceed by contradiction:  suppose there is some $x\in X$ which is not contained in any $B\in \topology{B}$.  Then, as $\topology{B}$ is a base for the topology, $X$ would not be open:  a contradiction.  Therefore, $\topology{B}$ covers $X$.

Now for the second property:  let $x\in X$ and let $B_1,B_2\in \topology{B}$ be such that $x\in B_1,B_2$.  By the definition of a base, we have that $\topology{B}\subseteq \topology{U}$.  In particular, $B_1\cap B_2\in \topology{U}$, and so because $\topology{B}$ is a base for $\topology{U}$, there must be some $x\in B_3\in \topology{B}$ such that $B_3\subseteq B_1\cap B_2$.

\blankline
\noindent
$(\Leftarrow )$ Suppose that (i)~$\topology{B}$ covers $X$, and that (ii)~for every for $x\in X$ and $B_1,B_2\in \topology{B}$ with $x\in B_1\cap B_2$, there is some $B_3\in \topology{B}$ such that $x\in B_3\subseteq B_1\cap B_2$.  We declare $U\subseteq X$ to be open iff for every $x\in U$ there is some $B\in \topology{B}$ with $x\in B\subseteq U$.  By the definition of bases, this was the only possibility.  We need only check that this is in fact a topology.  The empty-set is vacuously open.  $X$ is open because $\topology{B}$ covers $X$.  Let $\topology{V}$ be a collection of open sets and let $x\in \bigcup _{U\in \topology{V}}U$.  Then, $x\in U$ for some $U\in \topology{V}$, and so there is some $B\in \topology{B}$ such that $x\in B\subseteq U\subseteq \bigcup _{U\in \topology{V}}U$.  Thus, $\bigcup _{U\in \topology{V}}U$ is open.  Let $U_1,\ldots ,U_m$ be open and let $x\in \bigcap _{k=1}^mU_k$.  Then, there is some $B_k\in \topology{B}$ such that $x\in B_k\subseteq U_k$.  By \cref{enm4.1.4.ii}, there is some $B\in \topology{B}$ with $x\in B\subseteq B_1\cap \cdots \cap B_m\subseteq \bigcap _{k=1}^mU_k$, and so $\bigcap _{k=1}^mU_k$ is open.
\end{proof}
\end{prp}
\begin{exr}{}{exr4.1.7}
Let $X$ be a topological space and let $\topology{B}$ be a collection of subsets of $X$.  Show that $\topology{B}$ is a base for the topology of $X$ iff it has the property that $U\subseteq X$ open is equivalent to $U$ being a union of elements of $\topology{B}$.
\end{exr}
There is a similar way of defining a topology.  Instead of specifying a base for all open sets, you specify a base for all the open sets at a point (see the following definitions) for every point.
\begin{dfn}{Neighborhood}{Neighborhood}
Let $X$ be a topological space and let $S\subseteq N\subseteq X$.  Then, $N$ is a \term{neighborhood}\index{Neighborhood} of $S$ iff there is some open set $U\subseteq X$ such that $S\subseteq U\subseteq N$.  An \term{open neighborhood}\index{Open neighborhood} of $S$ is just an open set which contains $S$.  A(n open) neighborhood of a point $x$ is a(n open) neighborhood of $\{ x\}$.
\begin{rmk}
The intuition is that neighborhoods have the ``wiggle room'' that is characteristic of open sets, but are not necessarily open themselves.  For example, in $\R$, $D_{\varepsilon}(x_0)$ is a neighborhood of $x_0$, but isn't actually open.\footnote{Recall that $D_{\varepsilon}(x_0)\coloneqq \{ x\in \R :\abs{x-x_0}\leq \varepsilon \}$---see \cref{AbsoluteValue}.}
\end{rmk}
\end{dfn}
\begin{dfn}{Neighborhood base}{NeighborhoodBase}
Let $X$ be a topological space, let $x\in X$, and let $\topology{B}_x$ be a collection of neighborhoods\footnote{Not necessarily open!} of $x$.  Then, $\topology{B}_x$ is a \term{neighborhood base}\index{Neighorhood base} of $x$ iff for every neighborhood $N\subseteq X$ of $x$ there is some $B_x\in \topology{B}_x$ such that $B_x\subseteq N$.  If $\topology{B}_x$ is a neighborhood base of $x$ for all $x\in X$, then $\{ \topology{B}_x:x\in X\}$ is a \term{neighborhood base} of the topology.
\end{dfn}
\begin{exm}{A neighborhood base with no open sets}{exm4.1.7}
Take $X\coloneqq \R$, and for $x\in X$ define $\topology{B}_x\coloneqq \{ D_{\varepsilon}(x):\varepsilon >0\}$.  This is a neighborhood base but yet no $D_{\varepsilon}(x)$ is open.
\end{exm}
There are two important related facts about neighborhood bases---the first that they can be used to determine exactly which sets are open (\cref{prp3.1.11}) and the second that they can be used to define topologies (\cref{prp4.1.8}).
\begin{prp}{}{prp3.1.11}
Let $X$ be a topological space and for each $x\in X$ let $\topology{B}_x$ be a collection of neighborhoods of $x$.  Then, $\{ \topology{B}_x:x\in X\}$ is a neighborhood base of the topology iff the statement that $U\subseteq X$ is open is equivalent to the statement that for every $x\in U$ there is some $B_x\in \topology{B}_x$ such that $B_x\subseteq U$.
\begin{proof}
$(\Rightarrow )$ Suppose that $\topology{B}_x$ is a neighborhood base of $x$ for all $x\in X$.  Let $U\subseteq X$.  First suppose that $U$ is open.  Let $x\in U$.  Then, $U$ is a neighborhood of $x$, and so there is some $B_x\in \topology{B}_x$ such that $B_x\subseteq U$.  Conversely, suppose that for every $x\in U$ there is some $B_x\in \topology{B}_x$ such that $B_x\subseteq U$.  As $B_x$ is a neighborhood of $x$, there is some open $U_x\subseteq B_x$ with $x\in U_x$.  Then, $U=\bigcup _{x\in U}U_x$, and so $U$ is open.

\blankline
\noindent
$(\Leftarrow )$ Suppose that the statement that $U\subseteq X$ is open is equivalent to the statement that for every $x\in U$ there is some $B_x\in \topology{B}_x$ such that $B_x\subseteq U$.  Let $x\in X$ and let $N\subseteq X$ be a neighborhood of $x$.  Then, there is some open set $U\subseteq N$ with $x\in U$.  By the hypothesis, there is them some $B_x\in \topology{B}_x$ such that $B_x\subseteq U\subseteq N$.  Thus, by definition, $\topology{B}_x$ is a neighborhood base at $x$.
\end{proof}
\end{prp}
Once again, neighborhood bases allow us to \emph{define} topologies.
\begin{prp}{}{prp4.1.8}
Let $X$ be a set and for each $x\in X$ let $\topology{B}_x$ be a nonempty collection of subsets of $X$ which contain $x$.  Then, there exists a unique topology for which $\{ \topology{B}_x:x\in X\}$ is a neighborhood base iff for every $x\in X$ and $B_1,B_2\in \topology{B}_x$, there is a subset $U\subseteq B_1\cap B_2$ such that (i) $x\in U$ and (ii) for every $y\in U$ there is some $B_y\in \topology{B}_y$ such that $B_y\subseteq U$.
	
Furthermore, if every element of each $\topology{B}_x$ is open, then $\bigcup _{x\in X}\topology{B}_x$ is a base for this topology.
\begin{wrn}
Warning:  The elements of $\topology{B}_x$ \emph{need not be open}---see \cref{exm4.1.7}.  In particular, $\bigcup _{x\in X}\topology{B}_x$ will not be a base in general.
\end{wrn}
\begin{rmk}
This remark contains an explanation of my first failed attempt to formulate this result.  As such, it is not particularly important, and may be skipped (along with the following example, \cref{exm3.1.12}).
	
I first mistakenly thought that we could instead get away with the easier property that each $\topology{B}_x$ is a filter base.\footnote{$\topology{B}_x$ would be a ``filter base'' iff for every $B_1,B_2\in \topology{B}_x$, there is some $B_3\in \topology{B}_x$ such that $B_3\subseteq B_1\cap B_2$---see \cref{FilterBase}.}

For convenience of language, let us say that a collection $\{ \topology{B}_x:x\in X\}$ where each $\topology{B}_x$ is a collection of sets that contain $x$ `creates' a topology iff the statement that $U$ is open in this topology is equivalent to the statement that for every $x\in U$ there is some $B\in \topology{B}_x$ such that $B\subseteq U$.\footnote{This of course is just the definition of a neighborhood base, except I no longer want to require that elements of $\topology{B}_x$ be neighborhoods of $x$.  You should also note that this is ad hoc `throw away' terminology---besides the related \cref{exm3.1.12}, we shall never use this term again.}

Using this language, I initially thought that $\{ \topology{B}_x:x\in X \}$ would be a neighborhood base for a unique topology iff each $\topology{B}_x$ were a filter base.  If you only assume that each $\topology{B}_x$ is a filter base, however, then you will be unable to prove that elements of $\topology{B}_x$ are neighborhoods of $x$ (despite the fact that it would still ``create'' a unique topology)---see \cref{exm3.1.12}.

Instead, you might then try to relax the requirement that elements of $\collection{B}_x$ be neighborhoods of $x$, but if you do this, then you become unable to do the other direction, that is, you can have a topology ``created'' by $\{ \topology{B}_x:x\in X\}$ for which not every $\topology{B}_x$ is a filter base---see \cref{exm3.1.12} again.
\end{rmk}
\begin{rmk}
The previous remark notwithstanding, it is still true that if $\{ \topology{B}_x:x\in X\}$ is a neighborhood base for a unique topology, then for every $x\in X$ and $B_1,B_2\in \topology{B}_x$, there is some $B_3\in \topology{B}_x$ such that $B_3\subseteq B_1\cap B_2$---it's just that this is not \emph{equivalent} to being a neighborhood base for a unique topology.
\end{rmk}
\begin{proof}
$(\Rightarrow )$ Suppose that there exists a unique topology for which $\topology{B}_x$ is a neighborhood base at $x$.  Let $x\in X$ and let $B_1,B_2\in \topology{B}_x$.  Then, $B_1$ and $B_2$ are neighborhoods of $x$, and so there are open sets $U_1\subseteq U_2\subseteq B_2$ containing $x$.  Then, $U_1\cap U_2$ contains $x$ and is open.  As $\{ \topology{B}_x:x\in X\}$ is a neighborhood base for the topology, this means that for every $y\in U_1\cap U_2$, there is some $B_y\in \topology{B}_y$ such that $B_y\subseteq U_1\cap U_2$.
	
\blankline
\noindent
$(\Leftarrow )$ Suppose that for every $x\in X$ and $B_1,B_2\in \topology{B}_x$, there is some $U\subseteq B_1\cap B_2$ such that (i) $x\in U$ and (ii) for every $y\in U$ there is some $B_y\in \topology{B}_y$ such that $B_y\subseteq U$.

\begin{clm}[breakable=false]{}{}
For every $x\in X$ and $B_1,\ldots ,B_m\in \topology{B}_x$, there is some $U\subseteq B_1\cap \cdots \cap P_m$ such that (i) $x\in U$ and (ii) for every $y\in U$ there is some $B_y\in \topology{B}_y$ such that $B_y\subseteq U$.
\begin{proof}
By hypothesis, there is some $U_{12}\subseteq B_1\cap B_2$ such that (i) $x\in U$ and (ii) for every $y\in U$ there is some $B_y\in \topology{B}_y$ such that $B_y\subseteq U_{12}$  As of course $x\in U_{12}$, there is some $B_{12}\in \topology{B}_x$ such that $B_{12}\subseteq U_{12}$.  Applying the hypothesis again, there is some $U_{123}\subseteq B_{12}\cap B_3$ such that (i) $x\in U_{123}$ and (ii) for every $y\in U_{123}$ there is some $B_y\in \topology{B}_y$ such that $B_y\subseteq U_{123}$.  We have that $U_{123}\subseteq B_{12}\cap B_3\subseteq B_1\cap B_2\cap B_3$.  By repeating this process inductively, we find a subset $U_{1\dots m}\subseteq B_1\cap \cdots \cap B_m$ such that (i) $x\in U_{1\dots m}$ and (ii) for every $y\in U_{1\cdots m}$ there is some $B_y\in \topology{B}_y$ such that $B_y\subseteq U_{1\dots m}$, as desired.
\end{proof}
\end{clm}

We declare $U\subseteq X$ to be open iff for every $x\in U$ there is some $B_x\in \topology{B}_x$ with $B\subseteq U$.  By definition of neighborhood bases, this was the only possibility.  We need to check that this is in fact a topology, and that each element of every $\topology{B}_x$ is in fact a neighborhood of $x$ in this topology.  The empty-set is vacuously open.  $X$ is open because each $\topology{B}_x$ is nonempty.  Let $\topology{V}$ be a collection of open sets and let $x\in \bigcup _{U\in \topology{V}}U$.  Then, $x\in U$ for some $U\in \topology{V}$, and so there is some $B_x\in \topology{B}_x$ such that $B\subseteq U\subseteq \bigcup _{U\in \topology{V}}U$.  Thus, $\bigcup _{U\in \topology{V}}U$ is open.  Let $U_1,\ldots ,U_m$ be open and let $x\in \bigcap _{k=1}^mU_k$.  Then, $x\in U_k$ for each $k$, and so there is some $B_k\in \topology{B}_x$ such that $x\in B_k\subseteq U_k$.  By the claim, there is a subset $U\subseteq B_1\cap \cdots \cap B_m$ such that (i) $x\in U$ and (ii) for every $y\in U$ there is some $B_y\in \topology{B}_y$ such that $B_y\subseteq U$.  As in particular $x\in U$, there is some $B_x\in \topology{B}_x$ such that $B_x\subseteq U\subseteq U_1\cap \cdots \cap U_m$, and so $U_1\cap \cdots \cap U_m$ is open.

It remains to check that each $B\in \topology{B}_x$ is a neighborhood of $x$.  Taking $B_1\ceqq B\eqqc B_2$ in the hypothesis, we see that there is some subset $U\subseteq B$ such that (i) $x\in U$ and (ii) for every $y\in U$ there is some $B_y\in \topology{B}_y$ such that $B_y\subseteq U$.  The second condition is just the statement that $U$ is open.  Thus, as $x\in U\subseteq B$, $B$ is a neighborhood of $x$.
		
\blankline
\noindent
Now suppose that $\topology{B}_x$ consists purely of open sets.  Define $\topology{B}\coloneqq \bigcup _{x\in X}\topology{B}_x$.  We wish to show that $\topology{B}$ is a base for the topology.  To do that, we must show that $U\subseteq X$ is open iff for all $x\in U$ there is some $B\in \topology{B}$ such that $x\in B\subseteq U$.  One direction is easy:  if $U$ is open and $x\in X$, then in fact there is some $B\in \topology{B}_x$ such that $x\in B\subseteq U$.
		
Conversely, suppose that for all $x\in U$ there is some $B\in \topology{B}$ such that $x\in B\subseteq U$.  We wish to show that $U$ is open.  So, let $x\in U$.  We wish to find some $B\in \topology{B}_x$ such that $x\in B\subseteq U$.  By hypothesis and the definition of $\topology{B}$, we know that there is some $B_y\in \topology{B}_y$ such that $x\in B_y\subseteq U$ for $y\in X$.  However, we're assuming that every element in $\topology{B}_y$ is open, and so in fact there is some $B\in \topology{B}_x$ such that $x\in B\subseteq B_y\subseteq U$, as desired.
\end{proof}
\end{prp}
\begin{exm}{Counter-examples to a naive formulation of \cref{prp4.1.8}}{exm3.1.12}
We first give an example of a collection $\{ \topology{B}_x:x\in X\}$ where each $\topology{B}_x$ is a filter base containing $x$ for which the elements of $\topology{B}_x$ are not neighborhoods of $x$ in the unique topology ``created''\footnote{In the sense described in the remark of \cref{prp4.1.8}.} by $\{ \topology{B}_x:x\in X\}$.

Define $X\ceqq \{ 1,2,3\}$, $\topology{B}_1\ceqq \{ \{ 1,2\} \}$, $\topology{B}_2\ceqq \{ \{ 2,3\} \}$, and $\topology{B}_3\ceqq \{ \{ 3,1\} \}$.  Then, for every $x\in X$, we immediately have the property that for every $B_1,B_2\in \topology{B}_x$, there is some $B_3\in \topology{B}_x$ such that $B_3\subseteq B_1\cap B_2$.  On the other hand, if $\{ \topology{B}_x:x\in X \}$ were a neighborhood base for a topology on $X$, then this topology would have to be the indiscrete topology, in which case $\{ 1,2\}$, for example, would not be a neighborhood of $1$.

We now check that this does in fact ``create'' the indiscrete topology.  So, let $U\subseteq X$ be nonempty.  Then, without loss of generality, we would have $1\in U$, and hence $\{ 1,2\} \subseteq U$ as $B_1$ is (by assumption) a neighborhood base of $1\in X$.  But then $2\in U$, and hence $\{ 2,3\} \subseteq U$ as $B_2$ is (by assumption) a neighborhood base of $2\in X$.  But then $3\in U$, and so $U=X$.

\blankline
\noindent
We now give an example in which $\{ \topology{B}_x:x\in X\}$ ``creates'' a unique topology but for which not every $\topology{B}_x$ is a filter base (in which case we necessarily must have that elements of $\topology{B}_x$ are \emph{not} all neighborhoods of $x$---otherwise, \cref{prp4.1.8} implies they would have to be a filter base).

Let all the definitions be the same as before, except redefine
\begin{equation}
\topology{B}_1\ceqq \{ \{ 1,2\} ,\{ 1,3\} \} \text{ and }\topology{B}_3\ceqq \{ \{ 1,2,3\} \} .
\end{equation}
Again, this ``creates'' the indiscrete topology, but $\topology{B}_1$ is not a filter base because $\{ 1,2\} \cap \{ 1,3\} =\{ 1\}$ does not contain as a subset any element of $\topology{B}_1$.
\end{exm}
Sometimes we have a collection of sets that we would like to be open, but they do not necessarily form a base, and so in this case we cannot just invoke \cref{prp4.1.5}.  However, what we can do is take the `smallest' topology which contains these sets.
\begin{prp}{Generating collection (of a top\-ology)}{GeneratingCollection}
Let $X$ be a set and let $\collection{S}\subseteq 2^X$.  Then, there exists a unique topology $\topology{U}$ on $X$, the topology \term{generated}\index{Generate (a topology)} by $\collection{S}$, such that
\begin{enumerate}
\item \label{GeneratingCollection.i}$\collection{S}\subseteq \topology{U}$; and
\item \label{GeneratingCollection.ii}if $\topology{U}'$ is any other topology on $X$ containing $\collection{S}$, it follows that $\topology{U}\subseteq \topology{U}'$.
\end{enumerate}
Furthermore, the collection of all finite intersections of elements of $\collection{S}\cup \{ X\}$ is a base for this topology.\footnote{We throw $X$ in in case $\collection{S}$ did not cover $X$ (recall that bases (\cref{Base}) need to cover the space).}.  $\collection{S}$ is a \term{generating collection}\index{Generating collection (of a topology)}.
\begin{rmk}
Compare this with the definition of the integers, rationals, reals, closure, and interior (\cref{Integers,RationalNumbers,RealNumbers,Closure,Interior}).
\end{rmk}
\begin{proof}
Let $\topology{B}$ be the collection of all finite intersections of elements of $\collection{S}\cup \{ X\}$.  This definitely covers $X$ as $X\in \topology{B}$.  Furthermore, the intersection of any two elements of $\topology{B}$ is also an element of $\topology{B}$, by definition.  Therefore, there is a unique topology $\topology{U}$ on $X$ for which $\topology{B}$ is a base (\cref{prp4.1.5}).

By construction, $\collection{S}\subseteq \topology{B}\subseteq \topology{U}$.  On the other hand, if $\topology{U}'$ is any other topology for which every element of $\collection{S}$ is open, then, because topologies are closed under finite intersection, $\topology{U}'$ must contain $\topology{B}$, and hence it must contain $\topology{U}$ (because $\topology{U}'$ is closed under arbitrary union and every element of $\topology{U}$ is a union of elements of $\topology{B}$ (\cref{exr4.1.7})).

The uniqueness proof is one that is hopefully familiar by now:  if $\topology{V}$ also satisfies both these properties, then on one hand we already know that $\topology{U}\subseteq \topology{V}$, but on the other hand, \cref{GeneratingCollection.ii} applied to $\topology{V}$ gives $\topology{V}\subseteq \topology{V}$, and hence $\topology{U}=\topology{V}$.
\end{proof}
\end{prp}
Generating collections are actually quite nice because several things can be checked just by looking at generating collections, which are generally significantly `smaller'\footnote{Though not literally in the sense of cardinality.} than the entire topology---see \cref{exr4.1.27,exr4.2.41}, and the \namerefpcref{AlexanderSubbaseTheorem}.

\subsection{Some basic examples}

We can use the notion of a base to define the \emph{order topology}.
\begin{dfn}{Order topology}{OrderTopology}
Let $X$ be a totally-ordered set, and let $\topology{B}$ be the collection of sets of the form $(a,b)\coloneqq \left\{ x\in X:a<x<b\right\}$ for $a,b\in X$ with $a<b$.  (We also allow $a=-\infty$ and $b=+\infty$, in which case $(a,+\infty )\coloneqq \left\{ x\in X:x>a\right\}$ and similarly for $(-\infty ,b)$ and $(-\infty, +\infty)$.)
\begin{exr}[breakable=false]{}{}
Use \cref{prp4.1.5} to show that $\topology{B}$ is a base for a topology.
\end{exr}
The topology defined by $\topology{B}$ is the \term{order topology}\index{Order topology} on $X$.
\end{dfn}
\begin{exm}{A partially-ordered set whose open intervals do not form a base}{}
Define
\begin{equation}
X\coloneqq \{ x,y_1,y_2,y_3,z_1,z_2\} ,
\end{equation}
and declare that
\begin{equation}
x\leq y_k\text{ and }z_1\geq y_1,y_2\text{ and }z_2\geq y_2,y_3
\end{equation}
for all $k$.\footnote{Of course, we also have all additional relations needed to make this reflexive and transitive.}  Then,
\begin{equation}
(x,z_1)=\{ y_1,y_2\} \text{ and }(x,z_2)=\{ y_2,y_3\} .
\end{equation}
Thus, if the open\footnote{Warning:  Here, ``open'' is referring to the strict inequalities in the definition of $(a,b)$, and \emph{not} whether or not it is open in the relevant topology.} intervals formed a base for a topology, then
\begin{equation}
(x,z_1)\cap (x,z_2)=\{ y_2\} 
\end{equation}
would have to be open.  As this is a singleton, the only way this could happen is if $\{ y_2\}$ is an open interval itself.  However,, $\{ y_2\}$ is not an interval because if we had $\{ y_2\} =(a,b)$ for $a<y_2$ and $b>y_2$, we would necessarily have $a=x$ or $a=-\infty$, and $b=z_1,z_2,+\infty$.  In all of these $6$ cases, $(a,b)$ must contain at least either $y_1$ or $y_3$ as well, and so in particular, it would not be the case that $\{ y_2\} =(a,b)$.
\begin{rmk}
This is why we only define the order topology for \emph{totally}-ordered sets.
\end{rmk}
\end{exm}
\begin{exm}{$\N$, $\Z$, $\Q$, and $\R$}{exm3.1.21}
$\N$, $\Z$, $\Q$, and $\R$ are all totally-ordered and so we may (and do) equip them with the order topology.
\end{exm}
The order topology on $\N$ and $\Z$ are examples of a very special topology, the `finest' one possible, namely the \emph{discrete topology}.
\begin{dfn}{Discrete topology}{DiscreteTopology}
Let $X$ be any set.  The \term{discrete topology}\index{Discrete topology} is the topology in which \emph{every} subset is open.
\begin{rmk}
Note that every set is also \emph{closed}.
\end{rmk}
\begin{rmk}
In other words, the topology is just the power-set of $X$.  This is a topology for tautological reasons.
\end{rmk}
\begin{rmk}
Using the ``wiggle room'' intuition again, in the discrete topology, every singleton $\{ x\}$ is open, and so you might interpret this as saying that there is no ``wiggle room'' around $x$ at all.  Certainly, I think it's fair to say that you should imagine points of a discrete space as being, well, discrete.
\end{rmk}
\end{dfn}
\begin{prp}{}{prp4.1.11}
The order topologies on $\N$ and $\Z$ are both discrete.
\begin{proof}
We prove that the order topology on $\Z$ is discrete and leave the case of $\N$ as an exercise (a very similar argument will work).  We wish to show that every subset of $\Z$ is open.  To show this, it suffices to show that each singleton set is open because an arbitrary set is going to be a union of singletons.  So, let $m\in \Z$.  To show that $\{ m\}$ is open, it suffices to find $a,b\in \Z$ such that $(a,b)=\{ m\}$ (because every element in the base of a topology is automatically open).  Take $a\coloneqq m-1$ and $b\coloneqq m+1$.  Recall (\cref{exr1.2.14}) that there is no integer between $0$ and $1$.  It follows that there is no integer between $k$ and $k+1$ for all $k\in \Z$, and so indeed, $(m-1,m+1)=\{ m\}$.
\begin{exr}{}{}
Show that the order topology on $\N$ is discrete.
\end{exr}
\end{proof}
\end{prp}
\begin{exr}{}{}
Why is the topology on $\Q$ not discrete?
\end{exr}
The discrete topology is the `finest' topology you can have (finer means more open sets).  The `coarsest' topology you can have is the \emph{indiscrete topology}.
\begin{dfn}{Indiscrete topology}{dfnIndiscreteTopology}
Let $X$ be any set.  The \term{indiscrete topology}\index{Indiscrete topology} is just $\{ \emptyset ,X\}$.
\begin{rmk}
The definition of a topology requires that at least the empty-set and the entire set are open.  The indiscrete topology is when nothing else is open.  It is not particularly useful, but it can be an easy source of some counter-examples (e.g.~\cref{exm4.1.20}).
\end{rmk}
\end{dfn}

\subsection{Continuity}

We now finally vastly generalize our definition of continuity.
\begin{dfn}{Continuous function}{ContinuousFunction}
Let $f\colon X\rightarrow Y$ be a function between topological spaces and let $x\in X$.  Then, $f$ is \term{continuous}\index{Continuous (at a point)} at $x$ iff the preimage of every neighborhood of $f(x)$ is a neighborhood of $x$.  $f$ is \term{continuous}\index{Continuous} iff $f$ is continuous at $x$ for all $x\in X$.
\begin{wrn}
Warning:  Here, you \emph{cannot} replace ``neighborhood'' with ``open neighborhood''  For example, consider the function $f\colon \R \rightarrow \R$ defined by
\begin{equation}
f(x)\ceqq \begin{cases}0 & \text{if }x<0 \\ 1 & \text{if }x\geq 0\end{cases}.
\end{equation}
Then, this \emph{should} be continuous at $x=1$, but yet it is \emph{not} the case the preimage of every open neighborhood of $f(1)=1$ is open is an open neighborhood of $1$:  $f^{-1}(B_{\frac{1}{2}}(1))=[0,\infty )$.  While this is a neighborhood of $x=1$, it is not an \emph{open} neighborhood of $x=1$.
\end{wrn}
\end{dfn}
\begin{exr}{}{exr3.1.27}
Show that $f\colon X\rightarrow Y$ is continuous iff the preimage of every open set is open.
\end{exr}
In fact, we can do a bit better than this.
\begin{exr}{}{exr4.1.27}
Let $X$ and $Y$ be topological spaces and let $\collection{S}$ be generate the topology of $Y$.  Show that $f$ is continuous iff $f^{-1}(S)$ is open for all $S\in \collection{S}$.
\begin{rmk}
In particular, this is true for $\topology{S}$ a \emph{base} for the topology of $Y$
\end{rmk}
\end{exr}
\begin{exr}{}{}
Let $f\colon X\rightarrow Y$ be any function between two topological spaces.
\begin{enumerate}
\item Show that, if $X$ has the discrete topology, then $f$ is continuous.
\item Show that, if $Y$ has the indiscrete topology, then $f$ is continuous.
\end{enumerate}
\begin{rmk}
In other words, every function \emph{on} a discrete space is continuous.  Similarly, every function \emph{into} an indiscrete space is continuous.
\end{rmk}
\end{exr}

\begin{exm}{The category of topological \\ spaces}{}\index{Category of topological spaces}
The category of topological spaces is the category $\Top$\index[notation]{$\Top$}
\begin{enumerate}
\item whose collection of objects $\Obj (\Top )$ is the collection of all topological spaces;
\item with morphism set $\Mor _{\Top}(X,Y)$ precisely the set of all continuous functions from $X$ to $Y$;
\item whose composition is given by ordinary function composition; and
\item whose identities are given by the identity functions.
\end{enumerate}
\begin{exr}[breakable=false]{}{}
Show that the composition of two continuous functions is continuous.
\begin{rmk}
Note that this is something you need to check in order for $\Top$ to actually form a category.  You also need to verify that the identity function is continuous, but this is trivial (the preimage of a set is itself, so\textellipsis ).
\end{rmk}
\end{exr}
\end{exm}
\begin{dfn}{Homeomorphism}{Homeomorphism}
Let $f\colon X\rightarrow Y$ be a function between topological spaces.  Then, $f$ is a \term{homeomorphism}\index{Homeomorphism} iff $f$ is an isomorphism in $\Top$.
\end{dfn}
\begin{exr}{}{}
Show that a function is a homeomorphism iff (i)~it is bijective, (ii)~it is continuous, and (iii)~its inverse is continuous.
\end{exr}
\begin{exr}{}{exr3.1.34}
Find an example of a function that is (i)~bijective, (ii)~continuous, but (iii)~is not a homeomorphism.
\begin{rmk}
In other words, find a bijective continuous function whose inverse is not continuous.
\end{rmk}
\begin{rmk}
Contrast this with, for example, isomorphisms in $\Ring$.  It follows immediately from the definition that a function between groups is an isomorphism in $\Ring$ iff (i)~it is bijective, (ii)~it is a homomorphism, and (iii)~its inverse is a homomorphism.  However, by \cref{exrA.2.11x}, if the original function is a homomorphism, then we get that its inverse is a homomorphism for free, so we only need to actually check (i)~and (ii).  The point is, with rings (and other algebraic objects), the isomorphisms are just bijective homomorphisms, but this is \emph{not} the case in $\Top$.
\end{rmk}
\end{exr}
\begin{exr}{Embedding}{Embedding}
Let $f\colon X\rightarrow Y$ be a function between topological spaces.  Show that $f$ is an embedding in $\Top$\footnote{See \cref{EmbeddingAndQuotient}.} iff $f$ is a homeomorphism onto its image.
\end{exr}
\begin{exm}{An injective continuous function that is not an embedding}{exm3.1.36}
Let $X$ be any set.  Then, $\id _X\colon \coord{X,\text{discrete}}\rightarrow \coord{X,\text{indiscrete}}$ is an injective (in fact, bijective) function that is continuous, but will not be an embedding as long as $X$ has at least two points.
\begin{rmk}
While a trivial example, it has significance in that it tells us that the naive definition of an embedding in a concrete category as ``injective morphism'' is the \emph{wrong} definition of embedding---see \cref{EmbeddingAndQuotient}.
\end{rmk}
\end{exm}
\begin{exr}{}{}
Let $X$ be a set, and let $\topology{U}$ and $\topology{V}$ be two topologies on $X$.  Then, if $\coord{X,\topology{U}}$ is homeomorphic to $\coord{X,\topology{V}}$, must we have that $\topology{U}=\topology{V}$?
\end{exr}

There is a useful result that is applicable in general whenever you want to check that a ``piece-wise'' function is continuous.
\begin{prp}{Pasting Lemma}{PastingLemma}\index{Pasting Lemma}
Let $X$ and $Y$ be topological spaces, let $C_1,C_2\subseteq X$ be closed, and let $f_1:C_1\rightarrow Y$ and $f_2:C_2\rightarrow Y$ be continuous.  Then, if $\restr{f_1}{C_1\cap C_2}=\restr{f_2}{C_1\cap C_2}$, then the function $f\colon C_1\cup C_2\rightarrow Y$ defined by
\begin{equation}
f(x)\coloneqq \begin{cases}f_1(x) & \text{if }x\in C_1 \\ f_2(x) & \text{if }x\in C_2\end{cases}
\end{equation}
is well-defined and continuous.
\begin{proof}
We saw before in the reals (\cref{exr2.5.33}) that a function is continuous iff the preimage of every closed set is closed.  This is still true (\cref{prp3.2.11}) with essentially the same proof as before.  We use this to prove the result.

First of all, we needed to assume that $\restr{f_1}{C_1\cap C_2}=\restr{f_2}{C_1\cap C_2}$ in order for the definition of $f$ to actually be a function.  We now check that $f$ is indeed continuous.

Let $D\subseteq Y$ be closed.  Define $D_1\coloneqq f^{-1}(D)\cap C_1$ and $D_2\coloneqq f^{-1}(D)\cap C_2$, so that $f^{-1}(D)=D_1\cup D_2$.  It thus suffices to show that each $D_k$ is closed.  By $1\leftrightarrow 2$ symmetry, it suffice to show that just $D_1$ is closed.  However, this is the case because
\begin{equation}
D_1\coloneqq f^{-1}(D)\cap C_1=f_1^{-1}(D)\cap C_1
\end{equation}
and $f_1$ is continuous.
\end{proof}
\end{prp}

\subsection{Some motivation}

At this point, it is reasonable for one to ask ``Why do we care about such generality in an introductory \emph{real} analysis course?  Shouldn't we only be concerned with the real numbers for now?''.  I would argue that, even in the case where you really are only interested in the real numbers, abstraction can shed light onto such a specific example.  The real numbers are many things:  a field, a totally-ordered set, a totally-ordered field, a metric space, a uniform space, a topological space, a manifold, etc..  In mathematics, we are not just concerned about \emph{what} is true, but \emph{why} things are true.  Because the real numbers are so special, having so much structure, there are many things true about them.  But by the same token, because there is so much structure, unless you go through the proofs yourself in detail, it can be difficult to recall \emph{why} things are true---does property XYZ hold for the real numbers because they are a totally-ordered field or because they are a metric space or because they are a topological vector space or because\textellipsis ?  Instead, however, if we step back and only study $\R$ as a topological space, it becomes clearer why certain things are true.  This allows us to tell easily what is true about the real numbers because they are a topological space, as opposed to what is true about the real numbers because they are a field, etc..

For example, consider the following.
\begin{exm}{}{}
Though we have not technically defined it yet, hopefully you are familiar with the function $\arctan$ from calculus (see \cref{Arctan}).  $\arctan :\R \rightarrow (-\frac{\uppi}{2},\frac{\uppi}{2})$ is a homeomorphism (see \cref{ArctanHomeo}).
\end{exm}
\begin{exr}{}{}
Find a homeomorphism from $(-\frac{\uppi}{2},\frac{\uppi}{2})$ to $(0,1)$.
\end{exr}
Thus, from the point of view of general topology, there is no distinction between $\R$ and $(0,1)$.  This is completely analogous to the sense in which there is no difference between $\R$ and $2^{\N}$ at the level of sets.  ($\R$ and $(0,1)$ are isomorphic in $\Top$, and $\R$ and $2^{\N}$ are isomorphic in $\Set$.)  Recall from the very end of \crefnameref{chp1}:  morphisms matter.

Perhaps a good example of the more general context making it clearer \emph{why} things are true is the Intermediate Value Theorem.  Here is the `classical' statement you are probably familiar with from calculus.\footnote{See \cref{ClassicalIntermediateValueTheorem} for the proof.}
\begin{textequation}
Let $f\colon [a,b]\rightarrow \R$ be continuous.  Then, for all $y$ between $f(a)$ and $f(b)$ (inclusive) there is some $x\in [a,b]$ such that $f(x)=y$.
\end{textequation}
Compare that with the more general statement.\footnote{See \cref{IntermediateValueTheorem} for the proof.}
\begin{textequation}
The continuous image of a connected set is connected.
\end{textequation}
I think it's fair to say that the latter is much more elegant, and perhaps even easier to understand.\footnote{Admittedly, the proof to go from the general statement to the `classical' statement is not completely trivial (a subset of $\R$ is connected iff it is an interval---see \cref{thm4.5.14}), but the two results are still `morally' the same.}

\section{A review}

Quite a many things that we did with the real numbers in the last chapter carry over to general topological spaces no problem.  For convenience, we present here all definitions and theorems which carry over nearly verbatim to topological spaces.  We will try to point out when things do \emph{not} carry over identically (for example, limits no longer need be unique (\cref{exm4.1.20})).

One important thing to keep in mind is:  \emph{Open neighborhoods of a point in a general topological space play the same role that $\varepsilon$-balls did in $\R$}.  Not only do we use this to determine appropriate generalizations, but if you are feeling a bit uncomfortable with topological spaces, this should help your intuition.

For the entirety of this section, $X$ and $Y$ will be general topological spaces.  We recommend that this section be used mainly as a reference---the pedagogy of the concepts is contained in the last chapter.
\begin{dfn}{Net}{}
A \term{net}\index{Net} in $X$ is a function from a nonempty directed set $\coord{\Lambda ,\leq}$ into $X$.
\end{dfn}
\begin{dfn}{Sequence}{}
A \term{sequence}\index{Sequence} is a net whose directed set is order-isomorphic (i.e.~isomorphic in $\Pre$) to $\coord{\N ,\leq}$.
\end{dfn}
\begin{mdf}{Eventually XYZ}{EventuallyXYZInTop}
Let $\Lambda \ni \lambda \mapsto x_\lambda \in X$ be a net.  Then, $\lambda \mapsto a_\lambda$ is \term{eventually XYZ}\index{Eventually XYZ} iff there is some $\lambda _0$ such that $\{ \lambda \in \Lambda :\lambda \geq \lambda _0\} \ni \lambda \mapsto x_\lambda$ is XYZ.
\end{mdf}
\begin{mdf}{Frequently XYZ}{FrequentlyXYZInTop}
Let $\Lambda \ni \lambda \mapsto x_{\lambda}\in X$ be a net.  Then, $\lambda \mapsto x_{\lambda}$ is \term{frequently}\index{Frequently XYZ} iff for every $\lambda \in \Lambda$ there is some $\lambda '\geq \lambda$ such that $x_{\lambda '}$ is XYZ.
\end{mdf}
\begin{mpr}{}{mpr3.2.5}
Let XYZ be a property that is such that a net $\lambda \mapsto x_{\lambda}$ is XYZ iff each $x_{\lambda}$ is XYZ.  Then, a net $\lambda \mapsto x_{\lambda}\in \R$ be a net is frequently not XYZ iff it is not eventually XYZ.
\begin{rmk}
In particular, for $S\subseteq X$, a net is frequently contained in $S$ iff it is not eventually contained in $S^{\comp}$.
\end{rmk}
\end{mpr}
\begin{dfn}{Limit (of a net)}{LimitOfANet}
Let $\lambda \mapsto x_\lambda$ be a net and let $x_\infty \in X$.  Then, $x_\infty$ is a \term{limit}\index{Limit (of a net)} of $\lambda \mapsto x_\lambda$ iff for every open neighborhood $U$ of $x_\infty$, $\lambda \mapsto x_{\lambda}$ is eventually contained in $U$.  If a net has a limit, then we say that it \term{converges}\index{Convergence (of a net)}.
\begin{rmk}
Note that we no longer have the concept of \emph{diverge}, as this required to at least have a notion of ``unbounded'', a concept which doesn't make sense for general topological spaces.
\end{rmk}
\begin{wrn}
Warning:  Note that limits no longer need be unique---see the following example.  In particular, you have to be careful about using the notation $\lim _{\lambda}x_{\lambda}$---if there is more than one limit, to which element should the notation $\lim _{\lambda}x_{\lambda}$ be referring to?  That said, in the case limits are unique (e.g.\ in $T_2$ spaces---see \cref{prp4.5.37}), we will use the notation $\lim _{\lambda}x_{\lambda}$\index[notation]{$\lim _{\lambda}x_{\lambda}$} to denote \emph{the} limit of $\lambda \mapsto x_{\lambda}$.
\end{wrn}
\end{dfn}
\begin{exm}{Limits need not be unique}{exm4.1.20}
Define $X\coloneqq \{ 0,1\}$ and equip $X$ with the indiscrete topology (\cref{dfnIndiscreteTopology}).  Then, the constant net $\lambda \mapsto x_\lambda \coloneqq 0$ converges to both $0$ and $1$ (the only open neighborhood of both of these points is $X$ itself, and of course the net is eventually contained in $X$).
\end{exm}
\begin{prp}{}{}
Let $\lambda \mapsto x_{\lambda}$ be a net and let $x_{\infty}\in X$.  Then, it is \emph{not} the case that $\lambda \mapsto x_{\lambda}$ converges to $x_{\infty}$ iff there is an open neighborhood $U$ of $x$ such that $\lambda \mapsto x_{\lambda}$ is frequently contained in $U^{\comp}$.
\end{prp}
\begin{dfn}{Subnet}{Subnet}
Let $x:\Lambda \rightarrow X$ be a be a net.  Then, a \term{subnet}\index{Subnet} of $x$ is a net $y:\Lambda '\rightarrow X$ such that
\begin{enumerate}
\item for all $\mu \in \Lambda '$, $y_\mu =x_{\lambda _\mu}$ for some $\lambda _\mu \in \Lambda$; and
\item whenever $U\subseteq X$ eventually contains $x$, it eventually contains $y$.
\end{enumerate}
A \term{cofinal subnet}\index{Cofinal subnet} of $x$ is a net of the form $\restr{x}{\Lambda'}$ for $\Lambda '\subseteq \Lambda$ cofinal.  A \term{subsequence}\index{Subsequence} is a subnet that is a sequence.
\end{dfn}
\begin{prp}{}{prp3.2.7}
Let $\lambda \mapsto x_{\lambda}\in X$ be a net.  Then, $\mu \mapsto x_{\lambda _{\mu}}$ is a subnet of $\lambda \mapsto x_{\lambda _{\mu}}$ iff for all $\lambda _0$ there is some $\mu _0$ such that
\begin{equation}
\left\{ x_{\lambda _{\mu}}:\mu \geq \mu _0\right\} \subseteq \left\{ x_{\lambda}:\lambda \geq \lambda _0\right\} .
\end{equation}
\end{prp}
\begin{prp}{}{prp3.2.9}
Let $x\colon \Lambda \rightarrow X$ and $y\colon \Lambda '\rightarrow X$ be nets.  Then, if there is a function $\iota \colon \Lambda '\rightarrow \Lambda$ such that (i)~$y=x\circ \iota$ and (ii)~for all $\lambda \in \Lambda$ there is some $\mu _0\in \Lambda '$ such that, whenever $\mu \geq \mu _0$, it follows that $\iota (\mu )\geq \lambda$, then $y$ is a subnet of $x$.
\end{prp}
\begin{prp}{}{prp3.2.10}
Let $x\colon \Lambda \rightarrow X$ and $y\colon \Lambda '\rightarrow X$ be nets.  Then, if there is a function $\iota \colon \Lambda '\rightarrow \Lambda$ such that (i)~$y=x\circ \iota$, (ii)~is nondecreasing, and (iii)~has cofinal image, then $y$ is a subnet of $a$.
\end{prp}
\begin{thm}{Kelley's Convergence Axioms}{KelleysConvergenceAxioms}\index{Kelley's Convergence Axioms}
\begin{enumerate}
\item \label{enmKelleysConvergenceAxioms.i}Constant nets converge to the constant.
\item \label{enmKelleysConvergenceAxioms.iix}Let $\mu \mapsto x_{\lambda _\mu}$ be a subnet of a net $\lambda \mapsto x_\lambda$.  Then, if $\lim _{\lambda}x_{\lambda}=a_{\infty}$, then $\lim _{\mu}x_{\lambda _{\mu}}=a_{\infty}$.
\item \label{enmKelleysConvergenceAxioms.ii}Let $\lambda \mapsto x_{\lambda}$ be a net.  Then, if every cofinal subnet $\mu \mapsto x_{\lambda _\mu}$ has in turn a subnet itself $\nu \mapsto x_{\lambda _{\mu _\nu}}$ such that $\lim _\nu x_{\lambda _{\mu _\nu}}=x_\infty$, then $\lim _\lambda x_\lambda =x_\infty$.
\item \label{enmKelleysConvergenceAxioms.iii}Let $I$ be a directed set and for each $i\in I$ let $x^i:\Lambda ^i\rightarrow X$ be a convergent net.  Then, if $(x^\infty )_\infty \coloneqq \lim _i\lim _\lambda (x^i)_\lambda$ exists, then $I\times \prod _{i\in I}\Lambda ^i\ni \coord{i,\lambda}\mapsto (x^i)_{\lambda ^i}$ converges to $(x^\infty )_\infty$.
\end{enumerate}
\begin{rmk}
We will see later (\cref{KelleysConvergenceTheorem}) that these three properties can be used to define a topology (hence, the reason we refer to them as \emph{axioms}).
\end{rmk}
\begin{rmk}
As far as \namerefpcref{KelleysConvergenceTheorem} is concerned, we can\footnote{The proof of this fact is given in \cref{prp3.4.22}.} replace \cref{enmKelleysConvergenceAxioms.iix,enmKelleysConvergenceAxioms.ii} here with just the single axiom
\begin{textequation}
$\lim _{\lambda}x_{\lambda}=x_{\infty}$ iff every subnet $\mu \mapsto x_{\lambda _\mu}$ has in turn a subnet itself $\nu \mapsto x_{\lambda _{\mu _\nu}}$ such that $\lim _\nu x_{\lambda _{\mu _\nu}}=x_\infty$.
\end{textequation}
This is a bit cleaner, I suppose, if only because it cuts four axioms down to three.  That said, notice that the term ``cofinal'' has been dropped here.  This is because \cref{enmKelleysConvergenceAxioms.ii} without ``cofinal'' will imply \cref{enmKelleysConvergenceAxioms.iix} (so that, without the condition ``cofinal'' appearing in \cref{enmKelleysConvergenceAxioms.ii}, we needn't separately list \cref{enmKelleysConvergenceAxioms.iix}), but this should no longer be the case if you include the condition ``cofinal''.  As we will want to make use of the version with the ``cofinal'' appearing (and not just for the proof of \nameref{KelleysConvergenceTheorem}), we list the four-axiom version of Kelley's Convergence Axioms.
\end{rmk}
\end{thm}
\begin{dfn}{Limit point}{}
Let $S\subseteq X$ and let $x_0\in X$.  Then, $x_0$ is a \term{limit point}\index{Limit point} of $S$ iff there exists a net $\lambda \mapsto x_\lambda \in S$ with $x_\lambda \neq x_0$ that converges to $x_0$.
\begin{rmk}
Note how in $\R ^d$ (\cref{LimitPoint}) we wrote ``such that $\lim _{\lambda}x_{\lambda}=x$'' here.  We can no longer do this as limits might not be unique---see the remark in \cref{LimitOfANet}.  Similar tweaks apply to statements throughout this section.
\end{rmk}
\end{dfn}
\begin{dfn}{Limit (of a function)}{dfnLimitOfAFunction}
Let $f\colon X \rightarrow Y$ be a function between topological spaces, and let $x_0\in X$ and $y\in Y$.  Then, $y$ is a \term{limit}\index{Limit (of a function)} of $f$ at $x_0$ iff for every net $\lambda \mapsto x_\lambda$ with $x_\lambda \neq x_0$ that converges to $x_0$, $\lambda \mapsto f(x_\lambda )$ converges to $y$.
\begin{rmk}
Similarly as for limits of nets, we don't have uniqueness in general, in which case the notion $\lim _{x\to a}f(x)$ is ambiguous, though we will still make use of this notation when limits are unique.
\end{rmk}
\begin{wrn}
Warning:  While in the real numbers this might be true if you replace the word ``net'' with the word ``sequence'', however, in general topological spaces, this fails---see \cref{exm4.2.8}.
\end{wrn}
\end{dfn}
\begin{dfn}{Limit superior and limit inferior (of a function)}{LimitSuperiorAndLimitInferiorOfAFunction}
Let $D\subseteq \R ^d$, let $x_0\in \R ^d$ be a limit point of $D$, and let $f\colon D\rightarrow \R ^e$.  Then, the \term{limit superior}\index{Limit superior (of a function)} and \term{limit inferior}\index{Limit inferior (of a function)} of $f$ at $x_0$, $\limsup _{x\to x_0}f(x)$ and $\liminf _{x\to x_0}f(x)$ respectively, are defined by
\begin{subequations}
\begin{align}
\limsup _{x\to x_0}f(x) & \ceqq \lim _{U\in \topology{U}_{x_0}}\sup \left\{ f(x):x\in U\right\} \\
\liminf _{x\to x_0}f(x)& \ceqq \lim _{U\in \topology{U}_{x_0}}\inf \left\{ f(x):x\in U\right\} ,
\end{align}
\end{subequations}\index[notation]{$\limsup \limits _{x\to x_0}f(x)$}\index[notation]{$\liminf \limits _{x\to x_0}f(x)$}
where $\topology{U}_{x_0}$ is the collection of open sets containing $x_0$ regarded as a directed set with reverse inclusion.
\end{dfn}
\begin{prp}{}{}
Let $f\colon X\rightarrow Y$ be a function and let $x_0\in X$.  Then, $f$ is continuous at $x_0$ iff $f(x_0)$ is a limit of $f$ at $x_0$.  $f$ is continuous iff it is continuous at $x_0$ for all $x_0\in X$.
\begin{rmk}
If limits were unique, we could write this more transparently as $\lim _{x\to x_0}f(x)=f(x_0)$.
\end{rmk}
\end{prp}
\begin{prp}{}{prp3.2.11}
Let $f\colon X\rightarrow Y$ be a function.  Then, $f$ is continuous iff the preimage of every closed set is closed.
\end{prp}
\begin{dfn}{Accumulation point}{AccumulationPoint}
Let $S\subseteq X$ and let $x\in X$.  Then, $x$ is an \term{accumulation point}\index{Accumulation point} of $S$ iff every open neighborhood of $x$ intersects $S$ at a point distinct from $x$.
\end{dfn}
\begin{dfn}{Adherent point}{AdherentPoint}
Let $S\subseteq X$ and let $x_0\in X$.  Then, $x_0$ is an \term{adherent point}\index{Adherent point} of $X$ iff every open neighborhood of $x$ intersects $S$.
\end{dfn}
\begin{prp}{}{prp3.2.14}
Let $S\subseteq X$ and let $x_0\in X$.  Then, $x_0$ is an accumulation point of $S$ iff it is a limit point of $S$.
\begin{rmk}
If you replace ``net'' with ``sequence'' in the definition of a limit point, then this result will be \emph{false} in general---see \cref{exm4.2.8x}
\end{rmk}
\end{prp}
\begin{prp}{}{}
Let $S\subseteq X$ and let $x_0\in X$.  Then, $x_0$ is an adherent point of $S$ iff there is a net $\lambda \mapsto x_{\lambda} \in S$ that converges to $x_0$.
\end{prp}
\begin{dfn}{Isolated point}{IsolatedPoint}
Let $S\subseteq X$ and let $x\in X$.  Then, $x$ is an \term{isolated point}\index{Isolated point} iff there is an open neighborhood $U$ of $x$ such that $U\cap S=\{ x\}$.
\end{dfn}
\begin{prp}{}{}
Let $S\subseteq X$ and let $x_0\in X$.  Then, if $x_0$ is an adherent point of $S$, then either $x_0$ is an accumulation point of $S$ or $x_0$ is an isolated point of $S$.
\begin{rmk}
Note that this or is \emph{exclusive}.
\end{rmk}
\end{prp}
\begin{thm}{}{}
Let $C\subseteq X$.  Then, the following are equivalent.
\begin{enumerate}
\item $C$ is closed.
\item $C$ contains all its accumulation points.
\item $C$ contains all its limit points.
\item $C$ contains all its adherent points.
\item $C$ contains all points $x_0\in X$ for which there is a net $\lambda \mapsto x_{\lambda}\in C$ converging to $x_0$.
\item $C$ is equal to its set of adherent points.
\item $C$ is equal to the set of all points $x_0\in X$ for which there is a net $\lambda \mapsto x_{\lambda}\in C$ converging to $x_0$.
\end{enumerate}
\end{thm}
\begin{dfn}{Perfect set}{}
Let $C\subseteq X$.  Then, $C$ is \term{perfect}\index{Perfect} iff it is equal to its set of accumulation points.
\end{dfn}
There are several equivalent ways to state this condition.
\begin{prp}{}{}
Let $C\subseteq X$.  Then, the following are equivalent.
\begin{enumerate}
\item $C$ is perfect.
\item $C$ is closed and every element of $C$ is an accumulation point of $C$.
\item $C$ is closed and has no isolated points.
\end{enumerate}
\end{prp}

We proved that (\cref{prp3.4.27}) in $\R$ that $x\in \R$ is an accumulation point of a sequence iff there was a subsequence that converged to $x$.  We also gave an example (\cref{exm3.4.29}) of how this fails (even in $\R$) for general nets.  Not only this, but this also fails to hold (for sequences) in general topological spaces.
\begin{exm}{An accumulation point of a sequence to which no subsequence converges}{exm4.2.15}
Define $X\coloneqq \{ x_1,x_2,x_3\}$ and
\begin{equation}
\topology{U}\coloneqq \left\{ \emptyset ,X,\{ x_1,x_2\} \right\} .
\end{equation}
Consider the sequence $m\mapsto a_m$ defined by
\begin{equation}
a_m\coloneqq \begin{cases}x_2 & \text{if }m=0 \\ x_3 & \text{otherwise.}\end{cases}
\end{equation}
Then, it is true that every open neighborhood of $x_1\in X$ contains an element of the set $\{ a_m:m\in \N \}$, namely $a_0\coloneqq x_2$, distinct from $x_1$.  On the other hand, no subsequence of $m\mapsto a_m$ converges to $x_1$ as it is eventually outside an open neighborhood of $x_1$ (namely the open neighborhood $\{ x_1,x_2\}$).
\end{exm}
\begin{dfn}{Interior point}{}
Let $S\subseteq X$ and let $x_0\in X$.  Then, $x_0$ is an \term{interior point}\index{Interior point} of $S$ iff there is some open neighborhood $U$ of $x_0$ such that $U\subseteq S$.
\end{dfn}
\begin{prp}{Closure}{Closure}
Let $S\subseteq X$.  Then, there exists a unique set $\Cls (S)\subseteq X$\index[notation]{$\Cls (S)$}, the \term{closure}\index{Closure} of $S$, that satisfies
\begin{enumerate}
\item $\Cls (S)$ is closed;
\item $S\subseteq \Cls (S)$; and
\item if $C$ is any other closed set which contains $S$, then $\Cls (S)\subseteq C$.
\end{enumerate}
Furthermore, explicitly, we have
\begin{equation}
\Cls (S)=\bigcap _{\substack{C\subseteq X\text{ closed} \\ S\subseteq C}}C.
\end{equation}
\end{prp}
\begin{prp}{}{prp3.2.32}
Let $S\subseteq X$.  Then, $\Cls (S)$ is the union of $S$ and its set of accumulation points.
\end{prp}
\begin{thm}{Kuratowski Closure Axioms}{}\index{Kuratowski Closure Axioms}
Let $S,T\subseteq X$.  Then,
\begin{enumerate}
\item $\Cls (\emptyset) =\emptyset$;
\item $S\subseteq \Cls (S)$;
\item $\Cls (S)=\Cls \left( \Cls (S)\right)$; and
\item $\Cls (S\cup T)=\Cls (S)\cup \Cls (T)$.
\end{enumerate}
\end{thm}
\begin{prp}{Interior}{Interior}
Let $S\subseteq X$.  Then, there exists a unique set $\Int (S)\subseteq \R$\index[notation]{$\Int (S)$}, the \term{interior}\index{Interior} of $S$, that satisfies
\begin{enumerate}
\item $\Int (S)$ is open;
\item $\Int (S)\subseteq S$; and
\item if $U$ is any other open set which is contained in $S$, then $U\subseteq \Int (U)$.
\end{enumerate}
Furthermore, explicitly, we have
\begin{equation}
\Int (S)=\bigcup _{\substack{U\subseteq X\text{ open} \\ U\subseteq S}}U.
\end{equation}
\end{prp}
\begin{prp}{}{}
Let $S\subseteq X$.  Then, $\Int (S)$ is the set of interior points of $S$.
\end{prp}
\begin{thm}{Kuratowski Interior Axioms}{}\index{Kuratowski Interior Axioms}
Let $S,T\subseteq X$.  Then,
\begin{enumerate}
\item $\Int (X)=X$;
\item $\Int (S)\subseteq S$;
\item $\Int (S)=\Int \left( \Int (S)\right)$; and
\item $\Int (S\cap T)=\Int (S)\cap \Int (T)$.
\end{enumerate}
\end{thm}
\begin{prp}{}{prp4.2.21}
Let $S\subseteq X$.  Then, $S$ is closed iff $S=\Cls (S)$.
\end{prp}
\begin{prp}{}{}
Let $S\subseteq X$.  Then, $S$ is open iff $S=\Int (S)$.
\end{prp}

We postponed the following counter-examples because we technically had not introduced the notion of closure in a general topological space.
\begin{exm}{A limit point that is not a sequential limit point}{exm4.2.8x}
Define $X\coloneqq \R$.  We equip $\R$ with a nonstandard topology, the so-called \term{cocountable topology}\index{Cocountable topology}.\footnote{\term{Cocountable}\index{Cocountable} means that the complement is countable.  The name of the topology here derives from the fact that the \emph{open} sets have countable complement (except the empty-set of course).}  Let $C\subseteq X$ and declare that
\begin{textequation}
$C$ is closed iff either (i)~$C=X$ or (ii)~$C$ is countable.
\end{textequation}
Because the finite union of countable sets is countable and an arbitrary intersection of countable sets is countable (obviously), it follows that this defines a topology on $X$.

We first show that every convergent sequence is eventually constant.  So, let $m\mapsto x_m\in X$ be sequence that converges to $x_\infty \in X$.  We proceed by contradiction:  suppose that it is not eventually constant.  Then, the set
\begin{equation}
\{ m\in \N :x_m\neq x_\infty \}
\end{equation}
is cofinal, and hence defines a subsequence $n\mapsto x_{m_n}$ that (i)~converges to $x_\infty$ but (ii)~is never equal to $x_\infty$.  Hence, the set $C\coloneqq \{ x_{m_n}:n\in \N \}$ does not contain $x_\infty$, and so $C^{\comp}$ is an open neighborhood of $x_\infty$.  But of course $n\mapsto x_{m_n}$ is not eventually contained in $C^{\comp}$---no term of this subsequence is contained in $C^{\comp}$.  Therefore, $n\mapsto x_{m_n}$ cannot converge to $x_\infty$:  a contradiction.  Therefore, $m\mapsto x_m$ must be eventually constant.

Define $U\coloneqq \Q ^{\comp}$.  As $U^{\comp}=\Q$ is countable, $U$ is open.  We note that the closure of $U$ is all of $\R$:  no countable set can contain $U$, and so the only closed set which contains $U$ is $\R$ itself.  It thus follows that,\footnote{Because the closure of a set is that set union its accumulation points (\cref{prp3.2.32}), and accumulation points are the same as limit points (\cref{prp3.2.14}).} in particular, $0$ is a limit point of $U$.  On the other hand, as every convergent sequence is eventually constant, no sequence in $U\coloneqq \Q ^{\comp}$ can converge to $0$.
\end{exm}
\begin{exm}{Two distinct topologies with the same notion of sequential convergence}{exm4.2.25}
Let $X$ be as in \cref{exm4.2.8x} and denote the topology on $X$ given there (the cocountable topology) by $\topology{U}$.  Denote by $\topology{D}$ the discrete topology on $X$.  We saw in \cref{exm4.2.8x} that sequences converge iff they are eventually constant.  However, we also have the following result.
\begin{exr}[breakable=false]{}{}
Let $X$ be a discrete space and let $\lambda \mapsto x_\lambda \in X$ be net.  Show that $\lambda \mapsto x_\lambda$ iff it is eventually constant.
\end{exr}
Thus, a given sequence $m\mapsto x_m\in X$ converges with respect to $\topology{U}$ iff it converges with respect to $\topology{D}$, and in this case, they converge to the same limit.  On the other hand, the set $\{ 0\}$ is \emph{not} open with respect to $\topology{U}$\footnote{This uses the fact that the real numbers are uncountable!} but it is open with respect to $\topology{D}$.

Even though the topologies are distinct, perhaps it is the case that they are \emph{homeomorphic}?  We show that this cannot happen.  In fact, we show that no bijective function from $\phi :\coord{X,\topology{U}}\rightarrow \coord{X,\topology{D}}$ is continuous.  If $\phi$ were such a function, then $\phi ^{-1}(\Q ^{\comp})$ would be uncountable and proper, and hence would not be closed, a contradiction of the fact that $\phi$ is continuous and $\Q ^{\comp}$ is closed with respect to $\topology{D}$.
\end{exm}
\begin{exm}{A sequentially-continuous function that is not continuous}{exm4.2.8}
Let $X$ be as in \cref{exm4.2.8x} ($\R$ with the cocountable topology) and define $f\colon X\rightarrow X$ by
\begin{equation}
f(x)\coloneqq \begin{cases}1 & \text{if }x\in \Q \\ -1 & \text{if }x\in \Q ^{\comp}.\end{cases}
\end{equation}
(Note that this is just the \namerefpcref{DirichletFunction}!)

This function is certainly not continuous because, for example, the preimage of $\{ -1\}$ is not closed.  On the other hand, because every convergent sequence is eventually constant and the condition that $x_\lambda \neq x$ in the definition of a limit (\cref{dfnLimitOfAFunction}), it follows that $f$ vacuously satisfies the continuity condition for sequences.
\end{exm}
\begin{dfn}{Cover}{}
Let $S\subseteq X$ and let $\topology{U}\subseteq 2^{X}$.  Then, $\topology{U}$ is a \term{cover}\index{Cover} of $S$ iff $S\subseteq \bigcup _{U\in \topology{U}}U$.  $\topology{U}$ is an \term{open cover}\index{Open cover} iff every $U\in \topology{U}$ is open.  A \term{subcover} of $\topology{U}$ is a subset $\topology{V}\subseteq \topology{U}$ that is still a cover of $S$.
\end{dfn}
\begin{dfn}{Quasicompact}{Quasicompact}
Let $S\subseteq X$.  Then, $S$ is \term{quasicompact}\index{Quasicompact} iff every open cover of $S$ has a finite subcover.
\end{dfn}
\begin{exr}{}{exr4.2.33x}
Show that finite spaces are quasicompact.
\end{exr}
\begin{exr}{}{exr4.2.33}
Show that closed subsets of quasicompact spaces are quasicompact.
\end{exr}
\begin{dfn}{Finite-intersection property}{FiniteIntersectionProperty}
Let $S\subseteq X$ and let $\collection{C}\subseteq 2^X$ be a collection of subsets of $X$.  Then, $\collection{C}$ has the \term{finite-intersection property}\index{Finite-intersection property} with $S$ iff every finite subset $\{ C_1,\ldots ,C_m\} \subseteq \collection{C}$ intersects $S$:  $(C_1\cap \cdots \cap C_m)\cap S\neq \emptyset$.  For $S=X$, we simply say that $\collection{C}$ has the \term{finite-intersection property}
\end{dfn}
\begin{prp}{}{prp4.2.32}
Let $K\subseteq X$ and let $\collection{C}$ be a collection of closed subsets of $X$.  Then, $K$ is quasicompact iff whenever $\collection{C}$ has the finite-intersection property with $K$, the entire intersection $\bigcap _{C\in \collection{C}}C$ also intersects $K$.
\end{prp}
\begin{prp}{}{prp4.2.31}
Let $K\subseteq X$.  Then, $K$ is quasicompact iff every net $\lambda \mapsto a_\lambda \in K$ has a subnet that converges to a limit in $K$.
\end{prp}
Note that the Heine-Borel and Bolzano-Weierstrass Theorems (\cref{HeineBorelTheorem,BolzanoWeierstrassTheorem}) do not hold in general, but we will wait until after having studied integration before writing down counter-examples---see \cref{exm5.3.23,exm5.3.28}.

\subsection{A couple new things}

We present in this subsection a couple of facts that, while not technically review per se, make more sense to place here before we begin study of new topological concepts in earnest.

\begin{dfn}{Dense}{Dense}
Let $X$ be a topological space and let $S\subseteq X$.  Then, $S$ is \term{dense}\index{Dense} in $X$ iff $\Cls (S)=X$.
\begin{rmk}
In other words, every element of $X$ is either an accumulation point of $S$ or an element of $S$ (or both).  Recall (the remark in \cref{dfn3.4.20}) our intuition for accumulation points are points which are ``infinitely close'' to the set.  Thus, intuitively, the statement ``$S$ is dense in $X$.'' can be understood as saying that every point of $X$ is ``infinitely close'' to $S$.
\end{rmk}
\end{dfn}
\begin{exr}{}{exr4.2.38}
Show that $\Q$ and $\Q ^{\comp}$ are both dense in $\R$.
\begin{rmk}
We mentioned way back when we discussed `density' of $\Q$ and $\Q ^{\comp}$ in $\R$ (\cref{thm3.2.14,thm3.3.76}) that these results aren't literally the statements that $\Q$ and $\Q ^{\comp}$ are dense in $\R$.  When we say that $\Q$ is dense in $\R$, what we mean of course is that $\Cls (\Q )=\R$ (and similarly for $\Q ^{\comp}$).  People `abuse' language and refer to the properties of \cref{thm3.2.14,thm3.3.76} as ``density'' because density in this sense (that is, in the sense of the definition above) follows as an easy corollary of these theorems.
\end{rmk}
\end{exr}

The next couple of facts have to do with neighborhood bases, generating collections, and their relations to concepts reviewed in the previous subsection.
\begin{exr}{}{exr3.2.49}
Let $X$ be a topological space, let $\left\{ \collection{B}_x:x\in X\right\}$ be a neighborhood base for $X$, let $\lambda \mapsto x_{\lambda}\in X$ be a net, and let $x_{\infty}\in X$.  Show that $\lambda \mapsto x_{\infty}$ converges to $x_{\infty}$ iff $\lambda \mapsto x_{\lambda}$ is eventually contained in every element of $\collection{B}_{x_{\infty}}$.
\end{exr}
\begin{exr}{}{exr4.2.41}
Let $X$ be a topological space, let $\collection{S}$ generate the topology of $X$, let $\lambda \mapsto x_\lambda \in X$ be a net, and let $x_\infty \in X$.  Show that $\lambda \mapsto x_\lambda$ converges to $x_\infty$ iff $\lambda \mapsto x_\lambda$ is eventually contained in every element of $\collection{S}$ which contains $x_\infty$.
\begin{rmk}
In other words, for the purposes of convergence, it suffices to look only at a generating collection of the topology.
\end{rmk}
\end{exr}

There is another characterization of quasicompactness that we did not introduce in the previous chapter because it involves the concept of generating collections, something that we hadn't defined in that context.
\begin{thm}{Alexander Subbase Theorem}{AlexanderSubbaseTheorem}
Let $X$ be a topological space and let $\collection{S}$ be a generating collection for the topology on $X$.  Then, $X$ is quasicompact iff every cover by elements of $\collection{S}$ has a finite subcover.
\begin{rmk}
This is of course just the defining property of quasicompactness, the only change being that we need only check covers whose elements come from the generating collection $\collection{S}$.
\end{rmk}
\begin{rmk}
In particular, if $\collection{S}$ does not cover $X$, then $X$ is quasicompact.
\end{rmk}
\begin{rmk}
The term \term{subbase}\index{Subbase} is sometimes used for generating collections which cover the space.  In fact, people usually make the requirement that the generating collection cover the space, but there is no need.
\end{rmk}
\begin{proof}\footnote{Proof adapted from \cite[pg.~139]{Kelley}.}
$(\Rightarrow )$ Of course, if $X$ is quasicompact, then \emph{every} open cover has a finite subcover, and so certainly open covers that come from $\collection{S}$ will have finite subcovers.

\blankline
\noindent
$(\Leftarrow )$ 
\Step{Make hypotheses}
Suppose that every cover by elements of $\collection{S}$ has a finite subcover.  To show that $X$ is quasicompact, we prove the contrapositive of the defining condition of quasicompactness.  That is, we show that every collection of open sets that has the property that no finite subset covers $X$, also does not cover $X$.  So, let $\collection{U}$ be a collection of open sets that has the property that no finite subset covers $X$.  We show that $\collection{U}$ itself does not cover $X$.

\Step{Enlarge $\collection{U}$ to a maximal collection}
Let $\tilde{\collection{U}}$ be the collection of all collections of open sets which (i)~contain $\collection{U}$ (ii)~also have the property that no finite subset covers $X$.  This is a set that is partially-ordered by inclusion, and so we intend to apply Zorn's Lemma (\cref{ZornsLemma}) to extract a maximal such element.  So, let $\tilde{\collection{W}}$ be a well-ordered subset of $\tilde{\collection{U}}$ and define
\begin{equation}
\collection{W}_0\coloneqq \bigcup _{\collection{W}\in \tilde{\collection{W}}}\collection{W}.
\end{equation}
Certainly $\collection{W}_0$ is a collection of open sets, a collection which contains $\collection{U}$.  In order to be an upper-bound for $\tilde{\collection{W}}$, however, we need to check that no finite subset covers $X$.  So, let $W_1,\ldots ,W_m\in \collection{W}_0$.  Then, each $W_k\in \collection{W}_k$ for some $\collection{W}_k\in \tilde{\collection{W}}$.  Because $\tilde{\collection{W}}$ is in particular totally-ordered, one of $\collection{W}_1,\ldots ,\collection{W}_m$ must contain all the others, and in particular, all the $W_k$s are contained in a single $\collection{W}_k$.  It follows that $\{ W_1,\ldots ,W_m\}$ cannot cover $X$.

Therefore, by Zorn's Lemma, there is a maximal collection of open sets $\collection{U}_0$ that (i)~contains $\collection{U}$ and (ii)~has the property that no finite subset covers $X$.  To show that $\collection{U}$ does not
cover $X$, it suffices to show that $\collection{U}_0$ does not cover $X$.

\Step{Show that if an element of $\collection{U}_0$ contains an intersection of open sets, $\collection{U}_0$ must contain one of those sets}
Let $U\in \collection{U}_0$ and suppose that $U_1\cap \cdots \cap U_m\subseteq U$ for $U_1,\ldots ,U_m$ open.  We proceed by contradiction:  suppose that $U_k\notin \collection{U}_0$ for all $k$.  This means that, by maximality, $\collection{U}_0\cup \{ U_k\}$ must have the property that some finite subset covers $X$, and so, for each $U_k$, there are finitely many $U_k^1,\ldots ,U_k^{n_k}$ such that $X=U_k\cup U_k^1\cup \cdots \cup U_k^{n_k}$.  But then
\begin{equation}
\begin{split}
U\cup \bigcup _{k=1}^m\bigcup _{l=1}^{n_k}U_k^l & \supseteq \left( \bigcap _{k=1}^mU_k\right) \cup \left( \bigcup _{k=1}^m\bigcup _{l=1}^{n_k}U_k^l\right) \\
& \supseteq \bigcap _{k=1}^m\left( U_k\cup U_k^1\cup \cdots \cup U_k^{n_k}\right) =X,
\end{split}
\end{equation}
so that
\begin{equation}
U\cup \bigcup _{k=1}^m\bigcup _{l=1}^{n_k}U_k^l=X,
\end{equation}
a contradiction of that fact that $\collection{U}_0$ contains no finite subset which covers $X$.  Therefore, some $U_k\in X$.

\Step{Deduce that $\collection{U}_0$ does not cover $X$}
Define
\begin{equation}
\topology{V}_0\coloneqq \collection{U}_0\cap \collection{S}.
\end{equation}
First of all, as no finite subset of $\collection{U}_0$ covers $X$, in particular, $X\notin \collection{U}_0$, so that
\begin{equation}
\topology{V}_0=\collection{U}_0\cap (\collection{S}\cup \{ X\} ).
\end{equation}
As every element of $\topology{V}_0$ comes from $\collection{U}_0$, it follows that no finite subset of $\topology{V}_0$ covers $X$.  On the other hand, every element of $\topology{V}_0$ comes from $\collection{S}$, so that, by hypothesis, it in turn follows that $\topology{V}_0$ does not cover $X$.\footnote{If it did cover $X$, then there would have to be a finite subset which still covers $X$.}  Thus, we will be done if we can show that
\begin{equation}
\bigcup _{V\in \topology{V}_0}V=\bigcup _{U\in \collection{U}_0}U.
\end{equation}
The $\subseteq$ inclusion is obvious because $\topology{V}_0\subseteq \collection{U}_0$.  For the other inclusion, let $x\in \bigcup _{U\in \collection{U}_0}U$.  Then, $x\in U$ for some $U\in \collection{U}_0$.  Because $\collection{S}$ is a generating collection, the collection of all finite intersections of elements of $\collection{S}\cup \{ X\}$ is a base (\cref{GeneratingCollection}), and so there are $U_1,\ldots ,U_m\in \collection{S}$ such that $x\in U_1\cap \cdots \cap U_m\subseteq U$.  But then, by the previous step, $U_k\in \collection{U}_0$ for some $U_k$.  Of course, $U_k$ came from $\collection{S}\cup \{ X\}$, and so in fact $U_k\in \collection{U}_0\cap (\collection{S}\cup \{ X\} )=\topology{V}_0$, so that indeed $x\in \bigcup _{V\in \topology{V}_0}V$.
\end{proof}
\end{thm}

\section{Filter bases}\label{sct4.4}

This section is a bit of an aside and can probably be skipped without too much trouble.  Our motivation for covering filter bases is (i)~it will help us demonstrate our definition of subnet is the `correct' one, as opposed to, for example, the one currently\footnote{5 July 2015} given on Wikipedia, and (ii)~it is something that is important enough that you should probably at least be aware of its existence and will almost certainly eventually encounter it if you decide to become a mathematician.

Filter bases are actually an alternative to nets.  In principle, one could do the entirety of topology never speaking of nets and instead using only filter bases.  This would be one motivation for introducing them (though we have decided to primarily stick to nets).
\begin{dfn}{Filter base}{FilterBase}
Let $\coord{X,\leq}$ be a partially-ordered set and let $F\subset X$ be nonempty and not containing a minimum.\footnote{As mentioned in the remark, the primary case of interest is when the partially-ordered set is the power set $2^X$ of some other set $X$, in which case the minimum is the $\emptyset$.  Thus, the condition of disallowing the minimum (if one exists) is put in place so as to disallow $\emptyset \in \filter{F}$.}  Then, $F$ is a \term{filter base}\index{Filter base} of $X$ iff for $x_1,x_2\in F$, there is some $x_3\in F$ such that $x_3\leq x_1,x_2$.\footnote{This property is called being \term{downward-directed}\index{Downward-directed}.}
\begin{rmk}
This is the definition of an abstract filter base in any partially-ordered set.  For us, we will essentially only be interested in filter bases of the partially-ordered set $\coord{2^X,\subseteq}$ for $X$ a topological space:  if $F$ is a filter base of $\coord{2^X,\subseteq}$, $X$ a topological space, then we shall say that $F$ is a \term{filter base in $X$}.  (Note that the elements of a filter base do not have to be open sets.)
\end{rmk}
\begin{rmk}
This condition is exactly analogous to the condition for directed sets $\Lambda$, that for $x_1,x_2\in \Lambda$, there is some $x_3\in \Lambda$ with $x_3\geq x_1,x_2$.  In fact, you might even say that the definition is what it is because \emph{a filter base ordered by reverse-inclusion is a directed set}.
\end{rmk}
\end{dfn}
The relation between nets and filter bases is given by the following.
\begin{dfn}{Derived filter base}{DerivedFilterBase}
Let $X$ be a topological space, let $\lambda \mapsto x_\lambda \in X$ be a net, and define
\begin{equation}
\filter{F}_{\lambda \mapsto x_\lambda}\coloneqq \left\{ F\subseteq X:F\text{ eventually contains }\lambda \mapsto x_\lambda \right\} .
\end{equation}\index[notation]{$\filter{F}_{\lambda \mapsto x_\lambda}$}
\begin{exr}[breakable=false]{}{}
Show that $\filter{F}$ is a filter base.
\end{exr}
$\filter{F}_{\lambda \mapsto x_\lambda}$ is the \term{derived filter}\index{Derived filter} of $\lambda \mapsto x_\lambda$.
\begin{rmk}
Given a net, we just defined a canonically associated filter.  Of course, there will be many nets which give us this filter.  For example, I can change a single term of a net without affecting its derived filter.\footnote{In general at least.  In stupid cases, of course, e.g.~if the domain of the net is a single point, then this will change the derived filter.}
\end{rmk}
\begin{rmk}
Because of this, one might argue that filter bases are more fundamental than nets.  Nets somehow contain extra information that is completely irrelevant to topology.  An example of this is how one can always throw away finitely many terms of a sequence without affecting anything of importance.  Because the derived filter base of a net only contains information about what \emph{eventually} happens with the net, this extraneous information is lost when passing from the net to its derived filter.
\end{rmk}
\begin{rmk}
I personally find nets much more intuitive than filter bases (probably because they are a much more straightforward generalization of sequences than filter bases are), and thinking of how filter bases come from nets helps me understand some of the intuition of filter bases themselves.
\end{rmk}
\end{dfn}

At the bare minimum, in order for filter bases and nets to be effectively equivalent for the purposes of topology, we must at least (i)~define convergence of filter bases and (ii)~show that convergence of a net agrees with convergence of its derived filter.
\begin{dfn}{Limit (of a filter base)}{}
Let $X$ be a topological space, let $\filter{F}$ be a filter base in $X$, and let $x_\infty \in X$.  Then, $x_\infty$ is a \term{limit}\index{Limit (of a filter)} of $\filter{F}$ iff for every open neighborhood $U$ of $x_\infty$ there is some $F\in \filter{F}$ such that $F\subseteq U$.  If a filter base has a limit, then we say that it \term{converges}\index{Convergence (of a filter base)}.
\end{dfn}
\begin{prp}{}{prp4.3.6}
Let $X$ be a topological space, let $\lambda \mapsto x_\lambda \in X$ be a net, and let $x_\infty \in X$.  Then, $\lambda \mapsto x_\lambda$ converges to $x_\infty$ iff $\filter{F}_{\lambda \mapsto x_\lambda}$ converges to $x_\infty$.
\begin{proof}
$(\Rightarrow )$ Suppose that $\lambda \mapsto x_\lambda$ converges to $x_\infty$.  Let $U$ be an open neighborhood of $x_\infty$.  Then, $\lambda \mapsto x_\lambda$ is eventually contained in $U$.  Therefore, $\filter{F}_{\lambda \mapsto x_{\lambda}}\ni U\subseteq U$, and so $\filter{F}_{\lambda \mapsto x_\lambda}$ converges to $x_\infty$.

\blankline
\noindent
$(\Leftarrow )$ Suppose that $\filter{F}_{\lambda \mapsto x_\lambda}$ converges to $x_\infty$.  Let $U$ be an open neighborhood of $x_\infty$.  Then, there is some $F\in \filter{F}_{\lambda \mapsto x_\lambda}$ such that $F\subseteq U$.  By definition of derived filter bases, it follows that $\lambda \mapsto x_\lambda$ is eventually contained in $F$, and hence eventually contained in $U$.  Therefore, $\lambda \mapsto x_\lambda$ converges to $x_\infty$.
\end{proof}
\end{prp}

Now we turn to filterings and subnets.  A filtering is to a filter base as a subnet is to a net.  Recall that one of our motivations for talking about filter bases at all was to argue that our definition of subnet was the `correct' one.  One nice thing about filter bases is that there is no confusion about what the definition of a filtering should be.\footnote{Though evidently there is some confusion as to what they should be called---see the remark in the definition below.}  We will then show that our definition of subnet coincides with the notion of a filtering.
\begin{dfn}{Filtering}{Filtering}
Let $\filter{F}$ be a filter base on $X$.  Then, a \term{filtering}\index{Filtering} of $\filter{F}$ is a filter base $\filter{G}$ that has the property that, for every $F\in \filter{F}$, there is some $G\in \filter{G}$ such that $G\subseteq F$.
\begin{rmk}
Note that in some places\footnote{\emph{*cough*}---Wikipedia---\emph{*cough*}} this is called a \emph{refinement}.  This is poor terminology because it disagrees with the usual definition of refinements of covers---see \cref{dfnC.1}.  For comparison, we reproduce that definition here.
\begin{textequation}
$\filter{G}$ is a \term{refinement} of $\filter{F}$ iff for every $G\in \filter{G}$ there is some $F\in \filter{F}$ such that $G\subseteq F$.
\end{textequation}
\end{rmk}
\end{dfn}
\begin{exr}{}{exr4.4.8}
Let $\filter{F}$ and $\filter{G}$ be filter bases.  Show that if $\filter{F}\subseteq \filter{G}$, then $\filter{G}$ is a filtering of $\filter{F}$.
\end{exr}
In fact, for derived filter bases, every filtering is of this form.
\begin{prp}{}{prp4.4.8}
Let $X$ be a topological space, let $\Lambda \ni \lambda \mapsto x_\lambda \in X$ be a net, and let $\Lambda '\ni \mu \mapsto \lambda _\mu \in \Lambda$.  Then the following are equivalent.
\begin{enumerate}
\item \label{enm4.4.8.i}$\mu \mapsto x_{\lambda _\mu}$ is a subnet of $\lambda \mapsto x_\lambda$.
\item \label{enm4.4.8.ii}$\filter{F}_{\lambda \mapsto x_\lambda}\subseteq \filter{F}_{\mu \mapsto x_{\lambda _\mu}}$.
\item \label{enm4.4.8.iii}$\filter{F}_{\mu \mapsto x_{\lambda _\mu}}$ is a filtering of $\filter{F}_{\lambda \mapsto x_\lambda}$.
\end{enumerate}
\begin{rmk}
In particular, as the definitions of subnet given in \cite{Kelley} (\cref{prp3.3.92}) and on Wikipedia\footnote{As of 16 July 2015} (\cref{prp3.3.93}) are \emph{not} equivalent (\cref{exm3.3.93,exr3.3.94}) to our definition of subnet (\cref{Subnet}), they are in turn not equivalent to filterings of filter bases!
\end{rmk}
\begin{rmk}
Of course, in general there are filterings not of this form (\cref{exr3.3.11}), but for \emph{derived} filter bases, filtering is equivalent to containing (as sets).
\end{rmk}
\begin{proof}
$(\cref{enm4.4.8.i}\Rightarrow \cref{enm4.4.8.ii})$ Suppose that $\mu \mapsto x_{\lambda _\mu}$ is a subnet of $\lambda \mapsto x_\lambda$.  Let $F\in \filter{F}_{\lambda \mapsto x_\lambda}$.  Then, $\lambda \mapsto x_\lambda$ is eventually in $F$, and so\footnote{See the definition of subnet, \cref{Subnet}.} $\mu \mapsto x_{\lambda _\mu}$ is eventually in $F$, and so $F\in \filter{F}_{\mu \mapsto x_{\lambda _\mu}}$.

\blankline
\noindent
$(\cref{enm4.4.8.ii}\Rightarrow \cref{enm4.4.8.iii})$ \cref{exr4.4.8}

\blankline
\noindent
$(\cref{enm4.4.8.iii}\Rightarrow \cref{enm4.4.8.i})$
Suppose that $\filter{F}_{\mu \mapsto x_{\lambda _\mu}}$ is a filtering of $\filter{F}_{\lambda \mapsto x_\lambda}$.  Let $F\subseteq X$ be such that $F$ eventually contains $\lambda \mapsto x_\lambda$.  Then, $F\in \filter{F}_{\lambda \mapsto x_\lambda}$, and so there is some $F'\in \filter{F}_{\mu \mapsto x_{\lambda _\mu}}$ such that $F'\subseteq F$.  As $\mu \mapsto x_{\lambda _\mu}$ is eventually contained in $F'$, it is eventually contained in $F$, and so\footnote{Once again, by the definition of subnets, \cref{Subnet}.} $\mu \mapsto x_{\lambda _\mu}$ is a subnet of $\lambda \mapsto x_\lambda$.
\end{proof}
\end{prp}
\begin{exr}{}{exr3.3.11}
Find an example of a filter base $\filter{F}$ and a filtering $\filter{G}$ of $\filter{F}$, but for which $\filter{G}\not \subseteq \filter{F}$.
\end{exr}

In the next section, we present several new ways of defining topologies.  One of these ways will be by defining what it means for filters to converge, and so present here as theorems the results that will be used as axioms.
\begin{prp}{Kelley's Filter Convergence Axioms}{KelleysFilterConvergenceAxioms}\index{Kelley's Filter Convergence Axioms}
Let $X$ be a topological space.  Then,
\begin{enumerate}
\item \label{enmKelleysFilterConvergenceAxioms.i}$\filter{P}_x$ converges to $x$, where $\filter{P}_x\coloneqq \left\{ U\subseteq X:x\in U\right\}$;\footnote{``$\filter{P}$'' is for \emph{principal}, the etymology being from the use of the word ``principal'' in the context of ideals in ring theory (to the best of my knowledge anyways).}
\item \label{enmKelleysFilterConvergenceAxioms.iix}Let $\filter{G}$ be a filtering of the filter base $\filter{F}$.  Then, if $\filter{F}$ converges to $x$, then $\filter{G}$ converges to $x$.
\item \label{enmKelleysFilterConvergenceAxioms.ii}Let $\filter{F}$ be a filter.  Then, if every filtering $\filter{G}\supseteq \filter{F}$ has in turn a filtering $\filter{H}$ such that $\filter{H}$ converges to $x$, then $\filter{F}$ converges to $x$; and
\item \label{enmKelleysFilterConvergenceAxioms.iii}for all directed sets $I$ and filters $\filter{F}^i$ converging to $x^i\in X$, for $i\in I$, if $\filter{F}_{i\mapsto x^i}$ converges to $x^\infty$, then
\begin{equation*}
\begin{split}
\filter{F}_\infty & \coloneqq \left\{ U\subseteq X:\text{there exists }i_U\in I\text{ such that,}\right. \\
& \qquad \left. \text{whenever }i\geq i_U\text{, }U\supseteq F^i\text{ for some }F^i\in \filter{F}^i.\right\}
\end{split}
\end{equation*}
also converges to $x^\infty$.
\end{enumerate}
\begin{rmk}
In other words, $\filter{F}_{\infty}$ consists of those sets that eventually contain some element of $\filter{F}^i$.
\end{rmk}
\begin{rmk}
Note that these are completely analogous to Kelley's (Net) Convergence Axioms (\cref{KelleysConvergenceAxioms}), with one itsy-bitsy exception.  You'll recall that in the third axiom for nets, there is mention of \emph{cofinal} subnets.  I don't believe there is any concept for filterings that is analogous to cofinal subnets, and so there is no ``strict'' here.  That said, you'll notice I have written ``$\filter{G}\supseteq \filter{F}$''---this is not a big deal, but I write this instead of just ``$\filter{F}$ is a filtering of $\filter{G}$'' to make the axioms just a \emph{little} bit more parallel, even though the condition $\filter{G}\supseteq \filter{F}$ is \emph{not} the analogue of being strict.\footnote{Indeed, for derived filter bases, this is equivalent to being an (ordinary) subnet---see \cref{prp4.4.8}.}
\end{rmk}
\begin{rmk}
Perhaps this could be made into an argument that somehow nets are more fundamental---to define a topology by convergence of filters, you have to make use of nets (in this case, $i\mapsto x^i$).\footnote{Or rather, I am not aware of a way to state the axioms without at least implicitly using nets.}
\end{rmk}
\begin{proof}
\cref{enmKelleysFilterConvergenceAxioms.i} We leave this as an exercise.
\begin{exr}[breakable=false]{}{}
Show \cref{enmKelleysFilterConvergenceAxioms.i}.
\end{exr}

\blankline
\noindent 
\cref{enmKelleysFilterConvergenceAxioms.iix} We leave this as an exercise.
\begin{exr}[breakable=false]{}{}
Show \cref{enmKelleysFilterConvergenceAxioms.iix}$(\Rightarrow )$.
\end{exr}

\blankline
\noindent
\cref{enmKelleysFilterConvergenceAxioms.ii} Suppose that for every filtering $\filter{G}\supseteq \filter{F}$ there is a filtering $\filter{H}$ of $\filter{G}$ that converges to $x$.  We proceed by contradiction:  suppose that $\filter{F}$ doesn't converge to $x$.  Then, there is some open neighborhood $x\in U\subseteq X$ such that $F\not\subseteq U$ for all $F\in \filter{F}$.  For each $F\in \filter{F}$, let $x_F\in U^{\comp}\cap F$, so that $\coord{\filter{F},\supseteq}\ni F\mapsto x_F$ is a net.  We claim that $\filter{F}_{F\mapsto x_F}$ is a filtering of $\filter{F}$.

So, let $F_0\in \filter{F}$.  We wish to show that that in fact $F_0$ itself eventually contains $F\mapsto x_F$.  So, suppose that $F\geq F_0$, that is, $F\subseteq F_0$.  Then, $x_F\in F\subseteq F_0$.  Thus, whenever $F\geq F_0$, it follows that $x_F\in F_0$, so that indeed $F_0$ eventually contains $F\mapsto x_F$.

We wish to show that $\filter{F}_{F\mapsto x_F}$ itself has no filtering which converges to $x$.  This will be a contradiction, thereby proving the result.  To show this itself, we again proceed by contradiction:  suppose that there is some filtering $\filter{H}$ of $\filter{F}_{F\mapsto x_F}$ that converges to $x$.  As $\filter{H}$ converges to $x$, there must be some $H\in \filter{H}$ such that $x\in H\subseteq U$.  On the other hand, as $\filter{H}$ is a filtering of $\filter{F}_{F\mapsto x_F}$, there must be some $G\subseteq H$ which eventually contains $F\mapsto x_F$.  As $H\subseteq U$, of course $G\subseteq U$, which implies that at least one $x_F\in G\subseteq U$, a contradiction of the fact that $x_F\notin U$ (by construction).

\blankline
\noindent
\cref{enmKelleysFilterConvergenceAxioms.iii} Let $I$ be a directed set, for each $i\in I$ let $\filter{F}^i$ be a filter converging to $x^i\in X$, and suppose that $\filter{F}_{i\mapsto x^i}$ converges to $x^\infty$.  By \cref{prp4.3.6}, $i\mapsto x^i$ converges to $x^\infty$.  To show that $\filter{F}_\infty$ converges to $x^\infty$, it suffices to show that every open neighborhood of $x^\infty$ is an element of $\filter{F}_\infty$.  So, let $U$ be an open neighborhood of $x^\infty$.  Then, as $\filter{F}_{i\mapsto x^i}$ converges to $x^\infty$, it follows that there is some $V\in \filter{F}_{i\mapsto x^i}$ such that $V\subseteq U$.  As $i\mapsto x^i$ is eventually contained in $V$, there is some $i_0\in I$ such that, whenever $i\geq i_0$, it follows that $x^i\in V\subseteq U$.  Thus, for $i$ sufficiently large, $U$ is an open neighborhood of $x^i$.  But then, because $\filter{F}^i$ converges to $x^i$, it follows that there is some $F^i\in \filter{F}^i$ such that $F^i\subseteq U$, and so $U\in \filter{F}_{\infty}$ as desired.
\end{proof}
\end{prp}

\section{Equivalent definitions of topological spaces}

There is more than one way to define a topology.  By definition, the specification of a topology is just the specification of what sets are open.  Sometimes, however, what the open sets should be is not nearly as obvious as, for example, what convergence of nets should mean.  In this section, we present a couple of other ways you may define a topological space.  Which one is most useful, of course, will depend on the particular problem at hand.

To summarize, we have already shown that we may define a topology in the following ways.  We can define a topology
\begin{enumerate}
\item by specifying the open sets (\cref{TopologicalSpace});
\item by specifying the closed sets (\cref{exr4.1.2});
\item by specifying a base for the topology (\cref{prp4.1.5});
\item by specifying a neighborhood base for the topology (\cref{prp4.1.8}); or
\item by specifying a generating collection for the topology (\cref{GeneratingCollection}).
\end{enumerate}
Of course, there are many other ways to specify a topology as well, and it is these methods that are the topic of this section.

\subsection{Definition by specification of closures or interiors}

We have mentioned the Kuratowski Closure (Interior) Axioms as well as Kelley's (Filter) Convergence Axioms.  The former allow us to define a topology in the following way.
\begin{thm}{Kuratowski's Closure Theorem}{KuratowskisClosureTheorem}\index{Kuratowski's Closure Theorem}
Let $X$ be a set and let $\mrm{C}:2^X\rightarrow 2^X$ be a function on the power-set of $X$.  Then, if
\begin{enumerate}
\item \label{enmKuratowskiClosureTheorem.i}$\mrm{C}(\emptyset )=\emptyset$;
\item \label{enmKuratowskiClosureTheorem.ii}$S\subseteq \mrm{C}(S)$;
\item \label{enmKuratowskiClosureTheorem.iii}$\mrm{C}(S)=\mrm{C}\left( \mrm{C}(S)\right)$; and
\item \label{enmKuratowskiClosureTheorem.iv}$\mrm{C}(S\cup T)=\mrm{C}(S)\cup \mrm{C}(T)$,
\end{enumerate}
then there exists a unique topology on $X$ such that $\Cls (S)=\mrm{C}(S)$.
\begin{proof}
\Step{Make hypotheses}
Suppose that (i)~$\mrm{C}(\emptyset )=\emptyset$, (ii)~$S\subseteq \mrm{C}(S)$, (iii)~$\mrm{C}(S)=\mrm{C}\left( \mrm{C}(S)\right)$, and (iv)~$\mrm{C}(S\cup T)=\mrm{C}(S)\cup \mrm{C}(T)$.

\Step{Show that $S\subseteq T$ implies $\mrm{C}(S)\subseteq \mrm{C}(T)$}[stpKuratowskiClosureTheorem.2]
Suppose that $S\subseteq T$, so that $T=S\cup (T\setminus S)$, and hence $\mrm{C}(T)=\mrm{C}(S)\cup \mrm{C}(T\setminus S)$, and  so $\mrm{C}(S)\subseteq \mrm{C}(T)$.

\Step{Define what should be the closed sets}
The idea of the proof is that the closed sets should be precisely the sets that are equal to their closure (\cref{prp4.2.21}).  We thus make the definition
\begin{equation}
\collection{C}\coloneqq \left\{ C\in 2^X:C=\mrm{C}(C)\right\} .
\end{equation}

\Step{Verify that this defines a topology}
By \cref{enmKuratowskiClosureTheorem.i}, the empty-set is closed (by which we mean it is an element of $\collection{C}$).  By \cref{enmKuratowskiClosureTheorem.ii}, we have
\begin{equation}
X\subseteq \mrm{C}(X)\subseteq X,
\end{equation}
so that $X$ is closed as well.  Let $\collection{D}\subseteq \collection{C}$ and define
\begin{equation}
B\coloneqq \bigcap _{C\in \collection{D}}C
\end{equation}
Then, of course, $B\subseteq C$ for all $C\in \collection{D}$, and so by \cref{stpKuratowskiClosureTheorem.2}, we have that $\mrm{C}(B)\subseteq \mrm{C}(C)$ for all $C\in \collection{D}$, and so
\begin{equation}
\mrm{C}(B)\subseteq \bigcap _{C\in \collection{D}}\mrm{C}(C)=\bigcap _{C\in \collection{D}}C\eqqcolon B.
\end{equation}
As $B\subseteq \mrm{C}(B)$ by \cref{enmKuratowskiClosureTheorem.ii}, we have that $B=\mrm{C}(B)$, so that $\collection{C}$ is closed under arbitrary intersections.  We now check that it is closed under finite unions, so let $C,D\in \collection{C}$.  Then,
\begin{equation}
\mrm{C}(C\cup D)=\mrm{C}(C)\cup \mrm{C}(D)=C\cup  D,
\end{equation}
and so $C\cup D\in \collection{C}$.  Thus, $\collection{C}$ is closed under finite unions, and hence defines a topology by \cref{exr4.1.2}.\footnote{This is the exercise that tells us we can define a topology by specifying the closed sets.}

\Step{Show that $\Cls (S)=\mrm{C}(S)$}
Let $S\subseteq  X$.  By \cref{enmKuratowskiClosureTheorem.iii}, $\mrm{C}(S)$ is closed, and by \cref{enmKuratowskiClosureTheorem.ii}, it contains $S$.  Therefore, $\Cls (S)\subseteq \mrm{C}(S)$.  It follows that $\mrm{C}\left( \Cls (S)\right) \subseteq \mrm{C}(S)$.  On the other hand, because $S\subseteq \Cls (S)$, it follows that $\mrm{C}(S)\subseteq \mrm{C}\left( \Cls (S)\right) =\footnote{Because $\Cls (S)\in \collection{C}$, as it must be because $\Cls (S)$ is closed.}\Cls (S)$, and so indeed $\Cls (S)=\mrm{C}(S)$.

\Step{Demonstrate uniqueness}
If another topology $\topology{V}$ satisfies this property, that is, $\Cls _{\topology{V}}(S)=\mrm{C}(S)$, then we have that $\Cls _{\topology{V}}(S)=\Cls (S)$ (no subscript indicates the closure in the topology defined by $\collection{C}$).  It follows that a set $S$ is closed with respect to $\topology{V}$ iff it is equal to $\Cls _{\topology{V}}(S)$ iff it is equal to $\Cls (S)$ iff it is closed in the topology defined by $\collection{C}$.  Thus, the two topologies have the same closed sets, and hence are the same.
\end{proof}
\end{thm}
Similarly, we have a `dual' interior theorem.
\begin{thm}{Kuratowski's Interior Theorem}{KuratowskisInteriorTheorem}\index{Kuratowski's Interior Theorem}
Let $X$ be a set and let $\mrm{I}:2^X\rightarrow 2^X$ be a function on the power-set of $X$.  Then, if
\begin{enumerate}
\item $\mrm{I}(X)=X$;
\item $\mrm{I}(S)\subseteq S$;
\item $\mrm{I}(S)=\mrm{I}\left( \mrm{I}(S)\right)$; and
\item $\mrm{I}(S\cap T)=\mrm{I}(S)\cap \mrm{I}(T)$,
\end{enumerate}
then there exists a unique topology on $X$ such that $\Int (S)=\mrm{I}(S)$.
\begin{rmk}
We omit the proof as it is completely `dual' to the corresponding closure proof.
\end{rmk}
\end{thm}

\subsection{Definition by specification of convergence}

The previous two results showed that we can define a topology by defining what the closure or interior of every set should be.  The next result says that we can define a topology by defining what it means for nets to converge.
\begin{thm}{Kelley's Convergence Theorem}{KelleysConvergenceTheorem}\index{Kelley's Convergence Theorem}
Let $X$ be a set, denote by $\collection{N}$ the collection of all nets in $X$, and let $\to$ be a relation on $\collection{N}\times X$.  Then, if
\begin{enumerate}
\item \label{enmKelleysConvergenceTheorem.i}$(\lambda \mapsto x_\infty)\to x_\infty$;
\item \label{enmKelleysConvergenceTheorem.iix}if $(\lambda \mapsto x_{\lambda})\to x_{\infty}$, then $(\mu \mapsto x_{\lambda _{\mu}})\to x_{\infty}$ for every subnet $\mu \mapsto x_{\lambda _{\mu}}$ of $\lambda \mapsto x_{\lambda}$;
\item \label{enmKelleysConvergenceTheorem.ii}if every cofinal subnet $\mu \mapsto x_{\lambda _\mu}$ has in turn a subnet $\nu \mapsto x_{\lambda _{\mu _\nu}}$ such that $(\nu \mapsto x_{\lambda _{\mu _\nu}})\to x_\infty$, then $(\lambda \mapsto x_{\lambda})\to x_{\infty}$; and
\item \label{enmKelleysConvergenceTheorem.iii} for all directed sets $I$ and nets $x^i:\Lambda ^i\rightarrow X$, if $x^i\to (x^i)_\infty$ and $(i\mapsto (x^i)_\infty )\to (x^\infty )_\infty$, then $\left( I\times \prod _{i\in I}\Lambda ^i\ni \coord{i,\lambda}\mapsto (x^i)_{\lambda ^i}\right) \to (x^\infty )_\infty$,
\end{enumerate}
then there is a unique topology on $X$ such that $\lambda \mapsto x_\lambda$ converges to $x_\infty$ iff $(\lambda \mapsto x_\lambda )\to x_\infty$.
\begin{rmk}
People sometimes attempt to define a topology by defining what it means for \emph{sequences} to converge.  This is nonsensical.  For example, \cref{enmKelleysConvergenceTheorem.iii} doesn't even make sense in this context.  Moreover, because of examples like \cref{exm4.2.25} (the cocountable topology and discrete topology on $\R$ have the same notion of sequential convergence), trying to rectify this in general is hopeless.  My best guess is that this is because people tend to shy away from the usage of nets for some reason, and so they tend to define topologies using convergence of sequences.  In any case, I don't recall ever seeing a case where such a definition was given that \emph{didn't} make sense after you replaced the word ``sequence'' with the word ``net'.  That is to say, even though they are technically wrong, there is usually a very easy fix.
\end{rmk}
\begin{rmk}
We will see later that a topological space is what is called ``$T_2$'' iff limits are unique (\cref{prp4.5.37}).  Thus, defining a topology in this way is often a super easy way to guarantee at least this separation axiom.
\end{rmk}
\begin{proof}
\Step{Make hypotheses}
Suppose that \cref{enmKelleysConvergenceTheorem.i} $(\lambda \mapsto x_\infty)\to x_\infty$; \cref{enmKelleysConvergenceTheorem.iix} if $(\lambda \mapsto x_{\lambda})\to x_{\infty}$, then $(\mu \mapsto x_{\lambda _{\mu}})\to x_{\infty}$ for every subnet $\mu \mapsto x_{\lambda _{\mu}}$ of $\lambda \mapsto x_{\lambda}$; \cref{enmKelleysConvergenceTheorem.ii} if every cofinal subnet $\mu \mapsto x_{\lambda _\mu}$ has in turn a subnet $\nu \mapsto x_{\lambda _{\mu _\nu}}$ such that $(\nu \mapsto x_{\lambda _{\mu _\nu}})\to x_\infty$, then $(\lambda \mapsto x_{\lambda})\to x_{\infty}$; and \cref{enmKelleysConvergenceTheorem.iii} for all directed sets $I$ and nets $x^i:\Lambda ^i\rightarrow X$, if $x^i\to (x^i)_\infty$ and $(i\mapsto (x^i)_\infty )\to (x^\infty )_\infty$, then $\left( I\times \prod _{i\in I}\Lambda ^i\ni \coord{i,\lambda}\mapsto (x^i)_{\lambda ^i}\right) \to (x^\infty )_\infty$.

\Step{Define the notion of an \emph{adherent point}}
For a subset $S\subseteq X$ and $x\in X$, we say that $x$ is an \emph{adherent point}\footnote{This of course will turn out to agree with the notion of adherent point for the topology.} of $S$ iff there is some net $\lambda \mapsto x_\lambda \in S$ such that $(\lambda \mapsto x_\lambda )\to x$.

\Step{Define what should be the closure}
For a subset $S\subseteq X$, we define $\mrm{C}(S)$ to be the set of adherent points of $S$.

\Step{Show that $\mrm{C}$ satisfies Kuratowski's Closure Axioms}
The empty-set has no adherent points, and so trivially $\mrm{C}(\emptyset )=\emptyset$.

It follows from \cref{enmKelleysConvergenceTheorem.i} that $S\subseteq \mrm{C}(S)$.

We wish to show that $\mrm{C}\left( \mrm{C}(S)\right) \subseteq \mrm{C}(S)$.  So, let $(x^\infty )_\infty \in \mrm{C}\left( \mrm{C}(S)\right)$.  Then, there is a net $I\ni i\mapsto x^i\in \mrm{C}(S)$ such that $(i\mapsto x^i )\to (x^\infty )_\infty$.  As each $x^i\in \mrm{C}(S)$, there is a net $\Lambda ^i\ni \lambda ^i\mapsto (x^i)_{\lambda ^i}\in S$ such that $\left( \lambda ^i\mapsto (x^i)_{\lambda ^i}\right) \to (x^i)_\infty$.  By \cref{enmKelleysConvergenceTheorem.iii}, it then follows that $\left( I\times \prod _{i\in I}\Lambda ^i\ni \coord{i,\lambda}\mapsto (x^i)_{\lambda ^i}\right) \to (x^\infty )_\infty$.  As each $(x^i)_{\lambda ^i}\in S$, it follows that $(x^\infty )_\infty$ is an adherent point of $S$, that is, $(x^{\infty})_{\infty}\in \mrm{C}(S)$, and hence $\mrm{C}\left( \mrm{C}(S)\right) =\mrm{C}(S)$ (that $\mrm{C}(S)\subseteq \mrm{C}\left( \mrm{C}(S)\right)$ follows from the fact that $S\subseteq \mrm{C}(S)$).

We now show that $\mrm{C}(S\cup T)=\mrm{C}(S)\cup \mrm{C}(T)$.  We have that $S\subseteq \mrm{C}(S\cup T)$, and so $\mrm{C}(S)\subseteq \mrm{C}(S\cup T)$.  Similarly for $T$, and so we have $\mrm{C}(S)\cup \mrm{C}(T)\subseteq \mrm{C}(S\cup T)$.

We now show that $\mrm{C}(S\cup T)\subseteq \mrm{C}(S)\cup \mrm{C}(T)$.  So, let $x_\infty \in \mrm{C}(S\cup T)$, so that there is a net $\Lambda \ni \lambda \mapsto x_\lambda \in S\cup T$ such that $(\lambda \mapsto x_\lambda )\to x_\infty$.  Define
\begin{equation*}
\Lambda _S\coloneqq \left\{ \lambda \in \Lambda :x_\lambda \in S\right\} \text{ and }\Lambda _T\coloneqq \left\{ \lambda \in \Lambda :x_\lambda \in T\right\} .
\end{equation*}
As $\Lambda =\Lambda _S\cup \Lambda _T$, either $\Lambda _S$ or $\Lambda _T$ is cofinal in $\Lambda$.  Without loss of generality, suppose that $\Lambda _S$ is cofinal in $\Lambda$, so that $\restr{x}{\Lambda _S}$ is a cofinal subnet of $\lambda \mapsto x_\lambda$.  By hypothesis, this in turn must have a subnet $\mu \mapsto x_{\lambda _\mu}$, $\lambda _\mu \in \Lambda _S$, such that $(\mu \mapsto x_{\lambda _\mu})\to x_\infty$.  Thus, $x_\infty \in \mrm{C}(S)$, and so $\mrm{C}(S\cup T)\subseteq \mrm{C}(S)\cup \mrm{C}(T)$.

It follows by \namerefpcref{KuratowskisClosureTheorem} that there is a unique topology on $X$ such that $\Cls (S)=\mrm{C}(S)$.

\Step{Show that $\lambda \mapsto x_\lambda$ converges to $x_\infty$ iff $(\lambda \mapsto x_\lambda)\to x_\infty$}
Suppose that $\Lambda \ni \lambda \mapsto x_\lambda$ converges to $x_\infty$.  We proceed by contradiction:  suppose that it is not the case that $(\lambda \mapsto x_\lambda )\to x_\infty$.  Then, there must be some cofinal subnet $I\ni \mu \mapsto x_{\lambda _\mu}$ that has no subnet $\nu \mapsto x_{\lambda _{\mu _\nu}}$ such that $(\nu \mapsto x_{\lambda _{\mu _\nu}})\to x_\infty$.  To obtain a contradiction, we construct such a subnet.

For each $\mu _0$, define $S_{\mu _0}\coloneqq \{ x_{\lambda _\mu}:\mu \geq \mu _0\}$.  As $\lambda \mapsto x_\lambda$ converges to $x_\infty$, so does $\mu \mapsto x_{\lambda _\mu}$, and so $x_\infty \in \Cls (S_{\mu _0})$ for all $\mu _0$.  Hence, by the definition of our closure, for each $\mu \in I$ there is a net $\Lambda ^{\mu}\ni \nu ^{\mu}\mapsto x_{\lambda _{\mu _{\nu ^{\mu}}}}\in S_{\mu}$, so that $\mu _{\nu ^{\mu}}\geq \mu$, with $\left( \nu ^{\mu}\mapsto x_{\lambda _{\mu _{\nu ^{\mu}}}}\right) \to x_\infty$.\footnote{The superscript $\mu$ in $\nu ^{\mu}$ is just to help us keep track of which directed set $\nu ^{\mu}$ is contained in.}  From \cref{enmKelleysConvergenceTheorem.iii}, it follows that $\left( I\times \prod _{\mu \in I}\Lambda ^{\mu}\ni \coord{\mu ,\nu}\mapsto x_{\lambda _{\mu _{\nu ^{\mu}}}}\right) \to x_\infty$.  Thus, if this is in fact a subnet of $\mu \mapsto x_{\lambda _\mu}$, we will have our contradiction.  To show this, we apply \cref{prp3.3.92}.  So, let $\mu _0\in I$ be arbitrary.  We must find an index in $I\times \prod _{\mu \in I}\Lambda ^{\mu}$ that has the property that, whenever $\coord{\mu ,\nu}$ is at least as large as that index, it follows that $\mu _{\nu ^{\mu}}\geq \mu _0$.  However, we know that $\mu _{\nu ^{\mu}}\geq \mu$ for all $\mu$, and so we can take $\coord{\mu _0,\nu _0}$ as our index for any choice of $\nu _0$.

Now suppose that $(\Lambda \ni \lambda \mapsto x_\lambda )\to x_\infty$.  We proceed by contradiction:  suppose that $\lambda \mapsto x_\lambda$ does not converge to $x_\infty$.  Then, there is a cofinal subnet $\mu \mapsto x_{\lambda _\mu}$ which has no subnet converging to $x_\infty$.  In particular, $\mu \mapsto x_{\lambda _\mu}$ itself does not converge to $x_\infty$, and so there is an open neighborhood $U$ of $x_\infty$ which does not eventually contain $\mu \mapsto x_{\lambda _\mu}$.  It follows that $I\coloneqq \{ \lambda _\mu :x_{\lambda _\mu}\notin U\}$ is cofinal, so that $I\ni \mu \mapsto x_{\lambda _\mu}$ is a subnet contained in $U^{\comp}$.  On the other hand, we know that $(I\ni \mu \mapsto x_{\lambda _\mu})\to x_\infty$\footnote{By \cref{enmKelleysConvergenceTheorem.iix}, because it is a subnet of $\lambda \mapsto x_{\lambda}$ and $(\lambda \mapsto x_{\lambda})\to x_{\infty}$.}, so that $x_\infty \in \Cls (\{ x_{\lambda _\mu}:\mu \in I\})$, but this is a contradiction of the fact that $x_{\infty}$ has an open neighborhood ($U$) which contains no point of the set $\{ x_{\lambda _{\mu}}:\mu \in I\}$.

\Step{Demonstrate uniqueness}
Recall that sets are closed iff they contain all their limit points.  If we have two topologies with the same notion of convergence, then the set of limit points of a given set in each topology are the same, and consequently, a set is closed in one iff it is closed in the other.
\end{proof}
\end{thm}
We mentioned in a remark of \namerefpcref{KelleysConvergenceAxioms} that the second and third axioms are in fact equivalent to just a single axiom.  We now prove this.
\begin{prp}{}{prp3.4.22}
Let $X$ be a set, denote by $\collection{N}$ the collection of all nets in $X$, and let $\to$ be a relation on $\collection{N}\times X$.  Then, the following are equivalent.
\begin{enumerate}
\item \label{enm3.4.22.i}$(\lambda \mapsto x_\lambda )\to x_\infty$ iff every subnet $\mu \mapsto x_{\lambda _\mu}$ has in turn a subnet $\nu \mapsto x_{\lambda _{\mu _\nu}}$ such that $(\nu \mapsto x_{\lambda _{\mu _\nu}})\to x_\infty$.
\item \label{enm3.4.22.ii}
\begin{enumerate}
\item \label{enm3.4.22.ii.a}If $(\lambda \mapsto x_\lambda )\to x_\infty$ and $\mu \mapsto x_{\lambda _\mu}$ is a subnet of $\lambda \mapsto x_\lambda$, then $(\mu \mapsto x_{\lambda _\mu})\to x_\infty$; and
\item \label{enm3.4.22.ii.b}If every cofinal subnet $\mu \mapsto x_{\lambda _\mu}$ of $\lambda \mapsto x_{\lambda}$ has in turn a subnet $\nu \mapsto x_{\lambda _{\mu _\nu}}$ such that $(\nu \mapsto x_{\lambda _{\mu _\nu}})\to x_\infty$, then $(\lambda \mapsto x_\lambda )\to x_\infty$.
\end{enumerate}
\end{enumerate}
\begin{proof}
$(\Rightarrow )$ Suppose that $(\lambda \mapsto x_\lambda )\to x_\infty$ iff every subnet $\mu \mapsto x_{\lambda _\mu}$ has in turn a subnet $\nu \mapsto x_{\lambda _{\mu _\nu}}$ such that $(\nu \mapsto x_{\lambda _{\mu _\nu}})\to x_{\infty}$.

We first check \cref{enm3.4.22.ii.a}.  So, let $\lambda \mapsto x_\lambda$ be a net such that $(\lambda \mapsto x_\lambda )\to x_\infty$ and let $\mu \mapsto x_{\lambda _\mu}$ be a subnet.  To show that $(\mu \mapsto x_{\lambda _\mu})\to x_\infty$, we show that every subnet $\nu \mapsto x_{\lambda _{\mu _\nu}}$ has in turn a subnet $\xi \mapsto x_{\lambda _{\mu _{\nu _\xi}}}$ such that $(\xi \mapsto x_{\lambda _{\mu _{\nu _\xi}}})\to x_\infty$.  So, let $\nu \mapsto x_{\lambda _{\mu _\nu}}$ be a subnet of $\mu \mapsto x_{\lambda _\mu}$.  This itself is also a subnet of $\lambda \mapsto x_\lambda$, and so by hypothesis, it has a subnet $\xi \mapsto x_{\lambda _{\mu _{\nu _\xi}}}$ such that $(\xi \mapsto x_{\lambda _{\mu _{\nu _\xi}}})\to x_\infty$.

Now, suppose that every cofinal subnet $\mu \mapsto x_{\lambda _{\mu}}$ of $\lambda \mapsto x_{\lambda}$ has in turn a subnet $\nu \mapsto x_{\lambda _{\mu _{\nu}}}$ such that $(\nu \mapsto x_{\lambda _{\mu _{\nu}}})\to x_{\infty}$.  We wish to show that $(\lambda \mapsto x_{\lambda})\to x_{\infty}$.  To do that, we show that \emph{every} subnet (not just every cofinal subnet) $\mu \mapsto x_{\lambda _{\mu}}$ has in turn a subnet $\nu \mapsto x_{\lambda _{\mu _{\nu}}}$ such that $(\nu \mapsto x_{\lambda _{\mu _{\nu}}})\to x_{\infty}$.  So, let $M\ni \mu \mapsto x_{\lambda _{\mu}}$ be a subnet of $\Lambda \ni \lambda \mapsto x_{\lambda}$.  By \cref{prp3.2.7}, for every $\lambda \in \Lambda$, there is some $\mu _{\lambda}\in M$ such that
\begin{equation}
\left\{ x_{\lambda _{\mu}}:\mu \geq \mu _{\lambda}\right\} \subseteq \left\{ x_{\lambda '}:\lambda '\geq \lambda \right\} .
\end{equation}

Let $U\subseteq X$ eventually contain $\mu \mapsto x_{\lambda _{\mu}}$.  Then, there is some $\mu _U\in M$ such that, whenever $\mu \geq \mu _U$, it follows that $x_{\lambda _{\mu}}\in U$.  For $U\subseteq X$ that eventually contains $\mu \mapsto x_{\lambda _{\mu}}$ and $\lambda \in \Lambda$, let $\mu _{U,\lambda}\in M$ be at least as large as $\mu _U$ and $\mu _{\lambda}$.  In particular, $x_{\lambda _{\mu _{U,\lambda}}}\in \left\{ x_{\lambda '}:\lambda '\geq \lambda \right\}$, and so without loss of generality, assume that $\lambda _{\mu _{U,\lambda}}\geq \lambda$.  It follows that
\begin{equation*}
\left\{ \lambda _{\mu _{U,\lambda}}:U\subseteq X\text{ eventually contains }\mu \mapsto x_{\lambda _{\mu}},\ \lambda \in \Lambda \right\}
\end{equation*}
is cofinal in $\Lambda$, and hence defines a corresponding cofinal subnet.  By hypothesis, this has in turn a subnet $N\ni \nu \mapsto x_{\lambda _{\mu _{U_{\nu},\lambda _{\nu}}}}$ such that $(\nu \mapsto x_{\lambda _{\mu _{U_{\nu},\lambda _{\nu}}}})\to x_{\infty}$.

Now, for $\mu '\in M$ and $\nu '\in N$, let $\mu _{\mu ',\nu '}\in M$ be at least as large as both $\mu '$ and $\mu _{U_{\nu},\lambda _{\nu}}$.  It follows that
\begin{equation}
M\times N\ni \coord{\mu ',\nu '}\mapsto x_{\lambda _{\mu _{\mu ',\nu '}}}
\end{equation}
is a subnet of both $\mu \mapsto x_{\lambda _{\mu}}$ and $\nu \mapsto x_{\lambda _{\mu _{U_{\nu},\lambda _{\nu}}}}$.  The latter fact together with \cref{enm3.4.22.ii.b} imply that $(\coord{\mu ',\nu '}\mapsto x_{\lambda _{\mu _{\mu ',\nu '}}})\to x_{\infty}$.  Thus, we have constructed a subnet of $\mu \mapsto x_{\lambda _{\mu}}$ that is related by $x_{\infty}$ via $\to$, and hence, by hypothesis, $(\mu \mapsto x_{\lambda _{\mu}})\to x_{\infty}$, as desired.

\blankline
\noindent
$(\Leftarrow )$ Suppose that (a) if $(\lambda \mapsto x_\lambda )\to x_\infty$ and $\mu \mapsto x_{\lambda _\mu}$ is a subnet of $\lambda \mapsto x_\lambda$, then $(\mu \mapsto x_{\lambda _\mu})\to x_\infty$; and (b) if every cofinal subnet $\mu \mapsto x_{\lambda _\mu}$ of $\lambda \mapsto x_{\lambda}$ has in turn a subnet $\nu \mapsto x_{\lambda _{\mu _\nu}}$ such that $(\nu \mapsto x_{\lambda _{\mu _\nu}})\to x_\infty$, then $(\lambda \mapsto x_\lambda )\to x_\infty$.  The $(\Leftarrow )$ direction of \cref{enm3.4.22.i} is true by hypothesis, so it suffices to show the $(\Rightarrow )$ direction of \cref{enm3.4.22.i}.  So, suppose that $(\lambda \mapsto x_\lambda )\to x_\infty$ and let $\mu \mapsto x_{\lambda _\mu}$ be a subnet.  Then, by hypothesis, we also have that $(\mu \mapsto x_{\lambda _\mu})\to x_\infty$, and so this subnet $\mu \mapsto x_\mu$ has in turn a subnet (namely itself) that is related to $x_\infty$ by $\to$.
\end{proof}
\end{prp}

And of course, there is an analogous result for filters.
\begin{thm}{Kelley's Filter Convergence Theorem}{KelleysFilterConvergenceTheorem}\index{Kelley's Filter Convergence Theorem}
Let $X$ be a set, denoted by $\tilde{\filter{F}}$ the collection of all filter bases in $X$, and let $\to$ be a relation on $\tilde{\filter{F}}\times X$.  Then, if
\begin{enumerate}
\item \label{enmKelleysFilterConvergenceTheorem.i}$\filter{P}_x\to x$, where $\filter{P}_x\coloneqq \left\{ U\subseteq X:x\in U\right\}$;\footnote{``$\filter{P}$'' is for \emph{principal}, the etymology being from the use of the word ``principal'' in the context of ideals in ring theory (to the best of my knowledge anyways).}
\item \label{enmKelleysFilterConvergenceTheorem.iix}if $\filter{F}\to x$, then $\filter{G}\to x$ for every filtering $\filter{G}$ of $\filter{F}$;
\item \label{enmKelleysFilterConvergenceTheorem.ii}every filtering $\filter{G}\supseteq \filter{F}$ has in turn a filtering $\filter{H}$ such that $\filter{H}\to x$, then $\filter{F}\to x$; and
\item \label{enmKelleysFilterConvergenceTheorem.iii}for all directed sets $I$ and convergent filters $\filter{F}^i\to x^i\in X$, for $i\in I$, if $\filter{F}_{i\mapsto x^i}\to x^\infty$, then
\begin{equation}
\begin{split}
\filter{F}_\infty & \coloneqq \left\{ U\subseteq X:\text{there exists }i_U\in I\text{ such that,}\right. \\
& \qquad \left. \text{whenever }i\geq i_U\text{, }U\supseteq F^i\text{ for some}\right. \\ & \qquad \left. F^i\in \filter{F}^i\text{.}\right\} \to x^\infty ,
\end{split}
\end{equation}
\end{enumerate}
then there is a unique topology on $X$ such that $\filter{F}$ converges to $x_\infty \in X$ iff $\filter{F}\to x_\infty$.
\begin{proof}
\Step{Make hypotheses}
Suppose that \cref{enmKelleysFilterConvergenceTheorem.i}$\filter{P}_x\to x$, where $\filter{P}_x\coloneqq \left\{ U\subseteq X:x\in U\right\}$; \cref{enmKelleysFilterConvergenceTheorem.iix} if $\filter{F}\to x$, then $\filter{G}\to x$ for every filtering $\filter{G}$ of $\filter{F}$; \cref{enmKelleysFilterConvergenceTheorem.ii}; if every filtering $\filter{G}\supseteq \filter{F}$ has in turn a filtering $\filter{H}$ such that $\filter{H}\to x$, then $\filter{F}\to x$; and \cref{enmKelleysFilterConvergenceTheorem.iii} for all directed sets $I$ and convergent filters $\filter{F}^i\to x^i\in X$, for $i\in I$, if $\filter{F}_{i\mapsto x^i}\to x^\infty$, then $\filter{F}_\infty \to x^\infty$.

\Step{Define the notion of an \emph{adherent point}}
For a subset $S\subseteq X$ and $x\in X$, we say that $x$ is an \emph{adherent point}\footnote{This of course will turn out to agree with the notion of adherent point for the topology.} iff there is a filter base $\filter{F}$ such that $F\subseteq S$ for all $F\in \filter{F}$ with $\filter{F}\to x_{\infty}$.

\Step{Define what should be the closure}
For a subset $S\subseteq X$, we define $\mrm{C}(S)$ to be the set of adherent points of $S$.

\Step{Show that if $S\subseteq T$, then $\mrm{C}(S)\subseteq \mrm{C}(T)$}[stpKelleysFilterConvergenceTheorem.4]
Suppose that $S\subseteq T$.  Let $x\in \mrm{C}(S)$.  Then, there is a filter base $\filter{F}$ such that $F\subseteq S$ for all $F\in \filter{F}$ and such that $\filter{F}\to x$.  As $S\subseteq T$, every $F\in \filter{F}$ also satisfies $F\subseteq T$, and so of course we have that $x\in \mrm{C}(T)$ as well.

\Step{Show that $\mrm{C}$ satisfies Kuratowski's Closure Axioms}
The empty-set has no adherent points, and so trivially $\mrm{C}(\emptyset )=\emptyset$.

For $x\in S$, $\filter{P}_x\to x$.  Furthermore, $\{ \{ x\} \}$ is a filtering of $\filter{P}_x$, and so, $\{ \{ x\} \} \to x$, and hence $x$ is an adherent point of $S$, that is, $x\in \mrm{C}(S)$, so that $S\subseteq \mrm{C}(S)$.

We wish to show that $\mrm{C}(\mrm{C}(S))\subseteq \mrm{C}(S)$.  So, let $x^{\infty}\in \mrm{C}(\mrm{C}(S))$.  Then, there is a filter base $\filter{F}$ such that $F\subseteq \mrm{C}(S)$ for all $F\in \filter{F}_{\infty}$ with $\filter{F}\to x^{\infty}$.  For each $F\in \filter{F}$, let $x_F\in F$, so that $\filter{F}\ni F\mapsto x_F$ is a net.  Note that $\filter{F}_{F\mapsto x_F}\supseteq \filter{F}$, so that $\filter{F}_{F\mapsto x_F}\to x^{\infty}$.  As each $x_F\in \mrm{C}(S)$, for each $F\in \filter{F}$, there is a filter base $\filter{G}^F$ such that $G\subseteq S$ for all $G\in \filter{G}^F$ with $\filter{G}^F\to x_F$.  By \cref{enmKelleysFilterConvergenceTheorem.iii}, we then have that
\begin{equation}
\begin{split}
\filter{F}_{\infty} & \coloneqq \left\{ U\subseteq X:\text{ there exists }F_U\in \filter{F}\text{ such that },\right. \\
& \qquad \left. \text{whenever }F\subseteq F_u,\ U\supseteq G^F\text{ for some}\right. \\ & \qquad \left. G^F\in \filter{G}^F\text{.}\right\} \to x^{\infty}.
\end{split}
\end{equation}
Note that, as each $G^F\subseteq U$, if $U\in \filter{F}_{\infty}$, then so is $U\cap S$.  It follows that
\begin{equation}
\filter{G}_{\infty}\coloneqq \{ U\cap S:U\in \filter{F}_{\infty}\}
\end{equation}
is a filter base that is a filtering of $\filter{F}_{\infty}$.  As $\filter{F}_{\infty}\to x_{\infty}$, it follows that $\filter{G}_{\infty}\to x_{\infty}$.  As $G\subseteq S$ for all $G\in \filter{G}_{\infty}$, it follows that $x_{\infty}\in \mrm{C}(S)$, as desired.

We now show that $\mrm{C}(S\cup T)=\mrm{C}(S)\cup \mrm{C}(T)$.  By \cref{stpKelleysFilterConvergenceTheorem.4}, we have that $\mrm{C}(S)\subseteq \mrm{C}(S\cup T)$.  Likewise for $T$, and so $\mrm{C}(S)\cup \mrm{C}(T)\subseteq \mrm{C}(S\cup T)$.

We now show that $\mrm{C}(S\cup T)\subseteq \mrm{C}(S)\cup \mrm{C}(T)$.  So, let $x\in \mrm{C}(S\cup T)$, so that there is a filter base $\filter{F}$ such that $F\subseteq S\cup T$ for all $F\in \filter{F}$ with $\filter{F}\to x_{\infty}$.  If $F\cap S=\emptyset$ for all $F\in \filter{F}$, then $F\subseteq T$ for all $F\in \filter{F}$, in which case we will have that $x\in \mrm{C}(T)$, and we are done.  Thus, we may as well assume that $F\cap S$ is nonempty for all $F\in \filter{F}$.  It follows that $\filter{F}_S\coloneqq \{ S\cap F:F\in \filter{F}\}$ is a filtering of $\filter{F}$.  We have that $\filter{F}_S\to x$, and so $x\in \mrm{C}(S)$, as desired.

It follows by \namerefpcref{KuratowskisClosureTheorem} that there is a unique topology on $X$ such that $\Cls (S)=\mrm{C}(S)$.

\Step{Show that $\filter{F}$ converges to $x_{\infty}$ iff $\filter{F}\to x_{\infty}$}
Suppose that $\filter{F}$ converges to $x_{\infty}$.  We wish to show that $\filter{F}\to x_{\infty}$.  We proceed by contradiction:  suppose that it is not the case that $\filter{F}\to x_{\infty}$.  Then, there must be some filtering $\filter{G}$ of $\filter{F}$ that has no filtering $\filter{H}$ such that $\filter{H}\to x_{\infty}$.  To obtain a contradiction, we construct such a filtering.  As $\filter{F}$ converges to $x_{\infty}$, so does $\filter{G}$.  For every $G\in \filter{G}$, let $x_G\in G$, so that $\filter{G}\ni G\mapsto x_G\in X$ is a net, and furthermore, $\filter{F}_{G\mapsto x_G}$ is a filtering of $\filter{G}$, and so likewise converges to $x_{\infty}$.  Hence, $G\mapsto x_G$ converges to $x_{\infty}$.  For each $G_0\in G$, define $S_{G_0}\coloneqq \{ x_G:G\supseteq G_0\}$.  As $G\mapsto x_G$ converges to $x_{\infty}$, it follows that $x_{\infty}\in \Cls (S_{G_0})$ for all $G_0\in \filter{G}$.  Hence, by the definition of our closure, for each $G\in \filter{G}$, there is a filter base $\filter{H}_G$ such that $H\subseteq S_G$ for all $H\in \filter{H}_G$ and $\filter{H}_G\to x_{\infty}$.  From \cref{enmKelleysFilterConvergenceTheorem.iii}, it follows that
\begin{equation}
\begin{split}
\filter{H}_{\infty} & \coloneqq \left\{ U\subseteq X:\text{there exists }G_U\in \filter{G}\text{ such that ,} \right. \\
& \qquad \left. \text{whenever }G\subseteq G_U,\ U\supseteq H_G\text{ for some}\right. \\ & \qquad \left. H_G\in \filter{H}_G\text{.}\right\} \to x_{\infty}.
\end{split}
\end{equation}
Note that, as $H_G\subseteq S_G\subseteq G$, if $U\in \filter{H}_{\infty}$, then $U\cap G\in \filter{H}_{\infty}$ for all $G\in \filter{G}$.  Hence,
\begin{equation}
\filter{K}_{\infty}\coloneqq \{ U\cap G:G\in \filter{G}\}
\end{equation}
is a filtering of $\filter{H}_{\infty}$, and hence $\filter{K}_{\infty}\to x_{\infty}$.  However, it is likewise a filtering of $\filter{G}$, and so we have our contradiction.

Now suppose that $\filter{F}\to x$.  We proceed by contradiction:  suppose that $\filter{F}$ does not converge to $x$.  Then, there is a filtering $\filter{G}$ of $\filter{F}$ which has no filtering converging to $x_{\infty}$.  In particular, $\filter{G}$ itself does not converge to $x_{\infty}$, and so there is an open neighborhood $U$ of $x_{\infty}$ such that $G\not \subseteq U$ for all $G\in \filter{G}$.  So, for each $G\in \filter{G}$, let $x_G\in G\setminus U$.  $G\mapsto x_G$ is then a net contained in $U^{\comp}$.  On the other hand, $\filter{H}\coloneqq \left\{ F\cap U^{\comp}:F\in \filter{F}_{G\mapsto x_G}\right\}$ is a filtering of $\filter{G}$, and hence of $\filter{F}$, and so $\filter{H}\to x$.  As $H\subseteq U^{\comp}$ by construction for all $H\in \filter{H}$, it follows that $x\in \Cls (U^{\comp})=U^{\comp}$:  a contradiction.

\Step{Demonstrate uniqueness}
Note that $x\in X$ is an adherent point of $S\subseteq X$ iff there is a filter base $\filter{F}$ such that $F\subseteq S$ for all $F\in \filter{F}$ with $\filter{F}$ converging to $x$.  Recall that sets are closed iff they contain all their adherent points.\footnote{Because an adherent point of $S$ is either an element of $S$ of an accumulation point of $S$.}  If we have two topologies with the same notion of filter convergence, then the adherent points of a given set in each topology are the same, and consequently, a set is closed in one iff it is closed in the other.
\end{proof}
\end{thm}
And of course, there is an analogous result to \cref{prp3.4.22} which gives an axiom equivalent to the second and third.
\begin{prp}{}{prp3.4.17}
Let $X$ be a set, denoted by $\tilde{\filter{F}}$ the collection of all filter bases in $X$, and let $\to$ be a relation on $\tilde{\filter{F}}\times X$.  Then, the following are equivalent.
\begin{enumerate}
\item \label{prp3.4.17.i}$\filter{F}\to x$ iff for every filtering $\filter{G}$ of $\filter{F}$, there is some filtering $\filter{H}$ of $\filter{G}$ such that $\filter{H}\to x$.
\item \label{prp3.4.17.ii}
\begin{enumerate}
\item \label{prp3.4.17.ii.a}If $\filter{F}\to x$ and $\filter{G}$ is a filtering of $\filter{F}$, then $\filter{G}\to x$; and
\item \label{prp3.4.17.ii.b}If every filtering $\filter{G}\supseteq \filter{F}$ has in turn a filtering $\filter{H}$ such that $\filter{H}\to x$, then $\filter{F}\to x$.
\end{enumerate}
\end{enumerate}
\begin{proof}
$(\cref{prp3.4.17.i}\Rightarrow \cref{prp3.4.17.ii})$ Suppose that $\filter{F}\to x$ iff for every filtering $\filter{G}$ of $\filter{F}$, there is some filtering $\filter{H}$ of $\filter{G}$ such that $\filter{H}\to x$.

We first check \cref{prp3.4.17.ii.a}.  So, suppose that $\filter{F}\to x$ and let $\filter{G}$ be a filtering of $\filter{F}$.  We wish to show that $\filter{G}\to x$.  To show this, let $\filter{H}$ be a filtering of $\filter{G}$.  This is likewise a filtering of $\filter{F}$, and so, by hypothesis, this has a filtering $\filter{I}$ such that $\filter{I}\to x$.  Thus, we have shown that every filtering of $\filter{G}$ has in turn a filtering $\filter{I}$ such that $\filter{I}\to x$.  Thus, by hypothesis, $\filter{G}\to x$.  This shows \cref{prp3.4.17.ii.a}.

We now check \cref{prp3.4.17.ii.b}.  So, suppose that every filtering $\filter{G}\supseteq \filter{F}$ has in turn a filtering $\filter{H}$ such that $\filter{H}\to x$.  We wish to show that $\filter{F}\to x$.  To do that, we show that \emph{every} filtering $\filter{G}$ of $\filter{F}$ (not just those filter bases which are supersets of $\filter{F}$) have a filtering $\filter{H}$ such that $\filter{H}\to x$.

So, let $\filter{G}$ be a filtering of $\filter{F}$.  Define $\filter{H}\ceqq \filter{G}\cup \filter{F}$.  We first check that this is in fact a filter base.  So, let $U_1,U_2\in \filter{H}$.  If both $U_1$ and $U_2$ lie in either $\filter{G}$ or $\filter{F}$, then certainly there will be some $U_3\in \filter{H}$ with $U_3\subseteq U_1,U_2$ (simply because $\filter{G}$ and $\filter{F}$ are filter bases).  So, without loss of generality suppose that $U_1\in \filter{G}$ and $U_2\in \filter{F}$.  As $\filter{G}$ is a filtering of $\filter{F}$, there is some $V\in \filter{G}$ such that $V\subseteq U_2$.  As $\filter{G}$ is a filter base, there is some $U_3\in \filter{G}$ such that $U_3\subseteq U_1,V$, and hence $U_3\subseteq U_1,U_2$.  Thus, $\filter{H}$ is indeed a filter base.  Furthermore, $\filter{H}\supseteq \filter{F}$, and so, by hypothesis, has a filter $\filter{I}$ such that $\filter{I}\to x$.  However, $\filter{H}\supseteq \filter{G}$, and so certainly is a filtering of $\filter{G}$, so that in turn $\filter{I}$ is a filtering of $\filter{G}$.  Thus, we have shown that every filtering $\filter{G}$ of $\filter{F}$ has in turn a filtering $\filter{I}$ such that $\filter{I}\to x$, and hence, by hypothesis, $\filter{F}\to x$, as desired.

\blankline
\noindent
$(\cref{prp3.4.17.ii}\Rightarrow \cref{prp3.4.17.i})$ Suppose that (a) if $\filter{F}\to x$ and $\filter{G}$ is a filtering of $\filter{F}$, then $\filter{G}\to x$; and (b) if every filtering $\filter{G}\supseteq \filter{F}$ has in turn a filtering $\filter{H}$ of such that $\filter{H}\to x$, then $\filter{F}\to x$.  The $(\Leftarrow )$ direction \cref{prp3.4.17.i} is true by hypothesis, so it suffices to show the $(\Rightarrow )$ direction of \cref{prp3.4.17.i}.  So, suppose that $\filter{F}\to x$ and let $\filter{G}$ be a filtering.  Then, by hypothesis, we also have that $\filter{G}\to x$, and so this filtering has in turn a filtering (namely itself) that is related to $x$ by $\to$.
\end{proof}
\end{prp}

\subsection[Definition by specification of cont.~functions]{Definition by specification of continuous functions}

The following two results define a topology on a set by simply `declaring' that a collection of functions be continuous.  This is similar in nature to how we defined a topology with a generating collection (\cref{GeneratingCollection})---in that case, we started with a collection of subsets, and simply `declared' them to be open---here, we start with a collection of \emph{functions}, and simply `declare' them to be continuous.
\begin{prp}{Initial topology}{InitialTopology}
Let $X$ be a set, let $\collection{Y}$ be an indexed collection of topological spaces, and for each $Y\in \collection{Y}$ let $f_Y:X\rightarrow Y$ be a function.  Then, there exists a unique topology  $\topology{U}$ on $X$, the \term{initial topology}\index{Initial topology} with respect to $\{ f_Y:Y\in \collection{Y}\}$, such that
\begin{enumerate}
\item $f_Y:X\rightarrow Y$ is continuous with respect to $\topology{U}$ for all $Y\in \collection{Y}$; and
\item if $\topology{U}'$ is another topology for which each $f_Y:X\rightarrow Y$ is continuous, then $\topology{U}\subseteq \topology{U}'$.
\end{enumerate}
Furthermore, if $\collection{S}_Y$ generates the topology on $Y$, then the collection
\begin{equation}
\{ f_Y^{-1}(U):Y\in \collection{Y},\ U\in \collection{S}_Y\}
\end{equation}
generates $\topology{U}$.
\begin{rmk}
In other words, the initial topology is the smallest topology for which each $f_Y$ is continuous.
\end{rmk}
\begin{rmk}
But what about the largest such topology?  Well, the largest such topology is always going to be the discrete topology, which is not very interesting.  This is how you remember whether the initial topology is the smallest or largest---it can't be the largest because the discrete topology always works.
\end{rmk}
\begin{rmk}
In particular,
\begin{equation}
\{ f_Y^{-1}(V):Y\in \collection{Y},\ V\subseteq Y\text{ open}\}
\end{equation}
generates the initial topology.
\end{rmk}
\begin{wrn}
Warning:  $\{ f_Y^{-1}(U):Y\in \collection{Y},\ U\in \collection{S}_Y\}$ need not be a base for the topology even if each $\collection{S}_Y$ is (indeed, or even if each $\collection{S}_Y$ is the topology of $Y$ itself)---see, for example, the \namerefpcref{ProductTopology}.
\end{wrn}
\begin{rmk}
Compare this with the definition of the integers, rationals, reals, closure, interior, and generating collections (\cref{Integers,RationalNumbers,RealNumbers,Closure,Interior,GeneratingCollection}).
\end{rmk}
\begin{proof}
For $Y\in \collection{Y}$, let $\collection{S}_Y$ be any generating collection, define
\begin{equation}
\collection{S}\coloneqq \{ f_Y^{-1}(V):Y\in \collection{Y},\ V\in \collection{S}_Y\} ,
\end{equation}
and let $\topology{U}$ be the topology generated by $\collection{S}$ (\cref{GeneratingCollection}).

As every element of $\collection{S}$ is open, it is certainly the case that each $f_Y$ is continuous by \cref{exr4.1.27} (it suffices to check continuity on a generating collection).  On the other hand, if $\topology{U}'$ is another topology for each each $f_Y$ is continuous, then it must certainly contain $\collection{S}$, in which case $\topology{U}'$ contains $\topology{U}$ by the definition of a generating collection.

\begin{exr}[breakable=false]{}{}
Show that the initial topology is unique.
\end{exr}
\end{proof}
\end{prp}
There is a nice characterization of convergence in initial topologies.
\begin{prp}{}{prp4.4.5}
Let $X$ have the initial topology with respect to the collection $\{ f_Y:Y\in \collection{Y}\}$, let $\lambda \mapsto x_\lambda \in X$ be a net, and let $x_\infty \in X$.  Then, $\lambda \mapsto x_\lambda$ converges to $x_\infty$ in $X$ iff $\lambda \mapsto f_Y(x_\lambda )$ converges to $f_Y(x_\infty )$ in $Y$ for all $Y\in \collection{Y}$.  Furthermore, the initial topology is the unique topology that has this property.
\begin{proof}
$(\Rightarrow )$ Suppose that $\lambda \mapsto x_\lambda$ converges to $x_\infty$ in $X$.  Then, because each $f_Y$ is continuous, it follows that $\lambda \mapsto f_Y(x_\lambda )$ converges to $f_Y(x_\infty )$ in $Y$ for all $Y\in \collection{Y}$.

\blankline
\noindent
$(\Leftarrow )$ Suppose that $\lambda \mapsto f_Y(x_\lambda )$ converges to $f_Y(x_\infty )$ in $Y$ for all $Y\in \collection{Y}$.  To show that $\lambda \mapsto x_\lambda$ converges to $x_\infty$ in $X$, we apply \cref{exr4.2.41} (it suffices to check convergence on a generating collections).  We know that
\begin{equation}
\{ f_Y^{-1}(U):Y\in \collection{Y},\ U\subseteq Y\text{ open}\}
\end{equation}
generates the initial topology, and so we wish to show that $\lambda \mapsto x_\lambda$ is eventually contained in $f_Y^{-1}(U)$.  However, as $f_Y^{-1}(U)$ is an open neighborhood of $x_\infty$, then $U$ is an open neighborhood of $f_Y(x_\infty )$,\footnote{Of course we already knew that $U$ was open---the point is that $x_{\infty}\in U$.} and so $\lambda \mapsto f_Y(x_\lambda )$ is eventually contained in $U$ because $\lambda \mapsto f_Y(x_\lambda )$ converges to $f_Y(x_\infty )$.  But then $\lambda \mapsto x_\lambda$ is eventually contained in $f_Y^{-1}(U)$, and we are done.

\blankline
\noindent
Uniqueness follows from the uniqueness stated in \nameref{KelleysConvergenceTheorem}.
\end{proof}
\end{prp}
Initial topologies have the nice property that you can determine whether functions into $X$ are continuous or not by looking at their composition with each $f_Y$.
\begin{prp}{}{prp3.4.6}
Let $X$ have the initial topology with respect to the collection $\{ f_Y:Y\in \collection{Y}\}$, let $Z$ be a topological space, and let $f\colon Z\rightarrow X$ be a function.  Then, $f$ is continuous iff $f_Y\circ f$ is continuous for all $Y\in \collection{Y}$.  Furthermore, the initial topology is the unique topology with this property.
\begin{proof}
$(\Rightarrow )$ Suppose that $f$ is continuous.  Then, because each $f_Y$ is itself continuous and compositions of continuous functions are continuous, it follows that $f_Y\circ f$ is continuous for all $Y\in \collection{Y}$.

\blankline
\noindent
$(\Leftarrow )$ Suppose that $f_Y\circ f$ is continuous for all $Y\in \collection{Y}$.  Let $\lambda \mapsto x_\lambda$ converge to $x_\infty \in X$.  To show that $f$ is continuous, it suffices to show that $\lambda \mapsto f(x_\lambda)$ converges to $f(x_\infty )$.  However, because each $f_Y\circ f$ is continuous, $\lambda \mapsto f_Y(f(x_\lambda ))$ converges to $f_Y(f(x_\infty ) )$.  Therefore, by the previous result, we do indeed have that $\lambda \mapsto f(x_\lambda)$ converges to $f(x_\infty )$.

\blankline
\noindent
To show uniqueness, we apply the previous proposition.  We wish to show that if a topology on $X$ has the property that $f$ is continuous iff each $f_Y\circ f$ is continuous, then it also has the property that $\lambda \mapsto x_{\lambda}$ converges to $x_{\infty}$ iff each $\lambda \mapsto f(x_{\lambda})$ converges to $f(x_{\infty})$.  As the identity function $\id _X$ is continuous, each $f_Y=f_Y\circ \id _X$ is continuous, and so if $\lambda \mapsto x_{\lambda}$ converges to $x_{\infty}$, then certainly each $\lambda \mapsto f_Y(x_{\lambda})$ converges to $f_Y(x_{\infty})$.

Conversely, suppose that each $\lambda \mapsto f_Y(x_{\lambda})$ converges to $f_Y(x_{\infty})$.  Denote the index set of $\lambda \mapsto x_{\lambda}$ by $\Lambda$ and define $\Lambda _{\infty}\coloneqq \Lambda \sqcup \{ \infty \}$ and extend the order on $\Lambda$ to $\Lambda _{\infty}$ be declaring that $\lambda \leq \infty$ for all $\lambda \in \Lambda$.  $\Lambda _{\infty}$ is then a directed set.  Define $\collection{B}\coloneqq \left\{ (\lambda ,\infty ]\subseteq \Lambda _{\infty}:\lambda \in \Lambda \right\} \cup \left\{ \{ \lambda \} \subseteq \Lambda _{\infty}:\lambda \in \Lambda \right\}$.  As $\Lambda$ is directed, this is a base, and so defines a unique topology on $\Lambda _{\infty}$ (\cref{prp4.1.5}).  Furthermore, the topology is defined in such a way that $\Lambda \ni \lambda \mapsto \lambda \in \Lambda _{\infty}$ converges to $\infty \in \Lambda _{\infty}$.

Extend $x\colon \Lambda \rightarrow X$ to a function on $\Lambda _{\infty}$ be defining $x(\infty )\coloneqq x_{\infty}$.  As $\lambda \mapsto \lambda \in \Lambda _{\infty}$ converges to infinity, if we can show that $x\colon \Lambda _{\infty}\rightarrow X$ is continuous, we will have that $\lim x_{\lambda}\coloneqq \lim _{\lambda}x(\lambda )=x(\infty )\coloneqq x_{\infty}$, as desired.  By hypothesis, it suffices to show that $f_Y\circ x$ is continuous for all $Y\in \collection{Y}$.  The hypothesis that $\lambda \mapsto f_Y(x_{\lambda})$ converges to $f_Y(x_{\infty})$ for all $Y\in \collection{Y}$ tells us that $f_Y\circ x\colon \Lambda _{\infty}\rightarrow Y$ is continuous at $\infty \in \Lambda _{\infty}$.  On the other hand, for $\lambda \in \Lambda$, as $\{ \lambda \}$ is an open neighborhood of $\lambda$, every net in $\Lambda _{\infty}$ converging to $\lambda$ must be eventually constant, and so $f_Y\circ x$ is vacuously continuous at every other point.  Hence, $f_Y\circ x$ is continuous, and we are done.
\end{proof}
\end{prp}

There is a `dual' version of the initial topology, in which the functions map \emph{into} the set on which we would like to define a topology.
\begin{prp}{Final topology}{FinalTopology}
Let $X$ be a set, let $\collection{Y}$ be an indexed collection of topological spaces, and for each $Y\in \collection{Y}$ let $f_Y:Y\rightarrow X$ be a function.  Then, there exists a unique topology $\topology{U}$ on $X$, the \term{final topology}\index{Final topology} with respect to $\{ f_Y:Y\in \collection{Y}\}$, such that
\begin{enumerate}
\item $f_Y:Y\rightarrow X$ is continuous with respect to $\topology{U}$ for all $Y\in \collection{Y}$; and
\item if $\topology{U}'$ is another topology for which each $f_Y$ is continuous, then $\topology{U}\supseteq \topology{U}'$.
\end{enumerate}
Furthermore,
\begin{equation}\label{eqn3.4.27}
\topology{U}=\{ U\in 2^X:f_Y^{-1}(U)\text{ is open for all }Y\in \collection{Y}\text{.}\} .
\end{equation}
\begin{rmk}
In other words, the final topology is the largest topology for which each $f_Y$ is continuous.
\end{rmk}
\begin{rmk}
But what about the smallest such topology?  Well, the smallest such topology is always going to be the indiscrete topology, which is not very interesting.  This is how you remember whether the final topology is the smallest or largest---it can't be the smallest because the indiscrete topology always works.
\end{rmk}
\begin{proof}
Define
\begin{equation}\label{3.4.9}
\topology{U}\coloneqq \{ U\in 2^X:f_Y^{-1}(U)\text{ is open for all }Y\in \collection{Y}\text{.}\} .
\end{equation}
\begin{exr}[breakable=false]{}{}
Check that $\topology{U}$ is actually a topology.
\begin{rmk}
We didn't need to do any such checking in the construction of the initial topology because there we just took the topology \emph{generated} by the collection.
\end{rmk}
\end{exr}
Of course every $f_Y$ is continuous with respect to $\topology{U}$ as, by definition, the preimage of every element of $\topology{U}$ is open.  Furthermore, anything larger than $\topology{U}$ would necessarily have to contain some set for which the preimage under some $f_Y$ would not be open, and so that $f_Y$ would not be continuous.
\begin{exr}[breakable=false]{}{}
Show that the final topology is unique.
\end{exr}
\end{proof}
\end{prp}
There is likewise a `dual' result to \cref{prp3.4.6} which tells us when continuous functions \emph{on} a space equipped with a final topology are continuous.\footnote{I am not aware of a`dual' to \cref{prp4.4.5} that characterizes convergence in final topologies.}
\begin{prp}{}{prp3.4.34x}
Let $X$ have the final topology with respect to the collection $\{ f_Y:Y\in \collection{Y}\}$, let $Z$ be a topological space, and let $f\colon X\rightarrow Z$ be a function.  Then, $f$ is continuous iff $f\circ f_Y$ is continuous for all $Y\in \collection{Y}$.  Furthermore, the final topology is the unique topology with this property.
\begin{proof}
$(\Rightarrow )$ Suppose that $f$ is continuous.  Then, because each $f_Y$ is itself continuous and compositions of continuous functions are continuous, it follows that $f\circ f_Y$ is continuous for all $Y\in \collection{Y}$.

\blankline
\noindent
$(\Leftarrow )$ Suppose that $f\circ f_Y$ is continuous for all $Y\in \collection{Y}$.  Let $U\subseteq Z$ be open.  We must show that $f^{-1}(U)$ is open.  However, from \eqref{eqn3.4.27}, we know that $f^{-1}(U)$ will be open iff $f_Y^{-1}\left( f^{-1}(U)\right)$ will be open for all $Y\in \collection{Y}$.  However, $f_Y^{-1}\left( f^{-1}(U)\right) =[f\circ f_Y]^{-1}(U)$ is open because $f\circ f_Y$ is continuous.

\blankline
\noindent
Let $\topology{U}$ be another topology that has the property that $f\colon X\rightarrow Z$ is continuous iff $f\circ f_Y$ is continuous for all $Y\in \collection{Y}$.  To show that $\topology{U}$ is the final topology, by the definition (\cref{FinalTopology}), it suffices to show that each $f_Y:Y\rightarrow X$ is continuous with respect to $\topology{U}$ and that any other such topology is contained in $\topology{U}$.  As $\id _X:\coord{X,\topology{U}}\rightarrow \coord{X,\topology{U}}$ is continuous, by hypothesis, it follows that $\id _X\circ f_Y=f_Y$ is continuous with respect to $\topology{U}$.  Let $\topology{U}'$ be another topology such that $f_Y$ is continuous with respect to $\topology{U}'$ for all $Y\in \collection{Y}$.  We must show that $\topology{U}'\subseteq \topology{U}$.  To show this, it suffices to show that $\id _X:\coord{X,\topology{U}}\rightarrow \coord{X,\topology{U}'}$ is continuous.  By the defining property of $\topology{U}$, to show this, it suffices to show that the composition of this with each $f_Y$ is continuous.  In other words, it suffices to show that each $f_Y$ is continuous with respect to $\topology{U}'$, but this is true by hypothesis.
\end{proof}
\end{prp}

\subsection{Summary}

We now quickly recap all the ways in which we know how to specify a topology on a set.
\begin{enumerate}
\item We can specify the open sets (\cref{TopologicalSpace})..
\item We can specify the closed sets (\cref{exr4.1.2}).
\item We can specify a base (\cref{Base}).
\item We can specify a neighborhood base (\cref{NeighborhoodBase}).
\item We can generate a topology (\cref{GeneratingCollection}).
\item We can define the closure of each set (\cref{KuratowskisClosureTheorem}).
\item We can define the interior of each set (\cref{KuratowskisInteriorTheorem}).
\item We can define convergence of nets (\cref{KelleysConvergenceTheorem}).
\item We can define convergence of filters (\cref{KelleysFilterConvergenceTheorem}).
\item We can declare that functions on the space are continuous (the initial topology---see \cref{InitialTopology}).
\item We can declare that functions into the space are continuous (the final topology---see \cref{FinalTopology}.
\end{enumerate}

\section{New topologies from old}

The purpose of this section is to present several ways of constructing new topologies spaces from old.  In brief, the subspace topology will be the topology we put on subsets of topological spaces, the quotient topology will be the topology we put on quotients of topological spaces\footnote{Quotients in the sense of \cref{dfnA.1.42}.}, the product topology is the topology we put on Cartesian-products, the disjoint-union topology (surprise, surprise) is the topology we put on disjoint-unions of topological spaces.  They key to defining all of these topologies are the initial (for the subspace and product topologies) and final topologies (for the quotient and disjoint-union topologies) (\cref{InitialTopology,FinalTopology}).

\subsection{The subspace topology}

All \emph{subsets} of topological spaces have a canonically associated topology, called the \emph{subspace topology}.  Note that this is not completely immediate---for example, it is not the case that every subset of a ring is a ring.  There is definitely something to define and something to check (that the subspace topology is in fact a topology).
\begin{prp}{Subspace topology}{SubspaceTopology}
Let $X$ be a topological space and let $S\subseteq X$.  Then, there exists a unique topology $\topology{U}$ on $S$, the \term{subspace topology}\index{Subspace topology}, that has the property that a function into $S$ is continuous iff it is continuous regarded as a function into $X$.  Furthermore, the subspace topology is the initial topology with respect to the inclusion $\iota :S\hookrightarrow X$.  In particular,
\begin{equation}
\topology{U}=\{ U\cap S:U\subseteq X\text{ is open.}\} .
\end{equation}
\begin{rmk}
One fact to take note of is that, if $U\subseteq S$ is open in the subspace topology, then there is some open $U'\subseteq X$ such that $U=U'\cap S$.  Similarly, if $C\subseteq S$ is closed in the subspace topology, then there is some closed $C'\subseteq X$ such that $C=C'\cap S$.
\end{rmk}
\begin{wrn}
Warning:  Just because $U\subseteq S$ is open in $S$, does \emph{not} mean that $U$ is open in $X$---see \cref{exm4.1.14}.
\end{wrn}
\begin{rmk}
Unless otherwise stated, subsets of topological spaces are always equipped with the subspace topology.
\end{rmk}
\begin{proof}
All of this follows from the definition of the initial topology and its characterization of continuity of functions into initial topologies (\cref{InitialTopology,prp3.4.6}), with exception of the fact that $\topology{U}=\{ U\cap S:U\subseteq X\text{ is open.}\}$.  \cref{InitialTopology} tells us that this generates the initial topology, but it does not tell us that it is a topology itself.  Of course, however, if a generating collection is itself a topology, then the topology it generates is just itself.  Therefore, it suffices just to check that $\{ U\cap S:U\subseteq X\text{ is open.}\}$ is in fact a topology.
\begin{exr}[breakable=false]{}{}
Check that $\{ U\cap S:U\subseteq X\text{ is open.}\}$ is in fact a topology.
\end{exr}
\end{proof}
\end{prp}
\begin{exm}{}{exm4.1.14}
Consider the subspace topology on $[0,1]\subseteq \R$.  Note that $(\frac{1}{2},1]=(\frac{1}{2},\infty )\cap [0,1]$ is \emph{open in $[0,1]$}, despite the fact that it obviously not open in $\R$.
\end{exm}
\begin{exm}{}{}
Note that the order topology on $\Q$ is the space as the subspace topology inherited from $\R$.  This is of course because they are both equipped with the order topology with respect to the same order.  Likewise, for $\N \subseteq \Z$ and $\Z \subseteq \Q$.
\end{exm}

Here is a neat little application of the subspace topology that a priori doesn't seem like it would have to make use of the subspace topology at all.\footnote{And indeed, you can probably quite easily find a proof that doesn't use it.}
\begin{prp}{}{prp3.5.6}
Let $X$ be a topological space, let $K\subseteq X$ be quasicompact, and let $C\subseteq X$ be closed.  Then, $K\cap C$ is quasicompact and closed.
\begin{proof}
$K\cap C$ is closed in the subspace topology of $K$, and hence is quasicompact in $K$ by \cref{exr4.2.33}.

Now, let $\cover{U}$ be an open\footnote{Open in $X$, that is.} cover of $K\cap C$.  Then, $\left\{ U\cap K:U\in \cover{U}\right\}$ is an open cover of $K\cap C$ in $K$, and so as $K\cap C$ is quasicompact in $K$, there are finitely many $U_1,\ldots ,U_m\in \cover{U}$ such that
\begin{equation}
K\cap C\subseteq (U_1\cap K)\cup \cdots \cup \cdots (U_m\cap K).
\end{equation}
But then certainly
\begin{equation}
K\cap C\subseteq U_1\cup \cdots \cup U_m,
\end{equation}
that is, $\{ U_1,\ldots ,U_m\}$ is an open subcover of $\cover{U}$.
\end{proof}
\end{prp}

\subsection{The quotient topology}

Whenever we have a surjective function from a topological space $X$ onto a set $Y$, we can use this function and the topology on $X$ to place a topology on $Y$.  Recall (\cref{exrA.1.81}) that every surjective function can be viewed as a quotient function---this of course is the etymology of the term ``quotient topology''.
\begin{prp}{}{QuotientTopology}
Let $X$ be a topological space, let $Y$ be a set, and let $\q :X\rightarrow Y$ be surjective.  Then, there exists a unique topology $\topology{U}$ on $Y$, the \term{quotient topology}\index{Quotient topology}, that has the property that a function on $Y$ is continuous iff its composition with $\q$ is continuous.  Furthermore, the quotient topology is the final topology with respect to $\q :X\rightarrow Y$.  In particular,
\begin{equation}
\topology{U}=\{ U\subseteq Y:\q ^{-1}(U)\text{ is open.}\} .
\end{equation}
\begin{rmk}
Unless otherwise stated, quotients of topological spaces are always equipped with the quotient topology.
\end{rmk}
\begin{rmk}
If you ever hear mathematicians talking about some sort of ``gluing'' construction, what they're actually doing is constructing a topological space by first defining an equivalence relation $\sim$ on a topological space $X$ (points that are ``glued'' together are defined to be equivalent), and then equipping the quotient, $X/\sim$,\footnote{Recall that (\cref{dfnA.1.42}) $X/\sim$ is the set of equivalence classes.} with the quotient topology $X\rightarrow X/\sim$.  For example, if you take a square\footnote{Say, $[0,1]\times [0,1]$ endowed with the product topology (\cref{ProductTopology}).} and ``glue'' two opposite sides together,\footnote{Precisely, you would say that two distinct points are equivalent to one another iff they have $x$-coordinate $0$ and $x$-coordinate $1$ respectively and the \emph{same} $y$-coordinate.} you get a cylinder.
\end{rmk}
\begin{proof}
All of this follows from the definition of the final topology and its characterization in terms of continuity of functions defined on final topologies (\cref{FinalTopology,prp3.4.34x}).
\end{proof}
\end{prp}

\subsection{The product topology}

The product topology is the canonical topology we put on a Cartesian-product of topological spaces $X\times Y$.  While we do technically use this in places, we use it in ways where we could have gotten-away with not speaking of the product topology per se (for example, the product topology on $\R \times \R$ is the same as the topology defined by $\varepsilon$-balls).  One significant reason we go to the trouble of talking about the product topology explicitly is for the proof of producing a \emph{counter-example} to the statement
\begin{important}
A space is quasicompact iff every net has a convergent \emph{cofinal} subnet.\footnote{In case you're skimming and didn't read the context, this statement is \emph{false}.}
\end{important}
We mentioned when we defined subnets (\cref{Subnet}) that the notion of a cofinal subnet was the more obvious ``naive'' notion (that is, you just take terms from the original net, subject to the only condition that your indices get arbitrarily large), but that this ``naive'' notion was insufficient because it didn't allow us to prove certain theorems.  They key result, that spaces are quasicompact iff every net has a convergent subnet (\cref{prp4.2.31}), was precisely the example of a theorem I had in mind.

In brief, the counter-example will be
\begin{equation}
X\coloneqq \prod _{2^{\N}}\{ 0,1\} ,
\end{equation}
that is, an uncountable product (precisely, a product over the power set of $\N$) of the two-element set $\{ 0,1\}$.  Obviously $\{ 0,1\}$ is quasicompact (all finite spaces are), and then a theorem called \namerefpcref{TychonoffsTheorem}, which says that \emph{arbitrary} products of quasicompact spaces are quasicompact, will tell us that $X$ is quasicompact.

So before we do anything else then, we must first define the product topology.
\begin{prp}{Product topology}{ProductTopology}
Let $\collection{X}$ be an indexed collection of topological spaces.  Then, there exists a unique topology, the \term{product topology}\index{Product topology}, on $\prod _{X\in \collection{X}}X$, that has the property that a function into $\prod _{X\in \collection{X}}X$ is continuous iff each component of the function is continuous.  Furthermore, the product topology is the initial topology with respect to $\{ \pi _X:X\in \collection{X}\}$.\footnote{Recall (\cref{CartesianProductCollection}) that $\pi _X:\prod _{X\in \collection{X}}X\rightarrow X$ is just the projection.}  In particular, if $\collection{S}_X$ is a generating collection for the topology on $X\in \collection{X}$, then the collection
\begin{equation}\label{4.5.4x}
\{ \pi _X^{-1}(U):U\in \collection{S}_X\}
\end{equation}
generates the product topology, so that
\begin{equation}\label{4.5.4}
\begin{multlined}
\topology{B}\coloneqq \left\{ \prod _{X\in \collection{X}}S_X:S_X\in \collection{S}_X\text{ and}\right. \\ \left. \text{all but finitely many }S_X=X\right\}
\end{multlined}
\end{equation}
is a base for the product topology.
\begin{rmk}
The ``all but finitely many'' phrase is \emph{crucial}.  For example, in the space
\begin{equation}
\prod _{\N}\R ,
\end{equation}
that is, a countably-infinite product of $\R$, the set
\begin{equation}
(0,1)\times (0,1)\times (0,1)\times \cdots 
\end{equation}
is \emph{not} even open in the product topology on $\prod _{\N}\R$ (much less an element of any base).  The topology in which all sets of the form $U_0\times U_1\times U_2\times \cdots$ with each $U_k$ open (and not necessarily equal to all of $\R$)\footnote{This is called the \term{box topology}\index{Box topology}).} might be your more naive guess, but it is `wrong' in the sense that, if we allow things like this, then we lose the property that the continuity of a function is determined by the continuity of the components of the function---see \cref{exm4.5.20} below.
\end{rmk}
\begin{rmk}
Unless otherwise stated, products of topological spaces are always equipped with the product topology.
\end{rmk}
\begin{proof}
All of this follows from the definition of the initial topology, its characterization in terms of continuity of functions into initial topologies, and the defining result of generating collections (\cref{InitialTopology,prp3.4.6,GeneratingCollection}).
\end{proof}
\end{prp}
We mentioned in the remarks of this theorem that if you take as a base sets of the form $\prod _{X\in \collection{X}}S_X$ (with $S_X\neq X$ \emph{for all $X\in \collection{X}$} permissible), then you will lose the property that a function into the product is continuous iff each of its components is.  We now present a counter-example.
\begin{exm}{A discontinuous function into the box topology with each component continuous}{exm4.5.20}
Define \emph{as a set}
\begin{equation}
X\coloneqq \prod _{2^{\N}}\R 
\end{equation}
and consider the function
\begin{equation}
\id _X:\coord{X,\text{product topology}}\rightarrow \coord{X,\text{box topology}}.
\end{equation}
Then, each component of this function is continuous because the component of the identity is just the projection from $X$ onto the corresponding copy of $\R$ (the preimage of $U\subseteq \R$ under this projection is open in the product topology because in fact, according to the theorem, such an element is in a generating collection of the product topology).  On the other hand,
\begin{equation}
\prod _{2^{\N}}(0,1)
\end{equation}
is open in the box topology (by definition), but not open in the product topology by the theorem above.
\end{exm}
There is a relatively useful corollary of the definition of the product topology that characterizes convergence.
\begin{crl}{}{crl4.5.15}
Let $\collection{X}$ be an indexed collection of topological spaces, let $\lambda \mapsto x_\lambda \in \prod _{X\in \collection{X}}X$ be a net, and let $x_\infty \in \prod _{X\in \collection{X}}X$.  Then, $\lambda \mapsto x_\lambda$ converges to $x_\infty$ iff $\lambda \mapsto [x_\lambda ]_X$ converges to $[x_\infty ]_X$ in $X$ for all $X\in \collection{X}$.\footnote{To clarify, $[x_{\infty}]_X$ is the $X$-component of $x_{\infty}\in \prod _{X\in \collection{X}}X$.}
\begin{rmk}
In other words, nets converge to an element in a product iff every component converges to the corresponding component of that element.
\end{rmk}
\begin{proof}
As the product topology on $\prod _{X\in \collection{X}}X$ is the initial topology with respect to the projections, $\{ \pi _X:X\in \collection{X}\}$, this result follows from \cref{prp4.4.5}.
\end{proof}
\end{crl}
\begin{exr}{Closure of a product is the product of the closures}{}
Let $\collection{X}$ be an indexed collection of topological spaces, and for each $X\in \collection{X}$ let $S_X\subseteq X$.  Show that $\Cls \left( \prod _{X\in \collection{X}}S_X\right) =\prod _{X\in \collection{X}}\Cls (S_X)$.
\end{exr}
\begin{exr}{Projections are open}{ProjectionsAreOpen}
Let $\collection{X}$ be an indexed collection of topological spaces, let $X\in \collection{X}$, and let $U\subseteq \prod _{X\in \collection{X}}X$ be open.  Show that $\pi _X(U)\subseteq X$ is open.
\begin{rmk}
Functions that have the property that the image of open sets are open are \term{open functions}\index{Open function}.
\end{rmk}
\end{exr}

Now that we have defined the product topology, we prove a relatively difficult\footnote{At least compared to what we've been doing, though really the `meat' is contained in the \namerefpcref{AlexanderSubbaseTheorem}.} result concerning quasicompactness of products which will allow us to produce the desired counter-example.
\begin{thm}{Tychonoff's Theorem}{TychonoffsTheorem}\index{Tychonoff's Theorem}
Let $\collection{X}$ be a collection of quasicompact spaces.  Then, $\prod _{X\in \collection{X}}X$ is quasicompact.
\begin{proof}\footnote{Proof adapted from \cite[pg.~143]{Kelley}.}
Recall that the product topology on $\prod _{X\in \collection{X}}X$ has a generating collection of the form $\pi _X^{-1}(U_X)$ for $U_X\subseteq X$ open, where $\pi _X:\prod _{X\in \collection{X}}X\rightarrow X$ is the projection.  We apply the \namerefpcref{AlexanderSubbaseTheorem} to this generating collection.

So, let $\cover{U}$ be an open cover of $\prod _{X\in \collection{X}}X$ of subsets of the form $\pi _X^{-1}(U_X)$.  It suffices to show that if no finite subset of $\cover{U}$ covers $\prod _{X\in \collection{X}}X$, then $\cover{U}$ itself does not cover $\prod _{X\in \collection{X}}X$.  Define
\begin{equation}
\cover{U}_X\coloneqq \left\{ U\subseteq X\text{ open}:\pi _X^{-1}(U)\in \cover{U}\right\} .
\end{equation}
If a finite subset of $\cover {U}_X$ covered $X$, then its preimage would cover $\prod _{X\in \collection{X}}X$, and so by quasicompactness of $X$, it follows that $\cover{U}_X$ does not cover $X$, so choose $x_X\in X$ not contained in $\bigcup _{U\in \cover {U}_X}U$.  Then, the element $x\in \prod _{X\in \collection{X}}X$ whose coordinate at $X\in \collection{X}$ is $x_X$ is not contained in $\bigcup _{U\in \topology{U}}U$, so that $\cover{U}$ in fact does not cover $\prod _{X\in \collection{X}}X$.
\end{proof}
\end{thm}

And finally we are able to present our counter-example.
\begin{exm}{A quasicompact space with a net that has no convergent \emph{cofinal} subnet}{Wofsey}\footnote{This example was shown to me by Eric Wofsey on \href{http://mathoverflow.net/questions/210947/a-quasicompact-space-with-a-net-that-contains-no-convergent-strict-subnet}{mathoverflow.net}.}
Define
\begin{equation}
X\coloneqq \prod _{S\subseteq \N}\{ 0,1\} =\{ 0,1\} ^{2^{\N}}
\end{equation}
that is, a product of $2^{\aleph _0}$ copies of the two element set $\{ 0,1\}$.  In other words, it is the set of all functions from $2^{\N}$ into $\{ 0,1\}$---in fact, for the most of this example, we shall think of this space as the collection of functions.   This is quasicompact by Tychonoff's Theorem (and because finite sets are quasicompact---see \cref{exr4.2.33x}).  On the other hand, we may define a sequence $m\mapsto x_m\in X$ as follows.
\begin{equation}
x_m(S)\coloneqq \begin{cases}1 & \text{if }m\in S \\ 0 & \text{if }m\notin S,\end{cases}
\end{equation}
where $S\subseteq \N$.  ($x_m$ is a function from $2^{\N}$ into $\{ 0,1\}$, and so $x_m(S)$ is the value of this function at the element $S\in 2^{\N}$.)

We now show that this sequence has no convergent cofinal subnet (necessarily also a sequence).  We proceed by contradiction:  suppose that there were a cofinal subset $\Lambda '\subseteq \N$, $\Lambda '=\{ m_0,m_1,m_2,\ldots \}$ (with $m_n\leq m_{n+1}$), such that $n\mapsto x_{m_n}$ converges.  Then, by \cref{crl4.5.15} (nets in products converge iff each component does), for each $S\subseteq \N$, the sequence $n\mapsto x_{m_n}(S)\in \{ 0,1\}$ would have to converge.  Thus, for each $S\subseteq \N$, the sequence $n\mapsto x_{m_n}(S)$ would have to be either eventually $0$ or eventually $1$.  In other words, either (i)~for all but finitely many $m_n\in \Lambda '$, $m_n\notin S$;  or (ii)~for all but finitely many $m_n\in \Lambda '$, $m_n\in S$.  In other words, for all $S\subseteq \N$, either (i)~there is a cofinite\footnote{Just as cocountable means that the complement is countable, \term{cofinite}\index{Cofinite} means that the complement is finite.} subset of $\Lambda '$ that is contained in $S^{\comp}$ or (ii)~there is a cofinite subset of $\Lambda '$ that is contained in $S$.\footnote{Cofinite in $\Lambda '$, that is.}

So, take $S\coloneqq \{ m_0,m_2,m_4,\ldots \} \subset \Lambda '$.  Then, there is some cofinite subset $\Lambda ''\subseteq \Lambda '$ such that either $\Lambda ''\subseteq S$ or $\Lambda ''\subseteq S^{\comp}$.  In the former case, we have that
\begin{equation}
\text{finite set }=\Lambda '\setminus \Lambda '' \supseteq \Lambda '\setminus S=\{ m_1,m_3,m_5,\ldots \} :
\end{equation}
a contradiction.  Thus, we must have that $\Lambda ''\subseteq S^{\comp}$, and so
\begin{equation}
\text{finite set }=\Lambda '\setminus \Lambda ''\supseteq \Lambda '\setminus S^{\comp}=\{ m_0,m_2,m_4,\ldots \} :
\end{equation}
a contradiction.  As both possibilities resulted in a contradiction, this itself is a contradiction, and so our assumption that there was a convergent cofinal subnet must have been incorrect.  Therefore, $m\mapsto x_m$ contains no convergent cofinal subnet, despite the fact that $X$ is quasicompact.
\end{exm}

\subsection{The disjoint-union topology}

Our inclusion of the disjoint-union topology is mostly because of its obvious duality to the product topology---it feels incomplete not to include it.  On the other hand, while in principle, being completely dual to the product topology, it should be no more or less difficult work with, in practice it seems that it is \emph{much} easier to get a handle on, and so probably doesn't deserve as in-depth a treatment.
\begin{prp}{Disjoint-union topology}{DisjointUnionTopology}
Let $\collection{X}$ be an indexed collection of topological spaces.  Then, there exists a unique topology, the \term{disjoint-union topology}\index{disjoint-union topology}, on $\coprod _{X\in \collection{X}}X$, that has the property that a function defined on $\coprod _{X\in \collection{X}}X$ is continuous iff its restriction to each component is continuous.  Furthermore, the disjoint-union topology is the final topology with respect to $\{ \iota _X:X\in \collection{X}\}$.\footnote{Recall (\cref{DisjointUnionCollection}) that $\iota _X:X\rightarrow \coprod _{X\in \collection{X}}X$ is just the inclusion.}  In particular, a set $U\subseteq \coprod _{X\in \collection{X}}X$ is open iff $U\cap \iota _X(X)$ is open for all $X\in \collection{X}$.
\begin{rmk}
Unless otherwise stated, disjoint-unions of topological spaces are always equipped with the disjoint-union topology.
\end{rmk}
\begin{wrn}
Warning:  Just because $X=S\cup T$ and $S\cap T=\emptyset$, does \emph{not} mean that $X$ has the disjoint-union topology.  For example, by that logic, as every space is the disjoint-union of its points, every space would be discrete.  When I point this out to you now, it could very well appear that this deduction is so obviously incorrect as to not be worth mentioning, but I have definitely seen students make what is essentially this very same mistake in contexts where it is not as blatant that this is the (wrong) implication one is using.
\end{wrn}
\begin{proof}
All of this follows from the definition of the final topology and its characterization in terms of continuity of functions defined on final topologies (\cref{FinalTopology,prp3.4.34x}).
\end{proof}
\end{prp}

\section{Separation Axioms}\label{sct4.5}

We have mentioned the term ``$T_2$'' a couple of times now (see, for example, the definition of quasicompactness (\cref{Quasicompact}) and the \namerefpcref{HeineBorelTheorem}.  One of the purposes of this section is to explain what we meant by this.  The term ``$T_2$'' is a separation axiom, and you should know the other separation axioms as well if you plan to become a mathematician, but this is admittedly not a priority for this course (a study of them in detail is better suited for a course on general topology itself).\footnote{Says the person who decided to include the most detailed account I'm aware of (except for probably \cite{Steen}) \textellipsis .}

\subsection{Separation of subsets}

In this subsection, we will define various levels of ``separation'' of \emph{subsets} of a topological space.  In the next section, we will then define corresponding levels of separation for \emph{spaces}.

Throughout this subsection, let $S_1,S_2\subseteq X$ be \emph{disjoint} subsets of a topological space $X$.  In the various definitions will follow, we will say things like ``$S_1$ and $S_2$ are XYZ.''.  If $S_1=\{ x_1\}$ and $S_2=\{ x_2\}$ are singletons, then instead we will say that ``$x_1$ and $x_2$ are XYZ.''.  In fact, this\footnote{That is, the case when the sets are singletons.} is probably the case of most interest (though certainly not the only case).
\begin{dfn}{Topologically-distinguishable\hfill}{TopologicallyDistinguishable}
$S_1$ and $S_2$ are \term{topologically-distinguishable}\index{Topologically-distinguishable} iff there is a neighborhood of $S_1$ not intersecting $S_2$ \emph{or} there is a neighborhood of $S_2$ not intersecting $S_1$.
\begin{wrn}
Warning:  The term ``distinguishable'' comes from the fact that, in the case $S_1=\{ x_1\}$ and $S_2=\{ x_2\}$, then $x_1$ and $x_2$ being topologically-indistinguishable (i.e.~not topologically-distinguishable) means that $x_1$ and $x_2$ are contained in precisely the same open sets.  Thus, in this sense, the topology cannot tell the difference between $x_1$ and $x_2$, they are ``indistinguishable''.  However, if $S_1$ and $S_2$ are not singletons, then this need not be true, that is, the open sets containing $S_1$ and $S_2$ need not coincide---see \cref{exm3.6.4}.
\end{wrn}
\begin{rmk}
For example, in an indiscrete space, no two points are topologically-distinguishable.  In $\R$ on the other hand (and almost every space you work with that is not expressly cooked-up for the sole purpose of being a counter-example to something), any two points are topologically-distinguishable.
\end{rmk}
\end{dfn}
\begin{exm}{Two distinct points which are not top\-ologically-distinguishable}{exm4.5.2}
We just mentioned this in the remark above, but decided to place it in an example of its own to make it easier to spot if skimming.  Take $X\coloneqq \{ x_1,x_2\}$ and equip $X$ with the indiscrete topology, that is,
\begin{equation}
\topology{U}\coloneqq \{ \emptyset ,X\} .
\end{equation}
Then, $x_1\neq x_2$ but $x_1$ and $x_2$ are contained in precisely the same open sets.
\end{exm}
\begin{exm}{Two sets which are topo\-logically-indistinguish\-able but don't have the same open neighborhoods}{exm3.6.4}
Define $X\coloneqq \{ x_1,x_2,x_3\}$ and
\begin{equation}
\topology{U}\coloneqq \left\{ \emptyset ,X,\{ x_1,x_2\} \right\} .
\end{equation}
Define $S_1\coloneqq \{ x_1\}$ and $S_2\coloneqq \{ x_2,x_3\}$.  Then, every neighborhood of $S_1$ (there are only two, $\{ x_1,x_2\}$ and $X$) intersects $S_2$.  Similarly, every neighborhood of $S_2$ (there is only one, $X$) intersects $S_1$.  Thus, $S_1$ and $S_2$ are topologically-indistinguishable.  On the other hand, $\{ x_1,x_2\}$ is an open set which contains $S_1$ but not $S_2$.
\end{exm}
\begin{dfn}{Separated}{Separated}
$S_1$ and $S_2$ are \term{separated}\index{Separated} iff there is a neighborhood $U_1$ of $S_1$ not intersecting $S_2$ \emph{and} a neighborhood $U_2$ of $S_2$ not intersecting $S_1$.
\begin{rmk}
The difference between this and topological-distinguishability is that this has to happen to \emph{both} $S_1$ and $S_2$, whereas, in the case of topological-distinguishability, we only require that at least one of them has an open neighborhood that does not intersect the other.
\end{rmk}
\end{dfn}
\begin{exm}{Two points which are topolo\-gically-distinguish\-able but not separated}{exm4.5.3}
Define $X\coloneqq \{ x_1,x_2\}$ and
\begin{equation}
\topology{U}\coloneqq \left\{ \emptyset ,X,\{ x_1\} \right\} .
\end{equation}
Then, $x_1$ and $x_2$ are topologically-distinguishable as $\{ x_1\}$ is an open neighborhood of $x_1$ that does not contain $x_2$.  On the other hand, every neighborhood of $x_2$ contains $x_1$.
\begin{rmk}
This is the \term{Sierpinski Space}\index{Sierpinski Space}.
\end{rmk}
\end{exm}
\begin{dfn}{Separated by neighborhoods}{SeparatedByNeighborhoods}\index{Separated by neighborhoods}
$S_1$ and $S_2$ are \term{separated by neighborhoods}\index{Separated by neighborhoods} iff there is a neighborhood $U_1$ of $S_1$ and a neighborhood $U_2$ of $S_2$ with $U_1$ and $U_2$ disjoint.
\begin{rmk}
Equivalently, we may replace $U_1$ and $U_2$ with open neighborhoods.
\end{rmk}
\begin{rmk}
This is just like being separated, except that we may put $U_1$ around $S_1$ and $U_2$ around $S_2$ \emph{`simultaneously'} and have no intersection, whereas in the separated case, the $U_1$ and $U_2$ that `work' will in general intersect.
\end{rmk}
\end{dfn}
\begin{exm}{Two points which are separated but not separated by neighborhoods}{exm4.5.8}
Define $X\coloneqq \{ x_1,x_2,x_3\}$ and
\begin{equation}
\topology{U}\coloneqq \left\{ \emptyset ,X,\{ x_1,x_3\} ,\{ x_2,x_3\} ,\{ x_3\} \right\} .
\end{equation}
Then, $\{ x_1,x_3\}$ is an open neighborhood of $x_1$ which does not contain $x_2$ and $\{ x_2,x_3\}$ is an open neighborhood of $x_2$ which does not contain $x_1$.  On the other hand, every open neighborhood of $x_1$ intersects every open neighborhood of $x_2$ at $x_3$.
\end{exm}
\begin{dfn}{Separated by closed neighborhoods}{SeparatedByClosedNeighborhoods}
$S_1$ and $S_2$ are \term{separated by closed neighborhoods}\index{Separated by closed neighborhoods} iff there is a closed neighborhood $C_1$ of $S_1$ and a closed neighborhood $C_2$ of $S_2$ with $C_1$ and $C_2$ disjoint.
\begin{rmk}
This is the same as being separated by neighborhoods, except that we further require that the neighborhoods are closed.\footnote{Recall that neighborhoods do not have to be open---see \cref{Neighborhood}.}
\end{rmk}
\end{dfn}
\begin{exm}{Two points which are separated by neighborhoods but not by closed neighborhoods}{exm4.5.11}
Define $X\coloneqq \{ x_1,x_2,x_3\}$ and
\begin{equation}
\topology{U}\coloneqq \left\{ \emptyset ,X,\{ x_1\} ,\{ x_3\} ,\{ x_1,x_3\} \right\} .
\end{equation}
Then, $\{ x_1\}$ is a neighborhood of $x_1$, $\{ x_3\}$ is a neighborhood of $x_3$, and these two neighborhoods are disjoint, so that $x_1$ and $x_3$ are separated by neighborhoods.  On the other hand, $x_1$ only has four neighborhoods:  $\{ x_1\}$, $\{ x_1,x_3\}$, $X$, and $\{ x_1,x_2\}$.\footnote{Note that $\{ x_1,x_2\}$ is a neighborhood of $x_1$, but \emph{not} an open neighborhood.}  Of these, only two are closed, $X$ and $\{ x_1,x_2\}$.  $X$ certainly intersects every closed neighborhood of $x_3$, and so, if $x_1$ and $x_3$ are to be separated by closed neighborhoods, the closed neighborhood of $x_1$ must be $\{ x_1,x_2\}$.  By $1\leftrightarrow 3$ symmetry, the only closed neighborhood of $x_3$ that might `work' is $\{ x_2,x_3\}$, however, these two closed neighborhoods intersect, namely at $x_2$.  Therefore, we cannot separate $x_1$ and $x_3$ with \emph{closed} neighborhoods.
\end{exm}
There is an equivalent, alternative way to think about being separated by closed neighborhoods that you may find useful.
\begin{prp}{}{prp4.5.13}
$S_1$ and $S_2$ are separated by closed neighborhoods iff they are separated by open neighborhoods with disjoint closure.
\begin{proof}
$(\Rightarrow )$ Suppose that $S_1$ and $S_2$ are separated by closed neighborhoods $C_1$ and $C_2$ respectively.  By the definition of a neighborhood (\cref{Neighborhood}), there are then open neighborhoods $U_1$ and $U_2$ with $S_1\subseteq U_1\subseteq C_1$ and $S_2\subseteq U_2\subseteq C_2$.  Then, $\Cls (U_1)\subseteq C_1$ and $\Cls (U_2)\subseteq C_2$, and so $U_1$ and $U_2$ constitute open neighborhoods of $S_1$ and $S_2$ with disjoint closures.

\blankline
\noindent
$(\Leftarrow )$ Suppose that $S_1$ and $S_2$ are separated by open neighborhoods with disjoint closure.  Then, these closures constitute closed neighborhoods which separate $S_1$ and $S_2$.
\end{proof}
\end{prp}
\begin{dfn}{Completely-separated}{CompletelySeparated}
$S_1$ and $S_2$ are \term{complete\-ly-separated}\index{Completely-separated} or \term{separated by (continuous) functions}\index{Separated by continuous functions} iff there is a continuous function $f\colon X\rightarrow [0,1]$ such that $\restr{f}{S_1}=0$ and $\restr{f}{S_2}=1$.
\begin{rmk}
Why does being completely-separated imply being separated by closed neighborhoods?\footnote{All the other implications are true too (i.e.~separated implies topologically-distinguishable, separated by neighborhoods implies separated, etc.), this is just the first one that is not completely obvious, which is why it is the only one we have asked about.}
\end{rmk}
\end{dfn}
\begin{exm}{Two points which are separated by closed neighborhoods but not completely-separated}{ArensSquare}\footnote{This is significantly more nontrivial than the preceding counter-examples and comes from \cite[pg.~98]{Steen}.}
Define
\begin{equation}
\begin{split}
S\coloneqq & \left[ (0,1)\times (0,1)\right] \cap [\Q \times \Q ] \\
& =\left\{ \coord{x,y}\in (0,1)\times (0,1):x,y\in \Q \right\} ,
\end{split}
\end{equation}
\begin{equation}
\begin{split}
T\coloneqq & \{ \tfrac{1}{2}\} \times \{ r\sqrt{2}\in (0,1):r\in \Q \} \\
& =\{ \coord{\tfrac{1}{2},r\sqrt{2}}:r\in \Q ,\ r\sqrt{2}\in (0,1) \} 
\end{split}
\end{equation}
and
\begin{equation}
X\coloneqq S\cup T\cup \{ \coord{0,0})\} \cup \{ \coord{1,0})\}.
\end{equation}
We define a topology on $X$ by defining a neighborhood base at each point (see \cref{prp4.1.8}).  For $\coord{x,y}\in X$, there are four cases:  (i)~$\coord{x,y}=\coord{0,0}$, (ii)~$\coord{x,y}=\coord{1,0}$, (iii)~$\coord{x,y}\in T$, and (iv)~$\coord{x,y}\in S$.  We define
{\scriptsize
\begin{equation}
\topology{B}_{\coord{x,y}}\coloneqq \begin{cases}\left\{ U\subseteq S:U\text{ is open in }S\text{.}\footnote{Open in the usual topology (the subspace topology inherited from $(0,1)\times (0,1)$).}\right\} & \text{if }\coord{x,y}\in S \\ \left\{ U_{\coord{x,y}}^m:m\in \Z ^+\right\} & \text{if }\coord{x,y}\in T \\ & \enspace \text{or }x=\coord{0,0};\coord{1,0},\end{cases}
\end{equation}
}
where
{\scriptsize
\begin{equation}\label{4.5.20}
\begin{split}
U_{\coord{0,0}}^m & \coloneqq \{ \coord{0,0}\} \cup \left\{ \coord{x,y}\in (0,\tfrac{1}{4})\times (0,\tfrac{1}{m}):x,y\in \Q \right\} \\
U_{\coord{1,0}}^m & \coloneqq \{ \coord{1,0}\} \cup \left\{ (x,y)\in (\tfrac{3}{4},1)\times (0,\tfrac{1}{m}):x,y\in \Q \right\} \\
U_{\coord{\tfrac{1}{2},r\sqrt{2}}}^m & \coloneqq \left\{ \coord{x,y}\in (\tfrac{1}{4},\tfrac{3}{4})\times (r\sqrt{2}-\tfrac{1}{m},r\sqrt{2}+\tfrac{1}{m}):\right. \\ & \qquad \left. x,y\in \Q \right\} \text{ for }m\in \Z ^+\text{ with } \\
& \qquad \left( r\sqrt{2}-\tfrac{1}{m},r\sqrt{2}+\tfrac{1}{m}\right) \subseteq (0,1).
\end{split}
\end{equation}
}
By \cref{prp4.1.8}, there is a unique topology for which $\topology{B}_{\coord{x,y}}$ is a neighborhood base of $\coord{x,y}\in X$.

The closures of $\{ \coord{0,0}\} \cup (0,\frac{1}{4})\times (0,\frac{1}{n})$ and $\{ \coord{1,0}\} \cup (\frac{3}{4},1)\times (0,\frac{1}{n})$ in $X$ must be disjoint as, in particular, any point in the former cannot have $x$-coordinate exceeding $\frac{1}{4}$ and any point in the latter cannot have $x$-coordinate strictly less than $\frac{3}{4}$.

On the other hand, $\coord{0,0}$ and $\coord{1,0}$ cannot be separated by a function.  To see this, suppose that $f\colon X\rightarrow \R$ were a continuous function such that $f(\coord{0,0})=0$ and $f(\coord{1,0})=1$.  Then, $f^{-1}([0,\frac{1}{4}))$ would be an open neighborhood of $\coord{0,0}$, and so must contain $U_{\coord{0,0}}^m$ for some $m\in \Z ^+$.  Similarly, $f^{-1}((\frac{3}{4},1])$ must contain $U_{\coord{1,0}}^n$ for some $n\in \Z ^+$.  Let $r\in \Q$ be such that $r\sqrt{2}<\min \{ \frac{1}{m},\frac{1}{n}\}$.  Obviously, $f(\coord{\frac{1}{2},r\sqrt{2}})$ cannot be in both $[0,\frac{1}{4})$ and $(\frac{3}{4},1]$ as these sets are disjoint, so without loss of generality assume that it is not contained in $[0,\frac{1}{4})$, so let $U \subseteq [0,1]$ be an open neighborhood of $f(\coord{\frac{1}{2},r\sqrt{2}})$ with $\Cls (U)$ disjoint from $\Cls \left( [0,\frac{1}{4})\right) $, so that the preimages of $\Cls (U)$ and $\Cls \left( [0,\frac{1}{4})\right)$ are disjoint closed neighborhoods of $\coord{\frac{1}{2},r\sqrt{2}}$ and $\coord{0,0}$ respectively.  On the other hand, a closed neighborhood of $\coord{\frac{1}{2},r\sqrt{2}}$ must contain $U_{\coord{\frac{1}{2},r\sqrt{2}}}^o$ for $o\in \Z ^+$ with $r\sqrt{2}-\frac{1}{o}>0$ (and $r\sqrt{2}+\frac{1}{o}<1$).  As $r<\frac{1}{\sqrt{2}m}<\frac{1}{m}$, we have that $\frac{1}{o}<\frac{\sqrt{2}}{m}$, and hence $r\sqrt{2}-\frac{1}{o}<\frac{1}{m}-\frac{1}{o}<\frac{1}{m}$, and so there is some rational $q\in \Q$ with $r\sqrt{2}-\frac{1}{o}<q<\frac{1}{m}$.  But then, $\Cls \left( U_{\coord{0,0}}^m\right)$ and $\Cls \left( U_{\coord{\frac{1}{2},r\sqrt{2}}}^o\right)$ must intersect at $\coord{\frac{1}{4},q}$,  a contradiction of disjointedness.
\begin{rmk}
This is the \term{Arens Square}\index{Arens Square}.
\end{rmk}
\end{exm}
\begin{dfn}{Perfectly-separated}{}
$S_1$ and $S_2$ are \term{perfectly-separated}\index{Perfectly-separated} or \term{precisely-separated}\index{Precisely-separated} iff there is a continuous function $f\colon X\rightarrow [0,1]$ such that $S_1=f^{-1}(0)$ and $S_2=f^{-1}(1)$.
\begin{rmk}
Being completely-separated means that $f$ is $0$ on $S_1$ and $1$ on $S_2$.  Being perfectly-separated means that, furthermore, $f$ is $0$ \emph{nowhere else} except on $S_1$ and $f$ is $1$ \emph{nowhere else} except on $S_2$.
\end{rmk}
\end{dfn}
\begin{exm}{Two points which are completely-separated but not perfectly-separated}{UncountableFortSpace}\footnote{This comes from \cite[pg.~52]{Steen}.}
This example is fairly similar to the cocountable topology example---see \cref{exm4.2.8x}.

Define $X\coloneqq \R$.  Let $C\subseteq X$ and declare that
\begin{textequation}
$X$ is closed iff either (i)~$C$ contains $0$ or (ii)~$C$ is finite.
\end{textequation}
You can check for yourself that this satisfies the defining conditions for a topology in terms of closed sets (\cref{exr4.1.2}).

We wish to show that $0,1\in X$ are completely-separated, but not perfectly-separated.  We first check that they are completely-separated by producing a continuous function $f\colon X\rightarrow [0,1]$ such that $f(0)=0$ and $f(1)=1$.  Define
\begin{equation}
f(x)\coloneqq \begin{cases}1 & \text{if }x=1 \\ 0 & \text{otherwise.}\end{cases}
\end{equation}
We first check that $f$ is continuous.  Let $C\subseteq [0,1]$ be closed.  If $C$ contains $0\in [0,1]$, then $f^{-1}(C)$ contains $0\in X$, and so is closed.  If it does not contain $0$, then $f^{-1}(C)$ is finite---either $C$ contained $1$ in which case $f^{-1}(C)=\{ 1\}$ or it did not in which case $f^{-1}(C)=\emptyset$.

Now we show that $0,1\in X$ are not \emph{perfectly}-separated.  Suppose that there exists a continuous function $f\colon X\rightarrow [0,1]$ such that $\{ 0\} =f^{-1}(0)$ and $\{ 1\} =f^{-1}(1)$.  Then,
\begin{equation}\label{4.5.23}
\begin{split}
\{ 0\} & =f^{-1}(0)=f^{-1}\bigg( \bigcap _{m\in \Z ^+}[0,\tfrac{1}{m})\bigg) \\
& =\footnote{\cref{exrA.1.30}.\cref{enmA.1.30.ii}}\bigcap _{m\in \Z ^+}f^{-1}\bigg( [0,\tfrac{1}{m})\bigg) .
\end{split}
\end{equation}
That is, $\{ 0\}$ is a $G_\delta$ set.\footnote{Recall that this is just a fancy-shmancy term for a set which is a countable intersection of open sets---see \cref{GDeltaFSigma}.}  However, by definition, a set is open iff it does not contain $0\in X$ or its complement is finite.  Of course, all the sets appearing in \eqref{4.5.23} must be of the latter kind (because $0\in f^{-1}\left( [0,\frac{1}{m})\right)$).  However, taking the complement of this equation, we find
\begin{equation}
\R \setminus \{ 0\} \footnote{De Morgan's Laws---see \cref{DeMorgansLaws}}=\bigcup _{m\in \Z ^+}f^{-1}\left( [0,\tfrac{1}{m})\right) ^\comp ,
\end{equation}
so that $\R \setminus \{ 0\}$ is a countably-infinite union of finite sets:  a contradiction.
\begin{rmk}
This is the \term{Uncountable Fort Space}\index{Uncountable Fort Space}.
\end{rmk}
\end{exm}

Note that we obviously have the implications
\begin{textequation}
perfectly-separated $\Rightarrow$ completely-separated $\Rightarrow $ separated by closed neighborhoods $\Rightarrow $ separated by neighborhoods $\Rightarrow$ separated $\Rightarrow$ topologically-distinguishable $\Rightarrow$ distinct.
\end{textequation}
Here, ``distinct'' literally means that $S_1$ and $S_2$ are not the same thing, i.e. $S_1\neq S_2$.  Furthermore, we have presented counter-examples after each definition to show that each implication is strict.

\subsection{Separation axioms of spaces}\label{sbs3.6.2}

In the previous subsection, we defined several levels of ``separation'' between two disjoints subsets of a topological space.  We now use these definitions to put conditions on topological spaces themselves.

Throughout this section, $X$ will be a topological space.
\begin{important}
This is the first time in these notes that a result is stated `officially' that refers to material we have not yet covered.  For example, we are going to introduce a bunch of properties that a space may or may not possess, and for each of them, we are going to ask the question ``Is this preserved under subspaces, quotients, products, and disjoint unions?''.  In some cases, to answer the questions, we needed to refer to material coming as late as \cref{chp5x}, for example, to refer to the natural log.  However, the logical placement of such material is certainly in this section, in parallel with all the other analogous results, and so I didn't feel as if it made pedagogical sense to postpone such results by several chapters just because we don't `officially' know what $\ln$ is yet---the reality is that, if you're reading this, you almost certainly know\footnote{Or think you know anyways.} what $\ln$ is, and in any case, the development isn't actually circular, and you can flip to the definition (\cref{NaturalLogarithm}) and come back to the result when you feel comfortable.
\end{important}

\begin{dfn}{$T_0$}{T0}
$X$ is \term{$T_0$}\index{$T_0$} iff any two distinct points are topologically-distinguishable.
\begin{rmk}
Sometimes this condition is called \term{Kolmogorov}\index{Kolmogorov (topological space)}.  You will find that a lot (if not all) of the separation axioms of spaces have other names.  We have chosen the names we have because (i)~other terminology is less consistent and (ii)~it carries less information (the subscript $0$ in $T_0$ has some significance).
\end{rmk}
\begin{rmk}
This is an insanely reasonable condition.  I might even argue that if you have a space you're trying to study that is not $T_0$, you're doing something wrong.  If the topology cannot distinguish between two points, either (i)~you may as well identify those two points (see \cref{T0Quotient}) or (ii)~you should probably consider adding more structure to your space that does distinguish them.
\end{rmk}
\begin{rmk}
Apparently the ``$T$'' in all these separation axioms is for the German ``Trennungsaxiom'', or so Munkres (\cite[pg.~211]{Munkres}) informs me.
\end{rmk}
\end{dfn}
\begin{exm}{A space which is not $T_0$}{}
Any indiscrete space with at least two points.  The example above in \cref{exm4.5.2} worked.
\end{exm}
\begin{prp}{$T_0$ quotient}{T0Quotient}
Let $X$ be a topological space.  Then, there exists a unique\footnote{Up to homeomorphism.} topological space $\TZero (X)$, the \term{$T_0$ quotient}\index{$T_0$ quotient} of $X$, and a surjective map $\q :X\rightarrow \TZero (X)$ such that
\begin{enumerate}
\item \label{T0Quotient.i}$\TZero (X)$ is $T_0$; and
\item \label{T0Quotient.ii}if $Y$ is another $T_0$ space with a continuous map $\phi :X\rightarrow Y$, then there is a unique continuous map $\phi ':\TZero (X)\rightarrow Y$ such that $\phi =\phi '\circ \q$.
\end{enumerate}
\begin{rmk}
As $T_0$ is sometimes called Kolmogorov, so to this is sometimes called the \term{Kolmogorov quotient}\index{Kolmogorov quotient}.
\end{rmk}
\begin{rmk}
Compare this with the definitions of the integers, rationals, reals, closure, interior, generating collections, initial topology, and final topology (\cref{Integers,RationalNumbers,RealNumbers,Closure,Interior,InitialTopology,FinalTopology}).  Note how this is a bit different---the key difference here is that the map from $X$ to $\TZero (X)$ is now \emph{surjective} (i.e.~a quotient map) instead of in all the previous cases where it was \emph{injective} (i.e.~an inclusion).
\end{rmk}
\begin{proof}
Define $x_1\sim x_2$ iff the open sets which contain $x_1$ are precisely the same as the open sets which contain $x_2$.
\begin{exr}[breakable=false]{}{}
Show that $\sim$ is an equivalence relation.
\end{exr}
Define $\TZero (X)\coloneqq X/\sim$ and let $\q :X\rightarrow \TZero (X)$ be the quotient map (\cref{dfnA.1.42}).
\begin{exr}[breakable=false]{}{}
Show that $\TZero (X)$ satisfies \cref{T0Quotient.i} and \cref{T0Quotient.ii}.
\end{exr}
\begin{exr}[breakable=false]{}{}
Show that $\TZero (X)$ is unique.
\end{exr}
\end{proof}
\end{prp}
\begin{exr}{}{exr3.6.36}
Show that a subspace of a $T_0$ space is $T_0$.
\begin{wrn}
Warning:  While this might seem like it should be obvious, it's not true for other separation axioms (\cref{exm4.4.69}), and so there is indeed something to check.
\end{wrn}
\end{exr}
\begin{exr}{}{exr3.6.37}
Show that an arbitrary product of $T_0$ spaces is $T_0$.
\end{exr}
You'll notice that we didn't ask about quotients or disjoint unions.  This is because separation axioms are essentially never preserved under quotients\footnote{For example, $\R$ is perfectly-$T_4$ (see \cref{PerfectlyT4}), but it has a quotient which is not even $T_0$---see \cref{exm3.6.36}.} and are essentially always preserved under disjoint union.
\begin{exm}{A quotient of $\R$ that is not $T_0$}{exm3.6.36}
Consider the partition $\left\{ (-\infty ,0]\cup (1,\infty ),(0,1]\right\}$ of $\R$ and let $\sim$ denote the corresponding equivalence relation.\footnote{Recall that partitions define equivalence relations---see \cref{exrA.1.41}.}  We claim that $\R /\sim$ is not $T_0$.

Recall (\cref{QuotientTopology}) that a set is open in the quotient topology iff its preimage under the quotient map is open.  In this case, the quotient has just two points, one represented by $(-\infty ,0]\cup (1,\infty )$ and the other represented by $(0,1]$.  The preimage of both of these points are just the respective sets, neither of which are open, and hence neither of the points in the quotient is open.  As these are the only two nonempty proper subsets, there are no nonempty proper open sets in the quotient, that is, the quotient has the indiscrete topology, which is never $T_0$.\footnote{Except on a one point set and the empty-set.}
\end{exm}
\begin{dfn}{$T_1$}{T1}
$X$ is \term{$T_1$}\index{$T_1$} iff any two distinct points are separated.
\begin{rmk}
Sometimes this condition is called \term{accessible}\index{Accessible (topological space)} of \term{Fréchet}\index{Fréchet (topological space)}.  I also prefer $T_1$ over ``accessible'' because, not only does it carry slightly more information, but it's also a lot more common.  I would \emph{definitely} recommend not to use the term ``Fréchet'' to describe this, as the term ``Fréchet space'' is usually meant to describe something else entirely.
\end{rmk}
\begin{wrn}
Warning:  This is not the same as ``any two topologically-distinguishable points are separated''.  That condition is called $R_0$ and is rarely, if ever, used.\footnote{There is no difference between $R_0$ and $T_1$ for spaces which are $T_0$.}
\end{wrn}
\begin{rmk}
In contrast to the $T_0$ condition, there are \emph{many} important examples of spaces that are not $T_1$.
\end{rmk}
\end{dfn}
\begin{exm}{A space that is $T_0$ but not $T_1$}{}
The Sierpinski Space (\cref{exm4.5.3}) worked (briefly, it was a two point set with the only nontrivial open set being a singleton).
\end{exm}
While the above is probably the best way to state $T_1$ as a definition because it is phrased in a way that is analogous to other separation conditions, it is often best to think of $T_1$ spaces are spaces in which points are closed.
\begin{prp}{}{prp4.5.32}
Let $X$ be a topological space.  Then, $X$ is $T_1$ iff $\{ x\}$ is closed for all $x\in X$.
\begin{proof}
$(\Rightarrow )$ Suppose that $X$ is $T_1$.  Let $x\in X$.  For all $y\in X$ distinct from $x$, let $U_y$ be an open neighborhood of $y$ that does not contain $x$.  Then,
\begin{equation}
\bigcup _{y\neq x}U_y=X\setminus \{ x\}
\end{equation}
is open, and so
\begin{equation}
\bigcap _{y\neq x}U_y^{\comp}=\{ x\}
\end{equation}
is closed.

\blankline
\noindent
$(\Leftarrow )$ Suppose that $\{ x\}$ is closed for all $x\in X$.  Let $x_1,x_2\in X$.  Then, $\{ x_1\} ^{\comp}$ is an open neighborhood of $x_2$ that does not contain $x_1$ and $\{ x_2\} ^{\comp}$ is an open neighborhood of $x_1$ that does not contain $x_2$.
\end{proof}
\end{prp}
\begin{crl}{}{}
Every finite $T_1$ topological space is discrete.
\begin{rmk}
In particular, it is a waste of time to look for counter-examples in finite spaces if your space is $T_1$.
\end{rmk}
\begin{proof}
If the space is $T_1$, then every point is closed.  If the space is finite, then every subset is a union of finitely many points, that is, a finite union of closed sets, and hence closed.
\end{proof}
\end{crl}
\begin{exr}{}{exr3.6.45}
Show that a subspace of a $T_1$ space is $T_1$.
\end{exr}
\begin{exr}{}{exr3.6.46}
Show that an arbitrary product of $T_1$ spaces is $T_1$.
\end{exr}
\begin{dfn}{$T_2$}{T2}
$X$ is \term{$T_2$}\index{$T_2$} iff any two distinct points are separated by neighborhoods.
\begin{rmk}
This is quite often referred to as \term{Hausdorff}\index{Hausdorff (topological space)}.  In contrast to most of the alternative terminologies, this actually might be more common than the term ``$T_2$''.
\end{rmk}
\end{dfn}
\begin{dfn}{Compact}{Compact}
$X$ is \term{compact}\index{Compact} iff it is quasicompact and $T_2$.
\end{dfn}
\begin{prp}{}{prp4.5.37}
Let $X$ be a topological space.  Then, $X$ is $T_2$ iff limits are unique.
\begin{rmk}
Because of lack of uniqueness in general, we have been hesitant to write $\lim _\lambda x_\lambda$---if limits are not unique, then what limit does this symbol refer to?  Hereafter, however, in $T_2$ spaces, we will not hesitate to use this notation.
\end{rmk}
\begin{proof}
$(\Rightarrow )$ Suppose that $X$ is $T_2$.  Let $\lambda \mapsto x_\lambda \in X$ be a net and let $x_\infty ,x_\infty '\in X$ be limits of $X$.  We proceed by contradiction:  suppose that $x_\infty \neq x_\infty '$.  Then, there exist disjoint open neighborhoods $U$ of $x_\infty$ and $U'$ of $x_\infty '$.  As $\lambda \mapsto x_\lambda$ converges to $x_\infty$, it is eventually contained in $U$.  But then, as $U$ and $U'$ are disjoint, $\lambda \mapsto x_\lambda$ is not eventually contained in $U'$, a contradiction of the fact that $\lambda \mapsto x_{\lambda}$ converges to $x_{\infty}'$.

\blankline
\noindent
$(\Leftarrow )$ Suppose that limits are unique.  Let $x_\infty ,x_\infty '\in X$ be distinct.  We wish to show that there exist disjoint open neighborhoods of $x_\infty$ and $x_\infty '$.  We proceed by contradiction:  suppose that every open neighborhood of $x_\infty$ intersects every open neighborhood of $x_\infty '$.  Let $\collection{N}_{x_{\infty}}$ and $\collection{N}_{x_{\infty}'}$ denote the collection of all open neighborhoods of $x_{\infty}$ and $x_{\infty}'$ respectively.  Order them with respect to reverse-inclusion so as to form directed sets, and equip $\collection{N}_{x_{\infty}}\times \collection{N}_{x_{\infty}'}$ with the product order (\cref{ProductOrder}).  By hypothesis, for all $\coord{U,U'}\in \collection{N}_{x_{\infty}}\times \collection{N}_{x_{\infty}'}$, there is some $x_{\coord{U,U'}}\in U\cap U'$.  Then, $\coord{U,U'}\mapsto x_{U,U'}$ converges to both $x_{\infty}$ and $x_{\infty}'$, and hence, as limits are unique, $x_{\infty}=x_{\infty}'$:  a contradiction.
\end{proof}
\end{prp}
\begin{exr}{}{exr4.6.39}
Show that quasicompact subsets (which are in fact compact by \cref{exr4.6.37}) of a $T_2$ space can be separated by neighborhoods.
\end{exr}
\begin{exr}{}{exr4.6.40}
Show that quasicompact subsets of a $T_2$ space are closed.
\begin{wrn}
Warning:  The converse is false\footnote{This is in contrast to the $T_1$ situation in which $T_1$ \emph{is} equivalent to points being closed.}---see the following counter-example.
\end{wrn}
\end{exr}
\begin{exm}{A space for which every quasicompact subset is closed that is not $T_2$}{exm3.6.45}
Let $X$ be as in \cref{exm4.2.8x}, that is the real numbers with the cocountable topology.

Any open neighborhood of $0$, must intersect any open neighborhood of $1$, because both of these neighborhoods have countable complements and so, by the uncountability of $\R$, must intersect.  Therefore, $X$ is not $T_2$.

We claim that a subset of $X$ is quasicompact iff it is finite.  Finite subsets are always quasicompact.  On the other hand, take $K\subseteq X$ infinite.  Then, there is in particular a countably-infinite subset $\{ x_0,x_1,x_2,\ldots \} \subseteq K$.  Define $C_m\coloneqq \{ x_k:k\geq m\}$ and $\collection{C}\coloneqq \{ C_m:m\in \N \}$.  Then, $\collection{C}$ is a collection of closed subsets of $X$.  Furthermore, the intersection of any finitely many of them intersects $K$.  On the other hand, the intersection of all of them is empty.  Therefore, $K$ is not quasicompact.  That is, infinite subsets of $X$ are never quasicompact, and so in fact subsets of $X$ are quasicompact iff they are finite.

Now, if a subset of $X$ is quasicompact, it is finite, and hence closed (by the definition of the topology).
\end{exm}
On the other hand, the condition ``quasicompact subsets are closed'' is strictly stronger than being $T_1$ as well (this condition implies $T_1$ by \cref{prp4.5.32}).\footnote{To remember where the condition ``quasicompact subsets are closed'' fits into the series of implications (that is, strictly between $T_1$ and $T_2$), I secretly remember this condition as ``$T_{1\frac{3}{4}}$ ($T_{1\frac{1}{2}}$ is inappropriate because the $\frac{1}{2}$ should be reserved for some sort of ``separation by closed neighborhoods'').  I hesitate to make this an `official' term because, besides the fact that no one else uses it, I don't see how this property is really a \emph{separation} axiom.}
\begin{exm}{A $T_1$ space for which not every quasicompact subset is closed}{}
Define $X\coloneqq \N$.  We equip $X$ with a nonstandard topology, the so-called \term{cofinite topology}\index{Cofinite topology}.  Let $C\subseteq X$ and declare that
\begin{textequation}
$C$ is closed iff either (i)~$C=X$ or (ii)~$C$ is finite.
\end{textequation}
Because the finite union and arbitrary intersection of finite sets is finite (obviously), it follows that this defines a topology on $X$.  $X$ is $T_1$ (\cref{prp4.5.32} again) because points are finite, hence closed.

We claim that $\Z ^+\subseteq \N$ is quasicompact (of course this is not closed because it is not finite).  So, let $\topology{U}$ be an open cover of $\Z ^+$.  Let $U\in \topology{U}$ be an open set containing $1\in \Z ^+$.  From the definition of the topology, $\N \setminus U$ is finite, and so $\Z ^+\setminus U\subseteq \N \setminus U$ is finite, so let us write $\Z ^+\setminus U=\{ m_1,\ldots ,m_n\}$.  As $\topology{U}$ covers $\Z ^+$, there must be some $U_k\in \topology{U}$ with $m_k\in U_k$.  Then, $\{ U,U_1,\ldots ,U_n\}$ is a finite subcover of $\topology{U}$.
\end{exm}
\begin{exm}{A space that is $T_1$ but not $T_2$}{exm3.6.47}
We showed that the real numbers with the cocountable topology is not $T_2$ in \cref{exm3.6.45}.  On the other hand, by definition, countable sets are closed, and so certainly points are closed, and so $X$ is $T_1$.
\end{exm}
\begin{exr}{}{exr4.6.37}
Show that a subspace of a $T_2$ space is $T_2$.
\end{exr}
\begin{exr}{}{exr4.6.38}
Show that an arbitrary product of $T_2$ spaces is $T_2$.
\begin{rmk}
Thus, by \namerefpcref{TychonoffsTheorem} an arbitrary product of compact spaces is compact.
\end{rmk}
\end{exr}

Here is where the consistency of the terminology breaks-down.  If you guessed that the term $T_3$ refers to spaces in which any two distinct points are separated by closed neighborhoods, you'd be wrong.  Unfortunately, it seems that the term $T_3$ was already taken (we'll see what it means in a bit) when someone went to write down the following definition, and so it is called $T_{2\frac{1}{2}}$.
\begin{dfn}{$T_{2\frac{1}{2}}$}{T212}
$X$ is \term{$T_{2\frac{1}{2}}$}\index{$T_{2\frac{1}{2}}$} iff any two distinct points are separated by closed neighborhoods.
\begin{rmk}
The alternate terminology for this seems to be \term{Urysohn}\index{Urysohn}.
\end{rmk}
\end{dfn}
\begin{exm}{A space that is $T_2$ but not $T_{2\frac{1}{2}}$}{SimplifiedArensSquare}\footnote{This comes from \cite[pg.~100]{Steen}.}
This example is very similar to the Arens Square---see \cref{ArensSquare}.

Define
\begin{equation}
\begin{split}
S\coloneqq & ((0,1)\times (0,1))\cap (\Q \times \Q ) \\
& =\left\{ \coord{x,y}\in (0,1)\times (0,1):x,y\in \Q \right\}
\end{split}
\end{equation}
and
\begin{equation}
X\coloneqq S\cup \{ \coord{0,0}\} \cup \{ \coord{1,0}\} .
\end{equation}
(This is the rationals in the open unit square together with the bottom-left and bottom-right corners.)  We define a topology on $X$ by defining a neighborhood base at each point---see \cref{prp4.1.8}.  For $(x,y)\in X$, there are three cases:  (i)~$\coord{x,y}=\coord{0,0}$, (ii)~$\coord{x,y}=\coord{1,0}$, and (iii)~$\coord{x,y}\in S$.  We define
\begin{equation*}
\topology{B}_{\coord{x,y}}\coloneqq \begin{cases}\left\{ U\subseteq S:U\text{ is open in }S\text{.}\footnote{Open in the usual topology (the subspace topology inherited from $(0,1)\times (0,1)$).}\right\} & \text{if }\coord{x,y}\in S \\ \left\{ U_{\coord{x,y}}^m:m\in \Z ^+\right\} & \text{otherwise,}\end{cases}
\end{equation*}
where
\begin{equation*}
\begin{split}
U_{\coord{0,0}}^m & \coloneqq \{ \coord{0,0}\} \cup \left\{ \coord{x,y}\in (0,\tfrac{1}{2})\times (0,\tfrac{1}{m}):x,y\in \Q \right\} \\
U_{\coord{1,0}}^m & \coloneqq \{ \coord{1,0}\} \cup \left\{ \coord{x,y}\in (\tfrac{1}{2},1)\times (0,\tfrac{1}{m}):x,y \in \Q \right\} .
\end{split}
\end{equation*}
By \cref{prp4.1.8}, there is a unique topology for which $\topology{B}_{\coord{x,y}}$ is a neighborhood base of $\coord{x,y}\in X$.

The closure of any open neighborhood of $\coord{0,0}$ contains points with $x$-coordinate $\frac{1}{2}$ and $y$-coordinate arbitrarily small, and the same goes for any open neighborhood of $\coord{1,0}$.  Thus, these two points are not separated by closed neighborhoods.\footnote{Recall that two points are separated by closed neighborhoods iff they are separated by open neighborhoods with disjoint closure---see \cref{prp4.5.13}.}

On the other hand, the $x$-coordinate of every point in every open neighborhood of $\coord{0,0}$ is strictly less than $\frac{1}{2}$, and similarly the $x$-coordinate of every point in every open neighborhood of $\coord{1,0}$ is strictly greater than $\frac{1}{2}$.  In particular, these two points are separated by neighborhoods.

For any point $\coord{x,y}$ distinct from $\coord{0,0}$ and $\coord{1,0}$, we can separate $\coord{x,y}$ from $\coord{0,0}$ with neighborhoods by taking $U_{\coord{0,0}}^m\ni \coord{0,0}$ for $m\in \Z ^+$ with $\frac{1}{m}<y$ and any $\varepsilon$-ball about $\coord{x,y}$ of with radius less than $y-\frac{1}{m}$.  Similarly for $\coord{x,y}$ and $\coord{1,0}$.  Of course, any two points distinct from both $\coord{0,0}$ and $\coord{1,0}$ are separated by neighborhoods because they can be separated by neighborhoods in $S$.  Thus, $X$ is indeed $T_2$.
\begin{rmk}
This is the \term{Simplified Arens Square}\index{Simplified Arens Square}.
\end{rmk}
\end{exm}
\begin{exr}{}{exr3.6.67}
Show that a subspace of a $T_{2\frac{1}{2}}$ space is $T_{2\frac{1}{2}}$.
\end{exr}
\begin{exr}{}{exr3.6.68}
Show that an arbitrary product of $T_{2\frac{1}{2}}$ spaces is $T_{2\frac{1}{2}}$.
\end{exr}

You might be thinking to yourself ``Well, if $T_3$ wasn't separated by closed neighborhoods, then it must be completely-separated, right?''.  Sorry.  Wrong again.
\begin{dfn}{Completely-$T_2$}{CompletelyT2}
$X$ is \term{completely-$T_2$}\index{Completely-$T_2$} iff any two distinct points are completely-separated.
\begin{rmk}
Naturally, this is sometimes called \term{completely-Hausdorff}\index{Completely-Hausdorff}.
\end{rmk}
\begin{rmk}
Sometimes people will also say that \term{continuous functions separate points}.
\end{rmk}
\end{dfn}
\begin{exm}{A space that is $T_{2\frac{1}{2}}$ but not completely-$T_2$}{}
The Arens Square from \cref{ArensSquare} will do the trick.  There, we provided an example of two points which are separated by closed neighborhoods, but not completely-separated.  This is of course already enough to show that the Arens Square is not completely-$T_2$, but we still need to check that \emph{any} two points can be separated by closed neighborhoods.

Denote the Arens Square by $X$ and let $\coord{\frac{1}{2},r\sqrt{2}}\in X$ for $r\in \Q$.  Take $m\in \Z ^+$ such that $r\sqrt{2}-\frac{1}{m}>0$ and take $n\in \Z ^+$ such that $\frac{1}{n}<r\sqrt{2}-\frac{1}{m}$.  Then, using the definition \eqref{4.5.20}, we see that the closures of $U_{(1,0)}^m$ and $U_{\coord{\frac{1}{2},r\sqrt{2}}}$ are disjoint.  Similarly, a point of this form can be separated from $(1,0)$ by closed neighborhoods.

For $\coord{x,y}\in S$, just take $m\in \Z ^+$ with $\frac{1}{m}<y$.  Then, the closure of $U_{\coord{0,0}}^m$ and any $\varepsilon$-ball around $\coord{x,y}$ with radius less than $y-\frac{1}{m}$ will be disjoint.  A similar trick works for $\coord{1,0}$ of course.

For $r,q\in \Q$ with $r<q$, choose $m,n\in \Z ^+$ so that $r\sqrt{2}+\frac{1}{m}<q\sqrt{2}-\frac{1}{n}$ (and so that $r\sqrt{2}-\frac{1}{m}>0$ and $q\sqrt{2}+\frac{1}{n}<1$ so that the open neighborhoods are actually contained in $X$).  Then, the closures of $U_{\coord{\frac{1}{2},r\sqrt{2}}}^m$ and $U_{\coord{\frac{1}{2},q\sqrt{2}}}^n$ are disjoint.

For $\coord{x,y}\in S$ and $\coord{\frac{1}{2},r\sqrt{2}}\in T$, we must have that $y\neq r\sqrt{2}$, and so, without loss of generality, that $y<r\sqrt{2}$.  Then, choose $m\in \Z ^+$ large enough so that $r\sqrt{2}-\frac{1}{m}>y$ (and so that $U_{\coord{\frac{1}{2},\sqrt{2}}}^m$ is contained in $X$), and take a $\varepsilon$-ball about $\coord{x,y}$ with radius less than $(r\sqrt{2}-\frac{1}{m})-y$.  Then, the closure of these two neighborhoods will be disjoint.

Finally, can separate two points in $S$ by closed neighborhoods because we could do so in $(0,1)\times (0,1)$ (recall that $S$ just has the subspace topology).
\end{exm}
\begin{exr}{}{exr3.6.71}
Show that a subspace of a completely-$T_2$ space is completely-$T_2$.
\end{exr}
\begin{exr}{}{exr3.6.72}
Show that an arbitrary product of completely-$T_2$ spaces is completely-$T_2$.
\end{exr}

And of course, as now you probably saw coming, $T_3$ does not mean distinct points can be perfectly-separated.
\begin{dfn}{Perfectly-$T_2$}{PerfectlyT2}
$X$ is \term{perfectly-$T_2$}\index{Perfectly-$T_2$} iff any two distinct points can be perfectly-separated.
\begin{rmk}
Disclaimer:  I have never seen this term before.  Then again, I've never seen \emph{any} term to describe such spaces.  But honestly, if completely-$T_2$ means you can completely-separate points, then a space in which you can perfectly-separate points is going to be called \textellipsis
\end{rmk}
\end{dfn}
\begin{exm}{A space that is completely-$T_2$ but not perfectly-$T_2$}{exm4.5.48}
The Uncountable Fort Space from \cref{UncountableFortSpace} will do the trick.  There, we provided an example of two points which are completely-separated, but not perfectly-separated.  This is already enough to show that the Uncountable Fort Space is not perfectly-$T_2$, but we still need to check that \emph{any} two points can be completely-separated.

Recall that the Uncountable Fort Space was defined to be $X\coloneqq \R$ with the closed sets being precisely the finite sets and also the sets which contained $0$.  We showed above in \cref{UncountableFortSpace} that $1\in X$ can be completely-separated from $0\in X$.  Of course, there was nothing special about $1$, and so all we need to do is to show that we can completely-separate any two nonzero points in $X$.

So, let $x_1,x_2\in X$ be nonzero and distinct, and define $f\colon X\rightarrow [0,1]$ by
\begin{equation}
f(x)\coloneqq \begin{cases}0 & \text{if }x=x_1 \\ 1 & \text{if }x=x_2 \\ \tfrac{1}{2} & \text{otherwise.}\end{cases}
\end{equation}
Then, if $C\subseteq [0,1]$ closed contains $\frac{1}{2}$, $f^{-1}(C)$ contains $0\in X$, and is hence closed.  Otherwise, it is finite, as $f^{-1}\left( [0,1]\setminus \{ \frac{1}{2}\} \right) =\{ x_1,x_2\}$, and hence closed.  Thus, $f$ is continuous, and so completely-separates $x_1$ and $x_2$.
\end{exm}
\begin{exr}{}{exr3.6.76}
Show that a subspace of a perfectly-$T_2$ space is perfectly-$T_2$.
\end{exr}
Now should be the point where we say ``Exercise:  Show that an arbitrary product of perfectly-$T_2$ spaces is perfectly-$T_2$.''  Unfortunately, however, that is false.
\begin{exm}{A product of perfectly-$T_2$ spaces that is not perfectly-$T_2$}{exm3.6.73}
The counter-example will be an uncountable product of the real numbers.\footnote{The conclusion is actually true for \emph{countable} products---see \cref{prp3.6.73}.}

We first check that real numbers are perfectly-$T_2$.\footnote{In fact, they're uniformly-perfectly-$T_4$---see \cref{UniformlyPerfectlyT4,prp5.4.13}.}  Let $x_1,x_2\in \R$ be distinct.  Without loss of generality, suppose that $x_1<x_2$.  Define $f\colon X\rightarrow Y$ by
\begin{equation}
f(x)\coloneqq \begin{cases}\tfrac{1}{2} & \text{if }x\in (-\infty ,x_1-\tfrac{1}{2}] \\ 1+(x-x_1) & \text{if }x\in [x_1-\tfrac{1}{2},x_1] \\ -\tfrac{x-x_1}{x_2-x_1}+1 & \text{if }x\in [x_1,x_2] \\ x-x_2 & \text{if }x\in [x_2,x_2+\tfrac{1}{2}] \\ \tfrac{1}{2} & \text{if }x\in [x_2,\infty ).\end{cases}\footnote{The picture is that it starts at $\frac{1}{2}$, climbs up linearly to $1$ at $x_1$, decreases linearly down to $0$ at $x_2$, and then increases linearly back up to $\frac{1}{2}$, and stays there.}
\end{equation}
This is continuous by the \nameref{PastingLemma}, and furthermore satisfies $f^{-1}(1)=\{ x_1\}$ and $f^{-1}(0)=\{ x_2\}$, and so $\R$ is indeed perfectly-$T_2$.

Now, let $I$ be an \emph{uncountable} index set\footnote{``Index set'' is not a technical term, that is, it's not a special sort of set---it just means that it's some random set that we happen to be using for `indexing'.} and define $X\coloneqq \prod _I\R$.  We wish to show that $X$ is not perfectly-$T_2$.   It suffices to show that there is no continuous function $f\colon X\rightarrow [0,1]$ such that $\{ \coord{0,0,0,\ldots}\} =f^{-1}(0)$.\footnote{Because then we definitely cannot perfectly separate $\coord{0,0,0,\ldots}$ from any other point.  Also, note that, while my notation might suggest that this is a countable collection of coordinates, this is \emph{not} the case---I just mean that this is the element in $X$ all of whose coordinates are $0$.}  We proceed by contradiction:  let $f\colon X\rightarrow [0,1]$ be such a function.

For each $n\in \mathbb{Z}^+$, $f^{-1}([0,\frac{1}{n}))$ is an open neighborhood of $\coord{0,0,\ldots}$, and so there are finitely many indices $i_1^n,\ldots ,i_{m_n}^n\in I$ and open sets $U_{i_1^n}^n,\ldots ,U_{i_{m_n}^n}^n\subseteq \mathbb{R}$ containing $0$ with
\begin{equation*}
\coord{0,0,\ldots}\in U_{i_1^n}^n\times \cdots \times U_{i_{m_n}^n}^n\times \prod _{\substack{i\in I \\ i\neq i_1^n,\ldots ,i_{m_n}^n}}\mathbb{R}\subseteq f^{-1}([0,\tfrac{1}{n})).
\end{equation*}
Taking the intersection over all $n\in \mathbb{Z}^+$, we find that (because $\{ \coord{0,0,\ldots}\} =\bigcap _{n\in \Z ^+}f^{-1}([0,\frac{1}{n}))$).
\begin{equation*}
\bigcap _{n\in \mathbb{Z}^+}\bigg( U_{i_1^n}^n\times \cdots \times U_{i_{m_n}^n}^n\times \prod _{\substack{i\in I \\ i\neq i_1^n,\ldots ,i_{m_n}^n}}\mathbb{R}\bigg) =\{ \coord{0,0,\ldots}\} .
\end{equation*}
However, for each $n\in \mathbb{Z}^+$, we only have finitely many indices $i_1^n,\ldots ,i_{m_n}^n$, and so in total we have only countably many indices appearing at all.  This means that there are still uncountably many coordinates in the above intersection which are equal to $\mathbb{R}$:  a contradiction.
\end{exm}
Despite this, it \emph{is} true for countably-infinite products.
\begin{prp}{}{prp3.6.73}
A countable product of perfectly-$T_2$ spaces is perfectly-$T_2$.
\begin{wrn}
Warning:  Recall that this fails for arbitrary (not-necessarily-countable) products---see \cref{exm3.6.73}.
\end{wrn}
\begin{proof}
\forwardref

\noindent
Let $\{ X_m:m\in \mathbb{N}\}$ be a countable collection of perfectly-$T_2$ spaces and let $x,y\in \prod _{m\in \mathbb{N}}X_m$ be distinct.  Define $S:=\{ m\in \mathbb{N}:x_m=y_m\}$, so that $S^{\comp}$ is nonempty.  

First suppose that $S^{\comp}$ is finite.  For $m\in S$, let $g_m\colon X_m\rightarrow [0,1]$ be continuous and such that $g_m^{-1}(1)=\{ x_m\} =\{ y_m\}$.  For $m\in S^{\comp}$, let $f_m\colon X_m\rightarrow [-1,1]$ be such that $f_m^{-1}(1)=\{ x_m\}$ and $f_m^{-1}(-1)=\{ y_m\}$.  Write $S^{\comp}=\{ i_1,\ldots ,i_{m_0}\}$.  Finally, define $f\colon \prod _{m\in \mathbb{N}}X_m\rightarrow [-1,1]$ by
\begin{equation}
f\coloneqq \left( \prod _{k\in S}g_k\right) \cdot \left( \frac{f_{i_1}+\cdots +f_{i_{m_0}}}{m_0}\right) .\footnote{We have to work a bit harder to ensure that the product converges to a continuous function, but as that argument complicates the proof and is not part of the core idea, we postpone this correction until the end.}\footnote{Of course, each $g_k$ is a function on $X_k$ but yet $f$ is a function on $X$.  Thus, in order for this to make sense, when we write $g_k$ we don't mean $g_k(z)$ for $z\in X$, because that is nonsensical, but rather $g_k(z_k)$.  Similar comments apply throughout this example.}
\end{equation}

We claim that $f(z)=+1$ iff $z=x$ and $f(z)=-1$ iff $z=y$.  As the proofs are similar, we only prove the $x$ case.  One direction is easy:  if $z=x$, then each $g_k(z)=1$ and each $f_{i_k}=1$, and so $f(z)=1$.  Conversely, if $f=1$, then as the absolute value of the `average' term is bounded by $1$, if any $g_k(z)<1$, then we would have $\abs{f(z)}<1$, and so, we must have that each $g_k(z)=1$, which forces $z_k=x_k$ for all $k\in S$, as well as $\frac{f_{i_1}(z)+\cdots +f_{i_{m_0}}(z)}{m_0}=1$.  As $f_{i_k}(z)\in [-1,1]$, this forces each $f_{i_k}(z)=1$, which in turn forces $z_k=x_k$ for all $k\in S^{\comp}$, and so $z=x$.

Now suppose that $S^{\comp}$ is infinite and enumerate $S^{\comp}$ as $S^{\comp}=\{ i_1,i_2,\ldots \}$.  Let $g_m$ be as before, but now, for $i_k\in S^{\comp}$, let $f_{i_k}:X_{i_k}\rightarrow [0,\frac{1}{2^k}]$ be such that $f_{i_k}^{-1}(\frac{1}{2^k})=\{ x_{i_k}\}$ and $f_{i_k}^{-1}(0)=\{ y_{i_k}\}$.  Now define $f\colon \prod _{m\in \mathbb{N}}X_m\rightarrow [-1,1]$ by
\begin{equation}
f\coloneqq \left( \prod _{k\in S}g_k\right) \cdot \left( 2\sum _{k=1}^{\infty}f_{i_k}-1\right) .
\end{equation}

I claim once again that $f(z)=+1$ iff $z=x$ and $f(z)=-1$ iff $z=y$.  This time the cases are a bit different and so we do them each separately.  That $z=x$ implies $f(z)=+1$ and $z=y$ implies $f(z)=-1$ are easy.  Conversely, suppose that $f(z)=1$.  Similarly as before, this forces each $g_k(z)=1$, so that $z_k=x_k$ for all $k\in S$, and hence $2\sum _{k=1}^{\infty}f_{i_k}(z)-1=1$, and hence $\sum _{k=1}^{\infty}f_{i_k}(z)=1$.  As $f_{i_k}(z)\leq \frac{1}{2^k}$, this forces $f_{i_k}(z)=\frac{1}{2^k}$ for all $k\in S^{\mathrm{c}}$, and hence $z_k=x_k$ for all $k\in S^{\comp}$, and hence $z=x$, as desired.  Now suppose that $f(z)=-1$.  As the absolute value of the second term is bounded by $1$, we cannot have that any $g_k(z)<1$, and so once again we have each $g_k(z)=1$, so that $z_k=y_k$ for all $k\in S$, and hence $2\sum _{k=1}^{\infty}f_{i_k}(z)-1=-1$, and hence $\sum _{k=1}^{\infty}f_{i_k}(z)=0$.  As all of these terms are nonnegative, this forces each $f_{i_k}(z)=0$, so that $z_k=y_k$ for all $y\in S^{\comp}$, and hence $z=y$, as desired.

There is one important detail we have yet to check:  that our definition of $f$ in each case is actually \emph{continuous}.  To do this, we must show that $\prod _{k\in S}g_k$ and, in the second case, $\sum _{k=1}^{\infty}f_{i_k}$, actually converge to a continuous functions.  To do so, we make use of \emph{completeness} of the set of all bounded continuous functions on $X$ (\cref{exr4.4.22}).  Thus, we want to show that the partial sums of this series are Cauchy.  So, enumerate $S=\{ j_1,j_2,\ldots \}$ and take the codomain of $g_{j_k}$ to be $[2^{2^{-k}},1]$.  This ensures that the codomain of $\ln (g_{j_k})$\footnote{See \cref{NaturalLogarithm} for the definition of $\ln$.} will be $[-2^{-k},1]$, which is enough to ensure that the partial sums of $\sum _{k=1}^{\infty}\ln (g_{j_k})$ Cauchy, as desired.  As the codomain of $f_{i_k}$ is already $[0,\frac{1}{2^k}]$, we needn't modify the codomain in order that the partial sums be Cauchy.
\end{proof}
\end{prp}

But now we've gone through all the separation properties, right?  How could $T_3$ be a thing?  Well, now we enter a collection of separation axioms of a different nature---now we will focus on separating \emph{closed sets}.  One unfortunate fact, however, is that, if points are not closed, then these new separation axioms are \emph{not} strictly stronger than the separation axioms we just presented.  There are thus two versions of the following separation axioms:  ones in which the points are closed and ones in which they aren't.
\begin{dfn}{Regular}{Regular}
$X$ is \term{regular}\index{Regular (topological space)} iff any closed set and a point not contained in it can be separated by neighborhoods.
\begin{rmk}
You'll note that we skipped right over being topologically-distinguishable and (just) separated.  This is because a point is automatically separated from a closed set---the complement of the closed set is an open neighborhood of the point which does not intersect the closed set.
\end{rmk}
\begin{rmk}
Unfortunately, the terminology of separation axioms is so fucked that there's really no way of being consistent with all of the literature.  There are sources which reverse my conventions of regular and $T_3$.  We explain the motivation of our choice of convention in the definition of $T_3$ (\cref{T3}).
\end{rmk}
\end{dfn}
\begin{exm}{A space that is regular but not $T_0$}{}
The indiscrete topology on any set with at least two points is (almost) vacuously regular but not $T_0$:  there are only two closed sets, $\emptyset$ and $X$, and $X$ doesn't not contain any points, and for $\emptyset$ and $x\in X$, they are separated by the neighborhoods $\emptyset \subseteq \emptyset$ and $x\in X$.
\end{exm}
Stupid examples like this is why we almost always care about the case when we in addition impose the condition of being $T_1$.  This finally brings us to the definition of $T_3$.
\begin{dfn}{$T_3$}{T3}
$X$ is $T_3$ iff it is $T_1$ and regular.
\begin{rmk}
The $T_1$ condition (points are closed) is added so that this is a strict specialization of being $T_2$.  In fact, by the following proposition, we could have equally well said $T_0$.
\end{rmk}
\begin{rmk}
We choose to call this $T_3$ instead of regular because then we have $T_3\rightarrow T_2$.  If the terms were reversed, then there would be $T_3$ spaces (like indiscrete spaces) that were not $T_2$---ew.
\end{rmk}
\begin{wrn}
Warning:  Up until now, all of the $T_k$ axioms (including things like $T_{2\frac{1}{2}}$, completely-$T_2$, and perfectly-$T_2$) have been strictly comparable, that is, $T_1$ strictly implies $T_0$, $T_2$ strictly implies $T_1$, etc..  With the introduction of $T_3$, this is no longer the case.  It turns out that $T_3$ implies $T_{2\frac{1}{2}}$, but that $T_3$ is not comparable with either completely-$T_2$ or perfectly-$T_2$.
\end{wrn}
\end{dfn}
\begin{prp}{}{prp4.6.53}
If $X$ is $T_0$ and regular, then it is $T_2$.
\begin{proof}
Suppose that $X$ is $T_0$ and regular.  Let $x_1,x_2\in X$.  Because $X$ is $T_0$, without loss of generality there is some open neighborhood $U$ of $x_1$ which does not contain $x_2$.  Then, $U^{\comp}$ is closed and does not contain $x_1$, so because $X$ is regular, there are disjoint open neighborhoods $V_1$ and $V_2$ of $x_1$ and $U^{\comp}$ respectively.  Then, as $x_2\in U^{\comp}$, this implies that $x_1$ and $x_2$ are separated by neighborhoods.
\end{proof}
\end{prp}
Before having defined $T_3$, we have a perfect chain of strict implications:  perfectly-$T_2$ implies completely-$T_2$ implies $T_{2\frac{1}{2}}$ implies $T_2$ implies $T_1$ implies $T_0$.  We now have a new condition, $T_3$, and unfortunately it does not fit into this chain.\footnote{See \eqref{4.6.105} for the final diagram of implications.}  $T_3$ does not imply perfectly-$T_3$.  In fact, it doesn't even imply completely-$T_2$ (\cref{ThomasTentSpace}).  On the other hand, it \emph{does} imply $T_{2\frac{1}{2}}$---this is because $T_3$ is equivalent to $T_{3\frac{1}{2}}$ (\cref{prp4.5.70}) which certainly implies $T_{2\frac{1}{2}}$.  The other direction isn't true either---perfectly-$T_2$ does not imply $T_3$ (\cref{CocountableExtensionTopology}).
\begin{exm}{A space that is $T_3$ but not completely-$T_2$}{ThomasTentSpace}\footnote{Evidently, this example originally comes from \cite{Thomas}, though I first heard about it from Brian M.~Scott on \href{http://math.stackexchange.com/questions/386742/making-Tychonoff-corkscrew-in-counterexamples-in-topology-rigorous}{math.stackexchange}.}
For $m\in 2\Z$, define
\begin{equation}
L_m\coloneqq \{ m\} \times [0,\tfrac{1}{2}).
\end{equation}
For $n\in 1+2\Z$ and $k\in \Z$ with $k\geq 2$, let $T_{n,k}$ be the \emph{legs} of an isosceles triangle in $\R ^2\times \R ^2$ with apex at $p_{n,k}\coloneqq \coord{n,1-\frac{1}{k}}$ and base $[n-(1-\frac{1}{k}),n+(1-\frac{1}{k})]\times \{ 0\}$ (including the end-points of the legs on the base but not the apex $p_{n,k}$), precisely
\begin{equation}
T_{n,k}\coloneqq \left\{ \coord{n\pm t,1-t-\tfrac{1}{k}}:0<t\leq 1-\tfrac{1}{k}\right\} .
\end{equation}
Then, define
\begin{equation}
\begin{split}
Y_0 & \coloneqq \bigcup _{m\in 2\Z}L_m \\
Y_1 & \coloneqq \bigcup _{\substack{n\in 1+2\Z \\ k\in \Z ,k\geq 2}}\{ p_{n,k}\} \\
Y_2 & \coloneqq \bigcup _{\substack{n\in 1+2\Z \\ k\in \Z ,k\geq 2}}T_{n,k} \\
Y & \coloneqq Y_0\cup Y_1\cup Y_2.
\end{split}
\end{equation}
We define a topology on $Y$ by defining a neighborhood base at each point---see \cref{prp4.1.8}.  For $\coord{x,y}\in Y_2$, we declare a neighborhood base to be simply $\{ \coord{x,y}\}$.  For $\coord{x,y}=p_{n,k}$, we declare a neighborhood base of $\coord{x,y}$ to consist of subsets of the form $\{ p_{n,k}\} \cup S$ for $S\subseteq T_{n,k}$ cofinite in $T_{n,k}$.  For $\coord{x,y}\in L_m$ for $m\in 2\Z$, we declare a neighborhood base of $\coord{x,y}$ to consist of subsets of the form $\{ \coord{m,y}\}\cup S$ for $S\subseteq Y\cap \left( (m-1,m+1)\times \{ y\}\right)$ cofinite in $Y\cap \left( (m-1,m+1)\times \{ y\}\right)$. 

Finally define
\begin{equation}
X\coloneqq Y\sqcup \{ p_1,p_2\}
\end{equation}
for distinct new points $p_1,p_2$.  To define a topology on $X$, we once again use \cref{prp4.1.8}.  A neighborhood base at $\coord{x,y}\in Y$ consists of precisely the same sets as it did before.  We furthermore declare that
\begin{equation}
\begin{split}
\topology{B}_{p_1} & \coloneqq \left\{ U_{p_1}^\alpha :\alpha \in \R \right\} \\
\topology{B}_{p_2} & \coloneqq \left\{ U_{p_2}^\alpha :\alpha \in \R \right\}
\end{split}
\end{equation}
to be neighborhood bases of $p_1$ and $p_2$ respectively, where
\begin{equation}
\begin{split}
U_{p_1}^\alpha & \coloneqq \left\{ \coord{x,y}\in Y:x<\alpha \right\} \cup \{ p_1\} \\
U_{p_2}^\alpha & \coloneqq \left\{ \coord{x,y}\in Y:x>\alpha \right\} \cup \{ p_2\} .
\end{split}
\end{equation}
Note that in all cases, every element of $\topology{B}_p$ is \emph{open},\footnote{This does require a quick check.} which does not follow automatically from the definition of a neighborhood base---see the remark in \cref{prp4.1.8}.

We first check that $X$ is not completely-$T_2$ by showing that no continuous function can separate $p_1$ from $p_2$.  So, let $f\colon X\rightarrow [0,1]$ be continuous.

Write $\alpha \coloneqq f(p_{n,k})$.  Then,
\begin{equation}
\begin{split}
f^{-1}(\alpha )\cap T_{n,k} & =f^{-1}\left( \bigcap _{m\in \Z ^+}(\alpha -\tfrac{1}{m},\alpha +\tfrac{1}{m})\right) \cap T_{n,k} \\
& =\bigcap _{m\in \Z ^+}S_m,
\end{split}
\end{equation}
where
\begin{equation}
S_m\coloneqq f^{-1}((\alpha -\tfrac{1}{m},\alpha +\tfrac{1}{m}))\cap T_{n,k}
\end{equation}
is cofinite in $T_{n,k}$.  Hence,
\begin{equation}
T_{n,k}\setminus \left( f^{-1}(\alpha )\cap T_{n,k}\right) =\bigcup _{m\in \Z ^+}T_{n,k}\setminus S_m.
\end{equation}
In other words, the number of points in $T_{n,k}$ that map to a different value other than $f(p_{n,k})$ is at most countable.  Less precisely, $f$ is constant on $T_{n,k}$ modulo a countable set.

Let $C_{n,k}$ denote the set of $y$-values in $[0,\tfrac{1}{k})$ for which there is some point in $T_{n,k}$ with that $y$-coordinate and that maps to $f(p_{n,k})$.  We just showed that this set is cocountable in $[0,1-\frac{1}{k})$, and hence it is certainly cocountable in $[0,\frac{1}{2})\subseteq [0,1-\frac{1}{k})$.  Then, the intersection of them $\bigcap _{k\in \Z ,k\geq 2}C_{n,k}$ must in turn then be cocountable in $[0,\frac{1}{2})$, and in particular, be nonempty.  So, let $y_m\in L_m$ be such that for every $k\geq 2$ there is some $q_{m,k}^-\in T_{m-1,k}$ and some $q_{m,k}^+\in T_{m+1,k}$, each with the same $y$-coordinate a $y_m$, with $f(q_{m,k}^-)=f(p_{m-1,k})$ and $f(q_{m,k}^+)=f(p_{m+1,k})$.  Because the open neighborhoods of $y_m$ are cofinite in $((m-1,m+1)\times \{ y\} )\cap Y$, the sequence $k\mapsto q_{m,k}^-$ must eventually be in every neighborhood of $y_m$, and so we have $\lim _kq_{m-1,k}=y_m=\lim _kq_{m+1,k}$.  It follows that
\begin{equation}
\begin{split}
f(y_m) & =\lim _kf(q_{m,k}^+)=\lim _kf(p_{m+1,k}) \\
& =\lim _kf(q_{m+2,k}^-)=f(y_{m+2}).
\end{split}
\end{equation}
The crux:  for each $m\in 2\Z$, there exists points $y_m\in L_m$ and $y_{m+2}\in L_{m+2}$ with $f(y_m)=f(y_{m+2})$.  By the definition of the open neighborhoods of $p_1$ and $p_2$, it follows that $f(p_1)=\lim _{m\to -\infty}f(y_m)=\lim _{m\to +\infty}f(y_m)=f(p_2)$, so that $p_1$ and $p_2$ are not completely-separated, and so $X$ is not completely-$T_2$.

We now check that $X$ is $T_3$.  We first check that it is $T_1$.  $p_1$ is closed because
\begin{equation}
\{ p_1\} =\bigcap _{\alpha \in \R}(U_{p_2}^\alpha )^{\comp}.
\end{equation}
Similarly for $p_2$.  Each point $\coord{m,y}\in L_m$ is closed because the union of all open sets not contained in $\topology{B}_{\coord{m,y}}$, $\topology{B}_{p_1}$, or $\topology{B}_{p_2}$ is precisely the complement of $\coord{m,y}$.  Similarly for the $p_{n,k}$s.  For $\coord{x,y}\in T_{n,k}$, the intersection over $\{ \coord{x',y'}\} ^{\comp}$ for $\coord{x',y'}\neq \coord{x,y}$ in $T_{n,k}$\footnote{Recall that points of $T_{n,k}$ are open.} is $\{ \coord{x,y}\}$ along with at least one neighborhood of every other point that doesn't contain $\coord{x,y}$\footnote{The only points which have neighborhoods that contain $\coord{x,y}$ are $p_{n,k}$ and possibly elements of $L_{m-1}$ and $L_{m+1}$, but as in these cases all elements of the neighborhood case are cofinite, we can just pick a neighborhood that does not contain $\coord{x,y}$.} and so is closed.

Now we just need to show that we can separate closed sets from points with open neighborhoods.  One thing to note is that we need only separate closed sets that are complements of some element of some $\topology{B}_p$ from points, because every closed set is an intersection of sets of this form.\footnote{This uses the fact that every element in every $\topology{B}_p$ is open, so that all of these sets together actually form a base for the topology---see \cref{prp4.1.8,exr4.1.7}.}  So, let $p\in X$ and let $C\subseteq X$ be a complement of some element in the neighborhood base not containing $p$.  There is nothing to do but just break it down into cases.

First suppose that $p=p_1$ and $C\subseteq Y\cup \{ p_2\}$.  If $C$ contained points of arbitrarily small $x$-coordinate, then because it is closed, it would have to contain $p_1$.  Thus, there is some $x_0\in \R$ such that the $x$-coordinate of every point (besides $p_2$ of course, which doesn't actually have an $x$-coordinate) is at least $x_0$.  Then we can take $U_{p_1}^{\frac{x_0}{8}}$ as our open neighborhood of $p_1$ and $U_{p_2}^{\frac{x_0}{4}}$ as our open neighborhood of $C$.  Similarly if our point is $p_2$.

We can easily separate points of $T_{n,k}$ from closed sets as these points themselves are open.

Now take $p=\coord{m,y}\in L_m$ and $C\subseteq X$ not containing $p$.  If $C$ intersects $((m-1,m+1)\times \{ y\} )\cap Y$ at more than finitely many points, then $y$ would be an accumulation point of $C$, and hence would be contained in $C$.  Thus, it must intersect $((m-1,m+1)\times \{ y\})\cap Y$ at at most finitely many points, in which case we can simply remove them to obtain an open neighborhood of $\coord{m,y}$.  These finitely many points are elements of $T_{m-1,k}$ or $T_{m+1,k}$ for some $k$, and so are in particular open.  Thus, the union of these finitely many points together with $U_{p_1}^{(m-1)-(1-\frac{1}{k})}$ and $U_{p_2}^{(m+1)+(1+\frac{1}{k})}$ must contain $C$ (because these two neighborhoods of $p_1$ and $p_2$ contain all $T_{n,l}$ for $n\neq m-1,m+1$).

Finally if one of the points is some $p_{n,k}$, then using very similar logic as in the previous case, $C$ must intersect $T_{n,k}$ at at most finitely many points.  Removing these points from $T_{n,k}$ gives us an open neighborhood which is disjoint from the neighborhood formed from the union of these finitely many points (which are open) and the complement of $T_{n,k}$.
\begin{rmk}
I do not know of a name for this space, but in order to have something to refer to, I shall call it the \term{Thomas Tent Space}\index{Thomas Tent Space}.\footnote{The $T_{n,k}$s are like ``tents'', and indeed, that is the word Scott used to describe them.}
\end{rmk}
\end{exm}
\begin{exm}{A space that is perfectly-$T_2$ but not $T_3$}{CocountableExtensionTopology}
\forwardref

\noindent
Define $X\coloneqq \R$.  We equip $\R$ with the so-called \term{cocountable extension topology}\index{Cocountable extension topology}.  Let $C\subseteq X$ and declare that
\begin{textequation}
$C$ is closed iff it is the union of a countable set and a set closed in the usual topology on $\R$.
\end{textequation}

Now, let $x_1,x_2\in X$ be distinct, and let $\phi :\R \rightarrow (0,1)$ be any homeomorphism (in the usual topology).\footnote{Such a homeomorphism exists by \cref{Arctan} (hopefully you can construct a homeomorphism between $(-\frac{\uppi}{2},\frac{\uppi}{2})$ and $(0,1)$).}  Now define $f\colon X\rightarrow [0,1]$ by
\begin{equation}
f(x)\coloneqq \begin{cases}0 & \text{if }x=x_1 \\ 1 & \text{if }x=x_2 \\ \phi (x) & \text{otherwise.}\end{cases}
\end{equation}
The preimage of any set in $[0,1]$ which contains neither $0$ nor $1$ will be closed because $\phi$ is a homeomorphism.  If it does contain $0$ or $1$, then the preimage will be the union of a closed set in the usual topology on $\R$ and a finite set, and hence will be closed.  Thus, $f$ is continuous, and so $X$ is perfectly-$T_2$.  We know check that $X$ is not $T_3$.

$\Q \subseteq X$ is closed because it is countable.  Any open neighborhood of $\Q$ must be of the form of a usual open neighborhood of $\Q$ with countably many points removed.  Of course, the only usual open neighborhood of $\Q$ is all of $\R$, and so every open neighborhood of $\Q$ is just $\R$ with countably many points removed.  On the other hand, $\sqrt{2}\notin \Q$ and any open neighborhood of $\sqrt{2}$ is a usual open neighborhood with countably many points removed.  Thus, by uncountability of $\R$, these two must intersection, and so you cannot separate $\Q$ and $\sqrt{2}$ with neighborhoods.  Thus, $X$ is not $T_3$.
\end{exm}
Thus, once you hit $T_{2\frac{1}{2}}$, you can increase your separation in one of two noncomparable ways:  you can make your space completely-$T_2$ or you can make your space $T_3$.  At the end of the section, we will summarize how all the separation axioms relate to each other.
\begin{exr}{}{exr3.6.99}
Show that a subspace of a $T_3$ space is $T_3$.
\end{exr}
\begin{exr}{}{exr3.6.100}
Show that an arbitrary product of $T_3$ spaces is $T_3$.
\end{exr}

\begin{dfn}{$T_{3\frac{1}{2}}$}{T312}
$X$ is \term{$T_{3\frac{1}{2}}$}\index{$T_{3\frac{1}{2}}$} iff it is $T_1$ and any closed set can be separated by closed neighborhoods from a point it does not contain.
\begin{rmk}
This is not universally accepted terminology.  Often people use the term $T_{3\frac{1}{2}}$ to refer to what I call completely-$T_3$.  I imagine this is the case because it turns out that my $T_{3\frac{1}{2}}$ is equivalent to $T_3$ (see the next proposition).
\end{rmk}
\begin{rmk}
You will notice a pattern with the terminology.  If $T_k$ means you can separate XYZ from ABC with neighborhoods, then $T_{k\frac{1}{2}}$ means you can separate XYZ from ABC with \emph{closed} neighborhoods; completely-$T_k$ means you can completely-separate XYZ from ABC; perfectly-$T_k$ means you can perfectly-separate XYZ from ABC.  This admittedly creates conflict with some people's terminology, but the terminology of separation axioms is so varied from source to source that it was impossible to come up with a naming system that didn't conflict with something.  I chose what I did because it is the most systematic that does not completely depart from the established nomenclature.
\end{rmk}
\end{dfn}
\begin{prp}{}{prp4.5.70}
$X$ is $T_3$ iff $X$ is $T_{3\frac{1}{2}}$.
\begin{proof}
$(\Rightarrow )$ Suppose that $X$ is $T_3$.  Let $C\subseteq X$ be closed and let $x\in C^{\comp}$.  As $X$ is $T_3$, there is an open neighborhood $U$ of $C$ and an open neighborhood $V$ of $x$ which are disjoint.  Then, $V^{\comp}$ is a closed set not containing $x$, and so there is an open neighborhood $W_1$ of $V^{\comp}$ and an open neighborhood $W_2$ of $x$ which are disjoint.  Then,
\begin{equation}
C\subseteq U\subseteq \Cls (U)\subseteq V^{\comp}\subseteq W_1
\end{equation}
and
\begin{equation}
x\in W_2\subseteq \Cls (W_2)\subseteq W_1^{\comp},
\end{equation}
and so $U$ and $W_2$ are open neighborhoods of $C$ and $x$ respectively with disjoint closures, so that $X$ is $T_{3\frac{1}{2}}$ by \cref{prp4.5.13}.

\blankline
\noindent
$(\Leftarrow )$ This is immediate---if sets are separated by closed neighborhoods, then they are certainly separated by neighborhoods.
\end{proof}
\end{prp}
Now would be the time to ask about subspaces and products of $T_{3\frac{1}{2}}$ spaces, but as $T_{3\frac{1}{2}}$ is equivalent to $T_3$, we\footnote{And by ``we'' I mean ``you''.} have already addressed these questions in \cref{exr3.6.99,exr3.6.100}.

\begin{dfn}{Completely-$T_3$}{CompletelyT3}
$X$ is \term{completely-$T_3$}\index{Completely-$T_3$} iff it is $T_1$ and any closed set can be completely-separated from a point it does not contain.
\begin{rmk}
As noted above, sometimes people use the term $T_{3\frac{1}{2}}$ for this property.  This is also sometimes called \term{Tychonoff}\index{Tychonoff} (probably because of the \namerefpcref{TychonoffEmbeddingTheorem}.
\end{rmk}
\end{dfn}
\begin{exm}{A space that is $T_{3\frac{1}{2}}$ but not completely-$T_3$}{}
The Thomas Tent Space $X$ from \cref{ThomasTentSpace} will do the trick.  We showed there that it is $T_3$, but not completely-$T_2$.  From \cref{prp4.5.70}, it follows that $X$ is $T_{3\frac{1}{2}}$.  However, it if were completely-$T_3$, then it would also be completely-$T_2$---a contradiction.  Therefore, $X$ is likewise not completely-$T_3$.
\end{exm}
\begin{exr}{}{exr3.6.106}
Show that a subspace of a completely-$T_3$ space is completely-$T_3$.
\end{exr}
\begin{exr}{}{exr3.6.107}
Show that an arbitrary product of completely-$T_3$ spaces is completely-$T_3$.
\end{exr}
One characterization of completely-$T_3$ spaces that can be quite useful is that they are precisely the topological spaces which appear as (up to homeomorphism) subspaces of products of the closed interval $[0,1]$ (this is the \namerefpcref{TychonoffEmbeddingTheorem}, though we will have to wait for \nameref{UrysohnsLemma} to prove it.

\begin{dfn}{Perfectly-$T_3$}{PerfectlyT3}
$X$ is \term{perfectly-$T_3$} iff it is $T_1$ and any closed set can be perfectly-separated from a point it does not contain.
\end{dfn}
\begin{exm}{A space that is completely-$T_3$ but not perfectly-$T_3$}{exm4.6.80}
The Uncountable Fort Space from \cref{UncountableFortSpace} will do the trick.  In \cref{exm4.5.48}, we showed that this was completely-$T_2$ but not perfectly-$T_2$.  As it is not perfectly-$T_2$, it is certainly not perfectly-$T_3$, though we still need to check that it is completely-$T_3$.

Recall that the Uncountable Fort Space was defined to be $X\coloneqq \R$ with the closed sets being precisely the finite sets and also the sets which contained $0$.

So, let $C\subseteq X$ be closed and let $x_0\in C^{\comp}$.  First let us do the case where $x_0=0$.  Then, we may define $f\colon X\rightarrow [0,1]$ by
\begin{equation}
f(x)\coloneqq \begin{cases}1 & \text{if }x\in C \\ 0 & \text{otherwise.}\end{cases}
\end{equation}
Then, $f^{-1}(1)=C$ is closed and $f^{-1}(0)$ contains $0\in X$, and so is closed.  Thus, $f$ is continuous.  Now consider the case where $0\in C$.  Then, we may define $f\colon X\rightarrow [0,1]$ by
\begin{equation}
f(x)\coloneqq \begin{cases}1 & \text{if }x=x_0 \\ 0 & \text{otherwise.}\end{cases}
\end{equation}
$f^{-1}(1)$ is just a point and so is closed and $f^{-1}(0)$ contains $0\in X$ and so is closed.  Finally, let us consider the case where $x_0\neq 0$ and $0\notin C$.  Then, we may define $f\colon X\rightarrow [0,1]$ by
\begin{equation}
f(x)\coloneqq \begin{cases}1 & \text{if }x\in C \\ 0 & \text{if }x=x_0 \\ \tfrac{1}{2} & \text{otherwise.}\end{cases}
\end{equation}
Then, $f^{-1}(1)=C$ is closed, $f^{-1}(0)=\{ x_0\}$ is finite and hence closed, and $f^{-1}(\frac{1}{2})$ contains $0\in X$ and is hence closed.  Thus, indeed, $X$ is completely-$T_3$.
\end{exm}
\begin{exr}{}{exr3.6.108}
Show that a subspace of a perfectly-$T_3$ space is perfectly-$T_3$.
\end{exr}
Note that the same example (\cref{exm3.6.73}) that showed that an arbitrary product of perfectly-$T_2$ spaces need not be perfectly-$T_2$ also shows that an arbitrary product of perfectly-$T_3$ spaces need not be even perfectly-$T_2$, much less perfectly-$T_3$.  However, in fact, it's even worse:  not even a finite product of perfectly-$T_3$ spaces need be perfectly-$T_3$.
And now we return to the issue of the product of two perfectly-$T_3$ spaces.
\begin{exm}{A product of two perfectly-$T_3$ spaces that is not perfectly-$T_3$}{DoubleArrowSpace}\footnote{This example comes from (again) Brian M.~Scott's \href{http://math.stackexchange.com/questions/1859028/}{math.stackexchange answer}.}
\forwardref

\noindent
Define $X\coloneqq [0,1]\times [0,1]$ but do \emph{not} equip $X$ with the product topology.  Instead, we will equip it with the order topology (\cref{OrderTopology}) with respect to a particular total-order, the \emph{lexicographic ordering}.\footnote{We saw this briefly once before in \cref{exm1.2.28}.}  For $\coord{x_1,y_1},\coord{x_2,y_2}\in X$, define
\begin{equation*}
\coord{x_1,y_1}\leq \coord{x_2,y_2}\text{ iff }x_1<x_2\text{ or }(x_1=x_2\text{ and }y_1<y_2).
\end{equation*}

Now define $A\coloneqq \left\{ \coord{x,y}\in X:y=0\text{ or }y=1\text{.}\right\}$.  That is, $A$ is the union of the top and bottom edges of the unit square equipped with lexicographic ordering.  $A$ is our example of a perfectly-$T_3$ spaces such that $A\times A$ is not perfectly-$T_3$.

Note that the interior of the `top edge' of $A$ and the `bottom edge' of $A$ are homeomorphic copies of the Sorgenfrey line (\cref{SorgenfreyPlane}).  In that example, we showed that the Sorgenfrey Line is perfectly-$T_4$.
\begin{exr}{}{}
Use the fact that the Sorgenfrey Line is perfectly-$T_4$ to show that $A$ is perfectly-$T_4$.
\end{exr}

As $A$ contains a copy (in fact, it contains two) of the Sorgenfrey Line, $A\times A$ contains a copy of the Sorgenfrey Plane, which we showed was not $T_4$ in \cref{SorgenfreyPlane}.  Thus, as $A\times A$ has a subspace that is not $T_4$, $A\times A$ cannot be homeomorphic to a metric space.\footnote{Because subspaces of metric spaces are of course metric spaces, and metric spaces are uniformly-perfectly-$T_4$ (\cref{prp5.4.13}).}\footnote{Topological spaces that are homeomorphic to metric spaces are called \term{metrizable}\index{Metrizable}.}

\begin{exr}{}{}
Show that $A$ is compact.
\end{exr}
\begin{lma}{}{}
Let $X$ be a topological space.  Then, if $X$ is compact and $\left\{ \coord{x,x}\in X\times X:x\in X\right\}$ is a $G_{\delta}$ set, then $X$ is homeomorphic to a metric space.
\begin{proof}
We leave this as an exercise.
\begin{exr}{}{}
Prove this yourself.
\end{exr}
\end{proof}
\end{lma}
Define
\begin{equation}
\Delta \coloneqq \left\{ \coord{a,a}\in A\times A:a\in A\right\} .
\end{equation}
\begin{exr}{}{}
Show that $\Delta$ is closed.
\end{exr}
As $A$ is compact, if $\Delta$ were a $G_{\delta}$ set, then $A$ would be homeomorphic to a metric space by the lemma, and so $A\times A$ would be homeomorphic to a metric space:  a contradiction.  Therefore, $\Delta$ is not a $G_{\delta}$ set.  However, if $A\times A$ were perfectly-$T_3$, then all closed subsets of $A\times A$ should be $G_{\delta}$s.  As $\Delta$ is closed but not a $G_{\delta}$, it follows that $A\times A$ cannot be perfectly-$T_3$.
\begin{rmk}
$A$ is the \term{Double Arrow Space}\index{Double Arrow Space}.
\end{rmk}
\end{exm}

The separation axioms $T_0$ through perfectly-$T_2$ all had to do with separation of points.  All the $T_3$ separation axioms had to do with separating closed sets from points.  Finally, the $T_4$ axioms have to do with separating closed sets from closed sets.
\begin{dfn}{Normal}{Normal}
$X$ is \term{normal}\index{Normal (topological space)} iff any two disjoint closed subsets can be separated by neighborhoods.
\begin{rmk}
Similarly as with the definition of regular (\cref{Regular}), we do not need to consider topological-distinguishability and mere separatedness.
\end{rmk}
\begin{rmk}
Similarly as with the term ``regular'', some authors reverse my conventions of normal and $T_4$.  The motivation of our choice of convention is the same as it was for $T_3$.
\end{rmk}
\end{dfn}
\begin{exm}{A space that is normal but not $T_0$}{}
The indiscrete topology on any set with at least two points is (almost) vacuously normal but not $T_0$:  there are only two closed sets $\emptyset$ and $X$, and these can certainly be separated from each other by neighborhoods.
\end{exm}
Stupid examples like this is why we almost always care about the case when we in addition impose the condition of being $T_1$.
\begin{dfn}{$T_4$}{T4}
$X$ is \term{$T_4$}\index{$T_4$} iff it is $T_1$ and normal.
\begin{rmk}
Just as before, the condition of $T_1$ is imposed so that this is a strict specialization of being $T_3$.
\end{rmk}
\end{dfn}
There are at least two relatively large families of spaces that are $T_4$:  metric spaces (which are in fact perfectly-$T_4$---see \cref{prp5.4.13}) and \emph{compact} spaces.
\begin{prp}{}{prp4.6.83}
If $X$ is compact, then it is $T_4$.
\begin{proof}
Suppose that $X$ is compact.  Then, closed subsets are quasicompact (\cref{exr4.2.33}), and hence can be separated by neighborhoods by \cref{exr4.6.39}.
\end{proof}
\end{prp}
Recall that (\cref{ThomasTentSpace}) there are $T_3$ spaces that are \emph{not} completely-$T_2$.  This does not happen with $T_4$ and completely-$T_3$:  it turns out that every $T_4$ space is completely-$T_3$, as we shall see in a moment, in fact, they are what is called completely-$T_4$---see \cref{UrysohnsLemma}.  In particular, we shouldn't try to look for a space that is $T_4$ but not completely-$T_3$.  On the other hand, there are spaces that are perfectly-$T_3$ but not $T_4$.
\begin{exm}{A space that is perfectly-$T_3$ but not $T_4$}{NiemytzkisTangentDiskTopology}\footnote{This example comes from \cite[pg.~100]{Steen}.}
\forwardref

\noindent
Define $X\coloneqq \R \times \R _0^+$, the upper-half plane, and define a base for a topology on $X$ by
\begin{equation*}
\begin{split}
\cover{B} & \coloneqq \left\{ B_\varepsilon (\coord{x,y})\subseteq \R \times \R ^+:\varepsilon >0,\ \coord{x,y}\in \R \times \R ^+\right\} \\
& \qquad \cup \left\{ B_\varepsilon (\coord{x,\varepsilon})\cup \{ \coord{x,0}\} :x\in \R ,\ \varepsilon >0\right\} .
\end{split}
\end{equation*}
That is, we take all $\varepsilon$-balls contained in the (strict) upper half-plane together with all $\varepsilon$-balls in the upper half-plane which are `tangent' to the $x$-axis (together with the point of tangency).

We first check that $X$ is perfectly-$T_3$.  So, let $C\subseteq X$ be closed and let $\coord{x_0,y_0}\in C^{\comp}$.  The subspace topology of $\R ^+\times \R ^+\subseteq X$ is just the usual topology, which, being a metric space, is perfectly-$T_4$.  Thus, we only need to check the case where $y_0=0$.  Let $\varepsilon >0$ be such that $B_{\varepsilon}(\coord{x_0,\varepsilon})$ is disjoint from $C$.  Then let $\delta >0$ be sufficiently small and less than $1$ so that $\dist _C^{-1}([0,\delta ))\subseteq B_{\varepsilon}(\coord{x_0,\varepsilon})^{\comp}$.  Then define $f\colon X\rightarrow [0,1]$ by
{\small
\begin{equation*}
f(\coord{x,y})\coloneqq \begin{cases}0 & \text{if }\coord{x,y}=\coord{x_0,0} \\
\delta \cdot \left( \frac{(x-x_0)^2+y^2}{2\varepsilon y} \right) & \text{if }\coord{x,y}\in D_{\varepsilon}(\coord{x_0,\varepsilon})\setminus \{ \coord{x_0,0}\} \\
\delta & \text{if }\coord{x,y}\in B_{\varepsilon}(\coord{x_0,\varepsilon})\cap \\  & \ \enspace \dist _C^{-1}([0,\delta ))^{\comp}\cap (\R \times \R ^+) \\
\dist _C(\coord{x,y}) & \text{if }\coord{x,y}\in \dist _C^{-1}([0,\delta ]).\end{cases}
\end{equation*}
}
This is continuous by the \namerefpcref{PastingLemma}.

We now check that $X$ is not $T_4$.  To do this, we show that $\Q ,\R \setminus \Q \subseteq \R \times \R _0^+$ (as subsets of the $x$-axis) are closed and cannot be separated by neighborhoods.\footnote{We write $\Q$ and $\R \setminus \Q$ instead of $\Q \times \{ 0\}$ and $(\R \setminus \Q )\times \{ 0\}$.  Note that we cannot write $\Q ^{\comp}$ to denote the irrationals as usual because, in this context, $\Q ^{\comp}$ means $(\R \times \R _0^+)\setminus \Q$}  In fact, we check that every subset $S\subseteq \R \times \{ 0\}$ is closed.  To show that, we show that $S^{\comp}$ is open.  For $\coord{x,y}\in S^{\comp}$, if $y>0$, then certainly we can put an $\varepsilon$-ball around it that does not intersect the $x$-axis, and hence does not intersection $S$.  On the other hand, for $\coord{x,0}\in S^{\comp}$, $B_{\varepsilon}(\coord{x,\varepsilon})\cup \{ \coord{x,0}\}$ intersects the $x$-axis only at $\coord{x,0}$, and so does not intersect $S$.  Therefore, $S^{\comp}$ is open, and so $S$ is closed.

We now check that $\Q$ and $\R \setminus \Q$ cannot be separated by neighborhoods.  Let $U$ be an open neighborhood of $\R \setminus \Q$.  We show that there is some point $q_0\in \Q$ every neighborhood of which intersects $U$.

For $x\in \R \setminus \Q$, let $\varepsilon _x>0$ be such that
\begin{equation}\label{eqn4.4.83}
\R \setminus \Q \ni x\in B_{\varepsilon _x}(\coord{x,\varepsilon _x})\cup \{ x\} \subseteq U.
\end{equation}
For $m\in \Z ^+$, define
\begin{equation}\label{eqn4.4.84}
S_m\coloneqq \{ x\in \R \setminus \Q :\varepsilon _x>\tfrac{1}{m}\} .
\end{equation}
Then,
\begin{equation}
\R =\bigcup _{m\in \Z ^+}S_m\cup \bigcup _{x\in \Q}\{ x\},
\end{equation}
and so
\begin{equation}
\R =\bigcup _{m\in \Z ^+}\Cls _{\R}(S_m)\cup \bigcup _{x\in \Q}\{ x\} .\footnote{The subscript $\R$ here is to indicate that we are taking the closure with respect to the usual topology on $\R$.}
\end{equation}
As $\R$ is a complete metric space, by the \namerefpcref{BaireCategoryTheorem}, there must be some $m_0\in \Z ^+$ such that $\Cls _{\R}(S_{m_0})$ does \emph{not} have empty interior.  In particular, its interior must contains some rational point, so let $q_0\in \Cls _{\R}(S_{m_0})\cap \Q$.  Then, for every $\varepsilon >0$, $(q_0-\varepsilon ,q_0+\varepsilon )$ intersects $S_{m_0}$, say at $x_\varepsilon$.  That $x_{\varepsilon}\in S_{m_0}$ means that (from \eqref{eqn4.4.83} and \eqref{eqn4.4.84})
\begin{equation}
B_{1/m_0}(\coord{x_{\varepsilon},\tfrac{1}{m_0}})\subseteq \footnote{As $\frac{1}{m_0}<\varepsilon _{x_{\varepsilon}}$, the Triangle Inequality shows that if you are within $\frac{1}{m_0}$ of $\coord{x_{\varepsilon},\frac{1}{m_0}}$ then you must be within $\varepsilon _{x_{\varepsilon}}$ of $\coord{x_{\varepsilon},\varepsilon _{x_{\varepsilon}}}$.}B_{\varepsilon _{x_{\varepsilon}}}(\coord{x_{\varepsilon},\varepsilon _{x_{\varepsilon}}})\subseteq U.
\end{equation}
But then, for all $\varepsilon$ sufficiently small,
\begin{equation*}
\coord{x_\varepsilon ,\varepsilon}\in B_{\varepsilon}(\coord{q_0,\varepsilon})\cap B_{1/m_0}(\coord{x_\varepsilon ,\tfrac{1}{m_0}})\subseteq B_{\varepsilon}(\coord{q_0,\varepsilon})\cap U.
\end{equation*}
Thus, every neighborhood of $\coord{q_0,0}\in \Q$ intersects $U$.
\begin{rmk}
This is \term{Niemytzki's Tangent Disk Topology}\index{Niemytzki's Tangent Disk Topology}.
\end{rmk}
\end{exm}
It is at this point that we (should) turn to the issue of subspaces and products of $T_4$ spaces.  Unfortunately, probably a bit surprisingly, it is neither the case that a subspace of a $T_4$ space is necessarily $T_4$ nor the case that the product of two $T_4$ spaces is necessarily $T_4$.
We now finally return to the issue of subspaces and products of $T_4$ spaces.
\begin{exm}{A subspace of a $T_4$ space that is not $T_4$}{exm4.4.69}
By \namerefpcref{TychonoffsTheorem}, and the fact that compact spaces are $T_4$ (\cref{prp4.6.83}), arbitrary products of $[0,1]$ are $T_4$.  Furthermore, by the \namerefpcref{TychonoffEmbeddingTheorem}, every completely-$T_3$ space is homeomorphic to a subspace of a product of copies of $[0,1]$, that is, a $T_4$ space.  Thus, it suffices to exhibit a single space that is completely-$T_3$ but not $T_4$.  \cref{NiemytzkisTangentDiskTopology} (the previous example) is an example of such a space.
\end{exm}
We mentioned that the product of two $T_4$ spaces need not be $T_4$.  In fact, it's worse than this:  the product of two \emph{perfectly}-$T_4$ spaces need not even be $T_4$, and so we wait for this counter-example until having defined ``perfectly-$T_4$''---see \cref{SorgenfreyPlane}.

\begin{dfn}{$T_{4\frac{1}{2}}$}{T412}
$X$ is \term{$T_{4\frac{1}{2}}$}\index{$T_{4\frac{1}{2}}$} iff it is $T_1$ and any two disjoint closed subsets can be separated by closed neighborhoods.
\begin{rmk}
Just as with $T_{3\frac{1}{2}}$ (\cref{T312}), this terminology is not standard, presumably because it is actually just equivalent to $T_4$.
\end{rmk}
\begin{rmk}
It turns out that $T_4$ is equivalent to $T_{4\frac{1}{2}}$.  We don't say any more about this now, because in fact $T_4$ is equivalent to completely-$T_4$ (and as usual, the implication that completely-$T_4$ implies $T_{4\frac{1}{2}}$ is relatively easy).
\end{rmk}
\end{dfn}

\begin{dfn}{Completely-$T_4$}{CompletelyT4}
$X$ is \term{completely-$T_4$} iff it is $T_1$ and any two disjoint closed subsets can be completely-separated.
\end{dfn}
This is in fact \emph{equivalent} to being $T_4$, though this is relatively nontrivial, and the statement even has a name associated to it.  Before we prove that, however, we first present a useful 'lemma'.
\begin{prp}{}{prp4.5.91}
Let $X$ be a $T_1$ topological space.  Then,
\begin{enumerate}
\item \label{enm4.5.91.i}$X$ is $T_3$ iff whenever $U$ is an open neighborhood of $x\in X$, there is an open neighborhood $V$ of $x$ such that $x\in V\subseteq \Cls (V)\subseteq U$; and
\item \label{enm4.5.91.ii}$X$ is $T_4$ iff whenever $U$ is an open neighborhood of a closed subset $C\subseteq X$, there is an open neighborhood $V$ of $C$ such that $C\subseteq V\subseteq \Cls (V)\subseteq U$.
\end{enumerate}
\begin{proof}
We first prove \cref{enm4.5.91.i}.

$(\Rightarrow )$ Suppose that $X$ is $T_3$.  Let $U$ be an open neighborhood of $x\in X$.  Then, $U^{\comp}$ is a closed set that does not contain $x$, and so there are disjoint open neighborhoods $V$ and $W$ of $x$ and $U^{\comp}$ respectively.  Then,
\begin{equation}
x\in V\subseteq \Cls (V)\subseteq W^{\comp}\subseteq U.
\end{equation}

\blankline
\noindent
$(\Leftarrow )$ Suppose that whenever $U$ is an open neighborhood of $x\in X$, there is an open neighborhood $V$ of $x$ such that $x\in V\subseteq \Cls (V)\subseteq U$.  Let $C\subseteq X$ be closed and let $x\in C^{\comp}$.  Then, $C^{\comp}$ is an open neighborhood of $x$, and so there is an open neighborhood $V$ of $x$ such that
\begin{equation}
x\in V\subseteq \Cls (V)\subseteq C^{\comp},
\end{equation}
and so $V$ and $\Cls (V)^{\comp}$ are disjoint open neighborhoods of $x$ and $C$ respectively, so that $X$ is $T_3$.

\blankline
\noindent
\cref{enm4.5.91.ii}
\begin{exr}[breakable=false]{}{}
Prove \cref{enm4.5.91.ii}.
\end{exr}
\end{proof}
\end{prp}
\begin{thm}{Urysohn's Lemma}{UrysohnsLemma}\index{Urysohn's Lemma}\footnote{Proof adapted from \cite[pg.~207]{Munkres}.}
If $X$ is $T_4$, then $X$ is completely-$T_4$.
\begin{rmk}
This is often stated as ``If $C_1,C_2\subseteq X$ are disjoint closed subsets of a $T_4$ space $X$, then there is a continuous function $f\colon X\rightarrow [0,1]$ that is $0$ on $C_1$ and $1$ on $C_2$.''.
\end{rmk}
\begin{proof}
Suppose that $X$ is $T_4$.  Let $C_1,C_2\subseteq X$ be closed.

Let $\{ r_m:m\in \N \}$ be an enumeration of the rationals in $[0,1]$ with $r_0=1$ and $r_1=0$.  We define a collection of open sets $\{ U_{r_m}:m\in \N \}$ inductively.  During this process, we will apply \cref{prp4.5.91} repeatedly.

First of all, define $U_1\coloneqq C_2^{\comp}$.  Then, $U_1$ is an open neighborhood of $C_1$, and so there is some other open neighborhood $U_0$ of $C_1$ such that $C_1\subseteq U_0\subseteq \Cls (U_0)\subseteq U_1$.

Suppose that we have defined $U_{r_0},\ldots U_{r_m}$ such that $\Cls (U_p)\subseteq U_q$ if $r<q$ for $r,q\in \{ r_0,\ldots ,r_m\}$.  We wish to define $U_{m+1}$ so that this property still remains to be true.  As there are only finitely many rational numbers in $\{ r_0,\ldots ,r_m\}$ there is a largest $p_0\in \{ r_0,\ldots ,r_m\}$ with $p_0<r_{m+1}$ and similarly there is a smallest $q_0\in \{ r_0,\ldots ,r_m\}$ with $r_{m+1}<q_0$.  Take $U_{r_{m+1}}$ so that
\begin{equation}
\Cls (U_{p_0})\subseteq U_{r_{m+1}}\subseteq \Cls (U_{r_{m+1}})\subseteq U_{q_0}.\footnote{Here we are applying \cref{prp4.5.91}.\cref{enm4.5.91.ii}.}
\end{equation}
Proceeding inductively, this allows us to define $U_{r_m}$ for all $m\in \N$, and hence, we have defined $U_r$ for all $r\in \Q \cap [0,1]$.  By construction, it follows that
\begin{equation}
\Cls (U_p)\subseteq U_q
\end{equation}
for $p,q\in \Q \cap [0,1]$ for $p<q$.

For $x\in X$, define
\begin{equation}
Q_x\coloneqq \{ r\in \Q \cap [0,1]:x\in U_r\} .
\end{equation}
Note that $Q_x$ is empty iff $x\in C_2$.  We then in turn define
\begin{equation}
f(x)\coloneqq \begin{cases}1 & \text{if }x\in C_2 \\ \inf \left( Q_x\right) & \text{otherwise.}\end{cases}
\end{equation}
Of course $f(C_2)=\{ 1\}$.  Furthermore, if $x\in C_1$, then $x\in U_0$, and so indeed $f(x)=0$.  Thus, we need only check that $f$ is continuous.

So, let $x_0\in X$.  First suppose that $f(x_0)\neq 0,1$.  The other two cases are similar.  Let $\varepsilon >0$ be such that $B_\varepsilon (f(x_0))\subseteq [0,1]$ (if $f(x_0)=0$, for example, then you will instead use $[0,\varepsilon )$ in place of $B_\varepsilon (f(x_0))$).  Let $p,q\in \Q$ be such that
\begin{equation}
f(x_0)-\varepsilon <p<f(x_0)<q<f(x_0)+\varepsilon .
\end{equation}
Then, $U\coloneqq U_q\setminus \Cls (U_p)$ is open in $X$.  We claim that $f(U)\subseteq B_\varepsilon (x_0)$.  This will show that $f$ is continuous at $x_0$, and hence continuous as $x_0$ was arbitrary.

So, let $x\in U_q\setminus \Cls (U_p)$.  Then, $q\in Q_x$, and so $f(x)\coloneqq \inf (Q_x)\leq q$.  On the other hand, if $r\in Q$, so that $x\in U_r$, we cannot have that $U_r\subseteq U_p$ ($x\notin U_p$), and so we cannot have $r\leq p$, so that we must have $p<r$.  Therefore, $p$ is a lower-bound for $Q_x$, and so $f(x)=\inf (Q_x)\geq p$.  Hence,
\begin{equation}
f(x_0)-\varepsilon <p\leq f(x)\leq q<f(x_0)+\varepsilon ,
\end{equation}
so that $f(x)\in B_{\varepsilon}(x_0)$, so that $f(U)\subseteq B_{\varepsilon}(x_0)$, as desired.
\end{proof}
\end{thm}
Thus, as $T_4$ is equivalent to $T_{4\frac{1}{2}}$ is equivalent to completely-$T_4$, we needn't ask additional questions about subspaces or products of $T_{4\frac{1}{2}}$ or completely-$T_4$ spaces, and, as already mentioned, these issues are addressed in \cref{exm4.4.69,SorgenfreyPlane}.

We mentioned shortly after the definition of completely-$T_3$ that completely-$T_3$ spaces are precisely those spaces which embed into products of $[0,1]$.  Having proven \nameref{UrysohnsLemma}, we can now return to this result.
\begin{thm}{Tychonoff Embedding Theorem}{TychonoffEmbeddingTheorem}\index{Tychonoff Embedding Theorem}
Let $X$ a topological space.  Then, $X$ is completely-$T_3$ iff it is homeomorphic to a subspace of $\prod _S[0,1]$ for some set $S$.
\begin{proof}\footnote{Proof adapted from \cite[Theorem 34.2]{Munkres}.}
$(\Rightarrow )$ Suppose that $X$ is completely-$T_3$.  Define $\iota \colon X\rightarrow \prod _{\Mor _{\Top}(X,[0,1])}[0,1]$ by $\iota (x)_f\coloneqq f(x)$, that is, the $f$-component of $\iota (x)$ (for $f$ an element of the index set $\Mor _{\Top}(X,[0,1])$) is $f(x)\in [0,1]$.  We claim that $\iota$ is a homeomorphism onto its image.

We first show that $\iota$ is actually continuous.  So, let $\lambda \mapsto x_{\lambda}\in X$ converge to $x_{\infty}\in X$.  We wish to show that $\lambda \mapsto \iota (x_{\lambda})$ converges to $\iota (x_{\infty})$.  However, by our characterization of convergence in the product topology (\cref{crl4.5.15}), this is the case iff $\lambda \mapsto \iota (x_{\lambda})_f\coloneqq f(x_{\lambda})$ converges to $\iota (x_{\infty})_f\coloneqq f(x_{\infty})$ for all $f\in \Mor _{\Top}(X,[0,1])$, however, this is the case because each such $f$ is continuous.

We next check that $\iota$ is injective.  Let $x_1,x_2\in X$ be distinct.  Then, as $X$ is in particular completely-$T_2$, there is a continuous function $f\colon X\rightarrow [0,1]$ such that $f(x_1)=0$ and $f(x_2)=1$.  Thus, $\iota (x_1)_f\neq \iota (x_2)_f$, and so $\iota (x_1)\neq \iota (x_2)$.  Thus, $\iota$ is injective.

As $\iota$ is injective, it has an inverse $\iota (X)\rightarrow X$.  To show that $\iota$ is a homeomorphism onto its image, we wish to show that this inverse is continuous.  To show that, we want to show that the preimage of every open set under this inverse is closed.  However, the preimage of a set $U\subseteq X$ under this inverse is just $\iota (U)$, so it suffices to show that $\iota (U)\subseteq \prod _{f\in \Mor _{\Top}(X,[0,1])}[0,1]$ is open for every open $U\subseteq X$.

So, let $U\subseteq X$ be open.  Let $y_0\in \iota (U)$.  We find an open subset $V$ of $\iota (X)$ with $y_0\in V\subseteq \iota (U)$.

Write $y_0=\iota (x_0)$ for $x_0\in U$.  As $X$ is completely-$T_3$ there is a continuous function $f_0\colon X\rightarrow [0,1]$ such that $f_0(U^{\comp})=0$ and $f_0(x_0)=1$.  Define $V\coloneqq \pi _{f_0}^{-1}((0,1])\cap \iota (X)$, where $\pi _{f_0}\colon \prod _{f\in  \Mor _{\Top}(X,[0,1])}[0,1]\rightarrow [0,1]$ is the projection.  This is certainly open in $\iota (X)$, and so all that remains to be shown is that (i)~$y_0=\iota (x_0)\in V$ and that (ii)~$V\subseteq \iota (U)$.

However, as
\begin{equation}
\pi _{f_0}(y_0)\coloneqq \pi _{f_0}(\iota (x_0))\coloneqq f_0(x_0)=1\in (0,1],
\end{equation}
it follows that $y_0\in V$.  For the other condition, let $y\in V\subseteq \iota (X)$, and write $y=\iota (x)$ for some $x\in X$.  As $\iota (x)=y\in V\subseteq \pi _{f_0}^{-1}((0,1]))$, $f_0(x)>0$, and so $x\notin U^{\comp}$, and so $x\in U$, and so $y\in \iota (U)$.  Thus, $V\subseteq \iota (U)$, as desired.

\blankline
\noindent
$(\Leftarrow )$ Suppose that $X$ is homeomorphic to a subspace of $\prod _S[0,1]$ for some set $S$.  $\prod _S[0,1]$ is quasicompact by \namerefpcref{TychonoffsTheorem} and $T_2$ by \cref{exr4.6.38}, that is, compact, hence $T_4$ by \cref{prp4.6.83}, hence completely-$T_4$ by \namerefpcref{UrysohnsLemma}, hence completely-$T_3$.  As subspaces of completely-$T_3$ spaces are completely-$T_3$ (\cref{exr3.6.106}), it follows that $X$ is completely-$T_3$.
\end{proof}
\end{thm}

\begin{dfn}{}{PerfectlyT4}
$X$ is \term{perfectly-$T_4$} iff it is $T_1$ and any two disjoint closed can be perfectly-separated.
\begin{rmk}
For some reason, this is sometimes called $T_6$.  What is $T_5$ you ask?  Evidently $T_5$ means $T_1$ and every subspace is $T_4$.
\end{rmk}
\end{dfn}
There are other conditions equivalent to perfectly-$T_4$ that are sometimes easier to check.
\begin{prp}{}{prp3.6.131}
The following are equivalent.
\begin{enumerate}
\item \label{prp3.6.131.i}$X$ is perfectly-$T_4$.
\item \label{prp3.6.131.ii}$X$ is $T_1$ and for every closed $C\subseteq X$, there is a continuous function $f\colon X\rightarrow [0,1]$ such that $C=f^{-1}(0)$.
\item \label{prp3.6.131.iii}$X$ is $T_4$ and every closed subset of $X$ is a $G_{\delta}$.
\end{enumerate}
\begin{proof}\footnote{In case you're wondering ``Is there a good reason he didn't do this in the usual ``$(i)\Rightarrow (ii)\Rightarrow (iii)\Rightarrow (i)$' order?'', the answer is ``No, not really.''.  I just happened to think of the proofs in this order, and it's not clear to me that rewriting things in the `usual' order would necessarily make things clearer.}
\forwardref

\noindent
$(\cref{prp3.6.131.i}\Rightarrow \cref{prp3.6.131.ii})$ Suppose that $X$ is perfectly-$T_4$.  $X$ is certainly $T_1$.  Let $C\subseteq X$ be closed.  If $X\setminus C=\emptyset$, take $f$ to be the function the constant function $x\mapsto 0$.  Otherwise, let $x_0\in C^{\comp}$.  As $\{ x_0\}$ is closed, there is a continuous $f\colon X\rightarrow [0,1]$ such that $C=f^{-1}(0)$ and $\{ x_0\} =f^{-1}(1)$.

\blankline
\noindent
$(\cref{prp3.6.131.ii}\Rightarrow \cref{prp3.6.131.ii})$ Suppose that $X$ is $T_1$ and for every closed $C\subseteq X$, there is a continuous function $f\colon X\rightarrow [0,1]$ such that $C=f^{-1}(0)$.  Let $C_1,C_2\subseteq X$ be closed and disjoint.  Let $f_1,f_2\colon X\rightarrow [0,1]$ be continuous and such that $C_1=f_1^{-1}(0)$ and $C_2=f_2^{-1}(0)$.  Now, define $g\colon X\rightarrow [0,1]$ by
\begin{equation}
g(x)\ceqq \frac{f_1(x)}{f_1(x)+f_2(x)}.
\end{equation}
Notice that the denominator never vanishes as the zero sets of $f_1$ and $f_2$ are disjoint.  This is therefore a continuous function on $X$ taking values in $[0,1]$.  It is certainly $0$ on $C_1$ and $1$ on $C_2$.  Conversely, if $g(x)=0$, then $f_1(x)=0$, and so $x\in C_1$.  Thus, indeed, $g^{-1}(0)=C_1$.  Similarly, if $g(x)=1$, then $f_1(x)=f_1(x)+f_2(x)$, and so $f_2(x)=0$, and so $x\in C_2$.  Thus, $g^{-1}(1)=C_2$.  Thus, $X$ is perfectly-$T_4$

\blankline
\noindent
$(\cref{prp3.6.131.i}\Rightarrow \cref{prp3.6.131.iii})$ Suppose that $X$ is perfectly-$T_4$.  $X$ is certainly $T_4$.  On the other hand, let $C\subseteq X$ be closed and let $f\colon X\rightarrow [0,1]$ be continuous and such that $C=f^{-1}(0)$.  Then,
\begin{equation}
C=\bigcap _{m\in \Z ^+}f^{-1}\left( [0,\tfrac{1}{m})\right) ,
\end{equation}
and so $C$ is a $G_{\delta}$.

\blankline
\noindent
$(\cref{prp3.6.131.iii}\Rightarrow \cref{prp3.6.131.ii})$ Suppose that $X$ is $T_4$ and every closed subset of $X$ is a $G_{\delta}$.  $X$ is certainly $T_1$.  Let $C\subseteq X$ be closed.  Then, $C$ is a $G_{\delta}$, and so we can write $C=\bigcap _{m\in \N}U_m$, for $U_m\subseteq X$ open.  Without loss of generality, assume that $U_m\supseteq U_{m+1}$.\footnote{If necessary, replace $U_m$ with $\bigcap _{k=0}^mU_k$.}  As $X$ is $T_4$, by \namerefpcref{UrysohnsLemma}, there is a continuous function $f_m\colon X\rightarrow [0,1]$ such that $f_m(C)=0$ and $f_m(U_m^{\comp})=1$.  Define $f\colon X\rightarrow [0,1]$ by
\begin{equation}
f(x)\ceqq \sum _{m\in \N}\frac{f_m(x)}{2^{m+1}}.
\end{equation}
This defines a continuous function on $X$ by \cref{exr4.4.22} (completeness of bounded continuous functions) with values in $[0,1]$.  It is certainly $0$ on $C$.  On the other hand, if $x\notin C$, then $x\in U_m^{\comp}$ for some $m\in \N$, in which case $f(x)\geq \frac{f_m(x)}{2^{m+1}}=\frac{1}{2^{m+1}}>0$.  Thus, indeed, taking the contrapositive gives us $f(x)=0$ implies $x\in C$, so that $C=f^{-1}(0)$, as desired.
\end{proof}
\end{prp}
\begin{exm}{A space that is completely-$T_4$ but not perfectly-$T_4$}{}
The Uncountable Fort Space from \cref{UncountableFortSpace} will once again do the trick.  In \cref{exm4.5.48}, we show that it is not perfectly-$T_2$, and so it is certainly not going to be perfectly-$T_4$.  We still need to check that it is completely-$T_4$.

Recall that the Uncountable Fort Space was defined to be $X\coloneqq \R$ with the closed sets being precisely the finite sets and also the sets which contained $0$.

So, let $C_1,C_2\subseteq X$ be disjoint closed sets.  Let us first do the case where neither $C_1$ nor $C_2$ contains $0\in X$.  Then, we may define $f\colon X\rightarrow [0,1]$ by
\begin{equation}
f(x)\coloneqq \begin{cases}0 & \text{if }x\in C_1 \\ 1 & \text{if }x\in C_2 \\ \tfrac{1}{2} & \text{otherwise.}\end{cases}
\end{equation}
The preimage of $0$ is $C_1$ is closed, the preimage of $1$ is $C_2$ is closed, and the preimage of $\frac{1}{2}$ contains $0\in X$ and so is closed.  Thus, this function is continuous.  Now suppose that $0\in C_1$.  Then, we may define $f\colon X\rightarrow [0,1]$ by
\begin{equation}
f(x)\coloneqq \begin{cases}0 & \text{if }x\in C_1 \\ 1 & \text{if }x\in C_2.\end{cases}
\end{equation}
The preimage of $0$ is $C_1$ is closed and the preimage of $1$ is $C_2$ is closed.  Thus, this function is continuous.
\end{exm}
And now we turn to the usual issues of subspaces and products of perfectly-$T_4$ spaces.
\begin{prp}{}{prp3.6.130}
Let $X$ be a perfectly-$T_4$ space and let $S\subseteq X$.  Then, $S$ is perfectly-$T_4$.
\begin{proof}\footnote{Proof adapted from \href{http://math.stackexchange.com/questions/1840614/}{math.stackexchange}.}
Let $C,D\subseteq Y$ be closed and disjoint.  Write $C=C'\cap S$ and $D=D'\cap S$ for $C,D\subseteq Y$ closed.\footnote{See the defining result of the subspace topology \cref{SubspaceTopology}.}  As $X$ is perfectly-$T_4$, there is a continuous function $f\colon X\rightarrow [0,1]$ such that $C'=f^{-1}(0)$ and a continuous function $g\colon X\rightarrow [0,1]$ such that $D'=g^{-1}(0)$.  Define $h\colon S\rightarrow [0,1]$ by
\begin{equation}
h(s)\coloneqq \begin{cases}1 & \text{if }s\in D \\ \frac{f(s)}{f(s)+g(s)}& \text{otherwise.}\end{cases}
\end{equation}
Note that $f(x)+g(x)=0$ iff $f(x)=0=g(x)$ iff $s\in C'\cap D'$.  However, as $s\in S$, this would force $s\in C'\cap D'\cap S=C\cap D=\emptyset$.  Thus, the denominator is never $0$, and so defines a continuous function on $S$.

For $s\in D$, $g(s)=0$, in which case we have $h(s)=1$.  For $s\in C$, we have $f(s)=0$, in which case we have $h(s)=0$.  Conversely, if $h(s)=0$, then $f(s)=0$, and so $s\in C$.  Similarly, if $h(s)=1$, then $f(s)=f(s)+g(s)$, and so $g(s)=0$, and so $s\in D$.

Thus, $h$ perfectly separates $C$ and $D$ in $S$.

Finally, as subspaces of $T_1$ spaces are $T_1$ (\cref{exr3.6.45}), this shows that $S$ is perfectly-$T_4$.
\end{proof}
\end{prp}
Unfortunately, products of perfectly-$T_4$ spaces need not even be $T_4$, much less perfectly-$T_4$.
\begin{exm}{A product of two perfectly-$T_4$ spaces that is not $T_4$}{SorgenfreyPlane}
\forwardref

\noindent
Define $S\coloneqq \R$\footnote{``$S$'' is for \emph{Sorgenfrey}, some guy's name.} and equip $S$ with the topology defined by the base
\begin{equation}
\left\{ [a,b):a,b\in \R \cup \{ \pm \infty\} ,\ a<b\right\} .
\end{equation}
$S$ is called the \term{Sorgenfrey Line}\index{Sorgenfrey Line}.\footnote{It is also sometimes called the \term{lower-limit topology}\index{Lower-limit topology} because it has the property that a net $\lambda \mapsto x_{\lambda}$ converges to $x_{\infty}\in S$ iff it is eventually contained in $[x_{\infty},x_{\infty}+\varepsilon )$ for every $\varepsilon >0$, that is, iff it converges to $x_{\infty}$ ``from the right''.  I actually prefer this term for $S$ itself, as it is more descriptive, but as $S\times S$ is called the Sorgenfrey Plane (and no one calls this the ``lower-limit plane'' or something of the like), I have decided to just use ``Sorgenfrey'' for both.}  Note that
\begin{equation}
(a,b)=\bigcup _{b-a>\varepsilon >0}[a-\varepsilon ,b)
\end{equation}
is open.  Thus, every set that is open in the usual topology is also open in the Sorgenfrey Line.  In particular, points are closed because they are closed in the usual topology, and so $S$ is $T_1$ (\cref{prp4.5.32}).

We first prove that, if $C\subseteq S$ is closed and $x\in S$ is an accumulation point of $C$ \emph{in the usual topology}, then $x\in C$ or $x-\varepsilon \in C$ for every $\varepsilon >0$.  If $x\in C$, we're done.  Otherwise, $x\in C^{\comp}$, and so there is some $[x,a)\subseteq C^{\comp}$ for $a>x$.  On the other hand, as $x$ is an accumulation point of $C$ in the usual topology, we have that for every $\varepsilon >0$, $(x-\varepsilon ,x+\varepsilon )$ intersects $C$.  Choosing $\varepsilon$ sufficiently small so that $x+\varepsilon <a$, as $[x,x+\varepsilon )\subseteq [x,a)\subseteq C^{\comp}$, we in fact must have that $(x-\varepsilon ,x)$ intersects $C$, as desired.

We now show that $S$ is perfectly-$T_4$.  So, let $C,D\subseteq S$ be closed and disjoint.  Define $g_C,g_D\colon S\rightarrow [0,1]$ respectively by
\begin{equation}
g_C(x)\coloneqq \max \{ \dist _C(x),\tfrac{1}{2}\}
\end{equation}
and
\begin{equation}
g_D(x)\coloneqq \min \{ 1-\dist _D(x),\tfrac{1}{2}\} .
\end{equation}
As they are continuous with respect to the usual topology, they are continuous with respect to the Sorgenfrey Topology.

Define
\begin{equation*}
P\coloneqq \left\{ p\in (C\cup D)^{\comp}:\text{for every }\varepsilon >0,\ p-\varepsilon \in C\cup D\text{.}\right\} .
\end{equation*}
For $p\in P$, define $b_p\coloneqq \inf \left\{ x\in C\cup D:x\geq p\right\}$, and let us say that ``$C$ comes next'' iff $b_p$ is an accumulation point of $C$ in the usual topology, similarly for ``$D$ comes next'', and ``nothing comes next'' if this set is empty.  Finally pick $a_p$ such that $p<a_p<b_p$.\footnote{Such an $a_p$ exists because there must be some $[p,a_p)\subseteq (C\cup D)^{\comp}$.  If we happen to have $a_p=b_p$, instead replace $a_p$ with some element of $(p,a_p)$.}

Now, let $f\colon S\rightarrow [0,1]$ be the function that is defined to be $0$, on $C$, $1$ on $D$, $\tfrac{1}{2}$ on $[p,\infty )$ if ``nothing comes next'', given by $g_C$ or $g_D$ on $(-\infty ,\min \{ \inf (C),\inf (D)\} )$ according to whether $\inf (C)\leq \inf (D)$ or vice-versa, change linearly from $\frac{1}{2}$ at $p$ to $g_C(a_p)$ at $a_p$ over $[p,a_p)$ if ``$C$ comes next'' and then is given by $g_C$ on $[a_p,b_p)$, and changes linearly from $\frac{1}{2}$ at $p$ to $g_D(a_p)$ at $a_p$ over $[p,a_p)$ if ``$D$ comes next'' and then is given by $g_D$ on $[a_p,b_p)$.  First note that this actually gives a well-defined function on all of $S$.  As all of domains of this `piece-wise' definition are closed, by the \nameref{PastingLemma}, this defines a continuous function.

$f$ is obviously $0$ on $C$ and $1$ on $D$.  Conversely, the only way we might have $f(x)=0$ with $x\notin C$ is if $g_C(x)=0$ for $x\in [a_p,b_p)$ for some $p\in P$.  However, that $g_C(x)=0$ implies that $\dist _C(x)=0$, so that $x$ is an accumulation point of $C$ in the original topology.  The only time this would not automatically give that $x\in C$ is if $x-\varepsilon \in C$ for every $\varepsilon >0$, but then we would have $x\in P$, in which case $f(x)=\frac{1}{2}$.  Thus, $f(x)=0$ in fact implies that $x\in C$.  Similarly, $f(x)=1$ implies that $x\in D$.  Thus, $S$ is perfectly-$T_4$.

We now check that $S\times S$ is not $T_4$.\footnote{$S\times S$ is the \term{Sorgenfrey Plane}\index{Sorgenfrey Plane}.}

Define $L\coloneqq \left\{ \coord{x,-x}\in S\times S:x\in \R \right\}$.  Note that, as the topology of $S\times S$ is finer than that of $\R ^2$, $L$ is automatically closed.  Now define
\begin{equation}
Q\coloneqq \left\{ \coord{x,-x}\in S\times S:x\in \Q \right\}
\end{equation}
and
\begin{equation}
I\coloneqq \left\{ \coord{x,-x}\in S\times S:x\in \Q ^{\comp}\right\} .
\end{equation}

We first check that $Q$ and $I$ are closed.  We will then check that they cannot be separated by neighborhoods, thereby demonstrating that $S\times S$ it not $T_4$.\footnote{Note that the remainder of the argument is very similar to the end of the argument given in \cref{NiemytzkisTangentDiskTopology}, that is, the argument that show that Niemytzki's Tangent Disk Topology is not $T_4$.}

Note that $\left( [x,x+1)\times [-x,-x+1)\right) \cap L=\{ \coord{x,-x}\}$, so that $\{ \coord{x,-x}\}$ is open in $L$, so that the subspace topology of $L$ is the discrete topology.

Also note that as $L$ is closed in the usual topology, and the Sorgenfrey Topology is finer than the usual topology, it is closed in the Sorgenfrey Topology as well.  Thus, if $\lambda \mapsto x_{\lambda}\in Q$ is a net converging to $x_{\infty}\in S\times S$, in fact we must have that $x_{\infty}\in L$.  As $L$ has the discrete topology, this implies that the net $\lambda \mapsto x_{\lambda}\in Q$ must be eventually constant, which in turn implies that $x_{\infty}\in Q$.  Thus, $Q$ is closed.  The same exact argument works with $I$ in place of $Q$, and so $I$ is likewise closed.

We now check that $Q$ and $I$ cannot be separated by neighborhoods.  Let $U$ be an open neighborhood of $I$.  We show that there is some point $q_0\in Q$ every neighborhood of which intersects $U$.

For $\coord{x,-x}\in I$, let $\varepsilon _x>0$ be such that
\begin{equation}\label{eqn4.4.99}
I\ni \coord{x,-x}\in [x,x+\varepsilon _x)\times [-x,-x+\varepsilon _x)\subseteq U.
\end{equation}
For $m\in \Z ^+$, define
\begin{equation}\label{eqn4.4.100}
S_m\ceqq \left\{ x\in \Q ^{\comp}:\varepsilon _x>\tfrac{1}{m}\right\} .\footnote{Note that this is a subset of the real numbers with the \emph{usual} topology.}
\end{equation}
Then,
\begin{equation}
\R =\bigcup _{m\in \Z ^+}S_m\cup \bigcup _{x\in \Q}\{ x\} ,
\end{equation}
and so
\begin{equation}
\R =\bigcup _{m\in \Z ^+}\Cls (S_m)\cup \bigcup _{x\in \Q}\{ x\} .
\end{equation}
By the \namerefpcref{BaireCategoryTheorem}, there must be some $m_0\in \Z ^+$ such that $\Cls (S_{m_0})$ does \emph{not} have empty interior.  So, let $(a,b)\subseteq \Cls (S_m)$ and let $\Q \ni q_0\in (a,b)$.  Then, for every $\varepsilon >0$, $(q_0-\varepsilon ,q_0+\varepsilon )$ intersects $S_m$, say at $x_{\varepsilon}$.  That $x_{\varepsilon}\in S_{m_0}$ means that (from \eqref{eqn4.4.99} and \eqref{eqn4.4.100})
\begin{equation}
\begin{split}
\MoveEqLeft {}
[x_{\varepsilon},x_{\varepsilon}+\tfrac{1}{m_0})\times [-x_{\varepsilon},-x_{\varepsilon}+\tfrac{1}{m_0})\subseteq \\ & [x_{\varepsilon},x_{\varepsilon}+\varepsilon _{x_{\varepsilon}})\times [-x_{\varepsilon},-x_{\varepsilon}+\varepsilon _{x_{\varepsilon}})\subseteq U.
\end{split}
\end{equation}
But then, for all $\varepsilon$ sufficiently small, if $q_0\leq x_{\varepsilon}$,
\begin{equation*}
\begin{split}
\coord{x_{\varepsilon},-q_0} & \in \left( [q_0,q_0+\varepsilon )\times [-q_0,-q_0+\varepsilon )\right)  \\ & \qquad\cap \left( [x_{\varepsilon},x_{\varepsilon}+\tfrac{1}{m_0})\times [-x_{\varepsilon},-x_{\varepsilon}+\tfrac{1}{m_0})\right) \\ & \qquad \subseteq \left( [q_0,q_0+\varepsilon )\times [-q_0,-q_0+\varepsilon )\right) \cap U,
\end{split}
\end{equation*}
and similarly, if $x_{\varepsilon}\leq q_0$, we have that $[q_0,q_0+\varepsilon )\times [-q_0,-q_0+\varepsilon )$ intersects $U$ at $\coord{q_0,-x_{\varepsilon}}$.  Thus, every neighborhood of $\coord{q_0,0}\in \Q$ intersects $U$, and so $S\times S$ is in fact not even $T_4$.
\end{exm}

\subsection{Summary}

We summarize what we have covered so far in this section.

First of all, there are several levels of separation between two different objects in a space
\begin{important}
Distinct $\Leftarrow$\footnote{A two-point space with the indiscrete topology---see \cref{exm4.5.2}.} Topologically-distinguishable $\Leftarrow$\footnote{The Sierpinski Space---see \cref{exm4.5.3}.} Separated \\ $\Leftarrow$\footnote{A certain three-point space---see \cref{exm4.5.8}.} Separated by neighborhoods $\Leftarrow$\footnote{Another three-point space---see \cref{exm4.5.11}.} Separated by closed neighborhoods $\Leftarrow$\footnote{The Arens Square---see \cref{ArensSquare}.} Completely-separated $\Leftarrow$\footnote{The Uncountable Fort Space---see \cref{UncountableFortSpace}.} Perfectly-separated
\end{important}
The arrows indicated implication of course, and all these implications are strict, as indicated by the examples referenced in the footnotes.

There are three `families' of separation axioms of spaces:  (i)~separation of pairs of points, (ii)~separation of closed sets from points, and (iii)~and separation of disjoint closed sets.

In the first family:
\begin{enumerate}
\item Points being topologically-distinguishable is $T_0$ (\cref{T0}).
\item Points being separated is $T_1$ (\cref{T1}).
\item Points being separated by neighborhoods is $T_2$ (\cref{T2}).
\item Points being separated by closed neighborhoods is $T_{2\frac{1}{2}}$ (\cref{T212}).
\item Points being completely-separated is completely-$T_2$ (\cref{CompletelyT2}).
\item Points being perfectly-separated is perfectly-$T_2$ (\cref{PerfectlyT2}).
\end{enumerate}
In general, appending ``$\frac{1}{2}$'', ``completely'', or ``perfectly'' to a separation axiom in this way changes whatever the separation axiom was now to ``separated by closed neighborhoods'', ``completely-separated'', and ``perfectly-separated'' respectively.

Closed sets and points, as well as closed sets and closed sets, are automatically separated, and so there are no separation axioms analogous to $T_0$ and $T_1$ for families (ii)~and (iii).

For the other two ``families'', in order for them to be directly comparable with the first, we require that points be closed (that is, we explicitly require spaces in the second two ``families'' to be $T_1$---see \cref{prp4.5.32}).  Without this extra assumption, the separation axioms are called ``regular'' and ``normal'' respectively.

Thus, for the second family:\footnote{Remember that we require the spaces to be $T_1$ in addition to these properties.}
\begin{enumerate}
\item Closed sets and points being separated by neighborhoods is $T_3$ (\cref{T3}).
\item Closed sets and points being separated by closed neighborhoods is $T_{3\frac{1}{2}}$ (\cref{T312}).
\item Closed sets and points being completely-separated is completely-$T_3$ (\cref{CompletelyT3}).
\item Closed sets and points being perfectly-separated is perfectly-$T_3$ (\cref{PerfectlyT3}).
\end{enumerate}

And similarly for the third family:
\footnote{Remember that we require the spaces to be $T_1$ in addition to these properties.}
\begin{enumerate}
\item Closed sets being separated by neighborhoods is $T_4$ (\cref{T3}).
\item Closed sets being separated by closed neighborhoods is $T_{4\frac{1}{2}}$ (\cref{T412}).
\item Closed sets being completely-separated is completely-$T_4$ (\cref{CompletelyT4}).
\item Closed set being perfectly-separated is perfectly-$T_4$ (\cref{PerfectlyT4}).
\end{enumerate}

We now illustrate how all these axioms are related to each other.  All implications are strict (unless otherwise indicated by a $\Leftrightarrow$), in which case the offending counter-example is given in the indicated footnote.  Perhaps the only real surprise is the equivalence of $T_{4\frac{1}{2}}$ and completely-$T_4$:  \namerefpcref{UrysohnsLemma}.\footnote{Though perhaps it's worth nothing that $T_{k\frac{1}{2}}$ is equivalent to completely-$T_k$ for $k=3,4$ but \emph{not} $k=2$.}
{\small
\begin{savenotes}
\begin{equation}\label{4.6.105}
\begin{tikzcd}[column sep=tiny]
T_0 & & & & \\
T_1 \ar[u,Rightarrow,"\footnote{The Sierpinsi Space---see \cref{exm4.5.3}.}"] & & & & \\
T_2 \ar[u,Rightarrow,"\footnote{$\R$ with the cocountable topology---see \cref{exm4.2.8x}.}"] & & & & \\
T_{2\tfrac{1}{2}} \ar[u,Rightarrow,"\footnote{The Simplified Arens Square---see \cref{SimplifiedArensSquare}.}"] & T_3 \ar[l,Rightarrow,"\footnote{$\R$ with the cocountable extension topology---see \cref{CocountableExtensionTopology}.}"] & T_{3\tfrac{1}{2}} \ar[l,Leftrightarrow,"\footnote{See \cref{prp4.5.70}.}"] & T_4 \ar[l,Rightarrow,"\footnote{Niemytzki's Tangent Disk Topology---see \cref{NiemytzkisTangentDiskTopology}.}"] & T_{4\tfrac{1}{2}} \ar[l,Leftrightarrow,"\footnote{See \namerefpcref{UrysohnsLemma}.}"] \\
\text{Completely-}T_2 \ar[u,Rightarrow,"\footnote{The Arens Square---see \cref{ArensSquare}.}"] & & \text{Completely-}T_3 \ar[ll,Rightarrow,"\footnote{$\R$ with the cocountable extension topology---see \cref{CocountableExtensionTopology}.}"] \ar[u,Rightarrow,"\footnote{The Thomas Tent Space---see \cref{ThomasTentSpace}.}"] & & \text{Completely-}T_4 \ar[ll,Rightarrow,"\footnote{Niemytzki's Tangent Disk Topology---see \cref{NiemytzkisTangentDiskTopology}.}"] \ar[u,Leftrightarrow,"\footnote{Urysohn's Lemma, \cref{UrysohnsLemma}.}"] \\
\text{Perfectly-}T_2 \ar[u,Rightarrow,"\footnote{The Uncountable Fort Space--see \cref{UncountableFortSpace}.}"] & & \text{Perfectly-}T_3 \ar[ll,Rightarrow,"\footnote{$\R$ wit the cocountable extension topology---see \cref{CocountableExtensionTopology}.}"] \ar[u,Rightarrow,"\footnote{The Uncountable Fort Space---see \cref{UncountableFortSpace}.}"] & & \text{Perfectly-}T_4 \ar[ll,Rightarrow,"\footnote{Niemytzki's Tangent Disk Topology---see \cref{NiemytzkisTangentDiskTopology}.}"] \ar[u,Rightarrow,"\footnote{The Uncountable Fort Space---see \cref{UncountableFortSpace}.}"]
\end{tikzcd}.
\end{equation}
\end{savenotes}
}

Finally, we review which of these separation properties are preserved under subspaces and products.\footnote{Recall that (\cref{exm3.6.36}) none of them should be preserved under quotienting and all of them should be preserved under disjoint union.}

Every separation axiom we have seen is preserved under taking subspaces (\cref{exr3.6.36,exr3.6.45,exr4.6.37,exr3.6.67,exr3.6.71,exr3.6.76,exr3.6.99,exr3.6.106,exr3.6.108,prp3.6.130}) except for $T_4$ (which is of course equivalent to $T_{4\frac{1}{2}}$ and completely-$T_4$ by \namerefpcref{UrysohnsLemma} (\cref{exm4.4.69}).

The following separation axioms are preserved under products, and the products allowed are arbitrary unless otherwise stated.
\begin{enumerate}
\item $T_0$ (\cref{exr3.6.37})
\item $T_1$ (\cref{exr3.6.47})
\item $T_2$ (\cref{exr4.6.38}).
\item $T_{2\frac{1}{2}}$ (\cref{exr3.6.68})
\item Completely-$T_2$ (\cref{exr3.6.72})
\item Perfectly-$T_2$ (countably-infinite product works (\cref{prp3.6.73}), arbitrary products don't (\cref{exm3.6.73}))
\item $T_3\Leftrightarrow T_{3\frac{1}{2}}$ (\cref{exr3.6.100})
\item Completely-$T_3$ (\cref{exr3.6.107})
\end{enumerate}
On the other hand, products of perfectly-$T_3$ spaces, $T_4\Leftrightarrow T_{4\frac{1}{2}}\Leftrightarrow \text{completely-}T_4$, and perfectly-$T_4$ are in general never preserved under products, not even finite ones (\cref{DoubleArrowSpace,SorgenfreyPlane}).  Thus, the `rule' is that a separation axiom is preserved under products iff it is not a ``perfectly'' axiom, the sole exception being that $T_4$ (and it's equivalents) is not.\footnote{This should be easy to remember because $T_4$ is also the only separation axiom not preserved under subspaces.}\footnote{And in addition to this, perfectly-$T_2$ spaces happen to be preserved under countably-infinite products, but still not arbitrary ones.}

\section{Local properties}

For most topological properties, there is a ``local'' version.
\begin{mdf}{Locally XYZ}{LocallyXYZ}
A topological space is \term{locally XYZ}\index{Locally XYZ} iff each point has a neighborhood base consisting of sets that are XYZ.
\end{mdf}
The following result is probably the most useful for checking whether or not a space is actually locally XYZ.
\begin{mpr}{}{prp3.7.2}
Let $X$ be a topological space.  Then, $X$ is locally XYZ iff for every $x\in X$ and open set $U\subseteq X$ containing $x$, there is a neighborhood $N\subseteq U$ of $x$ that is XYZ.
\begin{proof}
$(\Rightarrow )$ Suppose that $X$ is locally XYZ.  Let $x\in X$ and let $U\subseteq X$ be an open set containing $X$.  By hypothesis, $x$ has a neighborhood base consisting of sets that are XYZ, and so by the definition of neighborhood base (\cref{NeighborhoodBase}), there is some neighborhood $N\subseteq U$ of $x$ that is XYZ.

\blankline
\noindent
$(\Leftarrow )$ Suppose that for every $x\in X$ and open set $U\subseteq X$ containing $x$, there is a neighborhood $N\subseteq U$ of $x$ that is XYZ.  Define
\begin{equation*}
\topology{B}_x\ceqq \left\{ N\subseteq X:N\text{ a neighborhood of }x\text{ that is XYZ.}\right\} .
\end{equation*}
We claim that this is a neighborhood base for the topology.  To show this, we must show that $U\subseteq X$ is open iff for every $x\in U$ there is some element $B\in \topology{B}_x$ with $x\in B\subseteq U$.

So, suppose that $U\subseteq X$ is open.  By hypothesis, there is a neighborhood $N\subseteq U$ of $x$ that is XYZ, that is, there is some $N\in \topology{B}_x$ with $x\in N\subseteq U$.  Conversely, suppose that for every $x\in U$ there is some element $B\in \topology{B}_x$ with $x\in B\subseteq U$.  Then, in particular, every point in $U$ has a neighborhood contained in $U$, and so $U$ is open.
\end{proof}
\end{mpr}

Of particular importance are the notions of local connectedness, local quasicompactness, and locally (completely/perfectly)-$T_k$.
\begin{exr}{}{}
Show that a space is $T_1$ iff it is locally $T_1$.
\end{exr}
\begin{exr}{}{}
Show that if a space is $T_2$ then it is locally $T_2$.  Find a counter-example to show that converse is false.
\end{exr}
\begin{exr}{}{}
Show that if a space is locally quasicompact and $T_2$, then it is locally compact.  Find a counter-example to show the converse is false.
\end{exr}
The following is the result related to local quasicompactness that will be used in the proof of the \nameref{HaarHowesTheorem}.
\begin{prp}{}{prp5.2.4}
Let $X$ be a locally compact space, let $K\subseteq X$ be quasicompact, and let $U\subseteq X$ contain $K$.  Then, there is an open set $V\subseteq X$ with compact closure such that
\begin{equation}
K\subseteq V\subseteq \Cls (V)\subseteq U.
\end{equation}
\begin{proof}
For each $x\in X$, let $B_x\subseteq X$ be an open neighborhood of $X$ with compact closure, and for $x\in U$ choose $B_x$ sufficiently small so that $B_x\subseteq U$.  Define $K_x\coloneqq K\cap B_x$ and $U_x\coloneqq U\cap B_x$.  As closed subsets of compact sets are compact, $K_x$ and $B_x\setminus U_x$ are compact in $B_x$, and, as compact spaces are $T_4$, we can find disjoint open (in $B_x$) neighborhoods $V_x\supseteq K_x$ and $W_x\supseteq B_x\setminus U_x$.  Write $V_x=V_x'\cap B_x$ and $W_x=W_x'\cap B_x$ for $V_x',W_x'\subseteq X$ open.  Note that
\begin{equation}
V_x\subseteq W_x^{\comp}\subseteq B_x^{\comp}\cup U_x
\end{equation}
and hence
\begin{equation}
\begin{split}
V_x & =V_x\cap \Cls (B_x)\subseteq W_x^{\comp}\cap \Cls (B_x)\subseteq \\
& U_x\cap \Cls (B_x)=U_x,
\end{split}
\end{equation}
so that $\Cls (U_x)\subseteq W_x^{\comp}\cap \Cls (B_x)\subseteq U_x$.

By quasicompactness, there are $x_1,\ldots ,x_m\in X$ such that $K\subseteq V_{x_1}\cup \cdots \cup V_{x_m}$.  Furthermore,
\begin{equation}
\begin{split}
\MoveEqLeft
\Cls (V_{x_1})\cup \cdots \cup V_{x_m})=\Cls (V_{x_1}\cup \cdots \cup \Cls (V_{x_m}) \\
& \subseteq (\Cls (B_{x_1})\setminus W_{x_1})\cup \cdots \cup (\Cls (B_{x_m})\setminus W_{x_m}) \\
& \subseteq U_{x_1}\cup \cdots \cup U_{x_m}\subseteq U.
\end{split}
\end{equation}
Thus, $V_{x_1}\cup \cdots \cup V_{x_m}$ works.
\end{proof}
\end{prp}

\subsubsection{Function spaces topologies}

Given topological spaces $X$ and $Y$, we haven't yet discussed much possible topologies one may put on $\Mor _{\Top}(X,Y)$.  An important one you are almost certain to encounter in your mathematical career is the \emph{quasicompact-open topology}\index{Quasicompact-open topology}\index{Compact-open topology}, which has the following natural generalization.
\begin{mdf}{XYZ-open topology}{XYZOpenTopology}
	Let $X$ and $Y$ be topological spaces.  Then, the \term{XYZ-open topology}\index{XYZ-open topology} on $\Mor _{\Top}(X,Y)$ is the topology generated by the collection
	\begin{equation}
		\begin{multlined}
		\left\{ \left\{ f\in \Mor _{\Top}(X,Y):f(N)\subseteq V\right\} :\right. \\ \left. N\subseteq X\text{ is XYZ and }V\subseteq Y\text{ is open.}\right\} .
		\end{multlined}
	\end{equation}
\end{mdf}
\begin{exr}{}{}
	Let $X$ and $Y$ be topological spaces, and let $\collection{T}$ be a generation collection for the topology of $Y$.  Show that
	\begin{equation}
		\begin{multlined}
			\left\{ \left\{ f\in \Mor _{\Top}(X,Y):f(N)\subseteq V\right\} :\right. \\ \left. N\subseteq X\text{ is XYZ and }V\in \collection{T}\right\}
		\end{multlined}
	\end{equation}
	generates the XYZ-open topology on $\Mor _{\Top}(X,Y)$.
	\begin{rmk}
		In other words, in the definition \cref{XYZOpenTopology}, we may as well just look at elements in a generating collection, instead of all open subsets of $Y$.
	\end{rmk}
\end{exr}
The relevance to local properties is given by the following result, which itself seems to be \cite{Arens} the original motivation for the introduction of the quasicompact-open topology.
\begin{mtm}{}{mtm3.7.14}
	Let $X$.  Then, $X$ is locally-XYZ,
	\begin{equation}
		\Mor _{\Top}(X,Y)\times X\ni \coord{f,x}\mapsto f(x)\in Y
	\end{equation}
	is continuous for all topologies spaces $Y$, where $\Mor _{\Top}(X,Y)$ is equipped with the XYZ topology.
	\begin{proof}
		Suppose that $X$ is locally-XYZ.  Let $Y$ be a topological space, let $\coord{f,x}\in \Mor _{\Top}(X,Y)\times X$, and let $V\subseteq Y$ be an open neighborhood of $f(x)\in Y$.  Then, $f^{-1}(V)$ is an open neighborhood of $x\in X$, and hence, as $X$ is locally-XYZ, there is a neighborhood $N\subseteq f^{-1}(V)$ of $x\in X$ that is XYZ.  Write
		\begin{equation}
			B_{N,V}\ceqq \left\{ \breve{f}\in \Mor _{\Top}(X,Y):\breve{f}(N)\subseteq V\right\} .
		\end{equation}
		As $N\subseteq f^{-1}(V)$, $f\in B_{N,V}$, and so $B_{N,V}\times N\subseteq \Mor _{\Top}(X,Y)\times X$ is a neighborhood of $\coord{f,x}\in \Mor _{\Top}(X,Y)\times X$.  Furthermore, if $\coord{\breve{f},\breve{x}}\in B_{N,V}\times N$, then $\breve{f}(\breve{x})\in V$, and so $B_{N,V}\times N$ is indeed a neighborhood of $\coord{f,x}$ that is mapped into $V$ via the evaluation map $\Mor _{\Top}(X,Y)\times X\rightarrow Y$, as desired.
	\end{proof}
\end{mtm}
\begin{exr}{}{}
	Let $X$ be a topological space.  If
	\begin{equation}
		\Mor _{\Top}(X,Y)\times X\ni \coord{f,x}\mapsto f(x)\in Y
	\end{equation}
	is continuous for all topologies spaces $Y$, where $\Mor _{\Top}(X,Y)$ is equipped with the XYZ topology, is it necessarily true that $X$ is locally-XYZ?
	\begin{rmk}
		In other words, is the converse of the previous result \cref{mtm3.7.14} true?
	\end{rmk}
\end{exr}
For an application of the quasicompact-open topology in particular, see \cref{prp4.2.99}.

\section[The IVT and EVT]{The Intermediate and Extreme Value Theorems}

\subsection[Connectedness and the IVT]{Connectedness and the Intermediate Value Theorem}

You'll recall from calculus that the classical statement of the Intermediate Value Theorem is
\begin{textequation}
Let $f\colon [a,b]\rightarrow \R$ be continuous.  Then, for all $y$ between $f(a)$ and $f(b)$ (inclusive), there exists $x\in [a,b]$ such that $f(x)=y$.
\end{textequation}
We will see that the proper way to interpret this statement is that the image of $f$ is connected, so that the image must contain everything in-between $f(a)$ and $f(b)$ as well.  Of course, in order to make this precise, we have to first define what it means to be connected.
\begin{dfn}{Connected and disconnected}{Connected}
Let $X$ be a topological space.  Then, $X$ is \term{disconnected}\index{Disconnected} iff there exist disjoint nonempty closed sets $C,D\subset X$ such that $X=C\cup D$.  $X$ is \term{connected}\index{Connected} iff it is not disconnected.  A subset $S$ of $X$ is connected iff it is connected in its subspace topology.
\begin{rmk}
The intuition for the definition of disconnected of course is that we `break up' the space into two separate pieces which have no overlap.
\end{rmk}
\begin{rmk}
Another way to say this is that $X$ is disconnected iff it has a partition into two closed sets.
\end{rmk}
\begin{rmk}
For $S\subseteq X$, note that it is \emph{not} the case that $S$ is disconnected iff $S=C\cup D$ for nonempty disjoint closed subsets $C,D\subseteq X$.  The reason for the difference is that \emph{subsets of $S$ which are closed in $S$ need not be closed in $X$}.  For example, $(0,1)\cup (2,3)$ is a disconnected space, but it cannot be written as the union of two nonempty disjoint closed subsets of $\R$.
\end{rmk}
\begin{rmk}
This is usually phrased in terms of open sets instead of closed sets.  It turns out the definitions are equivalent---see the following result.  The reason we state the definition in terms of closed sets is because there is a related concept called \emph{hyperconnected}\index{Hyperconnected} in which they are not equivalent and the `correct' notion is the one stated in terms of closed sets.
\end{rmk}
\end{dfn}
\begin{prp}{}{}
Let $X$ be a topological space.  Then, $X$ is disconnected iff there exist disjoint nonempty open sets $U,V\subset X$ such that $X=U\cup V$.
\begin{proof}
$(\Rightarrow )$ Suppose that $X$ is disconnected.  Then, by definition, there are disjoint nonempty closed sets $C,D\subset X$ such that $X=C\cup D$.  As $C$ and $D$ are disjoint, in fact $D=C^{\comp}$, and so $D$ is itself open.  Similarly $C$ is open.  This is the desired result.

\blankline
\noindent
$(\Leftarrow )$ Suppose that there exist disjoint nonempty open sets $U,V\subset X$ such that $X=U\cup V$.  Similarly as before, $V=U^{\comp}$, and hence is closed, etc. etc..
\end{proof}
\end{prp}
\begin{prp}{}{prp3.7.4}
Let $X$ be a topological space and let $\collection{U}$ be a collection of connected subsets of $X$ with nonempty intersection.  Then, $\bigcup _{U\in \collection{U}}U$ is connected.
\begin{proof}
To simplify notation, let us write $X'\coloneqq \bigcup _{U\in \collection{U}}U$.  We proceed by contradiction:  suppose that $\bigcup _{U\in \collection{U}}U$ is disconnected, so that we may write
\begin{equation}
X'=V\cup W
\end{equation}
for $V,W\subseteq X'$ open, nonempty, and disjoint.  Let $x_0\in \bigcap _{U\in \collection{U}}U$ and without loss of generality assume that $x_0\in V$.  For each $U\in \collection{U}$, let us write
\begin{equation}
U_V\coloneqq U\cap V\text{ and }U_W\coloneqq U\cap W,
\end{equation}
so that
\begin{equation}
U=U_V\cup U_W
\end{equation}
for all $U\in \collection{U}$.  As $U$ is connected, it follows that, for each $U$, either $U_V$ or $U_W$ is empty.  However, we know that $x_0\in U_V$, and so in fact, we must have that $U_W=\emptyset$ for all $U\in \collection{U}$, which in turn implies that $W=\emptyset$:  a contradiction.
\end{proof}
\end{prp}
\begin{dfn}{Connected component}{}
Let $X$ be a topological space and let $x_1,x_2$.  Then, $x_1$ and $x_2$ are \term{connected}\index{Connected (relation)} (to each other) iff there exists a connected set $U\subseteq X$ with $x_1,x_2\in U$.
\begin{prp}[breakable=false]{}{}
The relation of being connected to is an equivalence relation on $X$.
\begin{proof}
$x$ is connected to itself because $\{ x\}$ is connected.  The relation is symmetric because the definition of the relation is symmetric.  If $x_1$ is connected to $x_2$ and $x_2$ is connected to $x_3$, then there is some connected set $U$ which contains $x_1$ and $x_2$, and there is some connected set $V$ which contains $x_2$ and $x_3$.  As $U$ and $V$ both contain $x_2$, it follows from the previous proposition that $U\cup V$ is connected, and hence $x_1$ is connected to $x_3$.
\end{proof}
\end{prp}
A \term{connected component}\index{Connected component} of $X$ is an equivalence class of some point with respect to the relation of being connected to.
\end{dfn}
We have a pretty explicit description of connected components.
\begin{prp}{}{prp3.7.9}
Let $X$ be a topological space and let $x\in X$.  Then, the connected component of $X$ is
\begin{equation}
\bigcup _{\substack{U\subseteq X \\ U\text{ connected} \\ x\in U}}U.
\end{equation}
\begin{rmk}
In particular, by \cref{prp3.7.4}, every connected component of $X$ is connected.
\end{rmk}
\begin{proof}
Define
\begin{equation}\label{eqn3.7.10}
U_x\ceqq \bigcup _{\substack{U\subseteq X \\ U\text{ connected} \\ x\in U}}U.
\end{equation}
By \cref{prp3.7.4}, this is a connected set that contains $x$.  It follows that every element of $U_x$ is connected to $x$.  To show that it is the connected component of $x$, that is, the equivalence class of $x$ with respect to the relation of being ``connected to'', we must show that every other point that is connected to $x$ is contained in $U_x$.

So, let $y\in X$ be connected to $x$.  Then, there is a connected set $V$ with $x,y\in V$.  Then, $V$ appears in the union \eqref{eqn3.7.10}, and so $V\subseteq U_x$, and in particular, $y\in U_x$, as desired.
\end{proof}
\end{prp}
One particularly important property of connected components, at least in locally connected spaces, is that they are both open \emph{and} closed.  This is not a bad way of telling if a space is connected (a space is connected iff they only clopen sets are $\emptyset$ and itself).
\begin{prp}{}{prp3.7.12}
Let $X$ be a topological space.  Then, $X$ is locally connected iff every connected component of every open subset is open.
\begin{rmk}
In particular, as $X$ is the disjoint union of its connected components (by \cref{crlA.1.13}), if $[x]_{\sim}\subseteq X$ is a connected component of $X$, then $[x]_{\sim}^{\comp}$ is the union of all the other connected components, and hence is open.  Thus, $[x]_{\sim}$ is closed.  That is, in a locally connected space, all the connected components are clopen.
\end{rmk}
\begin{proof}
$(\Rightarrow )$ Suppose that $X$ is locally connected.  Let $U\subseteq X$ be open and for $x,y\in U$, let us write $x\sim y$ iff $x$ is connected to $y$.  Let $x\in [x_0]_{\sim}$.  By definition, this means that there is some connected set $V\subseteq U$ such that $x,x_0\in V$.  As the space is locally connected, $x$ has a neighborhood base consisting of connected sets.  So, let $N$ be a connected neighborhood of $x$.  Then, $V$ and $N$ intersect, namely at $x$, and so by \cref{prp3.7.4}, $V\cup N$ is again connected.  As $V\cup N$ certainly contains $x_0$, from the previous result, we have that $N\cup V\subseteq [x]_{\sim}$, and hence that $x\in N\subseteq [x]_{\sim}$.  Thus, $[x]_{\sim}$ is open, as desired.

\blankline
\noindent
$(\Leftarrow )$ Suppose that every connected component of every open subset is open.  To show that $X$ is locally connected, we apply \cref{prp3.7.2}, that is, we must show that for every $x\in X$ and $U\subseteq X$ be open containing $x$ there is a connected neighborhood $N\subseteq U$ of $x$.  So, let $x\in X$ and let $U\subseteq X$ be open and containing $x$.  By hypothesis, the connected component of $x$ in $U$ is open, and hence constitutes a connected neighborhood of $x$ contained in $U$, as desired.
\end{proof}
\end{prp}
\begin{exr}{}{}
Is it true that a space is locally connected iff every connected component is clopen?
\end{exr}
\begin{dfn}{Totally-disconnected}{}
Let $X$ be a topological space.  Then, $X$ is \term{totally-disconnected}\index{Totally-disconnected} iff every connected component of $X$ is a point.
\end{dfn}
\begin{prp}{}{}
Let $X$ be a discrete topological space.  Then, $X$ is totally-disconnected.
\begin{proof}
Let $U\subseteq X$ have at least two distinct points $x_1$ and $x_2$.  Then,
\begin{equation}
U=\{ x_1\} \cup (U\setminus \{ x_1\} ),
\end{equation}
and as every subset in a discrete space is open, it follows that $U$ is disconnected.
\end{proof}
\end{prp}
\begin{exm}{$\N$, $\Z$, and $\Q$ are totally-discon\-nected}{}
That $\N$ and $\Z$ are totally-disconnected follows from the fact that they are discrete.

$\Q$ is also totally-disconnected,\footnote{In particular, there are totally-disconnected spaces which are not discrete.} but this is more difficult to see.  Let $U\subseteq \Q$ have at least two distinct points $q_1$ and $q_2$.  Without loss of generality, suppose that $q_1<q_2$.  Then, by `density' of $\Q ^{\comp}$ in $\R$ (\cref{thm3.3.76}), there is some $x\in \Q ^{\comp}$ with $q_1<x<q_2$.  Define
\begin{equation}
V\coloneqq (-\infty ,x)\cap U \text{ and }W\coloneqq (x,\infty )\cap U .
\end{equation}
Both $V$ and $W$ are open by the definition of the subspace topology (\cref{SubspaceTopology}) of $U$, and both are nonempty because $q_1<x$ and $q_2>x$.  Thus, as $U=V\cup W$, $U$ is disconnected, and hence $\Q$ is totally-disconnected.
\end{exm}

We mentioned at the beginning of this section that the `proper' way to interpret the \nameref{IntermediateValueTheorem} is the statement that the image of connected sets are connected.  There is one other thing we first need to check though---we need to check that intervals in $\R$ are in fact connected.
\begin{thm}{}{thm4.5.14}
Let $I\subseteq \R$.  Then, $I$ is connected iff it is an interval.
\begin{rmk}
You might say that this result is to connectedness as the \nameref{HeineBorelTheorem} is to quasicompactness---this result characterizes connectedness in $\R$ and the \nameref{HeineBorelTheorem} characterizes quasicompactness in $\R$.  A key difference, however, is their generalization to $\R ^d$---there is no such thing as an interval in $\R ^d$,\footnote{I suppose you could equip $\R ^d$ with the product order, but this is not totally-ordered for $d\geq 2$, and so you cannot define the order-topology.} whereas the \nameref{HeineBorelTheorem} holds verbatim in $\R ^d$.
\end{rmk}
\begin{rmk}
In particular, $\R$ itself is connected, in contrast with $\N$, $\Z$, and $\Q$.
\end{rmk}
\begin{proof}
$(\Rightarrow )$ Suppose that $I$ is connected.  Let $a,b\in I$ with $a\leq b$ and let $x\in \R$ with $a\leq x\leq b$.  We must show that $x\in I$.  If either $x=a$ or $x=b$, we are done, so we may as well suppose that $a<x<b$.  We proceed by contradiction:  suppose that $x\notin I$.  Then,
\begin{equation}
I=(I\cap (-\infty ,x))\cup (I\cap (x,\infty )),
\end{equation}
and so as $I$ is connected, we must have that either $I\cap (-\infty ,x)$ is empty or $I\cap (x,\infty )$ is empty.  But $a$ is in the former and $b$ is in the latter:  a contradiction.

\blankline
\noindent
$(\Leftarrow )$ Suppose that $I$ is an interval.  As $I$ is an interval, by \cref{prp3.3.70}, we have that $I=[(a,b)]$ for $a=\inf (I)$ and $b=\sup (I)$.\footnote{Recall that this notation just means that the end-points can be either open or closed---see the remark in \cref{prp3.3.70}.}  We wish to show that $I$ is connected.  If $a=b$, then either $I=\emptyset$ or $I=\{ a\}$, in which case $I$ is trivially connected.  Therefore, we may assume without loss of generality that $a<b$.

We proceed by contradiction:  suppose that $I$ is disconnected.  The, we have that $I=U\cup V$ for $U,V\subseteq I$ open in $I$ disjoint and nonempty.  As $U$ and $V$ are open in $I$ and cover $I$, we must have that at least one of them contains an open neighborhood of $a$, so without loss of generality, suppose that $[(a,x_0)\subseteq U$ for some $x_0\in \R$ with $a<x_0\leq b$.  Now define
\begin{equation}
S\coloneqq \left\{ x\in I :[(a,x)\subseteq U\right\} .
\end{equation}
We just showed that this set is nonempty.  It is also bounded above by $b$ as $b$ is in particular an upper-bound of $I$.  Therefore, it has a supremum.  We wish to show that $\sup (S)=b$.

Note that $a<\sup (S)\leq b$.  If $\sup (S)=b$, we are done, otherwise $a<\sup (S)<b$, and so $sup (S)\in I=[(a,b)]$.

We show that in fact $\sup (S)\in U$.  We proceed by contradiction:  suppose that $\sup (S)\in V$ (here is where we use the fact that $\sup (S)\in I$).  As $V$ is open, there is a neighborhood of $\sup (S)$ completely contained in $V$.  On the other hand, by \cref{prp1.4.11}, this neighborhood has to contain some element of $S$, which in turn would imply that it would have to contain some element of $U$.  But then $U$ intersects $V$:  a contradiction.  Therefore, $\sup (S)\in U$.

Because $U$ is open, there is some $\varepsilon >0$ such that $(\sup (S)-\varepsilon ,\sup (S)+\varepsilon )\subseteq U$.  By \cref{prp1.4.11}, there must be some $x\in S$ with $\sup (S)-\varepsilon <x\leq \sup (S)$, so that $[(a,x)\subseteq U$.  But then,
\begin{equation}
[(a,x))\cup (\sup (S)-\varepsilon ,\sup (S)+\varepsilon )=[(a,\sup (S)+\varepsilon )\subseteq U,
\end{equation}
and so, in particular, there is some $x'>\sup (S)$ such that $[(a,x')\subseteq U$,  a contradiction as $S$ is the supremum of the set of all such elements.  Therefore, we must have that $\sup (S)=b$.

Now that we have finally succeeded in showing that $\sup (S)=b$, we finish the proof by coming to a contradiction of the assumption of disconnectedness.  As $\sup (S)=b$, this means that for every $x\in I$ with $a<x<b$, we have that $[(a,x)\subseteq U$, and hence
\begin{equation}
\bigcup _{a<x<b}[(a,x)=[(a,b)\subseteq U.
\end{equation}
Thus, either $V=\{ b\}$ or $V=\emptyset$:  a contradiction of being open in $I$ or being nonempty respectively.
\end{proof}
\end{thm}
As an application, we can use this to `classify' all open sets in the real numbers.
\begin{thm}{}{prp3.7.20}
Let $U\subseteq \R$ be open.  Then, $U$ is a countable disjoint union of open intervals.
\begin{rmk}
Note that this is very special to $\R$---I don't see how one could hope to generalize this to $\R ^d$ for $d\geq 2$.
\end{rmk}
\begin{proof}
For $x,y\in \R$, let us write $x\sim y$ iff $x$ is connected to $y$.  Then,
\begin{equation}\label{eqn3.7.21}
U=\bigcup _{x\in U}[x]_{\sim},
\end{equation}
that is, $U$ is the disjoint union of its connected components.\footnote{Because every equivalence relation determines a partition---see \cref{crlA.1.13}.}  By the previous result, each $[x]_{\sim}$ is an interval.  By \cref{prp3.7.12}, each $[x]_{\sim}$ is open, and hence an open interval.  Thus, $U$ is the disjoint union of open intervals, and so all that remains to be shown is that the union is countable.
\begin{exr}{}{}
Show that the union in \eqref{eqn3.7.21} is a countable one.
\end{exr}
\end{proof}
\end{thm}

And now we finally get to the statement of the `true' Intermediate Value Theorem.
\begin{thm}{Intermediate Value Theorem}{IntermediateValueTheorem}\index{Intermediate Value Theorem}
Let $f\colon X\rightarrow Y$ be a continuous function and let $S\subseteq X$ be connected.  Then, $f(S)$ is connected.
\begin{rmk}
	Sometimes this is abbreviated \term{IVT}\index{IVT}.
\end{rmk}
\begin{rmk}
In other words, the continuous image of a connected set is connected.
\end{rmk}
\begin{rmk}
In general, if a function $f\colon X\rightarrow Y$ has the property that the image of a connected set is connected, then we say that $f$ has the \term{intermediate value property}\index{Intermediate Value Property} (also referred to as \emph{Darboux continuous}---see \cref{DarbouxsTheorem}).  Thus, the Intermediate Value Theorem says that ``Continuous functions have the intermediate value property.''.  Continuous functions are not the only such functions with this property, however.  \namerefpcref{DarbouxsTheorem} says that any function that is the derivative of another function (from $\R$ to $\R$) has the intermediate value property.  See also \cref{exr3.8.32}.
\end{rmk}
\begin{proof}
We proceed by contradiction:  suppose that $f(S)$ is disconnected.  Then, $f(S)=U\cup V$ for $U,V\subseteq f(S)$ open in $f(S)$ disjoint and nonempty.  By continuity, we have that\footnote{Also recall that $f^{-1}(f(S))\supseteq S$---see \cref{exrA.1.47}.\cref{enmA.1.47.ii}.}
\begin{equation}
S\supseteq \footnote{This follows from the fact that we are applying the preimage of \emph{the restriction} of $f$ to $S$.}\restr{f}{S}^{-1}(U)\cup \restr{f}{S}^{-1}(V)\supseteq S,
\end{equation}
and so
\begin{equation}
S=\restr{f}{S}^{-1}(U)\cup \restr{f}{S}^{-1}(V).
\end{equation}
As $U$ and $V$ are open in $S$ and $f$ is continuous, $\restr{f}{S}^{-1}(U)$ and $\restr{f}{S}^{-1}(V)$ are open in $S$.  They also must be disjoint, for a point which lied in their intersection would be mapped into $U\cap V=\emptyset$ via $f$.  Therefore, because $S$ is connected, we have that either $\restr{f}{S}^{-1}(U)$ or $\restr{f}{S}^{-1}(V)$ is empty, which implies respectively that either $U$ or $V$ is empty:  a contradiction.
\end{proof}
\end{thm}
As a corollary of this (and the fact that a subnet of $\R$ is connected iff it is an interval---see \cref{thm4.5.14}), we have the classical statement of the Intermediate Value Theorem.
\begin{crl}{Classical Intermediate Value Theorem}{ClassicalIntermediateValueTheorem}\index{Classical Intermediate Value Theorem}
Let $f\colon [a,b]\rightarrow \R$ be continuous.  Then, $f([a,b])$ is an interval.  In particular, any element between $f(a)$ and $f(b)$ is in the image of $f$.
\begin{proof}
The ``in particular'' part follows from the definition of an interval (\cref{Interval}).

$[a,b]$ is connected by \cref{thm4.5.14}, and so, by the Intermediate Value Theorem, $f([a,b])$ is connected, and so by \cref{thm4.5.14} again, is an interval.
\end{proof}
\end{crl}
\begin{exr}{}{exr3.8.32}
Are there functions which are not continuous but which still have the intermediate value property?
\end{exr}

\subsection[Quasicompactness and the EVT]{Quasicompactness and the Extreme Value Theorem}

You'll recall from calculus that the classical statement of the Extreme Value Theorem is
\begin{textequation}
Let $f\colon [a,b]\rightarrow \R$ be continuous.  Then, there exists $x_1,x_2\in [a,b]$ such that $f(x_1)=\inf _{x\in [a,b]}\left\{ f(x)\right\}$ and $f(x_2)=\sup _{x\in [a,b]}\left\{ f(x)\right\}$.  In other words, continuous functions \emph{attain} their maximum and minimum on closed intervals.
\end{textequation}
This is actually a special case of (or follows easily from) a \emph{much} more general, elegant statement.
\begin{thm}{Extreme Value Theorem}{ExtremeValueTheorem}\index{Extreme Value Theorem}
Let $f\colon X\rightarrow Y$ be continuous and let $K\subseteq X$.  Then, if $K$ is quasicompact, then $f(K)$ is quasicompact.
\begin{rmk}
	Sometimes this is abbreviated \term{EVT}\index{EVT}.
\end{rmk}
\begin{rmk}
In other words, the continuous image of a quasicompact set is quasicompact.
\end{rmk}
\begin{rmk}
In general, if a function $f\colon X\rightarrow Y$ has the property that the image of a quasicompact set is quasicompact, then we say that $f$ has the \term{extreme value property}\index{Extreme Value Property}.  Thus, the Extreme Value Theorem says that ``Continuous functions have the extreme value property.''.
\end{rmk}
\begin{proof}
Suppose that $K$ is quasicompact.  Let $\topology{U}$ be an open cover of $f(K)$.  Then, $f^{-1}(\cover{U})\coloneqq \left\{ f^{-1}(U):U\in \cover{U}\right\}$ is an open cover of $K$, and therefore it has a finite subcover $\{ f^{-1}(U_1),\ldots f^{-1}(U_m)\}$.  In other words,
\begin{equation}
K\subseteq f^{-1}(U_1)\cup \cdots \cup f^{-1}(U_m),
\end{equation}
and hence
\begin{equation}
\begin{split}
f(K) & \subseteq f\left( f^{-1}(U_1)\cup \cdots \cup f^{-1}(U_m)\right) \\
& =\footnote{By \cref{exrA.1.30}.\cref{enmA.1.30.iii}.}f\left( f^{-1}(U_1)\right) \cup \cdots \cup f\left( f^{-1}(U_m)\right) \\
& \subseteq \footnote{By \cref{exrA.1.47}.\cref{enmA.1.47.i}.}U_1\cup \cdots \cup U_m,
\end{split}
\end{equation}
so that $\{ U_1,\ldots ,U_m\}$ is a finite subcover of $f(K)$, and hence $f(K)$ is quasicompact.
\end{proof}
\end{thm}
And now we can present the classical version of the theorem.
\begin{crl}{Classical Extreme Value Theorem\hfill}{ClassicalExtremeValueTheorem}\index{Classical Extreme Value Theorem}
Let $f\colon [a,b]\rightarrow \R$ be continuous.  Then, $f([a,b])$ is closed and bounded.  In particular, it attains a maximum and minimum on $[a,b]$.
\begin{proof}
The ``in particular'' is a result of \cref{exr3.4.27}, the statement that closed bounded sets contain their supremum and infimum.

By the \namerefpcref{HeineBorelTheorem}, $[a,b]$ is quasicompact.  Therefore, by the Extreme Value Theorem, $f([a,b])$ is quasicompact, and therefore, closed and bounded, again by the \nameref{HeineBorelTheorem}.
\end{proof}
\end{crl}

In fact, we may as well just combine the classical statements into one.
\begin{crl}{Classical Intermediate-Extreme \\ Value Theorem}{ClassicalIntermediateExtremeValueTheorem}\index{Classical Intermediate-Extreme Value Theorem}
Let $f\colon [a,b]\rightarrow \R$ be continuous.  Then, $f([a,b])$ is a closed, bounded, interval.
\end{crl}

\begin{exr}{}{}
Are there functions which are not continuous but which still have the extreme value property?
\end{exr}

Finally, we end (the subsubsection) with a handy little result that is a corollary of the \nameref{ExtremeValueTheorem}.  Hopefully you found an example in \cref{exr3.1.34} of a continuous bijective function that was not a homeomorphism.  In certain special cases, however, you can immediately make this deduction.
\begin{exr}{}{exr3.6.47}
Show that continuous injective function from a quasicompact space into a $T_2$ space is a homeomorphism onto its image.
\end{exr}

\subsubsection{Characterizations of quasicompactness}

What follows is a summary of all the equivalent characterizations of quasicompactness we are aware of.  We have seen several characterization so far, but we have one more to go.  IMHO, it is a bit more difficult to understand, and not as useful as the others, so we have procrastinated dealing with it until this subsection on quasicompactness.  While the real objective of this subsection was to do the \nameref{ExtremeValueTheorem}, it probably makes more sense to summarize these characterizations here than anywhere else.

First of all, we will need one last definition before getting to our final characterization.
\begin{dfn}{Ultranet}{UltraNet}
Let $X$ be a set and let $\lambda \mapsto x_{\lambda}$ be a net.  Then, $\lambda \mapsto x_{\lambda}$ is an \term{ultranet}\index{Ultranet} iff for every $S\subseteq X$, $\lambda \mapsto x_{\lambda}$ is eventually contained in $S$ or eventually contained in $S^{\comp}$.
\begin{rmk}
This is sometimes also called a \term{universal net}\index{Universal net}.  I prefer the term ``ultranet'' because (i) the word ``universal'' here isn't really being used in the sense in which it is usually meant and (ii) ``ultranet'' is meant to suggest a similarity with something called an \emph{ultra-filter}.  An \term{ultra-filter}\index{Ultra-filter} is just a fancy word for a proper maximal filter, and it turns out that\footnote{Disclaimer:  I didn't actually check this because it doesn't matter for us, so I don't guarantee that this statement is exactly true on those nose.} the derived filter of an ultranet is an ultra-filter, hence the terminology.
\end{rmk}
\end{dfn}
This allows us to present our final characterization (together with a couple of `lemmas' that can be of use in their own right).
\begin{prp}{}{prp3.8.42}
Let $X$ be a set, let $S\subseteq X$, and let $\lambda \mapsto x_{\lambda}\in X$ be an ultranet.  Then, if $\lambda \mapsto x_{\lambda}$ is frequently contained in $S$, then $\lambda \mapsto x_{\lambda}$ is eventually contained in $S$.
\begin{rmk}
Note that the converse is \emph{always} true (by the definitions \cref{EventuallyXYZInTop,FrequentlyXYZInTop}).
\end{rmk}
\begin{proof}
Suppose that $\lambda \mapsto x_{\lambda}$ is frequently contained in $S$.  If $\lambda \mapsto x_{\lambda}$ is eventually contained in $S$, we're done, so suppose that's not the case.  Then, as $\lambda \mapsto x_{\lambda}$ is an ultranet, it must be the case that $\lambda \mapsto x_{\lambda}$ is eventually contained in $S^{\comp}$.  But then it is \emph{not} the case that $\lambda \mapsto x_{\lambda}$ is frequently contained in $S$ (\cref{mpr3.2.5}):  a contradiction.
\end{proof}
\end{prp}
\begin{prp}{}{prp3.8.43}
Let $X$ be a topological space and let $\lambda \mapsto x_{\lambda}\in X$ be a net.  Then, there is a subnet $\mu \mapsto x_{\lambda _{\mu}}$ of $\lambda \mapsto x_{\lambda}$ that is a ultranet.
\begin{rmk}
In brief, every net has an ultra-subnet.
\end{rmk}
\begin{proof}\footnote{Proof adapted from \cite[pg.~90]{Howes}.}
\Step{Define $\collection{S}$}
Let $\widetilde{\filter{F}}$ be the collection of all filter bases $\filter{F}$ that contain $\filter{F}_{\lambda \mapsto x_{\lambda}}$ and have the property that $\lambda \mapsto x_{\lambda}$ is frequently contained in $F$ for all $F\in \filter{F}$.  Note that $\filter{F}_{\lambda \mapsto x_{\lambda}}\in \widetilde{\filter{F}}$ (this is the derived filter base of $\lambda \mapsto x_{\lambda}$ (\cref{DerivedFilterBase})).

\Step{Obtain a maximal element $\filter{F}_0$ of $\widetilde{\filter{F}}$}
Regard $\widetilde{\filter{F}}$ as a partially-ordered subset with respect to inclusion.  Let $\widetilde{\filter{W}}\subseteq \widetilde{\filter{W}}$ be a well-ordered subset, and define
\begin{equation}
\filter{F}\ceqq \bigcup _{\filter{W}\in \widetilde{\filter{W}}}\filter{W}.
\end{equation}
\begin{exr}[breakable=false]{}{}
Show that $\filter{F}$ is an upper-bound of $\widetilde{\filter{W}}$ in $\widetilde{\filter{F}}$.  Explicitly, show that $\filter{F}$ is (i) a filter base, (ii) contains $\filter{F}_{\lambda \mapsto x_{\lambda}}$, (iii) every element of $\filter{F}$ frequently contains $\lambda \mapsto x_{\lambda}$, and (iv) $\filter{F}$ is a superset of every element of $\widetilde{\filter{W}}$.
\end{exr}
By \namerefpcref{ZornsLemma}, $\widetilde{\filter{F}}$ has a maximal element.  Call such a maximal element $\filter{F}_0$.

\Step{Find a subnet $\mu \mapsto x_{\lambda _{\mu}}$ eventually contained in each element of $\filter{F}_0$}
Denote the index set of $\lambda \mapsto x_{\lambda}$ by $\Lambda$.  Order $\filter{F}_0$ by reverse inclusion and equip $\Lambda \times \filter{F}_0$ with the product order (\cref{ProductOrder}).  For $\lambda _0\in \Lambda$ and $F\in \filter{F}_0$, as $\lambda \mapsto x_{\lambda}$ is frequently contained in $F$, there is some $\lambda _{\lambda _0,F}\geq \lambda _0$ such that $x_{\lambda _{\lambda _0,F}}\in F$.  We claim that
\begin{equation}
\Lambda \times \filter{F}_0\ni \coord{\lambda _0,F}\mapsto x_{\lambda _{\lambda _0,F}}
\end{equation}
is a subnet of $\lambda \mapsto x_{\lambda}$ eventually contained in every element of $\filter{F}$.

First of all, suppose that $U$ eventually contained $\lambda \mapsto x_{\lambda}$.  As $\filter{F}_0$ contains $\filter{F}_{\lambda \mapsto x_{\lambda}}$, in fact $U\in \filter{F}_0$.  Also choose $\lambda _0\in \Lambda$ such that, whenever $\lambda \geq \lambda _0$, it follows that $x_{\lambda}\in U$.  Now, suppose that $\coord{\lambda ,F}\geq \coord{\lambda _0,F}$.  This means that $\lambda \geq \lambda _0$ and that $F\subseteq U$.  As $\lambda _{\lambda _0,F}\geq \lambda _0$, it follows that $x_{\lambda _{\lambda _0,F}}$, and hence as $F\subseteq U$, that $x_{\lambda _{\lambda _0,F}}\in U$.  Thus, $\coord{\lambda _0,F}\mapsto x_{\lambda _{\lambda _0,F}}$ is eventually contained in $U$, and hence this constitutes a subnet of $\lambda \mapsto x_{\lambda}$.

Now let $F\in \filter{F}_0$.  Fix any index $\lambda _0$.  Then, whenever $\coord{\lambda ,F'}\geq \coord{\lambda _0,F}$, it follows that $x_{\lambda _{\lambda ,F'}}\in F'\subseteq F$, and so indeed $\coord{\lambda ,F'}\mapsto x_{\lambda _{\lambda ,F'}}$ is eventually contained in $F$, as desired.


\Step{Show that $\mu \mapsto x_{\lambda _{\mu}}$ is an ultranet}
Let $S\subseteq X$.  We wish to show that $\mu \mapsto x_{\lambda _{\mu}}$ is eventually contained in $S$ or $S^{\comp}$.  If $S\in \filter{F}_0$, then by the previous part, $\mu \mapsto x_{\lambda _{\mu}}$ is eventually contained in $S$, and we are done, so suppose that $S\notin \filter{F}_0$.  This means that (\cref{mpr3.2.5}), $\lambda \mapsto x_{\lambda}$ is frequently in $S$.  Thus, there must be some $F\in \filter{F}_0$ such that $S\cap F=\emptyset$, otherwise $\filter{F}_0\cup \{ S\}$ would be strictly larger than $\filter{F}_0$ in $\widetilde{\filter{F}}$.  $S\cap F=\emptyset$ implies that $F\subseteq S^{\comp}$, and so $\lambda \mapsto x_{\lambda}$ is frequently contained in $S^{\comp}$ as it is frequently contained in $F$.  But then (\cref{mpr3.2.5} again) $\lambda \mapsto x_{\lambda}$ is eventually contained in $S$, and we are done. 
\end{proof}
\end{prp}
\begin{prp}{}{prp3.3.16}
Let $X$ be a topological space.  Then, $X$ is quasicompact iff every ultranet in $X$ converges.
\begin{proof}
$(\Rightarrow )$ Suppose that $X$ is quasicompact.  We first show that, for every net $\lambda \mapsto x_{\lambda}\in X$, there is some $x_{\infty}\in X$ that has the property that $\lambda \mapsto x_{\lambda}$ is frequently contained in $U$ for every open neighborhood $U$ of $x_{\infty}$.  We proceed by contradiction:  suppose that there is a net $\lambda \mapsto x_{\lambda}\mapsto x_{\lambda}\in X$ that has the property that for every $x\in X$ there is some open neighborhood $U_x\subseteq X$ of $x$ for which it is not the case that $\lambda \mapsto x_{\lambda}$ if frequently contained in $U_x$.  This means that $\lambda \mapsto x_{\lambda}$ is eventually contained in $U_x^{\comp}$ for all $x\in X$ (\cref{mpr3.2.5}), and hence every subnet of $\lambda \mapsto x_{\lambda}$ must eventually be contained in $U_x^{\comp}$ for all $x\in X$.  But then $\lambda \mapsto x_{\lambda}$ doesn't have any convergent subnet:  a contradiction.  Thus, it is indeed the case that for every net $\lambda \mapsto x_{\lambda}\in X$, there is some $x_{\infty}\in X$ that has the property that $\lambda \mapsto x_{\lambda}$ is frequently contained in $U$ for every open neighborhood $U$ of $x_{\infty}$.

Now take $\lambda \mapsto x_{\lambda}\in X$ be an ultranet.  It then follows by the previous proposition (\cref{prp3.8.42}) that in fact $\lambda \mapsto x_{\lambda}$ is \emph{eventually} contained in every open neighborhood $U$ of $x_{\infty}$, and hence converges to $x_{\infty}$, as desired.

\blankline
\noindent
$(\Leftarrow )$ Suppose that every ultranet in $X$ converges.  Then, by the previous result (\cref{prp3.8.43}), in particular every net in $X$ has a convergent subnet, and hence $X$ is quasicompact.
\end{proof}
\end{prp}
And now, for convenience, we collection all characterizations together.
\begin{thm}{}{thm3.8.41}
Let $X$ be a topological space.  Then, the following are equivalent.
\begin{enumerate}
\item \label{thm3.8.41.i}$X$ is quasicompact.\footnote{That is, every open cover of $X$ has a finite subcover.}
\item \label{thm3.8.41.ii}Every collection of closed sets in $X$ that has the property that every finite intersection is nonempty, has nonempty intersection.
\item \label{thm3.8.41.iii}Every net in $X$ has a convergent subnet.
\item \label{thm3.8.41.iv}Every ultranet in $X$ converges.
\end{enumerate}
\begin{proof}
$(\cref{thm3.8.41.i}\Leftrightarrow \cref{thm3.8.41.ii})$ \cref{prp4.2.32}

\blankline
\noindent
$(\cref{thm3.8.41.i}\Leftrightarrow \cref{thm3.8.41.iii})$ \cref{prp4.2.31}

\blankline
\noindent
$(\cref{thm3.8.41.i}\Leftrightarrow \cref{thm3.8.41.iv})$ \cref{prp3.3.16}
\end{proof}
\end{thm}